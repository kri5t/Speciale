\documentclass[twoside,11pt,openright]{report}

\usepackage[latin1]{inputenc}
\usepackage[american]{babel}
\usepackage{a4}
\usepackage{latexsym}
\usepackage{amssymb}
\usepackage{amsmath}
\usepackage{epsfig}
\usepackage[T1]{fontenc}
\usepackage{lmodern}
\usepackage[labeled]{multibib}
\usepackage{color}
\usepackage{datetime}
\usepackage{epstopdf}
\usepackage{graphicx}
\usepackage{subcaption}
\usepackage{hyperref}
\usepackage{enumitem} %Used to make lists smaller
\usepackage{float}
\usepackage[table]{xcolor}
\usepackage{todonotes}
\usepackage{menukeys}

\renewcommand*\ttdefault{txtt}

%\newcommand{\todo}[1]{{\color[rgb]{.5,0,0}\textbf{$\blacktriangleright$#1$\blacktriangleleft$}}}

% \newcites{A,B}{Primary Bibliography,Secondary Bibliography}

% see http://imf.au.dk/system/latex/bog/

\begin{document}
\subsubsection{Application Content:}
\textbf{Project title:} An Artificial Neural Network Approach: Combining Green Energy Production and Electricity Price Forecasts to Support Green Decision Making
\\[0.5cm]
It is our intent to forecast green energy production and electricity prices by modeling an Artificial Neural Network using the back propagation algorithm for training of the network. The dataset for training consists of historical data that is relevant to the specific task. For instance when dealing with forecasting of wind power production for a wind farm, the most influential factors are meteorological such as wind speed, wind direction and air density together with historical production development of that particular farm. 

The goal is to investigate whether or not Back Propagation Neural Networks are a proper technology for doing prediction of green energy and electricity prices. Secondly we are combining the two predictions to support green decision-making. By green decision-making is basically meant to highlight the points in time where the price is at its lowest and at the same time contains the highest percentage of green energy.

\end{document}