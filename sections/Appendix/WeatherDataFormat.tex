The format description is taken directly from The National Oceanic and Atmospheric Administration's (NOAA) National Climatic Data Center (NCDC)\footnote{\url{http://www.ncdc.noaa.gov/}} when receiving the data. The format is in fixed length and the highlighted lines are used in this thesis. \newline 

\noindent SURFACE HOURLY ABBREVIATED FORMAT \newline

\noindent ONE HEADER RECORD FOLLOWED BY DATA RECORDS: \newline

\noindent COLUMN  DATA DESCRIPTION \newline

\noindent 01-06   USAF = AIR FORCE CATALOG STATION NUMBER \newline 
08-12   WBAN = NCDC WBAN NUMBER \newline
14-25   YR--MODAHRMN = YEAR-MONTH-DAY-HOUR-MINUTE IN GREENWICH MEAN TIME (GMT) \newline
\textbf{27-29   DIR = WIND DIRECTION IN COMPASS DEGREES, 990 = VARIABLE, REPORTED AS
        '***' WHEN AIR IS CALM (SPD WILL THEN BE 000)} \newline
\textbf{31-37   SPD \& GUS = WIND SPEED \& GUST IN MILES PER HOUR} \newline   
39-41   CLG = CLOUD CEILING--LOWEST OPAQUE LAYER
        WITH 5/8 OR GREATER COVERAGE, IN HUNDREDS OF FEET,
        722 = UNLIMITED \newline
43-45   SKC = SKY COVER -- CLR-CLEAR, SCT-SCATTERED-1/8 TO 4/8,
        BKN-BROKEN-5/8 TO 7/8, OVC-OVERCAST,
        OBS-OBSCURED, POB-PARTIAL OBSCURATION \newline  
47-47   L = LOW CLOUD TYPE, SEE BELOW\newline
49-49   M = MIDDLE CLOUD TYPE, SEE BELOW\newline
51-51   H = HIGH CLOUD TYPE, SEE BELOW  \newline
53-56   VSB = VISIBILITY IN STATUTE MILES TO NEAREST TENTH
        NOTE: FOR SOME STATIONS, VISIBILITY IS REPORTED ONLY UP TO A
        MAXIMUM OF 7 OR 10 MILES IN METAR OBSERVATIONS, BUT TO HIGHER
        VALUES IN SYNOPTIC OBSERVATIONS, WHICH CAUSES THE VALUES TO 
        FLUCTUATE FROM ONE DATA RECORD TO THE NEXT.  ALSO, VALUES
        ORIGINALLY REPORTED AS '10' MAY APPEAR AS '10.1' DUE TO DATA
        BEING ARCHIVED IN METRIC UNITS AND CONVERTED BACK TO ENGLISH.\newline
58-68   MW MW MW MW = MANUALLY OBSERVED PRESENT WEATHER--LISTED BELOW IN PRESENT WEATHER TABLE\newline
70-80   AW AW AW AW = AUTO-OBSERVED PRESENT WEATHER--LISTED BELOW IN PRESENT WEATHER TABLE\newline
82-82   W = PAST WEATHER INDICATOR, SEE BELOW\newline
\textbf{84-92   TEMP \& DEWP = TEMPERATURE \& DEW POINT IN FAHRENHEIT} \newline
\textbf{94-99   SLP = SEA LEVEL PRESSURE IN MILLIBARS TO NEAREST TENTH} \newline
101-105   ALT = ALTIMETER SETTING IN INCHES TO NEAREST HUNDREDTH \newline
\textbf{107-112   STP = STATION PRESSURE IN MILLIBARS TO NEAREST TENTH}\newline
114-116  MAX = MAXIMUM TEMPERATURE IN FAHRENHEIT (TIME PERIOD VARIES)\newline
118-120 MIN = MINIMUM TEMPERATURE IN FAHRENHEIT (TIME PERIOD VARIES)\newline
122-126 PCP01 = 1-HOUR LIQUID PRECIP REPORT IN INCHES AND HUNDREDTHS --
        THAT IS, THE PRECIP FOR THE PRECEDING 1 HOUR PERIOD\newline
128-132 PCP06 = 6-HOUR LIQUID PRECIP REPORT IN INCHES AND HUNDREDTHS --
        THAT IS, THE PRECIP FOR THE PRECEDING 6 HOUR PERIOD\newline
134-138 PCP24 = 24-HOUR LIQUID PRECIP REPORT IN INCHES AND HUNDREDTHS
        THAT IS, THE PRECIP FOR THE PRECEDING 24 HOUR PERIOD\newline
140-144 PCPXX = LIQUID PRECIP REPORT IN INCHES AND HUNDREDTHS, FOR
        A PERIOD OTHER THAN 1, 6, OR 24 HOURS (USUALLY FOR 12 HOUR PERIOD
        FOR STATIONS OUTSIDE THE U.S., AND FOR 3 HOUR PERIOD FOR THE U.S.)
        T = TRACE FOR ANY PRECIP FIELD\newline
146-147 SD = SNOW DEPTH IN INCHES \newline