The ambition of this dissertation has been to examine the feasibility of a Back Propagation Artificial Neural Network as a technology for prediction of the electricity price and wind power production in the Danish electricity market. The feasibility is derived by the dataset analysis, experimental results and the potential for practical use. Feedforward Neural Network with Resilient Backpropagation has been used as prediction model. We divide the hypotheses into three categories and describe how they have been fulfilled one by one.

\begin{itemize}
\item Dataset analysis --- The purpose of the dataset analysis was to identify influential factors to be tested in the experiments. The dataset analysis have been done by identifying the characteristics of the electricity price and wind power based on related work within the field. The correlation between the individual properties for price and wind power have been established through a comprehensive analysis of the collected data. These data consists of hourly meteorological factors, demand and electricity prices as well as wind power. The analysis was conducted by establishing correlations statistically and systematically investigating the characteristics in relation to the electricity price and wind power in the Danish electricity market. The electricity price showed to be influenced by last known price, demand, time of day, seasonality and wind speed. Wind power was as expected highly influenced by wind speed and last known production but also by demand, air density, wind direction, time of day and seasonality. A volatile nature was established for both but price more significantly than wind power. Furthermore the need for transparency in input parameters and their representation was identified to be important for understanding the context. The dataset analysis approach clarified the influential factors sufficiently and the results helped shape the hypotheses of the experiments. 
\item Experimental results --- The experimental results have the purpose of either verifying or rejecting the established hypotheses to obtain the best Artificial Neural Network models for both wind power and electricity prediction. The experiments were divided into sub-experiments with the purpose of handling different aspects of the analysis such as best input and data representation. Further fine-tuning was included in sub-experiments such as black box optimization and the impact of step-ahead forecasting. The experiments was conducted to uncover as many scenarios as possible but also designed to improve from experiment to experiment. The best input combination for wind power was wind speed, temperature, last known wind power and time of day where temperature was found to substitute consumption. All inputs were normalized and used directly as input, but time of day was divided into a matrix representation. The black box optimization showed a training set size of three months and 300 epochs of training with an accuracy in MAE of 121,02 MWh out of an interval of 0-2800 MWh. The best inputs for electricity price was last known price, demand, temperature, wind speed, time of day, day of week and season of year. The influence of wind speed on the electricity price shows the connection to wind power in the Danish market. All timely and seasonal inputs were represented as matrices. The dataset was trimmed with 1\% in top and bottom to remove irregularities. Best number of epochs was 500 on a six months training set with a MAE of 45,11 DKK out of an interval of 61-632 DKK. The many experiments have made it possible to cover all input combinations and thereby establishing the correct combinations and their representations. The experiments was conducted on unseen data and contributed to the transparency of and trust in the system by simulating day-ahead predictions over an entire year.
\item Practical use --- The analysis and experiments have been used to relate the results to the potential of practical use. The experimental results indicate practical use by simulating day-ahead prediction for a whole year and still maintaining a steady result. The need for transparency and trust is of utmost importance when taking a decision in the electricity market due to the high-risk nature of the financial markets and therefore it is closely related to the feasibility. The price or wind power cannot stand alone without being accompanied by elaborating information of what and how it was obtained. The problem that we are facing while modelling can be transferred to the subject of decision making in terms of trusting the dataset and its representation by communicating the choices behind. It comes down to making the black box more transparent.
\end{itemize}

\noindent Our results show that the day-ahead predictions in both cases are capable of approaching its target over an entire year. Both wind power and electricity price follows the correct direction of the curve in 70\% of the predictions. The achieved Mean Average Error for wind power and electricity prices are 121,02 MWh and 45,11 DKK, respectively. The findings throughout the thesis show the feasibility of Artificial Neural Network as a technology for prediction as well as for decision support.  