The ambition of this dissertation is to examine the feasibility of a Resilient Backpropagation Feedforward Neural Network as a technology for prediction of the electricity price and wind power production in the Danish electricity market. The feasibility is derived based on this structure by the dataset analysis, experimental results and the potential for practical use. We divide the hypotheses into three categories and describe how they have been fulfilled one by one.

\subsubsection{Dataset analysis}
The purpose of the dataset analysis was to identify influential factors to be tested in the experiments. The dataset analysis have been carried out by identifying the characteristics of the electricity price and wind power based on related work within the field. The correlation between the individual properties for price and wind power have been established through a comprehensive analysis of the collected data. These data consist of hourly meteorological factors, demand and electricity prices as well as wind power. The analysis was conducted by establishing correlations statistically and then systematically investigating the characteristics in relation to the electricity price and wind power in the Danish electricity market. The electricity price showed to be influenced by last known price, demand, time of day, seasonality and wind speed. Wind power was as expected highly influenced by wind speed and last known production, but also by demand, air density, wind direction, time of day and seasonality. A volatile nature was established for both factors, but price was more significant than that of wind power. Furthermore the need for transparency in input parameters and their representation was identified to be important for understanding the context. The dataset analysis approach clarified the influential factors sufficiently and the results helped shape the hypotheses of the experiments. 
\subsubsection{Experimental results}
The experimental results have the purpose of either verifying or rejecting the established hypotheses to obtain the best Feedforward Neural Network models for both wind power and electricity prediction. The experiments were divided into sub-experiments with the purpose of handling different aspects of the analysis such as best input and data representation. Further fine-tuning was included in sub-experiments such as black box optimization and the impact of step-ahead forecasting. The experiments were conducted to uncover as many scenarios as possible but also designed to improve gradually through each experiment cycle. The best input combination for wind power was wind speed, temperature, last known wind power and time of day. All inputs were normalized and used directly as input, but time of day was divided into a matrix representation. The black box optimization showed a training set size of three months and 300 epochs of training with a mean average error of 121,02 MWh out of an interval of 0-2800 MWh. The best inputs for electricity price were last known price, demand, temperature, wind speed, time of day, day of week and season of year. The influence of wind speed on the electricity price shows its connection to wind power. All timely and seasonal inputs were represented as matrices. The dataset was trimmed with 1\% in top and bottom to remove irregularities. The best number of epochs was 500 and a training set size of six months with a mean average error of 45,11 DKK out of an interval of 61-632 DKK. The many experiments made it possible to cover all input combinations and thereby established the correct combinations and their representations. The experiments were conducted on unseen data and contributed to the transparency of and trust in the system by simulating day-ahead predictions over an entire year.
\subsubsection{Practical use}
The analysis and experiments have been used to relate the results to the potential of practical use. The experimental results indicate practical use by simulating day-ahead predictions for a whole year and still maintaining a steady result. The need for transparency and trust is of utmost importance when taking a decision in the electricity market due to the high-risk nature of the financial markets and therefore it is closely related to the feasibility of the system. The price or wind power cannot stand alone without being accompanied by information elaborating on what and how it was obtained. The problem that we are facing while modeling can be transferred to the subject of decision making in terms of trusting the dataset and its representation by communicating the choices behind. It all comes down to opening the black box.
\\[1cm]
The findings throughout the thesis show the feasibility of Feedforward Neural Networks as a technology for prediction as well as for decision support. The results of the thesis have been summed up as follows.

\begin{itemize}
\item The dataset analysis together with experiments have shaped the datasets into utilizing the inner workings of the Feedforward Neural Network models to achieve the best results under the given circumstances. The data is based on two years of hourly observations from the Danish electricity market and weather data obtained from Danish Meteorological Stations. The best inputs and their representation have been verified through comprehensive experimentation.   
\item The experiments illustrate how the modelled Feedforward Neural Networks are capable of predicting all days of the year. The experiments simulate real life scenarios based on real data identified through our analysis. The many experiments conducted establish robustness and stability in the results that is manifested in a Mean Average Error for wind power and electricity prices of 121,02 MWh and 45,11 DKK, respectively. Furthermore, wind power and electricity prices follow the correct direction in 70\% of all hours in the day-ahead predictions. 
\item The dataset analysis and experimental results combined emphasize the practical use; if both transparency and trustworthiness is achieved. We have shown the importance of having in-depth knowledge of the underlying data and verifying it when modeling the Feedforward Neural Networks. The importance can be directly transferred to the feasibility of decision making based on an Feedforward Neural Network due to the uncertain and volatile nature of the electricity price and wind power. Trustworthiness is ensured by the robustness of the experimental results but also by transparency in the underlying data through extensive analysis. 
\end{itemize}
