%\todo{SKAL VAERE HELT KLART AT DET BYGGER PAA SPECIALET}
%If this is not supposed to be used in a real life setting then what is it supposed to be?
%Because price cannot be presented on its own.
%Not always the best result. We have to communicate that.
The previous discussions emphasize the importance of simulating a real life setting when predicting with the Artificial Neural Network. It has been important for us to ensure trustworthiness in our results through transparency and day-ahead simulations over an entire year because the purpose of the ANN --- as we see it --- is to provide the fundamentals for decision making in practice. The results must therefore give a realistic picture of the performance. Many of the findings throughout the thesis can be directly mapped and related to issues that would be relevant for a system using our model for prediction --- such a system is a Decision Support System (DSS) as described in Section~\ref{sec:dssSection}. Some of the issues we have encountered during modelling of the network are transferable to the DSS which puts extra emphasize on why analysis and proper experimentation should be made explicit in the first place. We have found the following to be relevant for the discussion here:

\begin{itemize}
\item It has been established that results of the Artificial Neural Network is directly dependent on the dataset used for training --- the dataset was selected based on an extensive analysis in combination with different strategies and the experiments validating the dataset. The need for explicitly informing about what constituted the dataset (and thereby the result) can be transferred directly to the specific purpose of a Decision Support System, i.e. trust in result is necessary in decision making and must therefore be obtained through transparency of the underlying data and why has been included.
\item The necessity of performing exhaustive experiments on unseen data has been found to be of utmost importance to establishing realistic results. The trustworthiness of the Artificial Neural Network is much dependent on the performed experiments because many use cases exists. The results must be ensured by covering enough scenarios, e.g. it is not feasible to have a DSS that can run into a situation that cannot be handled by the system if it is not explicitly defined. The same applies for trimming which must also be handled. It is again a matter of ensuring trust through transparency.
\end{itemize}

The above points are related to trust and how it can be communicated through explicitly informing about the underlying information and how the result is obtained. The underlying information can be about the dataset but also what situations the model is capable of predicting and to what extend.

\subsubsection{Decision Support System based on ANN}
The ANN as a DSS can not stand on its own; a single price without any other information (about how this result was reached) is not enough to base a decision on. As we have pointed out in the input parameter discussion in Section \ref{sec:inputParameterDiscussion} and data manipulation discussion in Section \ref{sec:matrixTrimmingDiscussion} the need for transparency, clarification of the result and trustworthiness are just as important as the results when a decision has to be based on a DSS. What makes us trust our results when doing the actual modeling of the ANN is exactly what makes any others trust it.

Artificial Neural Networks can be used as the foundation for a prediction support system that is created in three steps\cite{shim2002past}. First step is to analyse the data and the problem at hand. Then find a suitable solution for the problem and lastly present the results in a digestible format to the client. In our analysis and experiments we have accounted for the first two steps in creating a DSS. The last step requires a whole survey on its own and are out of the scope of this thesis. Our goal with the discussion here is to clarify in what way our results show challenges in relation to decision making and what preconditions are necessary before the third step from\cite{shim2002past}. We do not come up with any solution but highlight the problematic areas based on our own results. 

The dataset used for a prediction can be both comprehensive and complex and a decision based on an ANN can only be as informed as the underlying dataset. A thorough understanding of both what kind of preprocessing has been applied to the dataset and where the data originates from are required to gain trust in the system as discussed in Section \ref{sec:inputParameterDiscussion} and \ref{sec:matrixTrimmingDiscussion}. Another factor is the insight in how the decision was reached by the system. This calls for transparency and contradicts the black-box nature of Artificial Neural Networks \cite{fromBlackBoxToTransparentBox}. We have in the earlier discussions elaborated on these problems in regard to our own ANN and in comparison to other peoples work in the field. In this section we will elaborate on these concerns and emphasize what must be considered when founding a DSS on an ANN.

\subsection{Trust in the system - the underlying data}
The underlying data of a prediction is what drives the Artificial Neural Network. It is the core of the prediction and without it we have no prediction see (Section \ref{sec:inputParameterDiscussion}). As introduced in Section \ref{sec:uncertainInformation} autonomous systems do not always guarantee the best result and can instead contribute to uncertainties in complex environments like the electricity market which makes the underlying data important for the best possible result to be obtained --- the point being that it is not sufficient to simply hand over the Artificial Neural Network after it has been modelled without including all the surrounding information that was established during the analysis and experiments. We state in Section~\ref{sec:inputParameterDiscussion} that for others to trust our results they have to be made aware of the underlying data since the electricity price or wind power predictions only constitute a number without these underlying informations. We argue that this information is equally important to a potential user of a DSS as it is for us to validate our hypotheses. The ANN must be accompanied with this information in order for anyone to base a decision on the price or wind power alone, i.e. the exact input data should always be presented together with the output. The system must communicate the input parameters and their origin at all times. Knowing where things come from and exactly what input parameters constitute a result is the basis for creating trust. Another aspect of the input data and creating trust is the need to know what is accounted for in the prediction. As a simple example consider air density as a substitute for both temperature and consumption in the wind power prediction (discussed in Section \ref{sec:inputParameterDiscussion}). Consumption can be substituted by temperature and air density can reflect temperature due to pressure being close to constant --- the trust is directly influenced by this knowledge because the intuition might look for consumption in that calculation. In continuation of this is the situations and scenarios manageable by the proposed network model in relation to what have actually been simulated in experiments which is directly related to the discussion about the size of the testing set in Section~\ref{sec:inputParameterDiscussion} and how different offsets affect the accuracy in Section~\ref{sec:offsetsDiscussion}. 
 
What kind of preprocessing the data has undergone before used as input in the ANN must also be taken into consideration. In the discussion about trimming in Section \ref{sec:matrixTrimmingDiscussion} and calculated inputs in Section \ref{sec:calculatedInputDiscussion} we point out how data manipulation and preprocessing can alter the results of the ANN --- trimming away price or wind power irregularities in the training set cannot be done for the testing set because it would mean trimming away values we do not know yet.  This is a way of handling spike prices as discussed in Section \ref{sec:matrixTrimmingDiscussion} but the outcome is the disability in predicting the highest values when these values are met during actual prediction. It opens up the potential for making such parameters configurable.

Prediction of wind power and the electricity price is based on information that is constantly changing. It will have an impact on the prediction and therefore the information must be communicated to the rightful owner because we are dealing with high-risk decisions. Often the data is uncertain information in the sense that it will always be an estimate of what the real price or wind power will be. Even some of the inputs to the ANN are based on forecasts e.g. meteorological factors and demand as discussed in Section \ref{sec:inputParameterDiscussion}. This leaves us with a trail of possible sources of error that has to be communicated in the DSS. It is important that we do not leave anything out to hide the fact that there are a lot of unknown dynamics in a neural network that we have no control over \cite{young2010using}. We just have to embrace these unknowns and try to enlighten the user in the best possible way. Error visualization is also important when talking about uncertain information. We need to show the user what the average error margins are on the selected configuration.

The importance of how accurate the result has to be are also reflected in the use of the system. If the user only needs to know the trend of the system the accuracy are not as important as how many times we guess the right direction of the curve movement. Also the closer we get to the hour that needs predicting the better the results get (see Section \ref{sec:stepAheadForecastingDiscussion}) and therefore a trader has to be able to accept a larger error margin when a result for the 24th hour is compared to a result for the 4th hour.

\subsection{Obtaining transparency - Opening the black-box}
We have obtained the best prediction through data manipulation and black box optimization which have been verified through our experiments in Chapter~\ref{ch:experimentalResults}. The network outputs a prediction which in itself could be used as decision support by the traders. The problematic is for the users to know when it can be trusted and why to even trust it. People are not in the habit of blindly trusting everything that is placed in front of them and especially not when it comes to high-risk decisions e.g. trading stocks, buying electricity etc.

Artificial Neural Networks are known to be a black box \cite{fromBlackBoxToTransparentBox} because the only knowledge consist in the input and output but nothing about the internal logic. There is a need for making the black box more transparent by communicating what is happening inside it. This is a field of research related to ANNs where the researchers attempt to open up the artificial neural network to give the user a result that also communicates how the decision was reached. This is called a grey-box and refers to different methods that seek to communicate how the result of an ANN was reached \cite{young2010using}. One of the methods is called decomposition and attempts to map the interconnections, weights and structures of the neural network to give the user a sense of trust and insight in how the result was conducted. Another approach is pedagogical where the inner workings of the neural networks are ignored but instead tries to map the input to the output data thus explaining the relationship between the two. The last method eclectic is a hybrid between the two but is not commonly implemented in practice. We do not want to distinguish between the methods and point out the best one --- since we have limited insight in these methods. We just want to emphasize the need for transparency in how the decisions are reached in ANNs and how these methods can help the user understand the inner workings.

\subsection{Concluding remarks}
In this section we have presented some of the problems and possible downfalls when modelling a DSS around an ANN based on our analysis and the experimental result discussions. We pointed out that the trust in the system heavily relies on the underlying data and what manipulation have been applied to it. We argue that an informed choice is not possible without it being accompanied by information about the influential factors due to the high-risk nature of electricity prices. One way to make transparency in the underlying data is by always explicitly describing what constituted the prediction in terms of input but there is also a potential in "opening" the black box by communicating what happened inside of it with decomposition where weights and structures are explained --- these methods involve communicating highly complex structures in a user interface. Furthermore our experimental results indicated the potential of decision making by simulating predictions for all days of a year.

We see the feasibility of the ANN directly connected to the potential of using it in a real system for decision making --- this is its rightful place. We argue that if taking the above challenges into consideration and making them preconditions when building a DSS on top of the proposed model then it will be more likely to succeed. We emphasize that it is not a final answer but considerations as a result of our analysis, experiments and discussions.

%The need for presenting uncertain information is presented in~\ref{sec:uncertainInformation}.
%f.x. the difference in accuracy when in the lowest and highest numbers.

%\subsection{How to make DSS a reality!?!?!?!}
%\begin{itemize}
%\item Improve understanding of the underlying data and the output;
%\item Transparency is the goal;
%\item Let the user take decisions based on the uncertain information;
%\item Visualize errors;
%\end{itemize}

%\todo{the success of the predictions in the dss rely heavily on the use.. users are important!}
%\todo{... videregive information og dele. Vores bedste resultat er ikke gaeldende alle steder jvf. 24 timer ahead fra forskellige starting points. Det underliggende bliver noedt til at vaere gjort klart for den, der bruger systemet}