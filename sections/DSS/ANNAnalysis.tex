Artificial neural networks are to some extent being used as the foundation for a prediction support system \cite{shim2002past}. In \cite{shim2002past} they present how a decision support system is created in three steps. First analyze the data and the problem at hand. Find a suitable solution for the problem and at last present the results in a digestible format to the client. In our analysis and experiments we have accounted for the first two steps in creating a decision support system. The last step requires a whole survey on its own and are out of the scope of this thesis. Our goal in this section is to clarify what preconditions must be met before an artificial neural network can be used for decision making and what challenges this introduces. A neural network conducts an analysis of a specific dataset and returns a result based on these parameters. The dataset used for a prediction can be both comprehensive and complex. Therefore a decision based on an ANN can only be as informed as the underlying dataset. A thorough understanding of both what kind of preprocessing has been applied to the dataset and where the data originates from are required to gain trust in the system. Another factor is the transparency of how the decision was reached by the system. This calls for transparency and contradicts the black-box nature of artificial neural networks \cite{fromBlackBoxToTransparentBox}. In this section we will elaborate on these concerns and propose solutions to how these goals can more easily be achieved.

\subsection{Trust in the system - the underlying data}
The underlying data of a prediction is what drives the artificial neural network. It is the core of the prediction and without it we have no prediction. This points out why trusting the dataset is important for the user to trust the predictions presented by the system. The underlying data should always be shown together with the output so the user gets an idea of what inputs resulted in what output. The system must communicate the input parameters and their origin at all times. Knowing where things come from and exactly what input parameters constitute a result is the basis for creating trust. As introduced in~\ref{sec:uncertainInformation} autonomous systems do not always guarantee the best result and the underlying data are important for the best possible result to be obtained. Users might have different preferred sources for the underlying data. Accommodating these needs and building different versions of the system on top of the data from different suppliers will help the system to gain the users trust.

Furthermore the user of the system has to know what kind of preprocessing the data has undergone before it was used as input in the ANN. In the discussion about trimming \ref{sec:matrixTrimmingDiscussion} and calculated inputs \ref{sec:calculatedInputDiscussion} we point out how data manipulation and preprocessing can alter the results of the ANN. These parameters are dynamic and can be applied or left out at the beginning of a prediction iteration. This emphasizes that these kinds of systems are primarily for experts that have prior knowledge in the field and knows (or can easily be introduced to) the benefits or disadvantages of data preprocessing.

\todo{discuss in relation to other texts. Many texts we did not know how to trust because we had no idea of the underlying data and what they used and how they manipulated}

\subsection{Obtaining transparency - Opening the black-box}
We have obtained the best possible prediction through data manipulation and black box optimization which have been verified through our experiments in Chapter~\ref{ch:experimentalResults}. The network outputs a prediction which in itself could be used as decision support by the traders. The problematic is for the users to know when it can be trusted and why to even trust it. People are not in the habit of blindly trusting everything that is placed in front of them and especially not when it comes to high-risk decisions e.g. trading stocks, buying electricity etc.

Artificial Neural Networks are known to be a black box \cite{fromBlackBoxToTransparentBox} because the only knowledge consist in the input and output but nothing about the internal logic. There is a need for making the black box more transparent by communicating what is happening inside it. This is a field of research related to ANNs where the researchers attempt to open up the artificial neural network to give the user a result that also communicates how the decision was reached. This is called grey-box and refers to different methods that seek to communicate how the result of an ANN was reached \cite{young2010using}. One of the methods is called decomposition and seeks to map the interconnections, weights and structures of the neural network to give the user a sense of trust and insight in how the result was conducted. Another approach is pedagogical where the inner workings of the neural networks are ignored but instead tries to map the input to the output data thus explaining the relationship between the two. The last method eclectic is a hybrid between the two but is not commonly implemented in practice. We do not want to distinguish between the methods and point out the best one - since we have limited insight in these methods. We just want to emphasize the need for transparency in how the decisions was reached in ANNs and how these methods can help the user understand the inner workings.

\subsection{Presenting uncertain information}
All of the above data has to be presented to the user. Often this data is uncertain information in the sense that it can either be an estimate or preprocessed data. We both present estimates to the user in the sense of our predictions but some of the input parameters are predictions as well. This leaves us with a trail of possible sources of error that has to be communicated to the user. It is important that we do not leave anything out to hide the fact that there are a lot of unknown dynamics in a neural network that we have no control over \cite{young2010using}. We just have to embrace these unknowns and try to enlighten the user in the best possible way. Error visualization is also important when talking about uncertain information. We need to show the user what the average error margins are on the selected configuration.

The importance of how accurate the result has to be are also reflected in the use of the system. If the trader only needs to know the trend of the system the accuracy are not as important as how many times we guess the right direction of the curve movement. Also the closer we get to the hour that needs prediction the better the results get \ref{sec:stepAheadForecastingDiscussion} therefore a trader has to be able to accept a bigger error margin when a result for the 24th hour is compared to a result for the 4th hour. This just further emphasizes the importance of the knowledge already held by the users of the system and how well educated in it they are. A neural network is highly configurable and has a lot of parameters that can be tweaked to give us the best result. We of course try to communicate the best possible solution in a given scenario. The user needs to have some expert knowledge to know whether this solution seems reasonable in the given situation or if some of the input parameters, network parameters or preprocessing parameters should be tweaked to obtain a better result. Again these tweak-able parameters can be perceived as uncertain information \cite{UncertainInformation} in the sense that they are not quantifiable and we do not know exactly the right combination that will give the best result in a given situation. 

\todo{Snak omkring vores emulation af et rigtig setup. Derfor mener vi godt at det kan bruges som et decision support system. Maaske som en konklusion paa denne sektion.}

%The need for presenting uncertain information is presented in~\ref{sec:uncertainInformation}.
%f.x. the difference in accuracy when in the lowest and highest numbers.

%\subsection{How to make DSS a reality!?!?!?!}
%\begin{itemize}
%\item Improve understanding of the underlying data and the output;
%\item Transparency is the goal;
%\item Let the user take decisions based on the uncertain information;
%\item Visualize errors;
%\end{itemize}

%\todo{the success of the predictions in the dss rely heavily on the use.. users are important!}
%\todo{... videregive information og dele. Vores bedste resultat er ikke gaeldende alle steder jvf. 24 timer ahead fra forskellige starting points. Det underliggende bliver noedt til at vaere gjort klart for den, der bruger systemet}