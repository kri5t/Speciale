\subsection{Experimental results}
This section contains experimental results from the experiment conducted to find the best possible price predicting solution. The results are divided into 5 experiments. The first experiment finds the best combination of input parameters for the network. These input parameters count meteorological, social and seasonal factors - that we identified in section~\ref{sec:Price}. The next experimental identifies are the need for trimming and reasons to why it is needed in this particular dataset. The third experiment tests different statistical strategies to incorporate immediate historical curve behavior in the dataset. This includes historical prices, curve behavior analysis, skewness analysis and historical EWMA(Exponentially-Weighted Moving Average). The fourth experiment takes care of the Artificial Neural Network parameters. This includes black-box optimization like pruning of the network and optimization of epochs. Dataset size will also be included in the fourth experiment.

\subsection{Experiment 1: Inputs}
In this section we experimented with the basic input parameters that we identified in section~\ref{sec:Price}. We did a cross comparison of all the inputs and possible combinations (For those that made sense). The Price and the Demand is such a basic measure when you define a price in any markets that they are included in every prediction. We did not make a cross comparison of the Month of Year and the Seasons of Year since they say the same but with different granularity. We did a cross comparison both with and without matrix inputs and with a mix of non-matrix and matrix inputs. This is done to determine whether an input makes better sense on matrix form or just as a simple normalized value. 

\subsubsection{Variables}
Experiment 1 is based on a dataset consisting of the last 3 months averaging to 2189 hours. We use 200 epochs for each training iteration.

The parameters being examined in this test are the following:
\begin{itemize}
	\item Last Hours Price (P)
	\item The Demand (D)
	\item Wind Speed(WS)
	\item Temperature(T)
	\item The Hourly Time of Day (ToD)
	\item The Day of the Week(DoW)
	\item The Month of the Year(MoY)
	\item The Season of the Year(SoY)
\end{itemize}

The (M) is for Matrix input and states if the input in the seasonal rows(ToD, WoD, MoY, SoY) are represented as a matrix.

The table~\ref{table:Top20Prices} contains the following information as well:
\begin{itemize}
	\item Number in table (\#)
	\item First hidden layer neuron count (N1)
	\item Second hidden layer neuron count (N2)
	\item \% Deviation from first entry (Dev)
	\item Mean Average Error (MAE)
	\item The Day of the Week(DoW)
	\item The Month of the Year(MoY)
	\item The Season of the Year(SoY)
\end{itemize}

\subsubsection{Hypothesis}
This is the first and most comprehensive of the tests conducted for price prediction. In this experiment we want to test all the different combinations of the parameters. We expect to see a better curve-fit as we include more variables and more complex variables. In regard to the dataset analysis in section~\ref{sec:ElectricityPriceAnalysis} we see the connection between the aforementioned variables and the price (that we are trying to predict). We have the following expectations based on the dataset analysis:
\begin{itemize}
	\item The Wind Speed will show an impact on the price because of the correlation between the two. Also the wind speed impacts the production of green energy which impacts the overall price of energy.
	\item The Temperature might have a small impact on the price but there is nearly no correlation between the price and the temperature. If it should have an impact it is probably because of the correlation between the temperature and the demand.
	\item The Hourly Time of Day is certainly going to have an effect on the hourly predicted price. Both because we can see a clear trend in price and what time of day it is and because we are predicting the hourly price; which logically indicates that this parameter is important for such a prediction. 
	\item The Day of the Week, The Month of the Year and The Seasons of the Year is expected to have an impact on the prediction. The impact are expected to get weaker as the inputs become more general e.g season of year; since our training dataset is based on year of data. 
	\item Matrix representation will be better than single node representation of the seasonality. This is due to the fact that we get a more granulated representation with the matrix and thus a more precise impact of the parameters. This has also been discussed in section~\ref{sec:Matrix}.
\end{itemize}

\subsubsection{Results}

\begin{table}[h!]
\centering  % used for centering table
\resizebox{\textwidth}{!}{
\begin{tabular}{|c|c|c|c|c|c|c|c|c|c|c|c|c|} % centered columns (7 columns)
\hline
{\#} & P & D & WS & T & ToD & DoW & MoY & SoY & N1 & N2 & MAE & {\%} Dev\\ [0.5ex] % inserts table 
%heading
\hline                  % inserts single horizontal line
1  &  \x    & \x    & \x    & \x    & \x\m  & \x\m  &       & \x\m  & 7  & 8  & 57.12 &  - \\ \hline
2  &  \x    & \x    & \x    & \x    & \x\m  & \x    &       & \x\m  & 5  & 9  & 58.09 & 1.7\% \\ \hline
3  &  \x    & \x    & \x    & \x    & \x\m  &       & \x\m  &       & 6  & 11 & 58.79 & 2.92\% \\ \hline
4  &  \x    & \x    & \x    &       & \x\m  & \x\m  & \x\m  &       & 9  & 3  & 60.14 & 5.29\% \\ \hline
5  &  \x    & \x    & \x    & \x    & \x\m  & \x    &       &       & 16 & 8  & 62.19 & 8.88\% \\ \hline
6  &  \x    & \x    & \x    & \x    & \x\m  &       &       & \x\m  & 6  & 5  & 62.26 & 9.0\% \\ \hline
7  &  \x    & \x    & \x    & \x    & \x\m  & \x    & \x\m  &       & 5  & 8  & 62.84 & 10.01\% \\ \hline
8  &  \x    & \x    & \x    & \x    & \x    & \x    &       & \x\m  & 5  & 0  & 63.94 & 11.94\% \\ \hline
9  &  \x    & \x    & \x    & \x    & \x    & \x\m  & \x\m  &       & 5  & 4  & 64.19 & 12.38\% \\ \hline
10 &  \x    & \x    & \x    &       & \x\m  & \x    & \x\m  &       & 6  & 8  & 64.72 & 13.31\% \\ \hline
11 &  \x    & \x    & \x    &       & \x    & \x    & \x\m  &       & 5  & 5  & 65.07 & 13.92\% \\ \hline
12 &  \x    & \x    & \x    & \x    & \x\m  & \x\m  &       &       & 6  & 6  & 65.95 & 15.46\% \\ \hline
13 &  \x    & \x    & \x    & \x    & \x\m  & \x\m  & \x\m  &       & 7  & 12 & 66.55 & 16.51\% \\ \hline
14 &  \x    & \x    & \x    &       & \x    &       &       & \x\m  & 7  & 0  & 67.21 & 17.66\% \\ \hline
15 &  \x    & \x    & \x    & \x    & \x    & \x    &       &       & 5  & 0  & 67.88 & 18.84\% \\ \hline
16 &  \x    & \x    & \x    &       & \x    & \x    &       & \x    & 6  & 0  & 68.21 & 19.42\% \\ \hline
17 &  \x    & \x    & \x    &       & \x\m  & \x\m  &       &       & 8  & 6  & 68.34 & 19.64\% \\ \hline
18 &  \x    & \x    & \x    &       & \x\m  &       &       & \x\m  & 7  & 4  & 68.35 & 19.66\% \\ \hline
19 &  \x    & \x    & \x    & \x    & \x    & \x    & \x    &       & 6  & 0  & 68.43 & 19.8\% \\ \hline
20 &  \x    & \x    & \x    & \x    & \x    &       &       & \x\m  & 5  & 0  & 68.45 & 19.84\% \\ \hline
 %\x    & \x    & \x    & \x    & \x\m  & \x\m  &       & \x\m  & 57.12 & 61.70 & \#1 \\ \hline
 %\x    & \x    & \x    & \x    & \x\m  & \x    &       & \x\m  & 58.09 & 68.23 & \#2 \\ \hline
 %\x    & \x    & \x    & \x    & \x\m  &       & \x\m  &       & 58.79 & 63.80 & \#3 \\ \hline
 %\x    & \x    & \x    &       & \x\m  & \x\m  & \x\m  &       & 60.14 & 67.82 & \#4 \\ \hline
 %\x    & \x    & \x    & \x    & \x\m  & \x    &       &       & 62.19 & 74.89 & \#5 \\ \hline
 %\x    & \x    & \x    & \x    & \x\m  &       &       & \x\m  & 62.26 & 61.86 & \#6 \\ \hline
 %\x    & \x    & \x    & \x    & \x\m  & \x    & \x\m  &       & 62.84 & 62.62 & \#7 \\ \hline
 %\x    & \x    & \x    & \x    & \x    & \x    &       & \x\m  & 63.94 & 72.37 & \#8 \\ \hline
 %\x    & \x    & \x    & \x    & \x    & \x\m  & \x\m  &       & 64.19 & 84.12 & \#9 \\ \hline
 %\x    & \x    & \x    &       & \x\m  & \x    & \x\m  &       & 64.72 & 72.29 & \#10 \\ \hline

 %\x    & \x    & \x    &       & \x    & \x    & \x\m  &       & 65.07 & 79.71 & \#11 \\ \hline
 %\x    & \x    & \x    & \x    & \x\m  & \x\m  &       &       & 65.95 & 58.19 & \#12 \\ \hline
 %\x    & \x    & \x    & \x    & \x\m  & \x\m  & \x\m  &       & 66.55 & 67.31 & \#13 \\ \hline
 %\x    & \x    & \x    &       & \x    &       &       & \x\m  & 67.21 & 74.67 & \#14 \\ \hline
 %\x    & \x    & \x    & \x    & \x    & \x    &       &       & 67.88 & 77.29 & \#15 \\ \hline
 %\x    & \x    & \x    &       & \x    & \x    &       & \x    & 68.21 & 72.45 & \#16 \\ \hline
 %\x    & \x    & \x    &       & \x\m  & \x\m  &       &       & 68.34 & 72.31 & \#17 \\ \hline
 %\x    & \x    & \x    &       & \x\m  &       &       & \x\m  & 68.35 & 76.25 & \#18 \\ \hline
 %\x    & \x    & \x    & \x    & \x    & \x    & \x    &       & 68.43 & 75.10 & \#19 \\ \hline
 %\x    & \x    & \x    & \x    & \x    &       &       & \x\m  & 68.45 & 75.97 & \#20 \\ \hline 
\end{tabular}
}
\caption{The top 20 results on training set 3 last months} % title of Table
\label{table:Top20Prices} % is used to refer this table in the text
\end{table}

If we take a look at the top 20 best input combinations shown in table~\ref{table:Top20Prices} we see some clear trends. If we start from the beginning the price and demand are static as mentioned earlier since they are fundamental market forces and thus not a changing factor in this analysis. The next input parameter is the Wind Speed. We see that every input combination in the top 20 includes the wind production and it is therefore a must for the prediction of the energy prices. Also we saw in section~\ref{sec:windPowerAnalysis}(table~\ref{table:pearsonCoeficientWindProduction}) that the Wind Speed heavily influences the green energy production and thus influencing the energy prices. \todo{Lav en sammenligning af wind speed og wind production for at udelukke den ene}

The temperature is a less obvious candidate for the prediction of price since the Pearson's correlation between the two only are 0.17. Nevertheless it is showing up in 8/10 top combinations in table~\ref{table:Top20Prices} and this might be because of the correlation between temperature/demand which is -0.59. The temperature is scattered all over the 144 combinations and thus it is hard to say anything about this input with confidence. \todo{Lav en analyse med og uden temperatur paa den bedste}.

The Hourly Time of Day (ToD) is included in every single combination in the top 20. This clearly shows that this input parameter is important for the prediction of the price. This is kind of obvious since what we are predicting is the hourly price. In section~\ref{sec:seasonality}(figure~\ref{fig:price_per_hour}) we saw that the price varied from 190 to 335 which strengthens the importance of the relationship between time of day and the price. Also the top 7 all have the ToD on matrix form which indicates that this is the best way of representing the ToD.

Next we have the Day of the Week (DoW) parameter. This parameter are present in 75\% of the 20 best results (8/10 and in 15/20 best combinations). We have to believe that it plays a significant role in the prediction of price. If we look at the analysis of the average price over weekdays in section~\ref{sec:seasonality}(figure~\ref{fig:price_over_weekdays}) we see that there is a significant difference in price on the different days especially the weekdays compared to the weekend. This parameter is mixed between matrix input and standard input. This might be due to the fact that the biggest difference between days are weekend and weekday thus minimizing the effect of a matrix representation. \todo{Maaske lav et forsoeg med weekday/weekend matrix.}

The last two parameters - Month of Year(MoY) and Season of Year (SoY) - are codependent and will be covered together. As mentioned before they cover the same information and we therefore only need one of them at any time \todo{Lav forsoeg der viser at det ikke giver mening at have baade MoY og SoY paa samme tid.}. The values are present in 9/10 of the best combinations in ~\ref{table:Top20Prices}. This is an indicator that the seasonality in the form of MoY and SoY plays a role in predicting the electricity price. Also we saw in the analysis in section~\ref{sec:seasonality}(figure~\ref{fig:monthlyAveragePrice} and ~\ref{fig:seasons}) that the price changes with seasonality and that it especially was more expensive in the winther than the rest of the year.

The matrix representation in the top 20 is the most common representation of the seasonal inputs (ToD, DoW, MoY and SoY) - with only 3/20 containing no matrix at all. The matrix format was expected to give a better result based on what we discussed in the matrix section~\ref{sec:Matrix}. The 3 combinations that does not contain a matrix representation are already present in the top 20 as pure matrix form. This might be the combination of inputs variables that are a good since both the matrix form and the non-matrix form are represented in the top 20 best combinations.

In the Wind Production Experiments section~\ref{sec:windProductionExperiments} we conducted a separate experiment for the matrix analysis. We did a combined experiment in the price section because we had 4 different parameters that could be represented as a matrix thus giving us a lot of different combinations. We also wanted to test all of the different combinations of matrix and non-matrix inputs because a mixed combination might render a better solution e.g. the season of year shifts so rarely it might be a better representation (together with the other parameters) on non-matrix form. Also there might be combinations with the rest of the inputs that work better with the matrix-format than others.

\todo{Matrix blandes fordi der er mange forskellige. Sammenlign med wind production, hvor der kun er 1. SKRIV HELT KONKRET OM MATRIX OG IKKE MATRIX I FORHOLD TIL AFSNITTET FOER}
\todo{Referer til Brians. Han har allrede skrevet om Matrix i forhold til at hans ikke virker specielt overbevisende, men at det i mit virker meget bedre. Diskuter.}

\begin{table}[H]
\centering  % used for centering table
\resizebox{\textwidth}{!}{
\begin{tabular}{|c|c|c|c|c|c|c|c|c|c|c|} % centered columns (7 columns)
\hline
P & D & WS & T & ToD & DoW & MoY & SoY & MAE & Rank\\ [0.5ex] % inserts table 
%heading
\hline                  % inserts single horizontal line
 \x    & \x    & \x    &       &       & \x\m  & \x\m  &       & 105.70 & \#134 \\ \hline
 \x    & \x    &       &       &       & \x\m  &       & \x\m  & 107.85 & \#135 \\ \hline
 \x    & \x    &       & \x    & \x    &       & \x    &       & 108.37 & \#136 \\ \hline
 \x    & \x    &       & \x    &       &       & \x    &       & 111.14 & \#137 \\ \hline
 \x    & \x    &       & \x    &       &       & \x\m  &       & 111.65 & \#138 \\ \hline
 \x    & \x    &       & \x    &       & \x\m  & \x\m  &       & 113.56 & \#139 \\ \hline
 \x    & \x    &       & \x    &       &       &       & \x    & 115.29 & \#140 \\ \hline
 \x    & \x    &       &       &       &       & \x\m  &       & 115.81 & \#141 \\ \hline
 \x    & \x    &       & \x    &       & \x\m  &       & \x\m  & 115.83 & \#142 \\ \hline
 \x    & \x    &       & \x    &       & \x    & \x    &       & 117.17 & \#143 \\ \hline
 \x    & \x    &       &       &       & \x\m  & \x\m  &       & 117.34 & \#144 \\ \hline
\end{tabular}
}
\caption{The bottom 10 input combinations for price prediction} % title of Table
\label{table:Bottom10Prices} % is used to refer this table in the text
\end{table}

If we take a look at the bottom 10 input combinations in terms of MAE we see some tendencies as well. We see that Wind Speed only appears in one of the combinations and so does the Time of Day(And not in the same combination). This further strengthens that these two variables are very important for the price prediction. We talked about matrix form earlier and its impact on the combinations. In the bottom 20 table we see that there is a lot of the combinations that contains variables on matrix form. This is because the Wind Speed and the Time of Day are the most influential parameters thus sending the combinations with these absent to the bottom of the chart. As these combinations also include matrix forms they are sent to the bottom of the chart. This does not indicate that the matrix format is worse than the others.

As we saw in the wind production experiments(WPE) section~\ref{sec:windPowerAnalysis} there was a problem regarding the way we conducted tests for the seasonality; specifically for the MoY and the SoY. We only use the last 3 months to train the network and that sort of eliminates the obvious purpose of the MoY and SoY. As we saw in the results for the first experiment in WPE if we include the same month as we are in from last year and reintroduce the purpose of the SoY and MoY it gave us a worse result than leaving it out. To make sure the same thing applies on the price prediction experiments we conducted to runs of all the combinations of inputs. First we ran it with the last 3 months including the same month that we are in from last year and after that we ran the experiment with only the last 3 months.
\todo{Er det blevet skrevet?}

Table~\ref{table:Top20Prices} shows us the top 20 MAE from the experiment with a training set containing the last 3 months(3Months). We have compared this to the experiment with a training set containing the last 3 months and the same month from last year(4Months). As we see in table~\ref{table:Top20Prices} the difference between the two datasets are not that big. This also shows in \todo{Ref Appendix for both} that the distribution of the MAE over the two datasets are quite similar. From this we can conclude that the effect of using last years month in our dataset does not make a significant difference and can be left out. This raises the question if the MoY and the SoY can be left out as well, since the obvious use for it is eliminated by never having a full month or full season in the training set - that is equal to the one we predict.

First we conducted an experiment containing a training set with a full year (to test the effect of seasonality on a full year) we included the rest of the parameters equal to rank \#1 in table~\ref{table:Top20Prices} and shifted the seasonality. The results can be seen in table~\ref{table:1YearTrain} where we can see the full effect of seasonality over a year. The table clearly shows that the MoY combination is the best of the three. If we compare the results to the results from table~\ref{table:Top20Prices} it shows us that with monthly seasonality the neural network will be able to do the same predictions as the predictions that only have a training set of 3 months. With the possibility of over training in too large datasets(see citation~\cite{1}) and that the neural network takes longer time training on a larger dataset. We can conclude that the smaller dataset of 3 months are better to use than the full year (We will elaborate on this in experiment X)\todo{Saet rigtigt experiment}. Further more we have to test if seasonality being an input parameter makes sense when our dataset is only 3 months big.

\begin{table}[H]
\centering  % used for centering table
\begin{tabular}{|c|c|c|c|c|c|c|c|c|c|c|} % centered columns (7 columns)
\hline
P & D & WS & T & ToD & DoW & MoY & SoY & MAE & Rank\\ [0.5ex] % inserts table 
\hline
\x    & \x    & \x    & \x    & \x\m  & \x\m  & \x\m  &       & 63.74 & \#1 \\ \hline
\x    & \x    & \x    & \x    & \x\m  & \x\m  &       &       & 90.79 & \#2 \\ \hline
\x    & \x    & \x    & \x    & \x\m  & \x\m  &       & \x\m  & 94.75 & \#3 \\ \hline
\end{tabular}
\caption{The top 20 results on training set 3 last months} % title of Table
\label{table:1YearTrain} % is used to refer this table in the text
\end{table}

%\begin{table}[H]
%\centering  % used for centering table
%\resizebox{\textwidth}{!}{
%	\begin{tabular}{|c|c|c|c|c|c| c c c c c} % centered columns (7 columns)
%	P & D & WS & T & ToD & WoD & MoY & SoY & MAE & Rank\\ [0.5ex] % inserts table 
%	\hline                  % inserts single horizontal line
%	x & x & x & x & x    & x(M) & x(M) &      & 61,95 & \#1 \\ \hline %newPredictions/TEN__MIXEDPrice_Consump_windSpeed_temperatureRow_timeOfDay_weekdaysMATRIX_monthOfYearMATRIX
%	x & x & x & x & x(M) & x    &      & x(M) & 62,76 & \#2 \\ \hline %newPredictions/TEN__MIXEDPrice_Consump_windSpeed_temperatureRow_timeOfDayMATRIX_weekdays",%
%	x & x & x & x & x(M) & x(M) &      & x(M) & 62,87 & \#3 \\ \hline %newPredictions/TEN__MIXEDPrice_Consump_windSpeed_temperatureRow_timeOfDayMATRIX_weekdays_monthOfYearMATRIX",
%	x & x & x &   & x(M) & x    & x(M) &      & 62,99 & \#4 \\ \hline %newPredictions/TEN__MIXEDPrice_Consump_windSpeed_timeOfDayMATRIX_weekdays_monthOfYearMATRIX",
%	x & x & x & x & x(M) & x    & x(M) &      & 64,24 & \#5 \\ \hline %newPredictions/TEN__MATRIX_Price_Consump_windSpeed_temperatureRow_timeOfDay_weekdays_seasonOfYear",
%	x & x & x & x & x(M) & x    &      &      & 65,18 & \#6 \\ \hline %newPredictions/TEN__MIXEDPrice_Consump_windSpeed_temperatureRow_timeOfDayMATRIX_weekdays_seasonOfYearMATRIX",
%	x & x & x & x & x(M) &      & x(M) &      & 65,53 & \#7 \\ \hline %newPredictions/TEN__MIXEDPrice_Consump_windSpeed_temperatureRow_timeOfDayMATRIX_monthOfYearMATRIX",
%	x & x & x & x & x    & x    &      & x(M) & 65,80 & \#8 \\ \hline %newPredictions/TEN__MIXEDPrice_Consump_windSpeed_temperatureRow_timeOfDayMATRIX_seasonOfYearMATRIX",
%	x & x & x & x & x(M) &      &      & x(M) & 67,21 & \#9 \\ \hline %newPredictions/TEN__MIXEDPrice_Consump_windSpeed_temperatureRow_timeOfDay_weekdays_seasonOfYearMATRIX",
%	x & x & x &   & x(M) & x(M) & x(M) &      & 70,25 & \#10 \\ \hline %newPredictions/TEN__MATRIX_Price_Consump_windSpeed_timeOfDay_weekdays_monthOfYear"
%	\hline %inserts single line
%	\end{tabular}
%}
%\caption{Average MAE of ten runs per entry} % title of Table
%\label{table:Top10Average} % is used to refer this table in the text
%\end{table}
%\begin{table}[H]
%\centering  % used for centering table
%\resizebox{\textwidth}{!}{
%\begin{tabular}{|c|c|c|c|c|c|c|c|c|c|c|} % centered columns (7 columns)
%\hline
%P & D & WS & T & ToD & DoW & MoY & SoY & 1 Run MAE & 10 Runs Average & Rank\\ [0.5ex] % inserts table 
%heading
%\hline                  % inserts single horizontal line
% \x    & \x    & \x    & \x    & \x    & \x\m  & \x\m  &       & 64.19 & 61.95 & \#9 \\ \hline
% \x    & \x    & \x    & \x    & \x\m  & \x    &       & \x\m  & 58.09 & 62.76 & \#2 \\ \hline
% \x    & \x    & \x    & \x    & \x\m  & \x\m  &       & \x\m  & 57.12 & 62.87 & \#1 \\ \hline
% \x    & \x    & \x    &       & \x\m  & \x    & \x\m  &       & 64.72 & 62.99 & \#10 \\ \hline
% \x    & \x    & \x    & \x    & \x\m  & \x    & \x\m  &       & 62.84 & 64.23 & \#7 \\ \hline
% \x    & \x    & \x    & \x    & \x\m  & \x    &       &       & 62.19 & 65.17 & \#5 \\ \hline
% \x    & \x    & \x    & \x    & \x\m  &       & \x\m  &       & 58.79 & 65.53 & \#3 \\ \hline
% \x    & \x    & \x    & \x    & \x    & \x    &       & \x\m  & 63.94 & 65.80 & \#8 \\ \hline
% \x    & \x    & \x    & \x    & \x\m  &       &       & \x\m  & 62.26 & 67.21 & \#6 \\ \hline
% \x    & \x    & \x    &       & \x\m  & \x\m  & \x\m  &       & 60.14 & 70.25 & \#4 \\ \hline
%\end{tabular}
%}
%\caption{Top 10 average MAE of 10 runs. Shown with their respective ranks in the single run and with the MAE from the single run.} % title of Table
%\label{table:Bottom10Prices} % is used to refer this table in the text
%\end{table}

\subsubsection{Conclusion}


\subsection{Experiment 2: Trimming}
As described in section~\ref{sec:Trimming} trimming is a solution to extreme outliers and a way to streamline your data; thus making it easier for neural networks to predict. The art is to get a good balance between the how much of the dataset you remove and how much of an accuracy boost your predictions will get. If you trim parts of the dataset that is actually common data it will get impossible to predict a very high price since the neural network never sees those values.

\subsubsection{Variables}
Experiment 2 is based on a dataset consisting of the last 3 months averaging to 2189 hours. We use 200 epochs for each training iteration.

In this experiment our variables are the grade of trimming we conduct on the training set. We trim from 1\% to 5\% in the lowest and highest part of the dataset thus totaling to 2\% to 10\% of dataset has been removed.

\subsubsection{Hypothesis}
This experiment is conducted to show the effect of trimming on a dataset. By removing the most extreme outliers we expect the predictions to stabilize since the randomness of the extreme outliers have been removed. This should result in a better overall MAE and a better curve fit.

\subsubsection{Results}

As we can see in figure~\ref{fig:NoTrim} (where no trimming was applied) the predicted values some times goes way out of line and hits the very maximum(about 1500) also when the actual value is nothing near that. This is due to the seldom occurrence of these high prices that the neural network is never able to get a full connection between these prices and the input parameters. Another reason for this to happen is that it is extremely hard to predict such sudden and very high fluctuations in price; thus removing these will not give us a performance hit, since we would not be able to predict them anyways and they just add the possibility of errors in the rest of the predictions.

\begin{figure}[H]
\centering
\includegraphics[width=0.85\linewidth,natwidth=898,natheight=587]{billeder/PriceExperimentalAnalysis/NoTrimming.jpg}
\caption{The \#1 forecast with no trimming of the dataset}
\label{fig:NoTrim}
\end{figure}

If we take a look at the same dataset but this time with a 1\% high and low trim (2\% in total) in figure~\ref{fig:1PTrim}; we clearly see that the actual prices in the beginning aren't spiking as heavily as they did in figure~\ref{fig:NoTrim}. This of course prevents us from predicting these high spikes but if we take a look at the rest of the set we see that the 5 faulty spikes have gone as well. This shows the trade off between being able to predict the outliers and the errors they introduce.

\begin{figure}[H]
\centering
\includegraphics[width=0.85\linewidth,natwidth=898,natheight=587]{billeder/PriceExperimentalAnalysis/1PTrim.jpg}
\caption{The \#1 forecast with 1\% trimming in both ends of the dataset}
\label{fig:1PTrim}
\end{figure}

In figure~\ref{fig:AllTrims} we see the dataset from figure~\ref{fig:1PTrim} with 1\% trim. The lines shows how much would be cut of if we applied 2\%(Purple), 3\%(Red), 4\%(Black) and 5\%(Blue) trimming. Here we clearly see that it is removing data - that is part of the norm - from the set. With every percent we go up we remove 363 entries so at 5\% trim (top and bottom) we will be removing 1815 entries from our dataset. This will result in us never being able to predict value higher or lower than the bars; which we of course are not interested in as most of the values we trim from 2\% and up are part of the standardized dataset.

\begin{figure}[H]
\centering
\includegraphics[width=0.85\linewidth,natwidth=898,natheight=587]{billeder/PriceExperimentalAnalysis/restOfTrims.jpg}
\caption{The \#1 forecast with 2\%(Purple), 3\%(Red), 4\%(Black) and 5\%(Blue) trimming in both ends of the dataset}
\label{fig:AllTrims}
\end{figure}

\todo{Skriv om tabellen. Vis sammenhaeng i forhold til 1815 entries vil mistes paa baggrund af en forbedring paa kun 8 MAE i bedste tilfaelde.}


\begin{table}[H]
\centering  % used for centering table
\resizebox{0.6\textwidth}{!}{
	\begin{tabular}{|c|c|c|c|c|c|} % centered columns (7 columns)
	\hline
	1PTrim & 2PTrim & 3PTrim & 4PTrim & 5PTrim & Number\\ [0.5ex] % inserts table 
	\hline                  % inserts single horizontal line
	47,21 & 42,90 & 44,48 & 42,46 & 41,79 & \#1 \\ \hline
	46,15 & 43,67 & 43,07 & 39,16 & 40,28 & \#2 \\ \hline
	47,14 & 45,10 & 43,38 & 40,50 & 39,41 & \#3 \\ \hline
	46,70 & 43,96 & 43,21 & 40,03 & 40,29 & \#4 \\ \hline
	45,96 & 43,25 & 45,51 & 40,74 & 40,42 & \#5 \\ \hline
	47,27 & 45,96 & 44,98 & 41,39 & 39,98 & \#6 \\ \hline
	45,93 & 44,66 & 43,39 & 41,02 & 40,40 & \#7 \\ \hline
	46,64 & 42,69 & 44,48 & 41,69 & 40,61 & \#8 \\ \hline
	45,98 & 44,51 & 43,71 & 40,74 & 41,08 & \#9 \\ \hline
	45,60 & 46,07 & 45,81 & 42,52 & 41,32 & \#10 \\ \hline
	\end{tabular}
}
\caption{Trims} % title of Table
\label{table:Top10Trimming} % is used to refer this table in the text
\end{table}

\subsection{Experiment 3: Statistical strategies}


\subsubsection{Variables}

\subsubsection{Hypothesis}

\subsubsection{Results}

\begin{table}[H]
\centering  % used for centering table
\resizebox{0.3\textwidth}{!}{
	\begin{tabular}{|c|c|} % centered columns (7 columns)
	\hline
	Smoothing factor & MAE \\ [0.5ex] % inserts table 
	\hline                   % inserts single horizontal line
	30\% & 44,69 \\ \hline
	90\% & 44,82 \\ \hline
	10\% & 44,87 \\ \hline
	20\% & 45,51 \\ \hline
	40\% & 45,51 \\ \hline
	80\% & 46,16 \\ \hline
	60\% & 46,29 \\ \hline
	70\% & 46,55 \\ \hline
	50\% & 46,92 \\ \hline
	\end{tabular}
}
\caption{Smoothing factor test} % title of Table
\label{table:SmoothingFactorTest} % is used to refer this table in the text
\end{table}

\begin{table}[H]
\centering  % used for centering table
\resizebox{0.3\textwidth}{!}{
	\begin{tabular}{|c|c|} % centered columns (7 columns)
	\hline
	Skewness stack & MAE\\ [0.5ex] % inserts table 
	\hline                  % inserts single horizontal line
	4 & 45,09 \\ \hline
	2 & 45,15 \\ \hline
	6 & 45,69 \\ \hline
	20 & 46,05 \\ \hline
	8 & 46,14 \\ \hline
	24 & 46,16 \\ \hline
	16 & 46,47 \\ \hline
	12 & 46,66 \\ \hline
	\end{tabular}
}
\caption{Skewness stack size} % title of Table
\label{table:SkewnessTest} % is used to refer this table in the text
\end{table}

\begin{table}[H]
\centering  % used for centering table
\resizebox{0.3\textwidth}{!}{
	\begin{tabular}{|c|c|} % centered columns (7 columns)
	\hline
	Curve stack & MAE\\ [0.5ex] % inserts table 
	\hline                  % inserts single horizontal line
	4 & 45,72 \\ \hline
	12 & 46,05 \\ \hline
	6 & 46,09 \\ \hline
	2 & 46,31 \\ \hline
	16 & 46,67 \\ \hline
	24 & 46,76 \\ \hline
	20 & 46,87 \\ \hline
	8 & 47,81 \\ \hline
	\end{tabular}
}
\caption{Curve behavior stack size} % title of Table
\label{table:CurveTest} % is used to refer this table in the text
\end{table}

MATRIX Price Consump windSpeed temperatureRow timeOfDay weekdays seasonOfYear PREDICT
\begin{table}[H]
\centering  % used for centering table
\resizebox{0.6\textwidth}{!}{
	\begin{tabular}{|c|c|c|c|c|c|} % centered columns (7 columns)
	\hline
	Curve & Skew & 1Historical & PAPER & MAE & Rank\\ [0.5ex] % inserts table 
	\hline                  % inserts single horizontal line
	       & \x    & \x    &       & 45.35 & - \\ \hline
	 \x    & \x    & \x    &       & 45.76 & 0.91\% \\ \hline
	 \x    &       &       & \x    & 45.83 & 1.06\% \\ \hline
	 \x    &       &       &       & 46.10 & 1.66\% \\ \hline
	 \x    & \x    &       &       & 46.40 & 2.32\% \\ \hline
	 \x    &       & \x    &       & 46.47 & 2.48\% \\ \hline
	       & \x    &       &       & 46.53 & 2.6\% \\ \hline
	       &       & \x    &       & 46.53 & 2.61\% \\ \hline
	       &       & \x    & \x    & 46.72 & 3.03\% \\ \hline
	       &       &       & \x    & 46.88 & 3.38\% \\ \hline
	 \x    & \x    & \x    & \x    & 47.40 & 4.54\% \\ \hline
	       & \x    &       & \x    & 47.65 & 5.09\% \\ \hline
	 \x    &       & \x    & \x    & 47.82 & 5.46\% \\ \hline
	 \x    & \x    &       & \x    & 47.96 & 5.76\% \\ \hline
	       & \x    & \x    & \x    & 48.45 & 6.83\% \\ \hline
	\end{tabular}
}
\caption{Trims} % title of Table
\label{table:Statistical1} % is used to refer this table in the text
\end{table}

MIXEDPrice Consump windSpeed temperatureRow timeOfDayMATRIX weekdays seasonOfYearMATRIX PREDICT
\begin{table}[H]
\centering  % used for centering table
\resizebox{0.6\textwidth}{!}{
	\begin{tabular}{|c|c|c|c|c|c|} % centered columns (7 columns)
	\hline
	Curve & Skew & 1Historical & PAPER & MAE & Rank\\ [0.5ex] % inserts table 
	\hline                  % inserts single horizontal line
	       & \x    & \x    &       & 45.11 & - \\ \hline
	       &       & \x    &       & 45.24 & 0.28\% \\ \hline
	 \x    & \x    & \x    &       & 45.43 & 0.7\% \\ \hline
	 \x    &       &       &       & 45.53 & 0.92\% \\ \hline
	 \x    & \x    &       & \x    & 46.45 & 2.98\% \\ \hline
	 \x    & \x    &       &       & 46.56 & 3.2\% \\ \hline
	       &       & \x    & \x    & 46.76 & 3.65\% \\ \hline
	 \x    &       & \x    &       & 46.87 & 3.9\% \\ \hline
	       & \x    & \x    & \x    & 47.25 & 4.73\% \\ \hline
	       & \x    &       &       & 47.88 & 6.14\% \\ \hline
	       &       &       & \x    & 47.88 & 6.14\% \\ \hline
	 \x    & \x    & \x    & \x    & 48.01 & 6.43\% \\ \hline
	 \x    &       & \x    & \x    & 48.13 & 6.68\% \\ \hline
	       & \x    &       & \x    & 48.40 & 7.3\% \\ \hline
	 \x    &       &       & \x    & 49.68 & 10.12\% \\ \hline
	\end{tabular}
}
\caption{Trims} % title of Table
\label{table:Statistical1} % is used to refer this table in the text
\end{table}

MIXEDPrice Consump windSpeed temperatureRow timeOfDayMATRIX monthOfYearMATRIX PREDICT
\begin{table}[H]
\centering  % used for centering table
\resizebox{0.6\textwidth}{!}{
	\begin{tabular}{|c|c|c|c|c|c|} % centered columns (7 columns)
	\hline
	Curve & Skew & 1Historical & PAPER & MAE & Rank\\ [0.5ex] % inserts table 
	\hline                  % inserts single horizontal line
	       &       & \x    & \x    & 45.14 & - \\ \hline
	       &       &       & \x    & 45.55 & 0.9\% \\ \hline
	       & \x    &       &       & 45.56 & 0.92\% \\ \hline
	       & \x    &       & \x    & 45.60 & 1.02\% \\ \hline
	 \x    &       & \x    &       & 45.91 & 1.7\% \\ \hline
	 \x    & \x    & \x    & \x    & 45.95 & 1.79\% \\ \hline
	 \x    & \x    &       &       & 45.99 & 1.89\% \\ \hline
	 \x    &       &       & \x    & 46.14 & 2.22\% \\ \hline
	 \x    &       &       &       & 46.58 & 3.18\% \\ \hline
	 \x    & \x    &       & \x    & 46.87 & 3.82\% \\ \hline
	       &       & \x    &       & 46.98 & 4.08\% \\ \hline
	       & \x    & \x    & \x    & 47.02 & 4.16\% \\ \hline
	 \x    & \x    & \x    &       & 47.08 & 4.3\% \\ \hline
	 \x    &       & \x    & \x    & 47.94 & 6.2\% \\ \hline
	       & \x    & \x    &       & 47.98 & 6.29\% \\ \hline
	\end{tabular}
}
\caption{Trims} % title of Table
\label{table:Statistical1} % is used to refer this table in the text
\end{table}


\todo{Kig paa price og consumption only med statistical features. Se at det ikke forbedrer. Referer tilbage til scatter teksten. Tag de bedste 3. Foerste vis ingen sammenhaeng. Anden og tredje heller ingen sammenhaeng. Brug helt simple til at verificere.}

\subsubsection{Experiment 4: Black box optimization}

Dataset size:
\begin{table}[H]
\centering  % used for centering table
\resizebox{0.6\textwidth}{!}{
	\begin{tabular}{|c|c|c|c|c|c|} % centered columns (7 columns)
	\hline
	Entries & Time & H1 & H2 & MAE & \% Deviation\\ [0.5ex] % inserts table 
	\hline                  % inserts single horizontal line
	4378 & 397.28 & 6 & 11 & 46.34 & - \\ \hline
	8756 & 844.68 & 12 & 5 & 47.82 & 3.19\% \\ \hline
	2189 & 232.87 & 6 & 18 & 49.33 & 6.46\% \\ \hline
	6567 & 677.33 & 14 & 6 & 49.67 & 7.19\% \\ \hline
	\end{tabular}
}
\caption{Trims} % title of Table
\label{table:Dataset} % is used to refer this table in the text
\end{table}

\begin{table}[H]
\centering  % used for centering table
\resizebox{0.6\textwidth}{!}{
	\begin{tabular}{|c|c|c|c|c|c|} % centered columns (7 columns)
	\hline
	Epochs & Time & H1 & H2 & MAE & \% Deviation\\ [0.5ex] % inserts table 
	\hline                  % inserts single horizontal line
	500 & 450.19 & 12 & 11 & 45.17 & - \\ \hline
	600 & 375.02 & 7 & 11 & 46.14 & 2.16\% \\ \hline
	900 & 415.28 & 8 & 3 & 47.32 & 4.77\% \\ \hline
	1100 & 566.83 & 10 & 5 & 47.40 & 4.95\% \\ \hline
	200 & 264.03 & 14 & 8 & 47.92 & 6.09\% \\ \hline
	100 & 213.95 & 14 & 11 & 48.13 & 6.55\% \\ \hline
	1600 & 807.21 & 6 & 17 & 48.23 & 6.78\% \\ \hline
	300 & 315.01 & 14 & 10 & 48.41 & 7.17\% \\ \hline
	1200 & 676.83 & 10 & 9 & 48.74 & 7.91\% \\ \hline
	400 & 345.01 & 12 & 8 & 49.37 & 9.3\% \\ \hline
	800 & 518.7 & 8 & 16 & 50.15 & 11.02\% \\ \hline
	1000 & 612.22 & 8 & 18 & 50.67 & 12.19\% \\ \hline
	1300 & 877.29 & 13 & 13 & 50.78 & 12.42\% \\ \hline
	1500 & 691.92 & 7 & 8 & 50.82 & 12.52\% \\ \hline
	700 & 471.98 & 7 & 20 & 52.26 & 15.7\% \\ \hline
	1400 & 978.75 & 12 & 18 & 54.41 & 20.46\% \\ \hline
	\end{tabular}
}
\caption{Trims} % title of Table
\label{table:Dataset} % is used to refer this table in the text
\end{table}



