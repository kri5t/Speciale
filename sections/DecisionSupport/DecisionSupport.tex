Since the late 1970s decision support has been a developing area both in the scientific research but also as a tool in the private sector. Decision support is important in that it helps the users to gain an advantage in complex domains and it helps them to take the better decision when it comes down to a crucial choice. Decision support has been applied in many forms over the years and as computers evolve and they become more sophisticated the same applies for the decision support systems (DSS).

In \cite{shim2002past} they argue that DSS are an inevitable evolution in decision making companies across the globe. They describe how DSS systems have evolved from being a conceptual framework to sophisticated software providing information for information heavy companies. In the beginning DSS only existed on standalone computers as a ressource to take into consideration when discussing something with your colleagues. It evolved with the introduction of data warehouses and the world wide web into a more analytic tool, called on-line analytic processing, that gathered information from data warehouses and compiled this information into behavioural information about customers or the like. This was decision support based on datamining and analysis of the data. Also the introduction of the internet (as we know it) reduced the costs of bringing DSS to smaller cooperations and firms and made it easier for people to use this software in their everyday work scenario. This also introduced collaborative support systems that are systems focused on helping groups of people rather than just a single individual. These types of decision support systems often spans more than just small groups of users and incorporates entire organisations into one decision support system. While these systems are interesting we wont be covering them here becaues they are out of the scope of this thesis. Instead we will be looking at Optimization-based support models. These DSS can be divided into three stages: formulation, solution and analysis\cite{shim2002past}:
\begin{quotation}
\textit{Formulation refers to the generation of a model in the form acceptable to a model solver. The solution stage refers to the algorithmic solution of the model. The analysis stage refers to the 'what-if' analyses and interpretation of a model solution or a set of solutions.}
\end{quotation}
The formulation support is about narrowing down the problem and normalizing the data so that it fits into a datamodel or an algorithm. The solution stage involves faster and better algorithms that will solve complex problems at a higher rate. At the same time people use more sophisticated methods to solve combinatorial problems. This is done by using genetic algorithms, neural networks and the likes\cite{shim2002past}. The last part is analysis where the DSS focuses on delivering the information from stage 2 and handing it over to the client in a useable form instead of just the analysis which the client then have to make sense off. This can be done by presenting the user for spreadsheets, graphs and report generation.

In the future \cite{shim2002past} argue that:
\begin{quotation}
\textit{(i) it should look for areas where the proven skills of DSS builders can be applied in new, emergent or overlooked areas; (ii) it should make an explicit effort to apply analytic models and methods; it should embody a far more prescriptive view of how decisions can be made more effectively; (iii) it should exploit the emerging software tools and
experience base of AI to build semi-expert systems, and (iv) it should re-emphasise the special value of DSS practitioners as being their combination of expertise in understanding decision making and knowing how to take advantage of developments in computer-related fields.}
\end{quotation}
By this they are saying that they predict a future with more sophisticated decision support systems that in a broader manner will use artificial intelligence to provide the support. They also argue that with the ability to distribute products over the internet and easily reach a broader audience more specialised DSS will pop up. They foresee that a lot of these services will be on a pay per use basis where you log into a system and get the information you are after and pay as you use it.
