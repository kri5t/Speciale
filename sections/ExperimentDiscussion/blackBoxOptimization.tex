The purpose of the black box optimization was to find the best number of epochs and best size of the training set. It showed in both predictions --- wind power and electricity price --- what was expected of it. It revealed that too many or too few number of epochs resulted in an over- or under-fitting of the networks function. Section~\ref{sec:annSection} describes common pitfalls when using Artificial Neural Networks where the generalization function of the network is overfitted. In short the generalization function is fitted only to the training set and therefore not able to predict beyond it, i.e. predict the values in the testing set. In this thesis we have dealt with over-fitting by early stopping and dividing the datasets into a training set and a testing set. We tried different epochs from 1-2800 and selected the number of epochs that achieved the best result on the testing set. It gives the best picture since predictions will naturally be done on data that has not been seen before. 

Over- and under-training of the network as presented in\cite{1} was more clear for wind power. Too few  and too large datasets resulted in a significant worsening in the error from 19,7\% to 48,64\% compared to the best size of data at 3 month. The same did not apply for electricity prices where the best size showed to be half a year and where the errors were closer to the best result varying from 3,19\% to 7,19\%. It emphasizes the need for experimenting with the right amount for the specific type of prediction.

The purpose of the time measurements is to illustrate the impact of training set size on pruning, training and prediction. The prediction time showed to be almost linear with the training set size in both cases. The times varied from 232 to 844 seconds for electricity prices and 462 to 1585 seconds for wind power for predictions over an entire year. The difference between electricity price and wind power numbers is found in the use of two computers with different computational power. A time-consuming element in the predictions are the identification of the network structure through pruning and the actual training of the network before doing all day-ahead predictions --- this procedure is always done before the predictions. The beginning procedure accounts for about 1/20 of the time whereas the subsequent 360 predictions (one for each day of the year) is the rest. The best electricity price prediction took 397 seconds which would result in approximately 20 seconds (397/20) of pruning and training before being able to predict. The individual predictions take seconds and in a real setting only one day is predicted at a time which makes the pruning and training the most significant time waste --- the approximately 20 seconds of wait time and the few seconds it takes to predict makes it possible to use the Artificial Neural Network model just before the time to predict. If time becomes an issue our results show the influence of computational power to be considerable and an increase could be used as a solution to bring down the time to predict. The focus on the time perspective would have another meaning if the algorithm was to be used in an autonomous trading system with the purpose of placing bids itself but this will not be discussed further.  