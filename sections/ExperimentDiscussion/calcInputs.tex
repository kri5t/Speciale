Calculated inputs showed very different results. Electricity price prediction showed an overall improvement when adding slope, skewness and volatility as input. The alternative scatter approach where previous hours are directly added as inputs for the network to calculate the relationship itself was outperformed by the direct input calculations. Skewness and volatility obtained the best result for prices with an improvement of 4\%. The results for wind power was different since only one of the calculated inputs, volatility, showed an improvement of 3,52\% compared to the same prediction without it. Using inputs that says something about the immediate past just before the 24 hours to predict can have both positive and negative results depending on what to predict. The purpose is of course to get a more precise prediction from start and thereby achieving an overall improvement from step to step. If this is not the case and the first predicted steps are inaccurate then the error will instead elevate from step to step and make the entire prediction more inaccurate. It is noticeable that the same calculations do not apply for both even though wind power and electricity price have similar characteristics and it is therefore necessary to experiment thoroughly with the different approaches in order to find the most applicable for your dataset. It was expected to see better accuracy when applying calculated inputs to the electricity prices since it relies more heavily on previous prices and price movements whereas wind power follows meteorological factors more significantly, namely wind speed. Price does not follow one parameter as close as wind power does with wind speed. What can be concluded from the results is the potential benefit from using calculated inputs to add additional characteristics about the movements of the curve in both cases and help the generalization function of the network to approach its target better if the correct calculation inputs are used. Alternative calculations from the field of economics could potentially help improve the predictions since it could include additional features and characteristics of the curve movements. 