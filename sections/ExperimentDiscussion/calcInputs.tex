Calculated inputs showed very different results. Electricity price prediction showed an overall improvement when adding slope, skewness and volatility as input. The alternative scatter approach where previous hours are directly added as inputs for the network to calculate the relationship itself was outperformed by the direct input calculations. Skewness and volatility obtained the best result for prices with an improvement of 4\%. The results for wind power was different since only one of the calculated inputs, volatility, showed an improvement of 3,52\% compared to the same prediction without it. What can be concluded from the results is the potential benefit from using calculated inputs to add additional characteristics about the movements of the curve and help the generalization function of the network to approach its target better. It is noticeable that the same calculations do not apply for both even though wind power and electricity price have similar characteristics. The reason can be found in how price relies more heavily on previous prices whereas wind power also follows wind speed significantly. Price does not have any input parameter to follow as directly as wind power. Alternative calculations from the field of economics could potentially improve the predictions since it could include additional features and characteristics of the curve movements.