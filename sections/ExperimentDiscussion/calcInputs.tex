\subsubsection{Influence of trend}
\label{sec:influenceOfTrendInCalcInput}
Calculated inputs showed very different results. Electricity price prediction showed an overall improvement when adding slope, skewness and volatility as input. The alternative scatter approach from Section~\ref{sec:priceExperimentThree} where previous hours are directly added as inputs for the network to calculate the relationship itself was outperformed by the direct input calculations. Skewness and volatility obtained the best result for prices with an improvement of 4\%. The results for wind power was different since only one of the calculated inputs, volatility, showed an improvement of 3,52\% compared to the same prediction without it. It was established in the analysis that both price and wind power are highly volatility and consists of different types of spikes during the year. The need for including characteristics about these spikes are discussed in Section~\ref{sec:usingStatisticalInput} but in brief it is related to the current price behaviour reflecting current market conditions and by including a calculation thereof we hope to add more characteristics of the output we are trying to predict. The need for adapting to changes in trends and seasons are also presented in \cite{forecastingSpotPricesAccountingForWindPower}. It is the only text included that also attempts to predict the day-ahead spot prices from nord pool spot for Western Denmark. They use instead a regression model and more sophisticated two-step models to capture and adapt to seasonal changes and shifts in trends. The specific models are Holt-Winters in two variations, ARIMAX and a seasonal persistence model but we wont go into greater details with those other than the purpose of it is to capture seasonal changes and trends. Since they are predicting the exact same market and use same electricity prices from nord pool spot we use it to exemplify other achievements of electricity price prediction within the same market. It must be contrasted with the discussion regarding comparison of neural networks in different markets --- this is the same market, the same currency and the exact same data format we are trying to predict. Furthermore their testing covers more than a year of unseen data from 2009-2011. The comparison is of course not definitive since they use older data than ours but it can give an indication of where we position ourselves. Table~\ref{table:resultComparisonWithOtherDanishText} show their results varying from 33,66 to 69,75. We position ourselves better than their standard persistence model but obtains close figures to ARIMA and standard Holt-Winters. Their two-step Holt-Winter model significantly outperforms all others and shows the room for improvements and future work of our approach. The text mentions that a model that might be just as suitable as theirs is a Adaptive Wavelet Neural Network. This has not been considered in this thesis but investigations could be made in future work. Their approach verifies our initial thoughts from Section~\ref{sec:usingStatisticalInput} and the importance of including seasonal changes and trend shifts to further characterise the output to predict. \todo{criticise calc inputs in relation text. Trend can potentially not be genralized over the dataset -- THERE IS A LOT OF DIFFERENT BEHAVIOURS which might be what we see --> wavelet .... or results indicate that this is more likely to be the case for wind power than electricity price}

\footnotesize
\begin{center}
\begin{longtable}{|c|c|}
\hline
\textbf{Model} & \textbf{MAE (DKK)} \\
\hline
\endfirsthead
\multicolumn{2}{c}%
{\tablename\ \thetable\ -- \textit{Continued from previous page}} \\
\hline
\textbf{Model} & \textbf{MAE (DKK)}  \\
\hline
\endhead
\hline \multicolumn{2}{r}{\textit{Continued on next page}} \\
\endfoot
\hline
\endlastfoot
\arrayrulecolor{light-gray}
Two-step Holt-Winters & 33,66\\ \hline
Holt-Winters & 40,34 \\ \hline
ARIMAX & 43,26 \\ \hline
Our ANN model & 45,17 \\ \hline
Seasonal persistence model & 69,75 \\ \hline
\caption{Results from various prediction models on unseen data.}
\label{table:resultComparisonWithOtherDanishText}
\end{longtable}
\end{center}
\normalsize

Using inputs that analyse on a predefined number of previous hours for every hour of the dataset (except from the first hours without enough previous hours) can have both positive and negative results depending on what to predict as seen in our experiments. The purpose is of course to get a more precise prediction from step to step in the 24-step-ahead prediction. The potential of elevating the error arises if the first steps are inaccurate because it will have an affect on some of the the following steps. This is reflected in slope calculation as input for wind power in Section~\ref{sec:windPowerSlopeCalc} and seen in the comparison without it in Figure~\ref{fig:basicCurveAnalysisGraphoForDiscussion} --- this will be elaborated in coming section about step ahead prediction in Section~\ref{sec:stepAheadDiscussion}. It is noticeable that the same calculations do not apply for both price and wind power even though they possess similar characteristics. It is therefore necessary to experiment thoroughly with the different approaches in order to find the most applicable for your dataset --- this emphasizes the discussion from the input parameters Section~\ref{sec:inputParameterDiscussion} regarding the need for describing and analysing your dataset because what worked for one dataset does not necessarily work for others if the two datasets are completely different. The calculated inputs are based on prices and productions from previous hours and because markets are different it must be stated clearly what factors are used. For instance, countries with many Cooling Degree Days or Heating Degree Days (following the model in Section~\ref{sec:ElectricityDemand}) have higher consumption which greatly influence the price and its behaviour. The same apply for the importance of wind power in the Danish electricity market due to huge amount of wind mills which in this thesis has shown to be significant. This is not applicable for countries without the same amount of mills. The documentation is a contributory factor to the potential of identifying similar markets in terms of influences. It can help identify both standard and calculated inputs for specific market conditions, e.g. if a marked is analysed to be similar to Denmark then it makes sense to use their ideas and models first.

\begin{figure}[H]
\centering
\includegraphics[width=0.99\linewidth]{billeder/curveAnalysisWindProduction.png}
\caption{Wind production prediction for 175 hours in 2012 with slope as input}
\label{fig:basicCurveAnalysisGraphoForDiscussion}
\end{figure}

It was expected to see better accuracy when applying calculated inputs to the electricity prices since it relies more heavily on previous prices and price movements as discussed in the price analysis in Section~\ref{sec:ElectricityPriceAnalysis} whereas wind power follows meteorological factors more significantly, namely wind speed. The problem can be related to the electricity price being extremely volatile and the network not being able to identify a general recognizable pattern in how immediate hours impact the electricity prices when calculated (mentioned in Section~\ref{sec:usingStatisticalInput}). The analysis establishes that previous hours impact the price and wind power but not if the actual skewness or historical volatility calculations have an influence. The reason is the many adjustable settings (see Section~\ref{sec:windProductionDev}) of the functions which in our opinion makes them more applicable for testing directly in experiments based on the conclusion that previous hours influence the output. 

What can be concluded from the results is the potential benefit from using calculated inputs to add additional characteristics about the movements of the curve and help the generalization function of the network to approach its target better if the correct calculated inputs are used as seen for wind power. Alternative calculations from the field of economics could be investigated to include additional features and characteristics of the curve movements but also hybrid versions of the Artificial Neural Network could be looked into. One such hybrid could be the Wavelet Artificial Neural Network \cite{•}

 This could involve alternative ways of calculating volatility, skewness but the electricity price is also sensitive to external influences that can result in electricity price spikes\cite{singhal2011electricity}. These influences are only shortly mentioned but covers socio-cultural factors as described in Section~\ref{sec:ElectricityDemand} but also natural gas, oil, fuel prices and system loads\cite{singhal2011electricity}. This is to be explored future work. 



