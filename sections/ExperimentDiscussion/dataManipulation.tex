De andre tekster undlader i stor omfang at snakke om input i forhold til konkret data, trimming og manipulation af det konkrete data. Sammenstil det med vores eget. Reflekter og konkluder.

== the best possible prediction. Hvad er med til at give det bedste result.

The experiments emphasize the need for manipulation of data before predicting. Not necessarily in the way but data always needs to be pre-processed.

Matrix is in both cases an improvement. Time of day matrix has been applied on both but the improvement is much more apparent in the electricity price forecast. The reason must be found in the co-relation between time of day and electricity price being much more significant due to the consumer market. Time of day in relation to wind power production is much more random since the production follows wind speed more directly. We can't control the weather. 

Trimming is most applicable when irregularities exist in the dataset that cannot be predicted. Wind power production showed no such irregularities and trimming it would make these values impossible to predict. Electricity prices on the other hand showed unpredictable values \todo{ref to drawing of trimming in both cases}. The overall prediction improved with 20\% with trimming applied. \todo{trimming and live data?}

Dataset manipulation and trimming of the dataset are two important techniques that can improve the predictions of an artificial neural network. We used a matrix representation (described in section~\ref{sec:Matrix}) for the input parameters that were applicable for such a representation - a finite and limited set of values the input parameter can represent and a significant difference between the input parameters influence on what you predict.