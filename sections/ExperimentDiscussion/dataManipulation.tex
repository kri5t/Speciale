De andre tekster undlader i stor omfang at snakke om input i forhold til konkret data, trimming og manipulation af det konkrete data. Sammenstil det med vores eget. Reflekter og konkluder.

== the best possible prediction. Hvad er med til at give det bedste result.

The experiments emphasize the need for manipulation of data before predicting. Not necessarily in the way but data always needs to be pre-processed.

Matrix is in both cases an improvement. Time of day matrix has been applied on both but the improvement is much more apparent in the electricity price forecast. The reason must be found in the co-relation between time of day and electricity price being much more significant due to the consumer market. Time of day in relation to wind power production is much more random since the production follows wind speed more directly. We can't control the weather. 

Trimming is most applicable when irregularities exist in the dataset that cannot be predicted. Wind power production showed no such irregularities and trimming it would make these values impossible to predict. Electricity prices on the other hand showed unpredictable values \todo{ref to drawing of trimming in both cases}. The overall prediction improved with 20\% with trimming applied. \todo{trimming and live data?}
\subsubsection{Matrix}
Dataset manipulation and trimming of the dataset are two important techniques that can improve the predictions of an artificial neural network. We used a matrix representation (described in section~\ref{sec:Matrix}) for the input parameters that were applicable for such a representation - a finite and limited set of values the input parameter can represent and a significant difference between the input parameters influence on what you predict. The inspiration for the matrix representation came from the way pictures are represented in ANNs. In \cite{knerr1992handwritten} they take every pixel (which has 16 different shades of grey) in a 16 pixel big image and map the pixels out as a matrix. This gives them 256 input variables for the neural network to represent this image. We do the same thing for the applicable input parameters but instead of pixels we map out inputs. The matrix format has limitations (as we saw in the wind power experiment two~\ref{sec:windProdExperimentTwo}). The input parameters you want to represent as a matrix has to be evenly distributed in the dataset. If this is not the case then some of the input values will be under trained or not trained at all since there are few to no representations of that value. This was the case with wind speed and the reason to why it was worse than the wind speed as a single input variable. When we applied the matrix format to the time of day input variable we saw an improvement in both cases. We did a simple matrix test in the wind speed experiment two~\ref{sec:windProdExperimentTwo} as we only had two different parameters that were applicable for the matrix format and because we saw that the wind speed as a matrix did perform poorly due to the prior mentioned limitations of the matrix format. In the price experiments we did a more elaborate test of the matrix format and did every combination (all with matrix, all without matrix and mixed). We did this because the price experiments had more parameters that was representable on the matrix format thus giving us more combinations. The results from the price experiment was not as lopsided regarding the matrix format as the wind production experiments. In the price experiments we could not say that one parameter had to be on one specific format but the combinations of the inputs both as matrix and non-matrix proved to be the best; but we saw that the matrix format was overrepresented in the top results giving us a hint about it being better. We have only seen one other \cite{crowley2005weather} use the matrix format (to some extend) for their seasonal inputs and the following articles has seasonal inputs but do not utilize the matrix format \cite{szkuta1999electricity, singhal2011electricity}.

\subsubsection{Trimming}
Trimming of the dataset is a technique to remove irregularities in the dataset. This can be done both for the highest values and the lowest values in the dataset. Trimming allows us to get a more standardized dataset but it comes with a cost. All the data that you remove from the dataset you will not be able to predict in the test dataset thus introducing limitations to the predictions. Therefore trimming should always be weighed up against how much of the dataset you will not be able to predict after trimming. If the irregularities are extreme enough and represent only small parts of the dataset then we will not be able to predict these values. If this is the case the outliers just introduce errors and no benefit since we were not able to predict them anyways thus justifying the removal of these values. In the wind power predictions experiment two~\ref{sec:windProdExperimentTwo} there was no benefit from trimming the dataset. Actually it introduced errors and made the predictions worse in average. This is because the dataset used for wind power production already consists of uniform data and the trimming will only scramble this data and introduce further volatility also described in~\ref{sec:windProdExperimentTwo} and shown in figure~\ref{fig:fivePercentTrimPrediction}. In the electricity price predictions experiment two~\ref{sec:priceExperimentTwo} we saw that the trimming helped the predictions to be better. This is because in the price prediction dataset there were some extreme outliers. When we trimmed the dataset by 1\% in the top we went from 1561 to 632 as the highest value in the dataset. The 1\% extreme values in this dataset only introduce more noise and the ANN was not able to predict so high values thus removing them did not make the performance worse it only helped the remainder of the results see figure~\ref{fig:1PTrim}. We saw an improvement of 20.99\% from a non-trimmed dataset compared to a trimmed dataset (in the price predictions). In \cite{singhal2011electricity} they also trim the dataset when predicting the electricity prices to get rid of the worst spikes in the price.
\todo{Trimming might be used in other articles but they do not mention it. How do we address this issue?}