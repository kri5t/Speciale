this section will discuss what can be done in relation to strategies or other attempts to improve accuracy of the predictions.

\subsubsection{Recalculate}
POTENTIAL STRATEGY
The re-calculation concept works by letting other neural networks re-calculate the prediction from the original network. The purpose is to identify places where the generalization function has problems and then divide it into new neural networks that only have the purpose of focusing on these problematic situations. For simplicity we are going to apply this to the small numbers and big numbers of the dataset. We will define what is small and big by taking the 2 and 98 percentile. Whenever the "original" network predicts something in or close to the big or small interval then the responsibility of that prediction is forwarded to one of the two. It works like a second opinion but here the network has been specialized to only focus on a specific part of the dataset instead of everything\todo{give example of corrections}. It creates a normalization function on a smaller dataset and the idea is to leave out a lot of unnecessary information. It does not need to account for anything other than its interval in the generalization function so it should be able to predict its target group better. We let the original network decide if it actually thinks we are in the low numbers --- if we are then we can possibly make that prediction even more accurate. These thoughts could basically we applied on all other parts of possible wind production values. 

Problem can occur if the intervals are too small which will cause the dataset to be too small and hard for the ANN to generalize upon.

\subsubsection{Prediction - Similar Days}
The similar days approach is described in Section~\ref{sec:sdmApproach} and addresses what to include in the dataset based on parameters. Wind power production follows wind speed and the expectation is that if filtering out days with completely different wind speeds and productions then a better accuracy can be obtained. The similar days analysis will take all wind speeds from the hours to predict and find all days within this interval plus/minus one in each end, e.g. an interval of wind speeds between 4-9 would result in similar days between 3-10. There is a potential for not getting the entire picture when using similar days. It can be related to the discussion about wind speed as matrix and how some values are under-represented in the dataset. In could happens in shifting seasons that wind speeds are never seen before and therefore run into a dataset without many hours. In those cases the entire dataset is used instead. The analysis in Section~\ref{sec:windProdSeasonality} describes the difference from season to season.

We can see from Table~\ref{table:theSimilarDaysApproach3monthTable} that the results are much equal to the best input combination from Table~\ref{table:windProdInputParamsTop10}. The purpose of the similar days is to filter out unnecessary noise from the dataset but since the dataset is of a manageable size and in itself contain a seasonal aspect it does not filter out many values as seen in the table. When applying it on an entire year more values are filtered out but it cannot make up for the increase in dataset as seen in Table~\ref{table:theSimilarDaysApproachYearTable}.