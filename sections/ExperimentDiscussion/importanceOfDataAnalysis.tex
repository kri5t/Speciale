\subsubsection{Identifying input parameters}
The strength of the prediction is very much dependent on the quality of the underlying data. As described in Section~\ref{sec:dataCollection} the data must satisfy certain criteria before it can be used as input and output for the Artificial Neural Network. It must be trustworthy and contain hourly observations for our specific purposes, e.g. day-ahead wind power and electricity price prediction based on a dataset that consists of hourly observations. All calculations in the ANN are based on this data which makes it of utmost importance for the forecast. The selection of the correct input parameters for testing is equally important which is described in the analysis in Chapter~\ref{ch:theANNs}. The analysis of both electricity price and wind production together with experiment one in Sections~\ref{sec:windPowerExperimentOne} and~\ref{sec:priceExperimentOne} show how the combination of the right parameters is the core to a good prediction. It is necessary to make a comprehensive analysis of different parameters to know what to include in the experiment and how to represent them. A good example of this is the seasonal aspect for electricity prices which showed difference in result when used as month or as the specific season (summer, winter, ect.). Consequently, the pre-requisites for the wind power and electricity price experiments to perform accurately is the the quality and selection of the input data since it is the basic foundation for the Artificial Neural Network to generalize upon. 

The importance of analysing exactly what input parameters constitutes the electricity price and how they are represented is much omitted in\cite{ShortTermWindPowerForecasting,singhal2011electricity} which is also discussed in Section~\ref{sec:priceExperimentThree}. It makes it very difficult to imitate the behaviour and use or test their knowledge and experience on how to include and represent inputs in the best possible way. The seasonal aspect can as mentioned be included in different ways, matrix can be used and the combination of the inputs have very different impact. Exactly what combination of inputs and how they are represented should always be documented for others to imitate and test on own datasets so that results can be compared based on the same data. Our experiments showed that even though some inputs was closely co-related to the output it was not necessarily included in the best prediction as seen with consumption and wind power in Section~\ref{sec:windPowerExperimentOne}. The black box nature of the ANN can identify unforeseen dependencies between the various input parameters. It is not enough to simply list the characteristics of the output to predict without mentioning what came out of the input analysis, if the experiments validated it and exactly what inputs was used. What influence the electricity price can differ from country to country due to different market conditions and different weather conditions since demand follows temperature. The focus should not be on the results alone but also how the result is obtained through the dataset, especially in the case of input parameters for the neural network which have shown to be of utmost importance in our analysis.

The electricity price can also be sensitive to other factors that have not been touched in this thesis. Such impacts could be socio-cultural as described in Section~\ref{sec:ElectricityDemand}. This could be explored future work. 

\subsubsection{Input combination}
As described above the different combinations of input parameters is found in experiment one for both electricity prices and wind power production. The experiments show the complexity in the black box nature of ANN since it identifies better connections of input than foreseen. In wind power production the substitution of temperature and air density with consumption showed better accuracy which was not expected, whereas wind speed showed to be much more crucial to the electricity price than foreseen. Wind power production impacts the electricity price as presented in \cite{dayAheadImpactOfWindPowerForecasts} but more than expected through the wind speed. This leads to the potential of feeding the electricity price with the prediction of wind power as input instead of wind speed to make the prediction more accurate. This connection has to be investigated in future work.

The seasonality characteristic of the wind power production and electricity price time-series is in both cases significant and for that reason expected to increase performance for both predictions. Seasonality is reflected in the month input parameter and showed different results. The month parameter obviously decreased the overall performance of the network in the case of wind power whereas it was increased in the electricity price. The first thought was the small size of the data only containing three months and therefore not reflecting the the seasons from the previous year which was the intention. When attempted on a whole years training set it showed an overall worsening in accuracy due to over-training due to the bigger training set. The electricity prices with the seasonal aspect was also tried with a training set containing a year but as opposed to wind power the accuracy of the forecasts stayed the same. What can be concluded from this is that more data is not necessarily equal to better results in both cases and 3 months of data is the most expressive to predict 24 hours ahead in our case. What was believed to be problematic was the shifting from one season to another where the new season was significantly different from the one we came from. This was proven wrong by the experiments since 3 months itself showed to contain enough information about the current season to be accurate and the the potential shortcomings from seasonal shifts were compensated by the omission of unnecessary data from the rest of the year. The high volatility results in many different cases during the year which can make it hard for the network to generalize when the data set becomes to huge.

\subsubsection{Unseen data}
The wind power section emphasized in connection with input parameter experiments the need for measuring accuracy of results only on the testing set (unseen data). It was seen in Table~\ref{table:predictionMAEUnseenVsTrainingSet} that the MAE could be the same across all predictions on the training set compared to a huge difference when applied on the testing set. The best MAE on unseen data showed the worst MAE on the training set. Furthermore, different seasons and consumptions during the year result in many different days where the electricity prices and wind power behaves differently. This is also discussed in the analysis in Chapter~\ref{ch:theANNs} and therefore the experiments must as a minimum be performed on an entire year of unseen data. All simulations in this thesis is performed once on a year but more testing on the same year should be conducted to further validate and strengthen results. Predicting the same hours will result in different results due to the nature of a prediction but it will still be indicative since it predicts many many hours during the year \todo{be more specific}. In \cite{1} the experiments are performed on 5 different days through out an entire year and then again on 2 weeks in February. Based on these days they conclude the following:

\begin{quotation}
\textit{The test results obtained through the simulation demonstrate that the proposed algorithm is robust, efficient, and accurate, and it produces better results for any day of the week.}
\end{quotation}

Omitting 340 days and concluding that it in general obtains better results is in contrast with our analysis of the seasonal aspect because they days differ much but also the results from experiment five in Section~\ref{sec:windPowerExperimentFive} and~\ref{sec:priceExperimentFive} where various starting points during the year greatly influences the overall results. This must be taken into consideration when claiming the feasibility and sufficiency of the prediction. When criticising the feasibility of others we must make aware of the fact that our testing set does not contain any predicted values. This can negatively affect the predictions when used in a real setting and testing of this must be included in future work. We still consider the actual values to be sufficient since they look exactly the same and can simulate the predicted values in our experiments. Furthermore, it gives us the opportunity to simulate all possible predictions on an entire year, since historical prediction data is not available, which our experiments show the importance of as described above. The analysis of the actual values show what influence the electricity price and wind power and in future work we must rely on other professionals to give us the best possible predictions of consumption and weather so that they are accurate enough to be consistent with the analysis. The weather predictions will deviate from the analysis \emph{only} if the they are inconsistent and inaccurate but as described in Section~\ref{sec:dataCollection} the accuracy of 24 hour weather prediction is 97\%. We consider the results obtained in this thesis valid because the potential decrease in accuracy would be applicable for all results, highly ranked or not. It correctly shows the co-relation between input/output, the ranking of results and the input combinations to be used.