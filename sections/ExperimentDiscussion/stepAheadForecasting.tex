The step-ahead forecasting showed how the less hours to predict the better mae. In electricity prices the error is linear to the number of steps to predict until a certain point (around 8) where it flattens out. It is necessary to identify a technique to reduce the elevation of errors from step to step. The problem is not as significant in wind power production because it heavily relies on wind speed and can follow it whereas the price relies greatly on the last known price --- thus making it more likely to elevate the error. 

The step-ahead experiments shows how the steps are based on the step before it; because we include the last-known price (in the price predictions) and the last-known wind production(in the wind production predictions). When steps (that the ANN has to predict) increases the accuracy will decline and the MAE will increase. This is because there is a certain elevation of error in our predictions because of the aforementioned last-known price and wind production. In both the of our results we see a big improvement in the MAE with fewer hours to predict ahead. We also see that the elevation of the error at some point flattens thus not rising uncontrollably. This is because the other inputs will also influence the prediction (not only the last-known price/wind production) thus guiding it in the right direction. This part of this source of error as a place to improve our algorithms since the error margin introduced by x hours-ahead are significant. Future work in this area includes predicting the same hour several times and taking the average of this. This should help the ANN to limit the elevation of the error. Another approach would be to include the predicted step into the dataset and train the network again including the newest prediction thus adjusting the weights to accommodate the most recent prediction.

Another reason for the flattening of the error margins(as described before) are the changing offsets. Some of the offsets in the dataset are better stating points for a prediction than others - reflected in the tables~\ref{table:stepAheadForecastingWindProductionStartingPositions} and in table~\ref{table:Offsets} with an improvement of 23,34\% and 15,04\% respectively. The offsets in the dataset are not something that is covered in other papers.
\todo{Retrain the network with the newest predictions}
\todo{Algorithmic analysis of the dataset. Find best offsets. Analyze the curve and apply to starting.}