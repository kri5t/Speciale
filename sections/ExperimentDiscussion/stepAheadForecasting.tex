The step-ahead forecasting showed how the less hours to predict the better mae. In electricity prices the error is linear to the number of steps to predict until a certain point (around 8) where it flattens out. It is necessary to identify a technique to reduce the elevation of errors from step to step. The problem is not as significant in wind power production because it heavily relies on wind speed and can follow it whereas the price relies greatly on the last known price --- thus making it more likely to elevate the error. 

The step-ahead experiments shows how the steps are based on the step before it; because we include the last-known price (in the price predictions) and the last-known wind production(in the wind production predictions). When steps (that the ANN has to predict) increases the accuracy will decline and the MAE will increase. This is because there is a certain elevation of error in our predictions because of the aforementioned last-known price and wind production. In both the of our results we see a big improvement in the MAE with fewer hours to predict ahead. We also see that the elevation of the error at some point flattens thus not rising uncontrollably. This is because the other inputs will also influence the prediction (not only the last-known price/wind production) thus guiding it in the right direction. We also see a place to improve our algorithms since the error margin introduced by hours-ahead are significant.
\todo{Retrain the network with the newest predictions}