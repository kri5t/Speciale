\subsection{Experimental Results}
This subsection describes the experiments that uncovers the best combination of hidden layers, neurons and epochs as well as the different strategies. Based on the above analysis of wind production influences the network will be tested with the following input parameters; 1) Wind speed; 2) Air Density; 3) Time of day 4) Temperature and 5) Consumption;. Furthermore, experiments are needed to investigate the statistical inputs as presented in section~\ref{sec:usingStatisticalInput}.

\todo{DRAWING OF NETWORK}

I FOUND RECURRENT LINK IN ARTICLE REF ID

Black art --- experimenting with hidden layers, momentum and learning rate. 

\todo{remember to compare temperature inclusion in relation to consumption}

\subsubsection{Experiment Series One - Selection of input parameters}
The first experiment series is an attempt to find the best input parameters of the network. The input parameters to experiment with is wind speed, air density, time of day, consumption and temperature. The most significant tests will be to determine if meteorological factors can stand as a substitute for consumption and if wind direction improves the prediction.


\begin{table}[H]
\centering  % used for centering table
\resizebox{\textwidth}{!}{
\begin{tabular}{c c c c c c c c c c} % centered columns (3 columns)
 & Wind & Air &  &  & Wind & Time & Last & \\ 
Test & Speed & Dens & Consump & Temp & Direction & of Day & Production & MAE & Rank \\
[0.5ex] % inserts table 
%heading
\hline                  % inserts single horizontal line
1. & x &  &  &  &  &  &  & 153.6 & 0 \\ %
2. & x & & x & & & & & 139.8 & 0 \\ 
3. & x & & x & & & x & & 138.7 & 0 \\
4. & x & & x & & & x & x & 139.0 & 0 \\ 
5. & x & & x & & x & x & & 147.7 & 0 \\
7. & x & & x & x &  &  & x & x 141.8 & 0 \\
8. & x & x & & & & & & 140.4 & 0 \\
9. & x & x & & & & x & & 139.2 & 0 \\
10. & x & x & x & & & x & & 138.3 & 0 \\
11. & x & x & x & & & x & x & 138.3 & 0 \\ [1ex] % [1ex] adds vertical space
\hline %inserts single line
\end{tabular}}
\caption{Table showing the difference between using consumption and meteorological factors} % title of Table
\label{table:consumptionInclusionTable} % is used to refer this table in the text
\end{table}

Since the co-relation between wind production and wind speed is 0.94 it has been tested on its own. The result clearly shows this relationship by only missing the ideal wind production with 143.9 in average in test number 1. Wind speed has been tested in combination with different input parameters. The addition of air density stands out with a MAE of 116.3 which can somehow be expected since it is directly proportional to the wind speed as described in~\ref{sec:airDensity}. 

The table clearly shows that wind direction does not have a correlation to the other parameters and therefore must be omitted in tests to come.
The importance of the wind production development is described in~\ref{sec:windProductionDev} and to account for this the network has been supplied with the wind production from one hour ago - more sophisticated attempts with statistics will come in other experiments.
It is obvious that using the consumption is much more accurate than substituting it with just meteorological factors. The wind direction does not give better prediction so this is omitted from the experiments to come.

\todo{The hours can be reflected as described in matrix section. Now each of the values will reflect one hour instead of one value reflecting all of the hours.}

The following experiments will be based on the input parameters:
\begin{itemize}
\item Wind Speed;
\item Air Density;
\item Wind production from last hour;
\item Consumption;
\end{itemize}


\subsubsection{Experiment Two - Data Manipulation}
The first experiment will show the difference in performance due to the manipulation of the time series data set. The three approaches are naive, trimming and matrix. These experiments do not include any statistical features. The point of this experiment is to find the best approach to simple data manipulation as presented in~\ref{sec:DataManipulation}.

\todo{also different size of dataset}

\begin{table}[H]
\centering  % used for centering table
\begin{tabular}{c c c c} % centered columns (3 columns)
ANN Type & MAE & MPE & Rank \\ [0.5ex] % inserts table 
%heading
\hline                  % inserts single horizontal line
ANN Naive & 0 & 0 & 0 \\ % inserting body of the table
ANN Trimming & 0 & 0 & 0 \\
ANN Matrix  & 0 & 0 & 0\\
ANN Matrix and Trimming  & 0 & 0 & 0 \\ 
ANN Statistical features  & 0 & 0 & 0\\ [1ex] % [1ex] adds vertical space
\hline %inserts single line
\end{tabular}
\caption{Table showing the performance of different data manipulation approaches when all run with 1500 epochs.} % title of Table
\label{table:dataManipulationApproaches} % is used to refer this table in the text
\end{table}

\todo{better with or without trimming?}

The purpose is to locate the 

\todo{do 1-12-24 step ahead to show the "carry-with"-error} 

Taking the most naive approach to prediction by simply including the important normalized input parameters and trying to predict the prices for an entire year. 

\subsubsection{Experiment Three - Size of data set}

\subsubsection{Experiment Four - Prediction Strategies}
Statistics


\subsubsection{Experiment Five - Performance Optimization}
The impact of different Layers/Neurons/Epochs combinations is huge and the results is shown here.

\begin{table}[H]
\centering  % used for centering table
\begin{tabular}{c c c c c} % centered columns (3 columns)
ANN Type & Epochs & MAE & MPE & Rank \\ [0.5ex] % inserts table 
%heading
\hline                  % inserts single horizontal line
ANN & 500 & 0 & 0 & 0 \\ % inserting body of the table
ANN & 1000 & 0 & 0 & 0 \\
ANN & 2000 & 0 & 0 & 0 \\
ANN & 2500 & 0 & 0 & 0 \\ [1ex] % [1ex] adds vertical space
\hline %inserts single line
\end{tabular}
\caption{Table showing the performance of different data manipulation approaches.} % title of Table
\label{table:performanceOpti} % is used to refer this table in the text
\end{table} 

\subsection{testlol}
\begin{table}[H]
\centering  % used for centering table
\resizebox{\textwidth}{!}{
\begin{tabular}{c c c c c c c c c c} % centered columns (3 columns)

Wind & Air &  &  & Wind & Time & Last & Time & \\ 
Speed & Dens & Consump & Temp & Direction & of Day & Production & Matrix & MAE & Rank \\
[0.5ex] % inserts table 
%heading
\hline                  % inserts single horizontal line
 x &  x &  &  &  x &  x &  x &  x & 115.9 & \#1 \\ \arrayrulecolor{light-gray} \hline
 x &  x &  x &  &  x &  x &  x &  x & 117.12 & \#2 \\ \hline
 x &  x &  &  x &  &  x &  x &  x & 119.51 & \#3 \\ \hline
 x &  x &  x &  x &  &  x &  &  x & 119.98 & \#4 \\ \hline
 x &  &  x &  x &  &  x &  &  x & 120.39 & \#5 \\
 x &  x &  x &  x &  &  x &  &  x & 120.46 & \#6 \\
 x &  x &  x &  x &  x &  x &  x &  x & 121.19 & \#7 \\
 x &  &  x &  &  &  x &  &  x & 121.5 & \#8 \\
 x &  &  x &  &  &  &  x &  & 121.72 & \#9 \\
 x &  &  &  &  &  x &  x &  & 121.75 & \#10 \\
 x &  x &  &  &  &  x &  x &  x & 122.05 & \#11 \\
 x &  &  &  &  &  x &  x &  x & 122.24 & \#12 \\
 x &  &  x &  x &  &  x &  x &  x & 122.28 & \#13 \\
 x &  x &  &  &  &  x &  x &  x & 122.97 & \#14 \\
 x &  x &  x &  &  x &  x &  &  x & 123.06 & \#15 \\
 x &  &  x &  x &  x &  x &  x &  x & 123.29 & \#16 \\
 x &  &  x &  &  x &  x &  &  x & 123.52 & \#17 \\
 x &  x &  x &  &  &  x &  &  x & 123.65 & \#18 \\
 x &  x &  &  x &  &  x &  x &  x & 123.78 & \#19 \\
 x &  x &  &  &  x &  x &  x &  x & 123.84 & \#20 \\
 x &  &  &  &  x &  x &  x &  & 124.53 & \#21 \\
 x &  x &  &  &  x &  x &  x &  x & 124.74 & \#22 \\
 x &  &  &  &  x &  &  x &  & 125.16 & \#23 \\
 x &  x &  &  &  x &  x &  &  x & 125.22 & \#24 \\
 x &  x &  &  &  &  x &  &  x & 125.26 & \#25 \\
 x &  x &  x &  x &  x &  x &  x &  x & 125.56 & \#26 \\
 x &  &  &  x &  &  x &  &  x & 125.59 & \#27 \\
 x &  x &  x &  &  &  x &  &  x & 126.77 & \#28 \\
 x &  x &  &  x &  &  x &  &  x & 127.25 & \#29 \\
 x &  x &  &  &  &  x &  &  x & 127.74 & \#30 \\
 x &  &  &  &  &  x &  &  x & 127.94 & \#31 \\
 x &  &  &  &  x &  x &  x &  x & 128.49 & \#32 \\
 x &  x &  &  x &  &  x &  &  x & 128.52 & \#33 \\
 x &  &  &  &  x &  x &  &  x & 128.85 & \#34 \\
 x &  x &  &  x &  &  x &  x &  x & 129.14 & \#35 \\
 x &  &  x &  &  x &  &  x &  & 129.32 & \#36 \\
 x &  x &  x &  &  x &  x &  x &  x & 130.57 & \#37 \\
 x &  &  &  x &  &  x &  x &  x & 131.17 & \#38 \\
 x &  x &  x &  x &  x &  x &  x &  x & 132.16 & \#39 \\
 x &  &  x &  x &  x &  x &  x &  & 133.79 & \#40 \\
 x &  x &  x &  &  &  x &  x &  x & 135.53 & \#41 \\
 x &  &  x &  &  x &  x &  x &  & 136.57 & \#42 \\
 x &  &  x &  &  x &  x &  x &  x & 137.08 & \#43 \\
 x &  &  x &  &  &  x &  x &  & 137.33 & \#44 \\
 x &  &  &  x &  &  x &  &  & 137.52 & \#45 \\
 x &  x &  &  x &  x &  x &  x &  x & 137.76 & \#46 \\
 x &  x &  &  x &  &  x &  x &  x & 138.41 & \#47 \\
 x &  &  x &  &  &  x &  x &  x & 138.42 & \#48 \\
 x &  &  x &  &  x &  x &  &  & 138.52 & \#49 \\
 x &  &  x &  &  x &  &  &  & 139.3 & \#50 \\
 x &  &  &  x &  &  &  &  & 139.33 & \#51 \\
 x &  &  &  x &  &  &  x &  & 139.41 & \#52 \\
 x &  &  x &  x &  x &  x &  x &  & 139.48 & \#53 \\
 x &  &  x &  x &  &  &  &  & 139.59 & \#54 \\
 x &  &  &  &  x &  x &  x &  & 139.99 & \#55 \\
 x &  &  &  x &  &  x &  x &  & 140.13 & \#56 \\
 x &  &  &  &  &  x &  &  & 140.24 & \#57 \\
 x &  x &  x &  x &  x &  x &  x &  x & 140.79 & \#58 \\
 x &  x &  &  x &  x &  x &  x &  x & 141.68 & \#59 \\
 x &  &  x &  &  &  &  &  & 141.77 & \#60 \\
 x &  &  &  &  x &  x &  &  & 141.79 & \#61 \\
 x &  &  &  x &  x &  &  x &  & 141.85 & \#62 \\
 x &  &  &  &  &  &  &  & 141.87 & \#63 \\
 x &  &  x &  &  &  x &  &  & 141.92 & \#64 \\
 x &  &  x &  x &  &  x &  &  & 142.4 & \#65 \\
 x &  &  x &  x &  &  x &  x &  & 142.76 & \#66 \\
 x &  &  &  x &  &  &  x &  & 143.35 & \#67 \\
 x &  &  &  &  x &  &  &  & 143.58 & \#68 \\
 x &  &  &  &  x &  &  x &  & 143.76 & \#69 \\
 x &  &  &  &  x &  x &  x &  x & 143.96 & \#70 \\
 x &  &  x &  x &  x &  &  x &  & 145.07 & \#71 \\
 x &  &  &  x &  &  x &  x &  x & 146.65 & \#72 \\
 x &  &  x &  x &  x &  x &  x &  & 146.78 & \#73 \\
 x &  &  x &  &  x &  x &  x &  & 148.84 & \#74 \\
 x &  &  x &  x &  x &  &  x &  & 152.48 & \#75 \\
 x &  &  x &  &  x &  x &  x &  x & 153.36 & \#76 \\
 x &  &  x &  x &  x &  x &  x &  x & 155.83 & \#77 \\
 x &  x &  x &  &  &  x &  x &  x & 156.78 & \#78 \\
 x &  &  &  x &  x &  x &  x &  & 159.09 & \#79 \\
 x &  &  &  x &  &  x &  x &  & 159.3 & \#80 \\
 x &  &  &  &  &  &  x &  & 159.57 & \#81 \\
 x &  &  x &  &  x &  &  x &  & 161.28 & \#82 \\ [1ex]
 x &  &  &  x &  x &  x &  x &  x & 167.37 & \#83 \\ [1ex]
 x &  &  x &  x &  &  &  x &  & 173.48 & \#84 \\ [1ex]
 x &  x &  x &  &  x &  x &  x &  x & 178.81 & \#85 \\ [1ex]
 x &  &  x &  x &  x &  &  x &  & 202.93 & \#86 \\ [1ex] % [1ex] adds vertical space
\hline %inserts single line
\end{tabular}}
\caption{Table showing the difference between using consumption and meteorological factors} % title of Table
\label{table:consumptionInclusionTable} % is used to refer this table in the text
\end{table}

