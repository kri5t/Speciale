When dealing with prediction of green energy production it is necessary to look at the weather in more detail. Wind energy prediction can be divided into two areas \cite{5}: 1) time-series analysis of power data and 2) wind speed prediction and conversion to power. What needs to be utilized in wind energy prediction is wind speed and wind direction which of course has a big influence on how much energy a wind turbine generate. An approach to analysis and prediction of wind power generation is presented in \cite{WindPowerGenerationUsingANN} where an Artificial Neural Network is used to predict the production. The prediction is based on weather data such as wind speed, relative humidity but also for how many hours the windmill is generating power.

Another source of green energy is solar panels. Solar
prediction algorithms are very accurate when the weather conditions are steady but in changing weather a large error margin is seen. To predict power output from a solar farm the time-series prediction algorithms and Estimated Weighted Moving Average (EWMA) models can be used \cite{5}.

Data centers is an example of a place where green energy prediction can be useful. It is often necessary for data centers to run long-running batch jobs where performance is measured in number of completed jobs and throughput. In \cite{5} they design an adaptive job scheduler that utilizes prediction of solar and wind energy production to scale workload (see introduction). Earlier work has focused on using immediate available green energy and then cancelling and rescheduling jobs thereafter. The point in \cite{5} is instead to scale the number of jobs to the expected availability of green energy production by predicting it beforehand. This helps reducing the number of cancelled jobs as the jobs are then scheduled for whenever the energy is available. If the amount of green energy production is not sufficient for an immediate or emergency job the remainder will be covered by brown energy. The system reduces the amount of wasted green energy and increases the overall throughput of the data center.
\\[0.5cm]
In this thesis we will only be focusing on energy that originates from windmills.