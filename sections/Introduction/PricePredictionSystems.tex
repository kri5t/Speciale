The ability of Neural Networks to forecast is seen in \cite{stockForecasting}. The stock price movements have similarities with electricity prices in terms of its non-linearity and chaotic/dynamic nature. Investors must take into account these factors when handling time series that are non-stationary, noisy and have structural breaks. In addition macro-economical elements significantly influence stock prices, e.g the economy in general, politics, bank rates and expectations of investors could be examples of such influences.

The solution for stock price forecasting in \cite{stockForecasting} is a hybrid forecasting model called Wavelet De-Noising-based Back Propagation (WDBP) Neural Network. In brief, the network is based on a wavelet transformation function that analyses non-stationary characteristics of the time series by decomposing the original data. The function produces a signal that is further decomposed into a low-pass and high-pass filter for every neuron in the network. The low-pass filter reflects main characteristics of the signal, opposed to the high-pass filter that represents noise. The purpose of the decomposition is to separate stock price characteristics from noise so that prediction will have better chances of accuracy when all undesirable has been discarded. The signal without the noise is then propagated through the Neural Network using back propagation.
\\[0.5cm]
A day-ahead forecasting algorithm that predicts electricity prices in the market based on Neural Network (ANN) and Similar Days Method (SDM) is described in \cite{pjmForecast}. The purpose is to give close estimates for several days to come. The estimates can be used by electricity traders in their decision making but also by transmission companies for different purposes. The companies can use it for scheduling a short-term generator outage in order to predict where it is most inexpensive. It can also be used by actual producers of energy to strategically bid into the market to increase prices. The price estimate itself plays a huge role in decision making in all of these examples.

The combination of ANN and SDM is an attempt to simplify the ANN and make the prediction more accurate. The algorithm forecasts by using a ANN that modifies price curves obtained by averaging five similar price days corresponding to the forecast day, i.e. The ANN corrects the received output from the similar days approach \cite{pjmForecast}. In other words the technique takes into consideration the influence of the most similar days and their price development in relation to the day we wish to forecast.

The ANN is trained with only 45 days from the day before the forecast and 45 days before and after the forecast day in the previous year \cite{pjmForecast}.
\\[0.5cm]
Prediction algorithms are not only used by electricity traders but can also be a part of various applications. In \cite{22} they introduce an intelligent electricity broker (IEB) that is integrated into Smart Grids where it; 1) provides provision of en energy storage; 2) attempts to lower the electricity bill, and 3) optimally utilizes electricity during peak and low-peak energy production periods. The prediction algorithm is used by a decision algorithm to locate points in time where it is most feasible for the owner of the smart grid to either sell stored energy or buy energy if storage is low. It helps the system to lower the total amount of energy costs to the owner and also utilize the energy in the most intelligent way.
 