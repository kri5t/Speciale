Decision Support Systems (DSS) can help users in critical environments where careful situation assessment and decision making is necessary. This is valid for the financial markets such as the stock and electricity market where multiple dynamic factors are present \cite{UncertainInformation}.
\\[0.5cm]
A combination of a Artificial Neural Network approach, an eXtended Classifier System (XCS) and the idea of cooperative learning is presented in \cite{groupLearningDS}. It is an attempt to incorporate the best of all three in an intelligent financial decision support system. XCS and ANN are both forecasting technologies and the system compares their results in order to get the best price estimate. The cooperative learning element shows itself in the way a decision is made. Each agent selects the best investment strategy from their own point of view based on the price estimates. The agents can simulate several strategies in each session and select the potential of highest profit. This information is shared with the rest and the final investment decision is made based on the decision of the majority of the other agents in the system. The DSS archives investment strategies that showed to be profitable in a premium strategy library in order to collect the best strategies. The agents don't necessarily have to use the premium strategies but it makes sense to first try what have succeeded in the past and if that fails move on to a random selection strategy. 

The intelligent system described above makes all decisions for the agent as opposed to the HCI approach to DSS discussed in \cite{UncertainInformation}. The focus is on uncertain information and how to present it to the user in the best possible way. They argue that uncertain information can't be denied and especially not in the domains with high risks such as the financials markets and military systems. The underlying reasoning algorithm used by the system will still make qualified estimates but the final decision is left to the user. The system will inform about uncertainties so the user can make trade-off's based on the information and then make a decision. Uncertainties can exist as different types such as bad data source reliability, conflict in data and data ambiguity but no matter the type it is presented to the user at every time. 

The paper further discuss different ways of presenting information including linguistic, textual and linguistic so that it is perceived better and faster by the user. Finally they present a number of guidelines that can used to develop a prototype of the interface for a DSS.