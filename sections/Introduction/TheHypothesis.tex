In this dissertation we take our outset in the growing need for predicting green energy prices. It is our intent to show that we can predict green energy prices by gathering weather data from online sources and combine it with historical price development data in a prediction algorithm based on a Back Propagation Artificial Neural Network. In our concrete case the prediction itself is to be used as decision support by energy traders in their attempt to beat the electricity market. We will be focusing on 1) building a Back Propagation Artificial Neural Network that is able to predict prices; 2) creating a web service for real-time fetching of the predicted prices. 
\\[0.5cm] 
The goal of the thesis is to investigate whether or not Back Propagation Artificial Neural Networks(BPNN) are a proper way to predict energy prices. The BPNN shall as a minimum be able to predict energy prices based on historical data like in \cite{2} and weather data. Based on this minimum requirement we will explore if 1) the prediction algorithm can come "some percentage" close to real-world data (the percentage needs to be defined!!); 2) the algorithm can be expanded to also include prediction of green energy prices; 3) real-time fetching of prices.