In this dissertation we take our outset in the growing need for green decision making. It is our intent to forecast green energy production and electricity prices within the electricity market by modelling an Artificial Neural Network using the back propagation algorithm for training of the network. The dataset for training consists of historical data that is relevant to the specific task. For instance when dealing with forecasting of wind power production for a wind farm, the most influential factors are meteorological such as wind speed, wind direction and air density together with historical production development of that particular farm. The energy prices are also influenced heavily by meteorological factors but it is instead combined with historical price development.
The goal is to investigate whether or not Back Propagation Artificial Neural Networks are a proper technology for doing prediction of green energy and electricity prices. Secondly, we are combining the two predictions in an attempt to support green decision making for traders. By green decision making is basically meant highlighting the points in time where the price is at its lowest and at the same time contains the highest percentage of green energy within a user defined time-interval.
\\[0.5cm]
We will be focusing on; 1) modelling and implementing a working Back Propagation Artificial Neural Networks that are capable of predicting price and energy production within the electricity market; 2) creating a Decision Support System (DSS) that will support green decision making.