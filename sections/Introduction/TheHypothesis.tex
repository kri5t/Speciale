Many before us have established that an Artificial Neural Network handles well time-series that are non-linear, volatile and noisy which applies for wind power and electricity prices \cite{stockForecasting,pjmForecast,yamin2004adaptive,windForecastPortugal}. It is our intent to build on this experience and model Artificial Neural Networks for prediction that use a variant of the back propagation algorithm for training of the networks. In this dissertation we focus on utilizing the inner the Artificial Neural Networks for prediction of electricity prices and wind power by identifying the influential input parameters through a comprehensive analysis validated through experiments. Wind power and electricity price predictions are key instruments for decision making in the electricity market\cite{dayAheadImpactOfWindPowerForecasts,21} and the actual use of Artificial Neural Networks for this purpose will be discussed.  

Artificial Neural Networks can be characterised as Machine Learning\cite{18} and historical data that is relevant for the specific task will be used for training. The goal is to investigate and identify (through analysis and experiments) the importance of the influential factors to be included and represented in these datasets. Based on this we examine the feasibility of a Back Propagation Artificial Neural Network as a technology for predicting electricity prices and wind power.
\\[0.5cm]
We will focus on; 1) identifying the influential factors for wind power and the electricity price; 2) modelling and implementing Back Propagation Artificial Neural Networks that are capable of predicting price and wind power within the Danish electricity market; 3) evaluating predictions originating from the Artificial Neural Networks in terms of both results and actual decision making.