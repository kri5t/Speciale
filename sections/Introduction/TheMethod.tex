The available data is prices, wind productions and meteorological factors measured every hour from earlier years. When variables are observed sequentially over time it constitutes a time series \cite[Chapter~7.1.2]{econometrics} where the values in the series evolve over time. Furthermore time series data are often strongly correlated over time which\cite[Chapter~7.1.2]{econometrics} which makes it usable for prediction. Based on the data and the non-linear nature of price forecasting and energy price prediction we want a system that is able to develop and learn from the past by analysing the time-series. Artificial Neural Networks can be categorized as Machine Learning \cite{18} and are networks that imitate the behaviour of the human brain \cite{1}. We choose to base our system on ANN because it gives us the power to be able to forecast energy prices based on how the prices have evolved in the past. Our decision is based on the nature of Artificial Neural Networks and how it is used in machine learning. It is often used in non-linear statistical analysis \cite{16} and has been used to predict various prices within the commodity market \cite{2,3,stockForecasting,pjmForecast}. We have chosen to use a multi-layered feed forward architecture since this is one of the most used and widely implemented in open-source frameworks \cite{17}.

We will base our work on Artificial Neural Networks and the resilient backpropagation (RPROP) algorithm used to train the network in order to predict electricity prices and wind power generation. It is faster than its predecessor Backpropagation \cite{8,15}.  The RPROP algorithm is a learning algorithm that uses weights to analyze the input dataset and to configure further learning \cite{17}. It is commonly used on the feedforward architecture and is the most used and implemented algorithm on ANNs \cite{14,17}. The historical data to include in the dataset will be identified through a systematic analysis of the electricity price and wind power characteristics together with experiments for verification. The result will be the optimal network settings for prediction of wind power and electricity prices in relation to best network structure, input, representation, size of training set and number of training epochs. The findings of our analysis and experimental results will be related to practical use in terms of decision making.