We will base our work on Artificial Neural Networks and the resilient backpropagation (RPROP) algorithm used to train the network in order to predict electricity prices and wind power generation. It is faster than its predecessor Backpropagation \cite{8,15}.  The RPROP algorithm is a learning algorithm that uses weights to analyze the input dataset and to configure further learning \cite{17}. It is commonly used on the feedforward architecture and is the most used and implemented algorithm \cite{14,17}.

The prediction algorithm will be based on historical data which potentially could be a very large dataset. The more historical data included in the algorithm the more data needs to be processed. We need to investigate how much of this data is necessary for precise prediction and perhaps make tradeoffs to ensure performance for decision support.

Based on the dataset that is available and the nonlinear nature of price forecasting and energy price prediction we want a system that is able to develop and learn from the past. Artificial Neural Networks can be categorized as Machine Learning \cite{18} and are networks that imitate the behaviour of the human brain \cite{1}. We choose to base our system on ANN because it gives us the power to be able to forecast energy prices based on how the prices have evolved in the past. Our decision is based on the nature of Artificial Neural Networks and how it is used in machine learning. It is often used in non-linear statistical analysis \cite{16} and has been used to predict various prices within the commodity market \cite{2,3,stockForecasting,pjmForecast}. We have chosen to use a multi-layered feed forward architecture since this is one of the most used and widely implemented in open-source frameworks \cite{17}.