Renewable energy has become increasingly important. Energy suppliers make it possible for their customers to choose between green and brown energy. Companies and persons promote themselves with green profiles to get a certain kind of image and it has become a choice of the individual company or home to be green. This is only one part of the picture, because the increased attention on green energy is influencing the market forces behind the scenes where acquisition of energy begins \cite{windPowerDanishLiberalized}. Various trading companies buy electricity on the deregulated energy market and since the amount of renewable energy in this market is increasing and will do for the years to come\cite{6, windPowerDanishLiberalized} the traders need to consider it carefully when buying and selling. The most essential task and basis for any decision making in the power market is to forecast the electricity price \cite{pjmForecast} and wind power production \cite{dayAheadImpactOfWindPowerForecasts}. Traditional producers can use the wind power predictions to strategically place their own energy in the market \cite{21}.
\\[0.5cm]
The electricity market consists of two instruments in order to facilitate trading between producers and consumers of energy: the pool which works as an online marketplace, and a framework to make physical contracts possible between both sides \cite{21}. The producers can make their electricity available and the consumers can bid correspondingly. An auction is made every hour to decide the market clearing price and which bids are accepted for that hour. The consumers and producers need to predict the hourly clearing prices to make plans for which bidding strategies to use. If a trader has a precise day-ahead prediction of the prices it will be easier to maximize profit by using the best possible strategy for exactly those conditions \cite{21}. The Electricity traders will attempt to buy electricity in a low-price market and sell it when the price has come up like in other commodity markets \cite{FIND REF}. With a precise prediction they know when to buy and when to sell. 

Approximately 25\% of the total power production in Western Denmark was covered by wind power in 2008 and influenced the prices on the power market accordingly \cite{windPowerDanishLiberalized}. The impact on price is seen in wind power variations in areas where wind power production covers a substantial amount of the total supply which apply for Denmark. For this reason it cannot be avoided when predicting electricity prices\cite{dayAheadImpactOfWindPowerForecasts}. Since the amount of green energy in the entire Danish liberalized market is targeted to increase to 30\% by 2025 which is almost a doubling since 2008 \cite{windPowerDanishLiberalized} the influence will only become greater. The price impact from windmills will be even greater in strong wind periods and in areas with congestions in the power transmission capacity \cite{windPowerDanishLiberalized}. The increase in renewable energy makes the ability of predicting green energy production and the corresponding market price more vital. It is not trivial to predict wind production because green energy by nature is unpredictable, e.g. wind power is highly influenced by wind speed and air density. If the traders are able to predict wind power when doing actual price forecasts it will give rise to a huge advantage in the market when deciding to buy or sell \cite{dayAheadImpactOfWindPowerForecasts}. Price forecasting itself contains many other factors than wind power that needs to be considered equally \cite{21}.
\\[0.5cm] Traders buy and sell in real-time, intra-day or day-ahead \cite{FIND REF}. This puts constraints on a price prediction algorithm --- but what constraints depend solely on the type of trade. If the trader wishes to buy in real-time or hour-ahead the algorithm must perform and deliver a result within minutes or seconds. If it is day-ahead then the time-interval can be increased. This might result in a compromise on the margin of error because fast results make less time for analysis and computation. Another fact is that the closer you get to the time of the trade the more accurate the weather predictions will be which directly impacts the price prediction algorithm using it. The ability to make both short- and long-term forecasts is very important in the deregulated competitive electricity market because it helps the trader to reduce risks in terms of under/over estimating the revenues from potential sales and this of course also makes it easier to manage risk\cite{21}.
\\[0.5cm]
Demand has a huge influence on the electricity price. No demand, no electricity price, but if these were the only two variables to consider, a linear regression model could be used for establishing a relationship and thereby a prediction. Other dynamic elements have an impact on the price and therefore the linear regression approach would result in the presence of serial correlation in the error \cite{21}. For instance will overestimated energy prices for one year most likely lead to overestimates the next year since it does not account for all the dynamic parameters. It is necessary to carefully consider all variables and characteristics of the price that we are trying to predict in the electricity market and then use a model that handles correlated errors based on this. The most noticeable characteristics are \cite{21}:
\begin{itemize}
\item High frequency;
\item High volatility;
\item Unusual prices in times of very high demand;
\item Calendar effect (weekend, holidays);
\item Multiple seasonality (daily and weekly periodicity);
\end{itemize}
When doing electricity price forecasting these characteristics must be taken into consideration when modelling a prediction algorithm. 

\todo{describe how green decision making is important due to the regulation of prices if the consumption can be covered by the amount of green energy. This makes it important to identify the green energy.}