This section discusses the results based on both electricity prices and wind production.

\subsection{The importance of the underlying data}
The strength of the prediction is very much dependent on the quality of the underlying data. As described in Section~\ref{sec:dataCollection} the data must satisfy certain criteria before it can be used as input and output in the Artificial Neural Network. It must be trustworthy and contain hourly observations for our specific purpose, e.g. day-ahead prediction based on a dataset that consists of hourly observations. All calculations in the ANN are based on this data which makes it of utmost importance for the output, the forecast.

\subsubsection{Input parameters}
\todo{this section is pretty obvious when looking at our analysis. You need to establish relationship between the output and the input parameters}

\todo{USE THE MODEL FROM UNCERTAIN INFORMATION}

\subsubsection{Data Manipulation - k}
De andre tekster undlader i stor omfang at snakke om input i forhold til konkret data, trimming og manipulation af det konkrete data. Sammenstil det med vores eget. Reflekter og konkluder.

== the best possible prediction. Hvad er med til at give det ebdsre result.

all our findings apply on both systems (green/price). It emphasizes the need for manipulation of data in all networks even though different measures to be met.

\subsection{ANNs ability to predict}


\todo{Neural works as basis for a decision support system. If ANN can predict can it then be used in the real life? Discussion about how this is related the text where the results are ambiguous }
Artificial Neural Networks can it itself be considered as a Optimization Based Support Model which is described in Section \todo{section and discuss further}. The trust of the forecast is critical and especially when dealing with ANNs because they are a black box. 

\subsection{Comparison with other results}

