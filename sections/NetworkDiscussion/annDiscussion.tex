This section discusses the results ..........

\subsection{The importance of the underlying data}

\subsubsection{Input parameters}

\subsubsection{Data Manipulation - k}
De andre tekster undlader i stor omfang at snakke om input i forhold til konkret data, trimming og manipulation af det konkrete data. Sammenstil det med vores eget. Reflekter og konkluder.

== the best possible prediction. Hvad er med til at give det ebdsre result.

\subsubsection{Black Box optimization}



\subsection{Trust - b}
We have obtained the best possible prediction through data manipulation and black box optimization. The network outputs a prediction which in itself could be used as decision support by the traders. The problematic here is for the users to know when it can be trusted and why to even trust it. People are not in the habit of blindly trusting everything that is placed in front of them and especially when it comes to money.


Communicate what we are building the prediction on. Show how we have performed etc.

\todo{show examples with in low and high numbers}

Or maybe rather say how much can this actually be trusted.

 
the best possible prediction, how do you get people to trust it. "Black"-magic. Hvad sker der.
How do we communicate it.

\subsubsection{The underlying data}
Artificial Neural Networks are known to be a black box\todo{ref}. Knowledge about the underlying data is only known by the ones who built the system. First and foremost the system must communicate the input parameters and where it comes from.

\subsubsection{Uncertain information}
f.x. the difference in accuracy when in the lowest and highest numbers.

\subsubsection{The output}


\subsection{WHAT SHOULD BE DONE}
\begin{itemize}
\item Improve understanding of the underlying data and the output;
\item Transparency is the goal;
\item Let the user take decisions based on the uncertain information;
\item Visualize errors;
\end{itemize}

\subsubsection{Transparency}

