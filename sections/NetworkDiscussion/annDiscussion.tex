This section discusses the results based on both electricity prices and wind production.

\todo{discuss how the applying of calculated inputs kind of "disturbs" the generalization function}

\subsection{The importance of the underlying data}
The strength of the prediction is very much dependent on the quality of the underlying data. As described in Section~\ref{sec:dataCollection} the data must satisfy certain criteria before it can be used as input and output in the Artificial Neural Network. It must be trustworthy and contain hourly observations for our specific purpose, e.g. day-ahead prediction based on a dataset that consists of hourly observations. All calculations in the ANN are based on this data which makes it of utmost importance for the output, the forecast.


\subsubsection{Input parameters}
\todo{this section is pretty obvious when looking at our analysis. You need to establish relationship between the output and the input parameters}

\todo{analysis of the network is very hard due to the black box. It finds connections that you cant foresee. "Its own life".}

\todo{data set and manipulation will always differ from task to task. Look at wind prediction vs. price. Volatility, skew ect. is different}

\todo{USE THE MODEL FROM UNCERTAIN INFORMATION}

\todo{A VERY DEEP UNDERSTANDING OF THE INPUT IS NEEDED --- talk about that}
\todo{discuss that we are not using predicted weather data}

\subsubsection{Data Manipulation - k}
De andre tekster undlader i stor omfang at snakke om input i forhold til konkret data, trimming og manipulation af det konkrete data. Sammenstil det med vores eget. Reflekter og konkluder.

== the best possible prediction. Hvad er med til at give det ebdsre result.

all our findings apply on both systems (green/price). It emphasizes the need for manipulation of data in all networks even though different measures to be met.

\subsection{ANNs ability to predict}


\todo{Neural works as basis for a decision support system. If ANN can predict can it then be used in the real life? Discussion about how this is related the text where the results are ambiguous }
Artificial Neural Networks can it itself be considered as a Optimization Based Support Model which is described in Section \todo{section and discuss further}. The trust of the forecast is critical and especially when dealing with ANNs because they are a black box. 

\todo{discuss this in relation to DSS and how it relates to it. It is here important to emphasize that in order to use this a lot of things around it has to be build. When used in a real life setting the user just need to know the input, a high level description of what happens and then the output}

\subsection{Comparison with other results}
\todo{compare matrix between price and production. It is obvious that the bigger difference between high and low during the day makes the benefits from matrix more clear for price than for production}.

\subsection{LITTERATURE?}
\todo{related to neural networks and what it is also used for and why this is a good approach here}
\todo{relate it to machine learning and why this fits our problem}

\subsection{Testing}
\todo{talk about how testing is time consuming. All tests run for long time}

\todo{the results are not definitive but what we can do is see tendencies in what parameters really matter.}

\todo{ideal world endless test --> neural network requires a lot of testing}



\todo{what makes this ai or machine learning}


