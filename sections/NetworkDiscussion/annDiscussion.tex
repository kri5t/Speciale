This section discusses the results based on both electricity prices and wind production.

\todo{discuss how the applying of calculated inputs kind of "disturbs" the generalization function}

\subsection{The importance of the underlying data}
The strength of the prediction is very much dependent on the quality of the underlying data. As described in Section~\ref{sec:dataCollection} the data must satisfy certain criteria before it can be used as input and output in the Artificial Neural Network. It must be trustworthy and contain hourly observations for our specific purpose, e.g. day-ahead prediction based on a dataset that consists of hourly observations. All calculations in the ANN are based on this data which makes it of utmost importance for the output, the forecast.


\subsubsection{Input parameters}
\todo{this section is pretty obvious when looking at our analysis. You need to establish relationship between the output and the input parameters}

\todo{analysis of the network is very hard due to the black box. It finds connections that you cant foresee. "Its own life".}

\todo{data set and manipulation will always differ from task to task. Look at wind prediction vs. price. Volatility, skew ect. is different}

\todo{USE THE MODEL FROM UNCERTAIN INFORMATION}

\todo{A VERY DEEP UNDERSTANDING OF THE INPUT IS NEEDED --- talk about that}
\todo{discuss that we are not using predicted weather data}

\subsubsection{Data Manipulation - k}
De andre tekster undlader i stor omfang at snakke om input i forhold til konkret data, trimming og manipulation af det konkrete data. Sammenstil det med vores eget. Reflekter og konkluder.

== the best possible prediction. Hvad er med til at give det ebdsre result.

all our findings apply on both systems (green/price). It emphasizes the need for manipulation of data in all networks even though different measures to be met.

\subsection{ANNs ability to predict}


\todo{Neural works as basis for a decision support system. If ANN can predict can it then be used in the real life? Discussion about how this is related the text where the results are ambiguous }
Artificial Neural Networks can it itself be considered as a Optimization Based Support Model which is described in Section \todo{section and discuss further}. The trust of the forecast is critical and especially when dealing with ANNs because they are a black box. 

\todo{discuss this in relation to DSS and how it relates to it. It is here important to emphasize that in order to use this a lot of things around it has to be build. When used in a real life setting the user just need to know the input, a high level description of what happens and then the output}

\subsection{Comparison with other results}
\todo{compare matrix between price and production. It is obvious that the bigger difference between high and low during the day makes the benefits from matrix more clear for price than for production}.

\subsection{LITTERATURE?}
\todo{related to neural networks and what it is also used for and why this is a good approach here}
\todo{relate it to machine learning and why this fits our problem}

\subsection{Testing}
\todo{talk about how testing is time consuming. All tests run for long time}

\todo{the results are not definitive but what we can do is see tendencies in what parameters really matter.}

\todo{ideal world endless test --> neural network requires a lot of testing}



\todo{what makes this ai or machine learning}


\subsection{Recalculate}
POTENTIAL STRATEGY
The re-calculation concept works by letting other neural networks re-calculate the prediction from the original network. The purpose is to identify places where the generalization function has problems and then divide it into new neural networks that only have the purpose of focusing on these problematic situations. For simplicity we are going to apply this to the small numbers and big numbers of the dataset. We will define what is small and big by taking the 2 and 98 percentile. Whenever the "original" network predicts something in or close to the big or small interval then the responsibility of that prediction is forwarded to one of the two. It works like a second opinion but here the network has been specialized to only focus on a specific part of the dataset instead of everything\todo{give example of corrections}. It creates a normalization function on a smaller dataset and the idea is to leave out a lot of unnecessary information. It does not need to account for anything other than its interval in the generalization function so it should be able to predict its target group better. We let the original network decide if it actually thinks we are in the low numbers --- if we are then we can possibly make that prediction even more accurate. These thoughts could basically we applied on all other parts of possible wind production values. 

Problem can occur if the intervals are too small which will cause the dataset to be too small and hard for the ANN to generalize upon.

\subsubsection{Prediction - Similar Days}
The similar days approach is described in Section~\ref{sec:sdmApproach} and addresses what to include in the dataset based on parameters. Wind power production follows wind speed and the expectation is that if filtering out days with completely different wind speeds and productions then a better accuracy can be obtained. The similar days analysis will take all wind speeds from the hours to predict and find all days within this interval plus/minus one in each end, e.g. an interval of wind speeds between 4-9 would result in similar days between 3-10. There is a potential for not getting the entire picture when using similar days. It can be related to the discussion about wind speed as matrix and how some values are under-represented in the dataset. In could happens in shifting seasons that wind speeds are never seen before and therefore run into a dataset without many hours. In those cases the entire dataset is used instead. The analysis in Section~\ref{sec:windProdSeasonality} describes the difference from season to season.

We can see from Table~\ref{table:theSimilarDaysApproach3monthTable} that the results are much equal to the best input combination from Table~\ref{table:windProdInputParamsTop10}. The purpose of the similar days is to filter out unnecessary noise from the dataset but since the dataset is of a manageable size and in itself contain a seasonal aspect it does not filter out many values as seen in the table. When applying it on an entire year more values are filtered out but it cannot make up for the increase in dataset as seen in Table~\ref{table:theSimilarDaysApproachYearTable}.