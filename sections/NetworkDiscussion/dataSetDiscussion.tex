In the analysis of the electricity price and the wind power we saw some similarities between the two but also some correlations between input data and the wind power/electricity price that surprised us.

First of all the demand showed to have an influence on the electricity price but also on the wind power. We did not anticipate a strong connection between demand and wind power because we only thought that the wind power would influence the market and not the other way around as well. All of the \fnurl{wind power}{http://en.wikipedia.org/wiki/Wind_power_in_Denmark\#Capacities_and_production} produced in Denmark are in large parts consumed in the danish electricity market and therefore also influenced by demand since it determines when the power produced can be used in the market.

The timely and seasonal factors showed to have an influence on the electricity price (Section \ref{sec:seasonality}) and the wind power(Section \ref{sec:windProdSeasonality}). The timely factor had a great influence on the price and the wind power was also influenced by time of day which was surprising since the wind power showed a pattern where it rose from 08 to 14 and declined again from 14 to 20. This is because of the connection to the market reflecting the demand during day-time which can also be seen in the price and time of day coherence. The seasonal factors also showed an impact on both datasets. The wind power is influenced greatly by wind speed which reflects the seasonal changes in the meteorological factors. The price is affected by seasonal changes in the demand for electric use which are also because of the meteorological factors thus solidifying that meteorological factors plays a great role when determining wind power and electricity price forecasts.

Another factor that had a great influence was the last-known price(Section \ref{sec:Price}) and wind production(Section \ref{sec:windProductionDev}). This shows that the curve behavior is important when we are predicting either wind power or electricity prices. If we were able to know the exact price/wind production before the one we are trying to predict the predictions would be more accurate because of the strong correlation between predictions and last-known value. This can be pretty hard when doing day-ahead predictions where we have to rely on earlier predictions to predict the 24th hour.

We saw that volatility played a role in both of the datasets. The electricity price were more volatile than the wind power. This can be explained by the sociological factors mentioned in Section \ref{sec:volatility} that are very hard to foresee because of their unpredictable nature. These factors also render a good place for improvements by analysing how they possibly can be foreseen and accounted for.

%Also talk about differences and similarities in the two data sets. Same data. Seasonality and hours impact both. ect.
%Demand i forhold til wind speed. Større link imellem markedet og vindkraft.
%Time of Day spiller ind paa begge to. Maaske surprising at den ogsaa har indvirkning paa wind power.
%Seasonality i forhold til de to.
%High correlation between last-known input and the price to predict.
%Volatilility i begge. Price mest.