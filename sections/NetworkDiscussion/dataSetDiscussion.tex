The analysis of the electricity price and the wind power revealed similarities between the two and correlations between input data and the wind power/electricity price that was surprising.

First of all, the demand was expected to influence electricity price but not the wind power as significantly as it was established. The expectation was that wind power would only influence the market and not the other way around\fnurl{wind power}{http://en.wikipedia.org/wiki/Wind_power_in_Denmark\#Capacities_and_production}. The connection indicates that the wind power is moved into the electricity market according to the demand.

The timely and seasonal factors showed to have an influence on the electricity price (Section \ref{sec:seasonality}) and the wind power(Section \ref{sec:windProdSeasonality}). The demand influence price and since demand is much controlled by consumer habits these will be reflected in the price. Wind power surprisingly showed a pattern in time of day productions due to the established connection to demand. The seasonal factors also showed an impact on both datasets. The wind power is influenced greatly by wind speed which reflects the seasonal changes in the meteorological factors. The price is affected by seasonal changes in the demand for electric use which are also because of the meteorological factors thus solidifying that meteorological factors plays a great role when determining wind power and electricity price forecasts.

Another factor that had a great influence was the last-known price(Section \ref{sec:Price}) and wind production(Section \ref{sec:windProductionDev}). This shows that the behaviour of the curve is important when we are predicting either wind power or electricity prices. If we are able to know the exact price/wind production before the one we are trying to predict the predictions could possibly obtain better accuracy due to the strong correlation to the last known value. This can be a challenging task when performing day-ahead prediction because all steps must rely on all previous predictions (except from the first hour). 

We saw that volatility played a role in both of the datasets but with more significance for the electricity price. The volatility for price can be explained by the sociological factors as mentioned in Section \ref{sec:volatility} that are very hard to foresee because of their unpredictable nature. Accounting for these factors in the network model is necessary to investigate because it will help add further characteristics of the price or wind power.

%Also talk about differences and similarities in the two data sets. Same data. Seasonality and hours impact both. ect.
%Demand i forhold til wind speed. Større link imellem markedet og vindkraft.
%Time of Day spiller ind paa begge to. Maaske surprising at den ogsaa har indvirkning paa wind power.
%Seasonality i forhold til de to.
%High correlation between last-known input and the price to predict.
%Volatilility i begge. Price mest.