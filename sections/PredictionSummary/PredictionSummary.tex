\subsubsection{Summary}
The characteristics of the electricity prices have been refreshed and different prediction methods have been presented. In the description of ANN prediction it is worth observing the input parameters of the network and the way it is modelled. The demand input can potentially consist of input parameters that represents the influential factors for the demand such as the meteorological and socio-economic factors. The price characteristics must be analysed for Western Denmark to identify the influential factors. The same applies for wind power and the influences of wind speed, air density, temperature amongst others.

ANNs can be used in combination with other technologies in an attempt to improve performance and accuracy. The ability to identify trend is also presented because the price is volatile and consists of many spikes \cite{singhal2011electricity}. It must be considered how we can obtain more information about the uncertainty arising from volatility. Furthermore, two other examples of predicting and how they relate to ANN is described. This further motivates our choice of ANN as prediction method. The ANN has shown to outperform ARIMA in the Spanish market which indicates its feasibility. 

Support Vector Machines (SVM) have been presented as an alternative approach to Machine Learning. The approach highlights four phases to create a SVM in Figure~\ref{fig:phasesOfSVM} that can be transferable to the use of Artificial Neural Networks. They emphasize that simulations must be done as (what they call) "out-of-sample"-forecasting to get realistic results. Data sampling and preprocessing is also presented as phases that also applies for Artificial Neural Networks.   

\subsection{Concluding Remarks}
This section introduced the concepts and approaches of electricity demand, wind power and electricity prices. Demand is important for price and the potential of substituting it with meteorological factors can be a way of letting the network calculate the influence itself. Section~\ref{sec:annSection} about Artificial Neural Networks established the importance of the dataset and thereby the necessity for investigating the influential factors for wind power and electricity price must be emphasized. The presented papers describe the characteristics and inputs but do not go into details with why it has been selected and why it works for that specific market. We must therefore perform a comprehensive analysis followed by experiments to establish all of these connections based on the characteristics. It applies for both wind power and the electricity price that are both dependent on weather data. It indicates that we can apply similar strategies when predicting both the wind power and electricity price --- this of course must be verified during the analysis and experimentation.