It is often necessary for data centers to run long-running batch jobs where performance is measured in number of completed jobs and throughput. In \cite{5} they design an adaptive job scheduler that utilizes prediction of solar and wind energy production to scale workload (see introduction). Earlier work has focused on using immediate available green energy and then cancelling and rescheduling jobs thereafter. The point in \cite{5} is instead to scale the number of jobs to the expected availability of green energy production by predicting it beforehand. This helps reducing the number of cancelled jobs as the jobs are then scheduled for whenever the energy is available. If the amount of green energy production is not sufficient for an immediate or emergency job the remainder will be covered by brown energy. The system reduces the amount of wasted green energy and increases the overall throughput of the data center.