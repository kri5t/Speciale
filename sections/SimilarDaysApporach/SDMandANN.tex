A day-ahead forecasting algorithm that predicts electricity prices in the market based on Neural Network (ANN) and Similar Days Method (SDM) is described in \cite{pjmForecast}. The purpose is to give close estimates for several days to come. The estimates can be used by electricity traders in their decision making but also by transmission companies for different purposes. The companies can use it for scheduling a short-term generator outage in order to predict where it is most inexpensive. It can also be used by actual producers of energy to strategically bid into the market to increase prices. The price estimate itself plays a huge role in decision making in all of these examples.

The combination of ANN and SDM is an attempt to simplify the ANN and make the prediction more accurate. The algorithm forecasts by using a ANN that modifies price curves obtained by averaging five similar price days corresponding to the forecast day, i.e. The ANN corrects the received output from the similar days approach \cite{pjmForecast}. In other words the technique takes into consideration the influence of the most similar days and their price development in relation to the day we wish to forecast.

The ANN is trained with only 45 days from the day before the forecast and 45 days before and after the forecast day in the previous year \cite{pjmForecast}. Results can be seen in table~\ref{table:sdmresult}

\begin{table}[h!]
\centering  % used for centering table
\begin{tabular}{c c c} % centered columns (3 columns)
Year 2006 \#1 & ANN (Avg. MAPE [\%]) \#2 & ANN ( FMSE [\$/MWh] ) \\ [0.5ex] % inserts table 
%heading
\hline                  % inserts single horizontal line
January 20 & 6.93 & 4.57  \\ % inserting body of the table
February 10 & 7.96 & 6.12  \\
March 05 & 7.88 & 5.39  \\
April 07 & 9.02 & 5.87 \\ [1ex] % [1ex] adds vertical space
\hline %inserts single line
\end{tabular}
\caption{Results of forecasting from \cite{pjmForecast}.} % title of Table
\label{table:sdmresult} % is used to refer this table in the text
\end{table}

The relatively low number of days could be the cause of higher error margin than \cite{FIND A SOURCE} because as mentioned in \cite{18} the more data the more complex problems can be handled. The reason for the days are not discussed in the paper.

Artificial Neural Networks (ANN) has also been used for electricity price forecasting. In \cite{singhal2011electricity} they use an ANN to predict the half-hourly price of electricity of 24 hours. They differentiate between three different kinds of days: Normal trend price, Price with small spikes and price with large spike. They present a prediction for each of these days and present us to the mean absolute error and the root mean square error, which are standard measures for how accurately the prediction is done. The neural network is fed with 13 inputs as follows \cite{singhal2011electricity}:
\begin{itemize}[noitemsep,topsep=3pt,parsep=2pt,partopsep=3pt]
\item Day of week
\item Time slot of Day
\item Forecasted Demand
\item Change in demand
\item Price (one day ago) - 3 inputs 
\item Price (one week ago) - 3 inputs
\item Price (two weeks ago) - 1 input 
\item Price (three weeks ago) - 1 input 
\item Price (four weeks ago) - 1 input
\end{itemize}
These inputs are fed into a 4-layer neural network: one input layer, two hidden layers and an output layer. The analysis of their ANNs shows that neural networks make a very precise prediction on normal trend price days but have difficulties forecasting the price with small and large spikes. They argue that if the reasons to the spikes in the price were taken into account as inputs in the network maybe the network would be better at forecasting the spikes prices. Also they argue that fuzzy logic, neural networks and dynamic clustering together will provide more efficient forecasting of the prices.