Predicting electricity prices of tomorrow is of the utmost importance for electric companies to adjust their daily bids. The energy service companies buy electricity on the market and then sell it to their customers. In order to maximize profits the traders of these companies need accurate short-term price predictions \cite{EnergyPriceForecasting}. The hourly price characteristics have been described in the introduction as; 1) highly frequent; 2) highly volatile; 3) calendar affected; 4) having unusual prices due to uncontrolled market events; 5) multiple seasonality (daily and weekly periodicity). These characteristics must be analysed in detail and verified through experiments for the market that the network attempts to predict in. 

\subsubsection{Artificial Neural Network Prediction}\label{sec:scatterPaper}
\cite{singhal2011electricity} use an ANN to predict the half-hourly price of electricity for the next 24 hours. They differentiate between three different kinds of days: Normal trend price, price with small spikes and price with large spike. They present a prediction for each of these days and present us to the mean absolute error and the root mean square error, which are standard measures for how accurately the prediction is done. The neural network is fed with 13 inputs as follows \cite{singhal2011electricity}:
\begin{itemize}[noitemsep,topsep=3pt,parsep=2pt,partopsep=3pt]
\item Day of week
\item Time slot of Day
\item Forecasted Demand
\item Change in demand
\item Price (one day ago) - 3 inputs 
\item Price (one week ago) - 3 inputs
\item Price (two weeks ago) - 1 input 
\item Price (three weeks ago) - 1 input 
\item Price (four weeks ago) - 1 input
\end{itemize}
These inputs are fed into a 4-layer neural network: one input layer, two hidden layers and an output layer. The analysis of their ANNs shows that neural networks make a very precise prediction on normal trend price days but have difficulties forecasting the price with small and large spikes. They argue that if the reasons to the spikes in the price were taken into account as inputs in the network maybe the network would be better at forecasting the spikes prices. Also they argue that fuzzy logic, neural networks and dynamic clustering together will provide more efficient forecasting of the prices.
\\[0.5cm]
Since price forecasts in electricity markets are such a volatile operation because of the shifting tides, price demand, holidays etc. that affects the price it can be a cumbersome problem to model. In \cite{amjady2006day} he proposes a new method based on neural networks and fuzzy logics to predict the electricity prices. He calls the new network approach a new fuzzy neural network. Fuzzy logic is basically a logic that has many values or many correct answers. Opposed to binary logic sets, where the answer can be only true or false, in fuzzy logic we have several grades of what we define as true, thus making it harder to decide what is really the truth but also makes us able to have a way larger scale of the data we are looking at.

The fuzzy logic is used within the nodes in the hidden layer to do evaluation of the data inputs. That is the activation function of the neurons in the hidden layer contains a fuzzification function that creates a square of the inputs compared to sinus activation functions that normally takes the sums of the inputs. This square over the inputs is used to classify the inputs into hyper cubes (input spaces) and then calculating how close they are together. Calculation of how far the inputs are from each other are used to calculate the output of the functions. By this we will get an upper and lower limit that the inputs can range between and thus we get an input that has the characteristics which turns out to give a better result than ARIMA (discussed below), wavelet-ARIMA, MultiLayered ANNs and Radial Basis based ANNs. The data however is not optimized in this paper and they claim this to be future work. They also state that the better performance is based on a limited dataset.