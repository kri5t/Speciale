This dissertation think of the feasibility of Artificial Neural Networks as dependent on both obtaining accurate results but also in its potential for practical use. Decision Support Systems (DSS) help users in complex domains that fits well with the electricity market. They mention the formulation, solution and analysis stages that incorporates intelligent systems that do big calculations and present them to the user/agent in a ready to use fashion. According to this we are conducting the first two steps of the implementation of the DSS. The third step deals with how to show the calculations to a user or agent through a Human Computer Interface. In the domain of the electricity market this would include showing uncertain information to the user to always make room for qualified guesses if the underlying data is not trustworthy. It is presented in \cite{UncertainInformation} that ignoring uncertain information in complex markets can result in creating uncertainties and thereby taking bad decisions. \cite{UncertainInformation} furthermore presents the types of uncertainty and how we are dealing with the objective ones in this thesis (see Figure~\ref{fig:typesOfUncertainty}. What is very important for us (as well as others relying on our prediction) is the reliability of the inputs in terms of data source but also how the different data is aggregated, presented and analysed.