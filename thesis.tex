\documentclass[twoside,11pt,openright]{report}

\usepackage[latin1]{inputenc}
\usepackage[american]{babel}
\usepackage{a4}
\usepackage{latexsym}
\usepackage{amssymb}
\usepackage{amsmath}
\usepackage{epsfig}
\usepackage[T1]{fontenc}
\usepackage{lmodern}
\usepackage[labeled]{multibib}
\usepackage{color}
\usepackage{datetime}
\usepackage{epstopdf}
\usepackage{graphicx}
\usepackage{subcaption}
\usepackage{hyperref}
\usepackage{enumitem} %Used to make lists smaller
\usepackage{float}
\usepackage{placeins}
\usepackage[table]{xcolor}
\usepackage{todonotes}
\usepackage{menukeys}
\usepackage{longtable}
\usepackage{rotating}
%\usepackage{amssymb}% http://ctan.org/pkg/amssymb
%\usepackage{pifont}% http://ctan.org/pkg/pifont

\newcommand{\x}{\scalebox{0.6}{x}}%
\newcommand{\m}{\scalebox{0.6}{(m)}}%
\renewcommand*\ttdefault{txtt}

\newcommand\fnurl[2]{%
  \href{#2}{#1}\footnote{\url{#2}}%
}

%\newcommand{\todo}[1]{{\color[rgb]{.5,0,0}\textbf{$\blacktriangleright$#1$\blacktriangleleft$}}}

% \newcites{A,B}{Primary Bibliography,Secondary Bibliography}

% see http://imf.au.dk/system/latex/bog/

\begin{document}

%%%%%%%%%%%%%%%%%%%%%%%%%%%%%%%%%%%%%%%%%%%%%%%%%%%%%%%%%%%%%%%%%%%%%%%
	
\pagestyle{empty} 
\pagenumbering{roman} 
\vspace*{\fill}\noindent{\rule{\linewidth}{1mm}\\[4ex]
{\Huge\sf Wind Power and Electricity Price Prediction Using Artificial Neural Networks to Support Decision Making in the Danish Electricity Market}\\[2ex]
{\huge\sf Brian Bak Laursen, 20071275}\\[2ex]
{\huge\sf Kristian Barrett, 20073457}\\[2ex]
\noindent\rule{\linewidth}{1mm}\\[4ex]	
\noindent{\Large\sf Master's Thesis, Department of Computer Science, ICT Product Development\\[1ex] 
\monthname\ \the\year  \\[1ex] Advisor: Niels Olof Bouvin\\[15ex]}\\[\fill]}
\epsfig{file=logo.eps}\clearpage

%%%%%%%%%%%%%%%%%%%%%%%%%%%%%%%%%%%%%%%%%%%%%%%%%%%%%%%%%%%%%%%%%%%%%%%

\pagestyle{plain}
\chapter*{Abstract}
\addcontentsline{toc}{chapter}{Abstract}
\definecolor{light-gray}{gray}{0.70}

In this thesis, we describe the foundation for modelling Artificial Neural Networks for prediction in the area of wind power and electricity pricing for the Danish electricity market. The thesis is divided into three areas of concern: (a) analysing the influential factors of wind power and electricity price; (b) constructing experiments for verification of the analysis and establishing the best Artificial Neural Network models for prediction; (c) determining feasibility of the approach by discussing experimental results as well as the potential for its practical use.

The main concepts of Artificial Neural Networks and time-series forecasting are presented. They include Machine Learning, Feedforward Neural Network, Supervised Learning with Backpropagation and Decision Support. The thesis highlights the characteristics of wind power and the electricity price based on concepts from existing prediction systems. 

We facilitate the Feedforward Neural Networks with Resilient Backpropagation towards the prediction models. The models learn from relevant data that consists of influential historical data of either wind power or the electricity price.

The thesis illustrates the feasibility of Artificial Neural Network in the context of wind power and electricity price prediction through the experimental results and discussions thereof.

\chapter*{Resum\'e}
\addcontentsline{toc}{chapter}{Resum\'e}

Vi beskriver i dette speciale fundamentet for at modellere Artificial Neural Networks til forudsigelse af vindenergi og elektricitets priser i det danske elektricitets-marked. Vi inddeler problemstillingen i tre omr\aa der: (a) analyse af de betydningsfulde faktorer for vindenergi og elektricitets priser; (b) opbygning af eksperimenter der har til form\aa l at validere analysen og etablere de bedste Artificial Neural Network modeller; (c) bestemme anvendeligheden af vores tilgang med udgangspunkt i en diskussion af de eksperimentelle resultater og potentialet for brug i praksis.

Vi har pr\ae senteret de koncepter, der omgiver Artificial Neural Networks og time series forudsigelser. Dette inkluderer Machine Learning, Feedforward Neural Network, Supervised Learning med Backpropagation og Decision Support. Vi fremh\ae ver karakteristika for vindenergi og elektricitets priser baseret p\aa ~koncepter fra eksisterende systemer.

Vi anvender Feedforward Neural Networks med Resilient Backpropagation til forudsigelse. Modellerne bliver tr\ae net med relevant data, der best\aa r af de influerende faktorer for enten vindenergi eller elektricitets prisen.

Specialet illustrerer anvendelsen af det Artificial Neural Network til forudsigelse af vindenergi og elektricitets priser gennem de eksperimentelle resultater og en diskussion deraf. 

\chapter*{Acknowledgements}
\addcontentsline{toc}{chapter}{Acknowledgments}

We are grateful for the guidance provided by our supervisor Niels Olof Bouvin throughout the entire process of writing our thesis. Time has never been an issue and the feedback has always been insightful which we very much appreciate.

We are also thankful to Jacob Styrup Bang for providing us with the computational power for achieving our experiments within a timely manner. At the same time we apologize for putting down the server.

Finally, we would like to thank our family and friends for always giving moral support and proofreading this thesis. Special thanks to the IT-guys for always joining us and listening to "boring" stuff about Artificial Neural Networks in the cafeteria at lunchtime. It will be dearly missed when entering real life.

\vspace{2ex}
\begin{flushright}
  \emph{Brian Bak Laursen,}\\
  \emph{Kristian Barrett,}\\
  \emph{Aarhus, \today.}
\end{flushright}

\tableofcontents
\listoffigures
\newpage
\pagenumbering{arabic}
\setcounter{secnumdepth}{2}

%%%%%%%%%%%%%%%%%%%%%%%%%%%%%%%%%%%%%%%%%%%%%%%%%%%%%%%%%%%%%%%%%%%%%%%

\chapter{Introduction}
\label{ch:intro}
\section{The Problem}
Renewable energy has become increasingly important. Energy suppliers make it possible for their customers to choose between green and brown energy. Companies and persons promote themselves with green profiles to attain a specific kind of image and it is the choice of the individual company or home to be green. This is only one fragment of the big picture, because the increased attention on green energy is influencing the market forces behind the scenes where acquisition of energy begins \cite{windPowerDanishLiberalized}. Various trading companies buy electricity on the deregulated energy market and since the amount of renewable energy in this market is increasing and will do for the years to come\cite{6, windPowerDanishLiberalized} the traders need to consider it carefully when trading. The most essential task and basis for any decision making in the electricity market is to forecast the electricity price \cite{pjmForecast} and wind power production \cite{dayAheadImpactOfWindPowerForecasts}. Traditional producers can use the wind power predictions to strategically place their own energy in the market \cite{21}.
\\[0.5cm]
The electricity market consists of two instruments in order to facilitate trading between producers and consumers of energy: the pool that works as an online marketplace, and a framework to make physical contracts possible between both parties \cite{21}. The producers can make their electricity available and the consumers can bid correspondingly. An auction is made every hour to decide the market clearing price and what bids are accepted for that hour. The consumers and producers need to predict the hourly clearing prices to make plans for what bidding strategies to use. If a trader has a precise day-ahead prediction of the prices it will be more manageable to maximize profit by using the best possible strategy for exactly those conditions \cite{21}. The electricity traders will attempt to buy electricity in a low-price market and sell it when the price has increased, like in other commodity markets \todo{\cite{FIND REF}}. With a precise prediction the traders know when to buy and when to sell. 

Approximately 25\% of the total power production in Western Denmark was covered by wind power in 2008 and influenced the prices on the electricity market accordingly \cite{windPowerDanishLiberalized}. The impact on price is seen in wind power variations in three areas where wind power production covers a substantial amount of the total supply in Denmark. For this reason the wind power cannot be avoided when predicting electricity prices\cite{dayAheadImpactOfWindPowerForecasts}. Since the amount of green energy in the entire Danish liberalized market is targeted to increase to 30\% by 2025, almost a doubling since 2008 \cite{windPowerDanishLiberalized}, the influence will only expand. The price impact from windmills will be even greater in strong wind periods and in areas with congestions in the electricity transmission capacity \cite{windPowerDanishLiberalized}. The increase in renewable energy makes the ability of predicting green energy production and the corresponding market price more vital. It is not trivial to predict wind production because green energy by nature is unpredictable, e.g. wind power is highly influenced by wind speed and air density. If the traders are able to predict wind power when doing actual price forecasts it will give rise to a huge advantage in the market when deciding to buy or sell \cite{dayAheadImpactOfWindPowerForecasts}. Price forecasting itself contains many other factors than wind power that need to be considered equally \cite{21}.
\\[0.5cm] Traders buy and sell in real-time, intra-day or day-ahead \todo{\cite{FIND REF}}. This puts constraints on a price prediction algorithm --- but what constraints are solely dependent on the type of trade. If the trader wishes to buy in real-time or hours-ahead the algorithm must perform and deliver a result within minutes or seconds. If trading is based on day-ahead trading then the time-interval can be increased. This might result in a compromise on the margin of error because fast results equals less time for analysis and computation. Another fact is that the closer you get to the time of the trade the more accurate the weather predictions will be, which directly impacts the price prediction algorithm using it. The ability to make both short- and long-term forecasts is very important in the deregulated competitive electricity market because it helps the trader to reduce risks in terms of under/over estimating the revenues from potential sales and this of course also makes it easier to manage risk\cite{21}.
\\[0.5cm]
Demand has a huge influence on the electricity price. No demand, no electricity price, but if these were the only two variables to consider, a linear regression model could be used for establishing a relationship and thereby generate a likely prediction. Other dynamic elements have an impact on the price and therefore the linear regression approach would result in the presence of serial correlation in the error margin \cite{21}. For instance overestimated energy prices for one year most likely lead to overestimates the next year since the estimation does not account for all the dynamic parameters. It is necessary to carefully consider all variables and characteristics of the price that we are trying to predict in the electricity market and then use a model that handles correlated errors based on this. The most noticeable characteristics are \cite{21}:
\begin{itemize}
\item High frequency;
\item High volatility;
\item Unusual prices in times of very high demand;
\item Calendar effect (weekend, holidays);
\item Multiple seasonality (daily and weekly periodicity);
\end{itemize}
The focus of the thesis will be on predicting the electricity price and wind power in the Danish electricity market. When performing electricity price forecasting the above characteristics must be taken into consideration when modeling the prediction algorithm. The characteristics of wind power is found in meteorological factors such as wind speed, air density and humidity \cite{WindPowerGenerationUsingANN}. The concepts will be elaborated throughout the thesis.
\section{Related Work}
This section will give an overview of related work within the area of price and green energy prediction. The presented concepts will be elaborated as the thesis progress.
\subsection{Price Prediction Systems}
The ability of Neural Networks to forecast is seen in \cite{stockForecasting}. The stock price movements have similarities with electricity prices in terms of its non-linearity and chaotic/dynamic nature. Investors must take into account these factors when handling time series that are non-stationary, noisy and have structural breaks. In addition macro-economical elements significantly influence stock prices, e.g the economy in general, politics, bank rates and expectations of investors could be examples of such influences.

The solution for stock price forecasting in \cite{stockForecasting} is a hybrid forecasting model called Wavelet De-Noising-based Back Propagation (WDBP) Neural Network. In brief, the network is based on a wavelet transformation function that analyses non-stationary characteristics of the time series by decomposing the original data. The function produces a signal that is further decomposed into a low-pass and high-pass filter for every neuron in the network. The low-pass filter reflects main characteristics of the signal, opposed to the high-pass filter that represents noise. The purpose of the decomposition is to separate stock price characteristics from noise so that prediction will have better chances of accuracy when all undesirable has been discarded. The signal without the noise is then propagated through the Neural Network using back propagation.
\\[0.5cm]
A day-ahead forecasting algorithm that predicts electricity prices in the market based on Neural Network (ANN) and Similar Days Method (SDM) is described in \cite{pjmForecast}. The purpose is to give close estimates for several days to come. The estimates can be used by electricity traders in their decision making but also by transmission companies for different purposes. The companies can use it for scheduling a short-term generator outage in order to predict where it is most inexpensive. It can also be used by actual producers of energy to strategically bid into the market to increase prices. The price estimate itself plays a huge role in decision making in all of these examples.

The combination of ANN and SDM is an attempt to simplify the ANN and make the prediction more accurate. The algorithm forecasts by using a ANN that modifies price curves obtained by averaging five similar price days corresponding to the forecast day, i.e. The ANN corrects the received output from the similar days approach \cite{pjmForecast}. In other words the technique takes into consideration the influence of the most similar days and their price development in relation to the day we wish to forecast.

The ANN is trained with only 45 days from the day before the forecast and 45 days before and after the forecast day in the previous year \cite{pjmForecast}.
\\[0.5cm]
Prediction algorithms are not only used by electricity traders but can also be a part of various applications. In \cite{22} they introduce an intelligent electricity broker (IEB) that is integrated into Smart Grids where it; 1) provides provision of en energy storage; 2) attempts to lower the electricity bill, and 3) optimally utilizes electricity during peak and low-peak energy production periods. The prediction algorithm is used by a decision algorithm to locate points in time where it is most feasible for the owner of the smart grid to either sell stored energy or buy energy if storage is low. It helps the system to lower the total amount of energy costs to the owner and also utilize the energy in the most intelligent way.
 

\subsection{Green Energy Production}
\label{sec:greeEnergyProductionIntroduction}
When dealing with prediction of green energy production it is necessary to look at the weather in more detail. Wind energy prediction can be divided into two areas \cite{5}: 1) time-series analysis of power data and 2) wind speed prediction and conversion to power. What needs to be utilized in wind energy prediction is wind speed and wind direction which of course has a big influence on how much energy a wind turbine generate. An approach to analysis and prediction of wind power generation is presented in \cite{WindPowerGenerationUsingANN} where an Artificial Neural Network is used to predict the production. The prediction is based on weather data such as wind speed, relative humidity but also for how many hours the windmill is generating power.

Another source of green energy is solar panels. Solar
prediction algorithms are very accurate when the weather conditions are steady but in changing weather a large error margin is seen. To predict power output from a solar farm the time-series prediction algorithms and Estimated Weighted Moving Average (EWMA) models can be used \cite{5}.

Data centers is an example of a place where green energy prediction can be useful. It is often necessary for data centers to run long-running batch jobs where performance is measured in number of completed jobs and throughput. In \cite{5} they design an adaptive job scheduler that utilizes prediction of solar and wind energy production to scale workload (see introduction). Earlier work has focused on using immediate available green energy and then cancelling and rescheduling jobs thereafter. The point in \cite{5} is instead to scale the number of jobs to the expected availability of green energy production by predicting it beforehand. This helps reducing the number of cancelled jobs as the jobs are then scheduled for whenever the energy is available. If the amount of green energy production is not sufficient for an immediate or emergency job the remainder will be covered by brown energy. The system reduces the amount of wasted green energy and increases the overall throughput of the data center.
\\[0.5cm]
In this thesis we will only be focusing on energy that originates from windmills.


\subsection{Decision Support Systems}
Decision Support Systems (DSS) can help users in critical environments where careful situation assessment and decision making is necessary. This is valid for the financial markets such as the stock and electricity market where multiple dynamic factors are present \cite{UncertainInformation}.
\\[0.5cm]
A combination of an Artificial Neural Network approach, an eXtended Classifier System (XCS) and the idea of cooperative learning is presented in \cite{groupLearningDS}. It is an attempt to incorporate the best of all three in an intelligent financial decision support system. XCS and ANN are both forecasting technologies and the system compares their results in order to get the best price estimate. The cooperative learning element shows itself in the way a decision is made. Each agent selects the best investment strategy from their own point of view based on the price estimates. The agents can simulate several strategies in each session and select the potential of highest profit. This information is shared with the rest and the final investment decision is made based on the decision of the majority of the other agents in the system. The DSS archives investment strategies that showed to be profitable in a premium strategy library in order to collect the best strategies. The agents don't necessarily have to use the premium strategies but it makes sense to first try what have succeeded in the past and if that fails move on to a random selection strategy. 

The intelligent system described above makes all decisions for the agent as opposed to the HCI approach to DSS discussed in \cite{UncertainInformation}. The focus is on uncertain information and how to present it to the user in the best possible way. They argue that uncertain information can't be denied and especially not in the domains with high risks such as the financial markets and military systems. The underlying reasoning algorithm used by the system will still make qualified estimates but the final decision is left to the user. The system will inform about uncertainties so the user can make trade-off's based on the information and then make a decision. Uncertainties can exist as different types such as bad data source reliability, conflict in data and data ambiguity but no matter the type it is presented to the user at every time. 

In \cite{UncertainInformation} is further discussed how to present information including linguistic, textual and graphical so that it is perceived better and faster by the user. Finally they present a number of guidelines that can used to develop a prototype of the interface for a DSS.

\section{The Method}
The available data are prices, wind productions and meteorological factors measured every hour from earlier years. When variables are observed sequentially over time it constitutes a time series where the values in the series evolve over time. Furthermore time series data are often strongly correlated over time which make it usable for prediction \cite[Chapter~7.1.2]{econometrics}. Based on the data and the non-linear nature of price forecasting and energy price prediction we want a system that is able to develop and learn from the past by analysing the time-series. Artificial Neural Networks can be categorized as Machine Learning \cite{18} and are networks that imitate the behaviour of the human brain \cite{1}. We choose to base our system on ANN because it gives us the power to be able to forecast energy prices based on how the prices have evolved in the past. Our decision is based on the nature of Artificial Neural Networks and how it is used in machine learning. It is often used in non-linear statistical analysis \cite{16} and has been used to predict various prices within the commodity market \cite{2,3,stockForecasting,pjmForecast}. We have chosen to use a multi-layered feed forward architecture since this is one of the most used and widely implemented in open-source frameworks \cite{17}.

We will base our work on Artificial Neural Networks and the Resilient Backpropagation (RPROP) algorithm used to train the network in order to predict electricity prices and wind power generation. The RPROP algorithm is a learning algorithm that uses weights to analyze the input dataset\cite{17} and it is a faster algorithm than its predecessor Backpropagation \cite{8,15}. It is commonly used on the feedforward architecture and is the most used and implemented algorithm on ANNs \cite{14,17}.

The prediction algorithm will be based on historical data which potentially could be a very large dataset. The more historical data included in the algorithm the more data needs to be processed. We need to analyse how much of this data is necessary for precise prediction and perhaps make tradeoffs to ensure performance for decision support.

\section{The Hypothesis}
\label{sec:theHypothesis}
Work in the field have established that an Artificial Neural Network handles time-series well that are non-linear, volatile and noisy (which applies for wind power and electricity prices) in an effective manner \cite{stockForecasting,pjmForecast,yamin2004adaptive,windForecastPortugal}. It is our intent to build on this experience and model Feedforward Neural Networks for prediction that uses the Resilient Backpropagation algorithm for training of the networks. In this dissertation we focus on utilizing the Feedforward Neural Networks for prediction of electricity prices and wind power by identifying the influential input parameters through a comprehensive analysis validated through experiments. Wind power and electricity price predictions are key instruments for decision making in the electricity market\cite{dayAheadImpactOfWindPowerForecasts,21} and the potential of Artificial Neural Networks for this purpose will be discussed.

Feedforward Neural Networks can be characterised as Machine Learning\cite{18} and historical data that is relevant for the specific task will be used for training. The goal is to investigate and identify (through analysis and experiments) the importance of the influential factors to be included and represented in these datasets. Based on this we examine the feasibility of a Resilient Backpropagation Feedforward Neural Network as a technology for predicting electricity prices and wind power. The feasibility will be examined by analysing the experimental results and the potential for practical use.
\\[0.5cm]
We will focus on; 1) identifying the influential factors for wind power and the electricity price; 2) modeling and implementing Feedforward Neural Networks with Resilient Backpropagation that are capable of predicting electricity price and wind power within the Danish electricity market; 3) evaluating predictions originating from the Artificial Neural Networks in terms of experimental results and decision making.

\newpage
\section{The Structure of the Thesis}
\label{sec:structureOfTheThesis}
The rest of the thesis is structured as follows to uncover the presented hypotheses from Section~\ref{sec:theHypothesis}. 

\emph{Chapter 2, Main Concepts:} The chapter presents the main concepts within the area of wind power and electricity prediction. It includes a description of Machine Learning and Artificial Neural Networks that are used for prediction in this thesis. Different approaches to prediction of wind power and electricity prices have been presented and finally decision support is described. 

\emph{Chapter 3, Data Analysis:} A comprehensive analysis of the influential factors for wind power and electricity prices is conducted in this chapter. It includes description of the dataset and how it is collected together with the statistical tool to identify correlations between the influential factors and the output to predict.

\emph{Chapter 4, Forecasting Model:} The model used for prediction is presented here. The structure of the network, the training algorithm and the framework is described. Secondly, we point out strategies for our experiments with the network in order to identify the best approach to prediction of wind power and electricity price. 

\emph{Chapter 5, Experimental Results:} This chapter describes, analyses and concludes on the conducted experiments. The various experiments are based on the analysis and it has the purpose of rejecting or verifying the established hypotheses by simulating actual day-ahead predictions. The result of the chapter will be the best Artificial Neural Network models for electricity price and wind power (one for each) within the scope of the thesis.

\emph{Chapter 6, Experimental Result Discussion:} The experimental results is discussed here. The focus is to discuss the findings from Chapter 5 and simultaneously relate it to practical use (decision making) and other literature within the area of Artificial Neural Networks. It will be discussed here how the findings relate to the established hypotheses in the introduction.  

\emph{Chapter 7, Conclusion:} The results and findings of the thesis is concluded in this chapter.

\emph{Chapter 8, Appendix:} The appendix contains results and graphs from the experimental results.  


%%%%%%%%%%%%%%%%%%%%%%%%%%%%%%%%%%%%%%%%%%%%%%%%%%%%%%%%%%%%%%%%%%%%%%%

\chapter{Main Concepts}
This chapter introduces and explains the technologies and concepts that are the foundation for the thesis.
\label{ch:foundations}
\section{Machine Learning}
\label{sec:machineLearning}
Machine learning algorithms can figure out how to perform tasks by generalizing, where the goal is to generalize beyond the examples of a training set \cite{18}. The programs using the algorithm learn from data and as more data becomes available, more complex problems can be handled. Systems such as web search, spam filters and stock trading uses this approach.
\\[0.5cm]
Many learning algorithms exist and they are all based on combinations of three components \cite{18}; 1) the \textbf{representation} component makes an attempt to discover how to represent the input during training of the algorithm. What kind of information expressed in the data decides what representation algorithm to use; 2) the \textbf{evaluation} component has the objective of distinguishing between the best and the worst output from the representation component and let the next component decide which choice is the best. Basically it attempts to map the output to a specific value so that the next component can analyze it; 3) finally, the \textbf{optimization} component needs to search for the highest scoring output from the evaluation and then deliver it as a result. A greedy search could be an example of such a search (see figure ~\ref{fig:threeComponents}).
\begin{figure}[h!]
\centering
\includegraphics[width=0.6\linewidth,natwidth=898,natheight=587]{billeder/Table1-TheComponentsOfLearningAlgorithms.png}
\caption{The three components of learning algorithms \cite{18}}
\label{fig:threeComponents}
\end{figure}
\\[0.5cm] 
Machine learning cannot rely on data alone. Every learner must make assumptions beyond the data that is given to generalize beyond it \cite{18} because it is not possible to map the real world uniformly to a possible mathematical function. When meteorologists try to model the weather they make assumptions based on historical datasets that are similar to the current weather conditions. If enough historical occurrences resulted in the same weather condition the meteorologists must make the assumption that it can be classified as what have been seen before. The same is applicable when predicting electricity prices. We examine historical data and locate situations that are similar to the current and if enough examples with the same result is found it can be assumed that they are of the same type.
\\[0.5cm]
It can be the case that the system is not at all able to deliver a correct answer. This could result in the system guessing to the best of its ability or simply randoming an answer. This problem is called overfitting \cite{18}. Overfitting can be divided into bias and variance, where bias is when the learner consistently learns the same wrong thing over and over again. Variance is the learner's tendency to learn random things and ignoring the possibility of a more correct answer. See figure ~\ref{fig:biasandvariance} for an example with dart throwing.
\begin{figure}[h!]
\centering
\includegraphics[width=0.5\linewidth,natwidth=898,natheight=587]{billeder/biasVSvariance.png}
\caption{Example of bias and variance in a dart throw \cite{18}}
\label{fig:biasandvariance}
\end{figure}

Machine learning is not only about the technical stuff but it also relies a great deal on intuition, creativity and  black art \cite{18}. 
\newline A way to achieve machine learning is by using a Artificial Neural Network. One way to model an ANN is to base it on the
Levenberg-Marquardt algorithm \cite{7,9,10}. It is a least squares algorithm that,
opposed to Backpropagation algorithm\cite{8}, is very fast at calculating the
results. It is a curve fitting algorithm mostly used on nonlinear problems with
a lot of unknowns[Need citation]. The algorithm can be used with Artificial Neural Networks and information approximation\cite{8} and it can be used in combination with the backpropagation algorithm to be used on a feedforward architecture of the neural network\cite{13}.

%%%%%%%%%%%%%%%%%%%%%%%%%%%%%%%%%%%%%%%%%%%%%%%%%%%%%%%%%%%%%%%%%%%%%%%
\newpage
\section{Artificial Neural Network}
\label{sec:annSection}
This section is based on information from the book AI techniques for game programming\cite{buckland2002ai} chapter 7, 8 \& 9. Also from the book Neural networks: a systematic introduction\cite{rojas1996neural} chapter 7.
\\[0.5cm]
Neural networks are models that imitate the brains behavior. They have been created as an option to model artificial intelligence and analyze machine learning. The human brain is made up of billions of neurons that are interconnected in a big grid. They communicate by firing electrical shocks through the network of neurons. The human brain is extremely complex and can calculate vast amounts of data in no time. This is why scientists and mathematicians have been trying to emulate this behavior to create artificial intelligence.

\begin{figure}[!ht]
\centering
\includegraphics[width=0.8\linewidth]{billeder/ANN.png}
\caption{A simple neural network with 3 layers. \cite{stockForecasting}}
\label{fig:ANN}
\end{figure}

Artificial neural networks (ANN) are artificial neurons (nodes) that are connected in a network. The network consists of an arbitrary number of layers that are interconnected. The most common structure in these networks are a feed forward structure. This kind of structure has the characteristic that it only flows data from the input layer through the hidden layers to the output layer. There are no loops in the network thus making it unable to reiterate any information. Normally all of the nodes in the input layer is connected to all of the nodes in the second layer. The same connections are done in the next layers until we hit the output layer. This will give us the sum of all the previous nodes($x_i$) and their weights($w_i$)($\sum_{i=0}^{i=n+1} w_i x_i$) in every node in the next layer. See ~\ref{fig:weight_of_layers}.
\begin{figure}[!ht]
\centering
\includegraphics[width=0.8\linewidth]{billeder/weight_of_layers.png}
\caption{How the weight is calculated from one layer to the next. \cite[P. 251]{buckland2002ai}}
\label{fig:weight_of_layers}
\end{figure}
All of these connections carry a weight that dictates how data flows through the network and reflects the relations between the inputs and the outputs of the network. The inputs of the network should be all the factors that has an influence on the output we want from the network. In our example we would want to input; weather data, temperature, demand, availability etc. \cite{21} to get the price as an output from the network.

Every node in our network contains an activation function. This function, when calculated, tells us whether the artificial neuron should fire or not. That is, if the neuron should transmit the data from the current layer to the next layer. There are a lot of different activation functions. The simplest form is the binary step function which either fires or it does not. This depends on the input and gives us a low threshold of complexity which is good in simple neural networks as the relations between the nodes do not need to be that fine grained. In more complex systems we want activation functions with a broader output range than binary. In many cases a sigmoid function is used as the activation function. This is because of the ''S''-shape which enables it to compute outputs in a non-linear way. The non-linear nature of the sigmoid functions is what makes the neural network able to compute non-trivial problems in reasonable sized networks. The sigmoid function allows the activation functions of the neurons to have a broader range of inputs which will produce an output compared to step activation functions. To be able to calculate a non-trivial problem in these kind of networks we need what is called a training algorithm. This algorithm depicts how the network evolves over time also known as learning. There are two kinds of learning; unsupervised learning and supervised learning.

\subsection{Unsupervised learning}
Unsupervised learning is when we do not have output data for the input data. Instead we have a problem where we want the neural network to try and estimate some behavior relative to a specific task based on some assumptions we have about the system it performs on. It is commonly used with estimation problems like "Cluster Analysis", which in short is an attempt to find matching criterion in data and group the data in clusters. This is often done by exploring the dataset and that is what unsupervised learning is good for. It also works with Artificial Intelligence(AI) that have to explore parts of the (virtual)world. In \cite{buckland2002ai} he explains how an unsupervised learning feed-forward artificial neural network trains itself using a genetic algorithm to keep track of the fitness function of the AI. The fitness function is used to tell the AI if it is doing good or bad according to some requirements established before the experiment. This fitness function is what trains the network. The AI will get a plus score if it encounters what we are looking for and get negative if it hits something that we defined as "wrong". Based on this fitness function it will update the weights of the neural network accordingly to what is most beneficial for the network as a total. After it has been allowed to do a lot of runs it begins to get a sense of what it is exploring and should be able to make better choices for each run.

\subsection{Supervised learning}
Supervised learning are a set of algorithms that use a dataset which contains both the inputs and outputs. This dataset is used to train the neural network to make it able to do calculations on current data and predict the outcome. An example of an algorithm used for supervised learning is the back-propagation algorithm. 
It starts out by randomly assigning weights to all of the connections between the neurons. It then calculates the output of the network and compares it to the expected output. From the comparison of output and expected output it calculates the error margin and adjusts the weights accordingly. This is done for all the hidden layers until we hit the input layer. All of these steps are called an epoch. We will repeat as many epochs as we need until the sum of all the errors are within a given threshold. The name of the algorithm originated from these epochs where it propagates the error backwards in the network. We use Resilient Back-propagation (RPROP) which is based on the traditional back-propagation algorithm as described in \cite{rpropForGeometricDilution,adaptiveRprop}. It is a first-order algorithm better at escaping local minima and at the same time scales linearly to the number of input parameters. In short the the difference is in how the weights are being updated through evaluating the behaviour of the error function. The traditional backpropagation adjust its values based on the actual value of the derivative and since the derivative value decreases exponentially as further it gets from the output layer it will take longer time for the last neurons to learn. In RPROP the derivative sign decide the weight adjustment and not the magnitude of the derivative as with traditional backpropagation. RPROP use the same amount of computational effort but is faster due to the derivative sign evaluation. Furthermore, it is not necessary to define the learning rate or momentum because they are automatically adapted and therefore we avoid spending time on fine-tuning these parameters.

Supervised learning can be thought of as learning with a teacher. As an example we can use the XOR table:

\begin{table}[!ht]
\centering  % used for centering table
\begin{tabular}{c c c} % centered columns (3 columns)
Input \#1 & Input \#2 & Output \\ [0.5ex] % inserts table 
%heading
\hline                  % inserts single horizontal line
0 & 0 & 0  \\ % inserting body of the table
1 & 0 & 1  \\
0 & 1 & 1  \\
1 & 1 & 0 \\ [1ex] % [1ex] adds vertical space
\hline %inserts single line
\end{tabular}
\caption{This training set is very simple yet it illustrates a training set for a supervised learning algorithm very well.} % title of Table
\label{table:xor-table} % is used to refer this table in the text
\end{table}

In this dataset we have both of the inputs and we are given the expected output. This gives the back-propagation algorithm a direction to follow when minimizing the error function of the network. We can do this because we a predicted output that we can compare to the expected output which allows us to predict the direction that the error correction should take to close in on the best possible solution. Also the sigmoid activation function helps us closing in on the target output since the sigmoid function always has a positive derivative which ensures that we will always have a direction to follow\cite[p. 153]{rojas1996neural}. This is because the derivative of the sigmoid function always will point us towards the global minimum of the approached function. If we take a look at ~\ref{table:xor-table} we are given inputs and outputs. We take the input 1 and 0 and we expect the output to be 1. The neural network will initialize weights and start to run the back-propagation. The back-propagation algorithm will compare the output it predicts to the output we expected e.g. we may get an output of 0.5 (based on the random weight initialization) and this is compared to the expected output 1. Because of the sigmoid function we know what direction the weights should be corrected to come closer to the output of 1. This allows the back-propagation algorithm to recalculate the weights and be sure that it is heading for the global minimum of the error function thus heading for the right output.
%[FORKLAR MATEMATIKKEN EVT MED EKSEMPEL s. 332 i bogen]

In neural networks bias neurons are often added to the layers to help them learn patterns. The bias neurons are added to give the activation functions the ability to change its output even if x is zero. If we look at the graph in figure ~\ref{fig:activationFunctions} it uses the following activation function: \begin{math} \frac{1}{(1+e^{(-cx)})} \end{math} where c is the weight.

\begin{figure}[!ht]
\centering
\includegraphics[width=0.8\textwidth ,natwidth=410,natheight=237]{billeder/ActivationFunctions.png}
\caption{}
\label{fig:activationFunctions}
\end{figure}

The figure shows how the gradient of the function alters with different weight values. Even though the gradient of the function is clearly altered by the weights the function is still outputting the same result for zero thus we cannot alter the output for x equal to zero just by altering the weights. This is what we use the bias for. If we apply a bias of one to all of the neurons we will be able to shift it either to the left or the right. In figure ~\ref{fig:activationFunctionsWithBias} we see the same function as before where the weight is set to 2. The difference is that we added a bias (b) to this function: \begin{math} \frac{1}{(1+e^{(-2x+b)})} \end{math} \cite[p. 165]{rojas1996neural,inductiveBias}

\begin{figure}[!ht]
\centering
\includegraphics[width=0.8\textwidth ,natwidth=410,natheight=237]{billeder/ActivationFunctionsWithBias.png}
\caption{}
\label{fig:activationFunctionsWithBias}
\end{figure}

\subsection{Common pitfalls}
When we are trying to fit our algorithm and make it recognize patterns we will encounter several possible pitfalls. First of all there is the chance of ending up in a local minima. This is when the the back-propagation algorithm attempts to find the global minimum of the error curve thus having reduced the error as much as possible. The algorithm works by trying to reduce this error margin a little step at a time. If it encounters a local minimum on the curve and thinks it has reached the global minimum it gets stuck and we will get an inaccurate result \ref{fig:localMinimum}. 

\begin{figure}
\centering
\begin{minipage}{.5\textwidth}
  \centering
  \includegraphics[width=.4\linewidth]{billeder/globalMinimum.png}
  \captionof{figure}{At the global minimum \cite[P. 318]{buckland2002ai}}
  \label{fig:globalMinimum}
\end{minipage}%
\begin{minipage}{.5\textwidth}
  \centering
  \includegraphics[width=.4\linewidth]{billeder/localMinimum.png}
  \captionof{figure}{Stuck in a local minimum \cite[P. 318]{buckland2002ai}}
  \label{fig:localMinimum}
\end{minipage}
\end{figure}

To avoid the backpropagation algorithm to falsely accept a local minimum as the global minimum we can give the algorithm momentum. This is done by adding a bit of the last error correction from the earlier layer to the next layers error correction. This way the algorithm, so to say, will scoot right by any small deviations in the error correction face.
\newline
Another pitfall when working with neural networks is over fitting the algorithm. This is when the algorithm instead of finding a generalized pattern in the inputs it will find an over-fit pattern that will fit exactly the input resulting in problems when presenting the network for unseen data. This is better shown in Figure~\ref{fig:overfitting}.
\begin{figure}[!ht]
\centering
\includegraphics[width=0.8\linewidth,natwidth=1262,natheight=415]{billeder/overfitting.png}
\caption{A. The plot graph of the input B. The generalized function C. An over-fit function \cite[P. 319]{buckland2002ai}}
\label{fig:overfitting}
\end{figure}
This can be avoided by some simple techniques. First of all we want to reduce the neurons as much as possible as long as it does not interfere with our performance of the system (also known as pruning). This is a trial and error problem and has to be tweaked along with evolving the neural network. Another option is to add noise to avoid this problem. By adding noise(random data values) we prevent the algorithm from fitting the function to closely to the given data. Thus giving us a more generalized function where it hopefully will be able to give a better result on unseen data. Early stopping is another method to avoid over-fitting. This is only doable with large datasets where you can split it into two equal datasets. The first will work as a training set and the second will work as a validation set. We will keep training the dataset and checking with the validation set until the difference between those two start to increase.

\subsection{Summary}
This section touched the basics of neural networks and where the inspiration for these networks came from. Feed-forward was presented the most common architecture used for Artificial Neural Networks and  it will also be applied in this thesis. This architecture allows data to only flow from input to output with no loops in it. We elaborated what activation functions are and how they are used in neural networks. Since neural networks are all about the computer learning to calculate some very complex problem we need a learning algorithm. We explained that unsupervised learning algorithms are about exploring possibilities in a space and try to come close to a goal that we set up. It has no definitive goal (opposed to supervised learning) but a success factor that it tries to maximize. We also talked about supervised learning and how these algorithms try to achieve the best possible result. This is done by trial-and-error where the algorithm learns until it satisfies an error margin we chose or for a specific number of epochs. Furthermore we talked about why and what bias is used for. It is a constant that is added to the calculations to prevent the algorithm to stall while trying to find the lowest error margin in our prediction. We also explained common pitfalls and how to avoid these. Among these were over-fitting of the function we are trying to a achieve which means it only fits the training set and not a general set of data. Also we talked about the algorithm falsely believing that the local minimum was the global minimum and therefore coming up with a wrong result.
We also talked about the training set and how this data is handled by the training algorithm. We mainly focus on supervised training since this is what we are gonna use in this thesis. A very important point to make here is that the training set is of utmost importance when it comes to performance and accuracy of our ANN. The training set has to be developed and refined during a training period of a network and relies a lot on experience and tinkering with the inputs. We need to normalize the input set so that it fits each other and all the parameters get the same weighting from the beginning. This is done so that some inputs unintentionally gets a higher significance. Also we can only have assumptions of on which form the inputs will give the best result and we might have to try different forms like numbers, bits, matrices etc. to get the best results.

%%%%%%%%%%%%%%%%%%%%%%%%%%%%%%%%%%%%%%%%%%%%%%%%%%%%%%%%%%%%%%%%%%%%%%%
\newpage
\section{Prediction}
\label{sec:predictionSection}
Being able to predict electricity prices and wind power is the basis for our thesis. Demand has an impact on the electricity price so this is handled here. This section will introduce and give selected examples of these concepts within the different areas. 

\subsection{Electricity Demand}
\label{sec:ElectricityDemand}
Weather conditions have a huge impact on the electricity industry in terms of network infrastructure and electricity consumption. In \cite{19} they describe a multiple regression model that accurately predicts the monthly electricity demand based on weather and sociocultural conditions. The monthly electricity demand from this model shows a clear cyclic pattern which reflects the temperature changes during the year \cite{19}. Besides weather conditions, social and economic factors also affect the monthly demand, e.g. the demand was decreased in Denmark during the financial crisis in 2009 \cite{20}. 

\subsubsection{Parametric multiple regression}
Parametric multiple regression is preferred in \cite{19} as opposed to an Artificial Neural Network used in this thesis. From a statistical error indices point of view the Artificial Neural Network (ANN) is a valid choice for prediction (see figure ~\ref{fig:anncomparison}) in relation to the error. The argument for multiple regression is the simplicity in adjusting input values for each analysis of electricity demand in the prediction model and in the end they show similar results to ANN. 
\begin{figure}[h!]
\centering
\includegraphics[width=0.8\linewidth,natwidth=898,natheight=587]{billeder/StatisticalErrorOfNeuralNetworksAndRegression.png}
\caption{Shows ANN compared to B\&J and SE \cite{19} }
\label{fig:anncomparison}
\end{figure}
To make a demand prediction based on weather conditions they have established a relationship between the two. To get an understanding of this relationship they have made a plot of the average monthly demand as a function of Central England Temperature (CET) \cite{19}. The plot (see figure ~\ref{fig:CET}) shows an inverse relationship where it can be seen that a lower temperature in general results in an increased load consumption. Winter gives rise to lighting and heating load which is consistent with lower temperatures, and conversely in the summer with temperatures above 18 degrees the consumption tends to increase again due to the need for cooling and air-condition \cite{19}.
\begin{figure}[h!]
\centering
\includegraphics[width=0.8\linewidth,natwidth=898,natheight=587]{billeder/MeanMonthlyDemandEngland.png}
\caption{Function of monthly CET as a function from 1970 to 1995 \cite{19}}
\label{fig:CET}
\end{figure}

\subsubsection{Heating and Cooling Degree Days}
The nonlinear relationship between temperature and load consumption is turned into a concept called degree days. They introduce two categories of days; 1) Heating Degree Days (HDD) that is used to quantify when heating is required; 2) Cooling Degree Days (CDD) which is then used to quantify the need for cooling. The days are calculated from the CET data and give a more indicative picture than the temperature-load relationship \cite{19}. A simple explanation of the calculation follows.
\begin{center}
$CDD=\sum\limits_{d=1}^{N_{d}}(\gamma_{d})(T_{dm}-T_{base_{C}})$
\end{center} 
 
where $T_{base_{c}} = 20^{\circ}$ is the base temp and $T_{dm}$ is the mean daily temperature. $\gamma_{d} = 0$ if $T_{dm}-T_{base_{c}} < 0$ and $\gamma_{d} = 1$ if $T_{dm}-T_{base_{c}} > 0$. In other words if the temperature is above $20^{\circ}$ the day can be characterized as CDD and there is a need for cooling.
\begin{center}
$HDD=\sum\limits_{d=1}^{N_{d}}(1-\gamma_{d})(T_{base_{H}}-T_{dm})$
\end{center} 

where $T_{base_{h}} = 15.5^{\circ}$ is the base temp and again $T_{dm}$ is the mean daily temperature. $\gamma_{d} = 1$ if $T_{base_{h}}-T_{dm} < 0$ and $\gamma_{d} = 0$ if $T_{base_{h}}-T_{dm} > 0$. This means that on a day with temperatures below 15.5 degrees there is a need for heating. In both cases where $\gamma_{d} != 0$ a big difference has a greater impact on the demand.

Temperature is the most influential factor but other weather conditions can also be used when calculating the electricity demand, e.g. wind speed and rainfall can impact heating and lighting demand whereas direct sunshine can decrease the need for heating \cite{19}. The model can be expressed by the relation between all factors by adding them together, e.g. the HDD value automatically increase the total electricity demand if it is not zero and this apply to all other factors in the model seen below.

\begin{center} \^E$_{A}=\alpha_{0}+\alpha_{1}CDD+\alpha_{2}HDD+\alpha_{3}ELD
+\alpha_{4}V_{w}+\alpha_{5}M_{s}+\alpha_{6}M_{r}$ 
\end{center} 
 
Where $\alpha_{n}$ are constants, ELD is humidity, $V_{w}$ is wind speed, $M_{s}$ sunshine and $M_{r}$ rainfall.
The calculation model can also include socio-economic factors. Specifically the population growth has a natural impact on the electricity demand over time. The more people the higher the demand gets. This would give rise to an additional factor in the model considering the growth. The prediction model from \cite{19} comes within 3 percentage of the actual demand when it is run on data from 1996-2003 which is pretty close (~\ref{fig:predicteddemand}).
\newline
\begin{figure}[h!]
\centering
\includegraphics[width=0.8\linewidth,natwidth=898,natheight=587]{billeder/PredictionOfDemand.png}
\caption{The actual values and the predicted demand in a comparison \cite{19}}
\label{fig:predicteddemand}
\end{figure}

\subsubsection{Summary}
The electricity demand has great influence on the energy price. It is important to identify the influences of the demand because those inevitably also will have an affect on the electricity prices. 

The concept of Cooling Degree Days (CDD) and Heating Degree Days (HDD) has been described. They might give us an quantifiable parameter to measure the need for cooling and heating in Denmark. What it brings forward is the necessity for considering demand when predicting price and a potential of substituting demand with the parameters given --- wind speed, temperature, ect. The demand function (\^E$_{A}$) is presented. This function gives a picture of the factors that directly influence the demand and the possible mappings to our ANN as the demand part --- every parameter is an input.

\subsection{Wind Power Production}
The identification of rich wind resources have become important together with the increasing focus on green energy \cite{WindPowerGenerationUsingANN}. It can come as no surprise that the meteorological factors like wind speed and air density have an impact on the wind power. Figure~\ref{fig:energyGeneration} shows how the monthly wind power increases with the monthly average wind speed for their location --- we expect to see a similar relationship. 

\begin{figure}[h!]
\centering
\includegraphics[width=0.8\linewidth,natwidth=898,natheight=587]{billeder/EnergyGenerationVsWindSpeed.png}
\caption{The influence of wind speed on the wind power \cite{WindPowerGenerationUsingANN}}
\label{fig:energyGeneration}
\end{figure} 

\subsubsection{Windmill Placement}
\label{sec:windmillPlacement}
It is important to analyse and predict the wind power at a certain location before placing actual windmills. Before making a substantial investment in a new wind farm it is of utmost importance to put it in the best possible location in relation to wind statistics, e.g. wind speed and how often the winds change in power and direction at the position. In \cite{4} they use a Measure-Correlate-Predict (MCP) method to predict wind statistics based on large amount of geographical-, weather- and historical data so that the farms can be placed best possible. These algorithms are heavy and will do calculations on sensor data directly from the location for months before giving any results. Time is not an issue because the company need to know for certain that the location will make the wind mills produce to the best or their ability.

Another approach is seen in \cite{WindPowerGenerationUsingANN} where wind speed, relative humidity and generation hours of the windmills are used as input for an Artificial Neural Network. Wind energy is proportional to air density where the more "heavy" air contributes to the windmill turbine receiving more energy, thus the wind power will vary together with the specific density and wind speed by the formula:

\begin{center}
$Power from Wind=\frac{1}{(2)}*\rho*A*V^3$
\end{center}

\noindent where $\rho = $ air density, A = area of the wind turbine and V = wind speed.
\newline
\noindent The wind speed is strengthened by the air density. Moist air is lighter than dry air because water molecules are less dense than the molecules in dry air such as oxygen and nitrogen. This basically means that the more air molecules like oxygen and nitrogen the more wind energy \cite{AirDensityInForecast}.
The air density depends on pressure and temperature but is also influenced by relative humidity. The parameters are necessary to investigate when attempting to forecast the wind power.

The last parameter in their prediction algorithm is generation hours that is the period in which the turbines produce power. The number of hours are influenced by for instance mechanical breakdowns, scheduled maintenance and low wind speeds. It is clear that the more generation hours the more energy is produced as seen in Figure~\ref{fig:energyGenerationFromHours}. The generation hours are hard to predict but can be calculated from past years uptime together with the expectations of the company delivering the windmills.  

\begin{figure}[h!]
\centering
\includegraphics[width=0.8\linewidth,natwidth=898,natheight=587]{billeder/GenerationHourVSGeneration.png}
\caption{The influence of generation hours on energy production \cite{WindPowerGenerationUsingANN}}
\label{fig:energyGenerationFromHours}
\end{figure} 

The Artificial Neural Network trains on a 3-year dataset containing the mentioned input parameters. The input parameters are during the training compared to the output variable which is the wind power output of the turbine. See figure~\ref{fig:annArchitecture} for the architecture.
\\[0.5cm]
\begin{figure}[h!]
\centering
\includegraphics[width=0.7\linewidth,natwidth=898,natheight=587]{billeder/ANNwindSpeedPrediction.png}
\caption{Artificial Neural Network architecture from \cite{WindPowerGenerationUsingANN}}
\label{fig:annArchitecture}
\end{figure}

\subsubsection{Nearest Neighbour Approach}
Prediction can be done by using an algorithm that produces weighted nearest-neighbour tables to generate wind power curves based on available wind speed and direction from an online weather-data source. The weighted approach allows the algorithm to adapt to seasonal changes by weighting newest results highest, and the power curves makes it possible to use the algorithm on different wind farms. This prediction is used to schedule jobs in data centres which is described in Section~\ref{sec:greeEnergyProductionIntroduction}.

The time perspective and data of the algorithm is comparable to our specific purpose. The ANN dataset will consist of enough historical data fetched from online sources to make prediction possible.

\subsubsection{Summary}
It is described how the weather influences the wind power and what parameters are necessary to consider when doing the prediction. An Artificial Neural Network for prediction of wind power is presented and the concept of generation hours is discussed. The generation hours could help the network to forecast the wind power but since we are attempting to forecast the wind power for all of Western Denmark it seems unrealistic due to the many windmills in Denmark 

Furthermore, examples of how these predictions are used today is described. They are mainly used for placement planning of wind mill farms. Some knowledge can be transferred to our specific purpose but decision making in real-time cannot rely on heavy algorithms that will do calculations for days on data directly from the location. We must deliver accurate estimates without delaying the traders work.

\subsection{Electricity Prices}
\label{sec:electriciyPrices}
A day-ahead forecasting algorithm that predicts electricity prices in the market based on Neural Network (ANN) and Similar Days Method (SDM) is described in \cite{pjmForecast}. The purpose is to give close estimates for several days to come. The estimates can be used by electricity traders in their decision making but also by transmission companies for different purposes. The companies can use it for scheduling a short-term generator outage in order to predict where it is most inexpensive. It can also be used by actual producers of energy to strategically bid into the market to increase prices. The price estimate itself plays a huge role in decision making in all of these examples.

The combination of ANN and SDM is an attempt to simplify the ANN and make the prediction more accurate. The algorithm forecasts by using a ANN that modifies price curves obtained by averaging five similar price days corresponding to the forecast day, i.e. The ANN corrects the received output from the similar days approach \cite{pjmForecast}. In other words the technique takes into consideration the influence of the most similar days and their price development in relation to the day we wish to forecast.

The ANN is trained with only 45 days from the day before the forecast and 45 days before and after the forecast day in the previous year \cite{pjmForecast}. Results can be seen in table~\ref{table:sdmresult}

\begin{table}[h!]
\centering  % used for centering table
\begin{tabular}{c c c} % centered columns (3 columns)
Year 2006 \#1 & ANN (Avg. MAPE [\%]) \#2 & ANN ( FMSE [\$/MWh] ) \\ [0.5ex] % inserts table 
%heading
\hline                  % inserts single horizontal line
January 20 & 6.93 & 4.57  \\ % inserting body of the table
February 10 & 7.96 & 6.12  \\
March 05 & 7.88 & 5.39  \\
April 07 & 9.02 & 5.87 \\ [1ex] % [1ex] adds vertical space
\hline %inserts single line
\end{tabular}
\caption{Results of forecasting from \cite{pjmForecast}.} % title of Table
\label{table:sdmresult} % is used to refer this table in the text
\end{table}


An explanation for the choice of days is not discussed in the paper. As  mentioned in \cite{18} the more data the more complex problems can be handled

Artificial Neural Networks (ANN) has also been used for electricity price forecasting. In \cite{singhal2011electricity} they use an ANN to predict the half-hourly price of electricity of 24 hours. They differentiate between three different kinds of days: Normal trend price, Price with small spikes and price with large spike. They present a prediction for each of these days and present us to the mean absolute error and the root mean square error, which are standard measures for how accurately the prediction is done. The neural network is fed with 13 inputs as follows \cite{singhal2011electricity}:
\begin{itemize}[noitemsep,topsep=3pt,parsep=2pt,partopsep=3pt]
\item Day of week
\item Time slot of Day
\item Forecasted Demand
\item Change in demand
\item Price (one day ago) - 3 inputs 
\item Price (one week ago) - 3 inputs
\item Price (two weeks ago) - 1 input 
\item Price (three weeks ago) - 1 input 
\item Price (four weeks ago) - 1 input
\end{itemize}
These inputs are fed into a 4-layer neural network: one input layer, two hidden layers and an output layer. The analysis of their ANNs shows that neural networks make a very precise prediction on normal trend price days but have difficulties forecasting the price with small and large spikes. They argue that if the reasons to the spikes in the price were taken into account as inputs in the network maybe the network would be better at forecasting the spikes prices. Also they argue that fuzzy logic, neural networks and dynamic clustering together will provide more efficient forecasting of the prices.
\\[0.5cm]
Since price forecasts in electricity markets are such a volatile operation because of the shifting tides, price demand, holidays etc. that affects the price it can be a cumbersome problem to model. In \cite{amjady2006day} he proposes a new method based on neural networks and fuzzy logics to predict the electricity prices. He calls the new network approach a new fuzzy neural network. Fuzzy logic is basically a logic that has many values or many correct answers. Opposed to binary logic sets, where the answer can be only true or false, in fuzzy logic we have several grades of what we define as true, thus making it harder to decide what is really the truth but also makes us available to have a way larger scale of the data we are looking at.

The fuzzy logic is used within the nodes in the hidden layer to do evaluation of the data inputs. That is the activation function of the neurons in the hidden layer contains a fuzzification function that creates a square of the inputs compared to sinus activation functions that normally takes the sums of the inputs. This square over the inputs is used to classify the inputs into hyper cubes (input spaces) and then calculating how close they are together. Calculation of how far the inputs are from each other are used to calculate the output of the functions. By this we will get an upper and lower limit that the inputs can range between and thus we get a input that has the characteristics which turns out to give a better result than ARIMA, wavelet-ARIMA, MultiLayered ANNs and Radial Basis based ANNs. The data however is not optimized in this paper and they claim this to be future work. They also state that the better performance is based on a limited dataset.

\subsubsection{ARIMA Prediction}
 The Auto Regressive Integrated Moving Average (ARIMA) model is a time series model where the ARIMA analyse time series with a class of stochastic processes \cite{EnergyPriceForecasting,ARIMA}. The model has been applied to forecast of commodity prices such as oil, gas and electricity \cite{ARIMA}. 

The success of the presented ARIMA model is dependent on the linear relationship of the underlying data generating process, whereas the Artificial Neural Network can handle non-linear relationships \cite{1}. The Artificial Neural Networks are simple but a very powerful tool when it comes to forecasting, provided that the training set contains enough data and that enough computational resources are available. 
In \cite{1} the Artificial Neural Network outperforms the ARIMA model in terms of both time consumption and accuracy of the predicted price as shown in ~\ref{fig:ArimaVSNN}. where the error percentage to the actual price is shown.
\begin{figure}[h!]
\centering
\includegraphics[width=0.8\linewidth,natwidth=898,natheight=587]{billeder/ARIMAvsNN.png}
\caption{Comparison between Neural Network and ARIMA in terms of error \cite{1}}
\label{fig:ArimaVSNN}
\end{figure}
\subsubsection{Support Vector Machine Prediction}
\label{sec:svmPrediction}
Support vector machines (SVM) can be used for forecasting in the commodity market. In \cite{xie2006new} they use SVMs for predicting the crude oil prices. SVMs are by nature a linear learning machine which means SVMs always use linear functions to solve the regression analysis. However they can be expanded to be able to solve nonlinear problems. This is done by mapping the data into a high-dimensional feature space using a nonlinear mapping. Afterwards it is possible to use linear regression on this space to solve nonlinear problems. The SVMs undergo four different phases before being able to make predictions. These can be seen in figure ~\ref{fig:phasesOfSVM}.
\begin{figure}[weight!]
\centering
\includegraphics[width=0.8\textwidth ,natwidth=410,natheight=237]{billeder/phases_of_SVM.png}
\caption{The steps taken to create a Support Vector Machine}
\label{fig:phasesOfSVM}
\end{figure}
Data sampling is done daily but due to inconsistencies in these data they adopt weekly and monthly data as alternatives. Data preprocessing is done by transforming the data into more appropriate data for learning purposes. This can be done by using logarithmic transformation or other data transformations. The training and learning step is used for determining the architecture and the parameters of the SVM. There is no criterion for deciding these other than just trial-and-error and the developers experience in the field. Out-of-sample forecasting is done on new data and the prediction is made. As for evaluation they use the Root Mean Square Error (RMSE) to describe the estimated deviation from the real values. Their results and analysis shows that their SVM performs better than both ARIMA and Back Propagation Neural Networks. They argue that SVMs gives better predictions than ARIMA and BPNNs in most cases but neural networks might perform better with data that is optimized for neural networks. Also the neural network outperforms the SVM in one of their sub-period comparisons.

SVMs have also been used for load forecasting of electricity demand. In \cite{chen2004load} they use SVMs to make short-term predictions such as one-day ahead predictions. The goal of this study was to win a competition held by the EUNITE (EUropean Network on Intelligent TEchnologies for Smart Adaptive Systems). This was the winning proposal in the competition. They first examined the data which was half-hourly load demands which had been recorded from 1997 to 1998. In the pursuit of winning the competition they analyzed the data and figured out that in the wintertime load demand was higher than in the summertime thus indicating a connection to weather data and a separation of data was possible. The weekends could also be separated from the weekdays since the weekend load was lower than regular weekdays. When analysis was done they started setting up the model. First of all they prepared the data and selected the data needed for the prediction. They selected calendar attributes to map the holidays and which day it was to account for lower demands. Temperature was included in the vectors used for prediction of the electricity load demand and historical data was incorporated as well. The data segmentation in the steps of preparing a SVM allowed them to take only a subset of the data since most of it could be generalized in the data analysis thus making it easier to do computations. They argue that their model for forecasting demand loads were the best possible model used in this competition and to give a more varied view on this they try out other methods; among these Neural Networks. They configured a neural network that first performed work on the same data as their Support Vector Machine and it provided less than satisfactory feedback with an error margin of 6-8\%. If they took the basic values used in the competition without doing any precomputation on it they received a much better result of 3.64\%. From this they conclude that their SVM performed better than Neural Networks in the specific problem and that neural networks performance depends heavily on the data used as input. 
\subsubsection{Summary}
Artificial Neural Network examples have been presented in the related work section. The characteristics of the electricity prices have been refreshed and different predictions methods have been presented. 
In the description of ANN prediction it is worth observing the input parameters of the network and the way it is modelled. The demand input could potentially consist of input parameters that represents the influential factors for the demand such as the meteorological and socio-economic factors.

ANN can be used in combination with other technologies in an attempt to improve performance and accuracy. This should be considered.
Furthermore, two other examples of predicting and how they relate to ANN is described. This further motivates our choice of ANN as prediction method.   

\subsection{Summary}
Seem shitty with two summaries just after each other! What to do?

This section introduced the concepts and approaches of electricity demand, green energy production and electricity prices. What is somewhat noticeable is the similarity between the influential factors --- especially when it comes to price and demand. They are all dependent on weather data. This indicate that we can approach the wind power production and electricity price prediction with the same Artificial Neural Network simply by replacing the training set and the number of inputs.

\newpage

\section{Decision support}
This section introduces the concept of a Decision Support System and how it relates to Artificial Neural Network. Secondly, it presents how uncertain information (that is highly represented in the electricity market) can be handled through transparency and analysis.  
\label{sec:dssAndUncertain}
\subsection{Decision Support Systems}
\label{sec:dssSection}
Since the late 1970s decision support has been a developing area both in the scientific research but also as a tool in the private sector. Decision support is important in that it helps the users to gain an advantage in complex domains and it helps them to take the better decision when it comes down to a crucial choice. Decision support has been applied in many forms over the years and as computers evolve and they become more sophisticated the same applies for a Decision Support System (DSS).

In \cite{shim2002past} they argue that DSSs are an inevitable evolution in decision making companies across the globe. They describe how DSSs has evolved from being a conceptual framework to sophisticated software providing information for information heavy companies. In the beginning DSSs only existed on standalone computers as a ressource to take into consideration when discussing something with your colleagues. It evolved with the introduction of data warehouses and the world wide web into a more analytic tool (called on-line analytic processing) that gathered information from data warehouses and compiled this information into behavioural information about customers or the like. This was decision support based on datamining and analysis of the data. Also the introduction of the internet reduced the costs of bringing DSSs to smaller cooperations and firms and made it easier for people to use this software in their everyday work scenario. This also introduced collaborative support systems that are systems focused on helping groups of people rather than just a single individual. These types of decision support systems often spans more than just small groups of users and incorporates entire organisations into one decision support system. While these systems are interesting we will not be covering them here becaues they are out of the scope of this thesis. Instead we will be looking at optimization-based support models. These DSSs can be divided into three stages: formulation, solution and analysis\cite{shim2002past}:
\begin{quotation}
\textit{Formulation refers to the generation of a model in the form acceptable to a model solver. The solution stage refers to the algorithmic solution of the model. The analysis stage refers to the 'what-if' analyses and interpretation of a model solution or a set of solutions.}
\end{quotation}
The formulation support is about narrowing down the problem and normalizing the data so that it fits into a datamodel or an algorithm. The solution stage involves faster and better algorithms that will solve complex problems at a higher rate. At the same time people use more sophisticated methods to solve combinatorial problems. This is done by using genetic algorithms, neural networks and the likes\cite{shim2002past}. The last part is analysis where the DSS focuses on delivering the information from stage two and handing it over to the client in a useable form instead of just the analysis which the client then have to make sense off. This can be done by presenting the user for spreadsheets, graphs and report generation.

In the future \cite{shim2002past} argue that:
\begin{quotation}
\textit{(i) it should look for areas where the proven skills of DSS builders can be applied in new, emergent or overlooked areas; (ii) it should make an explicit effort to apply analytic models and methods; it should embody a far more prescriptive view of how decisions can be made more effectively; (iii) it should exploit the emerging software tools and
experience base of AI to build semi-expert systems, and (iv) it should re-emphasise the special value of DSS practitioners as being their combination of expertise in understanding decision making and knowing how to take advantage of developments in computer-related fields.}
\end{quotation}
By this they are saying that they predict a future with more sophisticated decision support systems that in a broader manner will use artificial intelligence to provide the support. They also argue that with the ability to distribute products over the internet and easily reach a broader audience more specialised DSS will occur. They foresee that a lot of these services will be on a pay per use basis where you log into a system and get the information you are after and pay as you use it.

\subsection{Presentation of Uncertain Information}
\label{sec:uncertainInformation}
It is necessary to get a complete understanding of uncertainties in the data you wish to present and an adequate way of visualizing it to the user. This becomes even more important when the information is to be used in dynamic environments where high-risk decisions have to be made \cite{UncertainInformation}. Furthermore, new technologies have made it possible to analyze and compare data from multiple sources which can create an information overload that makes decision-making even more difficult. The Decision Support System (DSS) as described above can combine and present all of this information in order to support users in their attempt to make the best decision. However, autonomous systems does not always guarantee the best result or performance. It can actually contribute to creation or exposing of uncertainties when dealing with complex environments such as financial markets \cite{UncertainInformation} if not handled carefully. It necessary to present the uncertain information in the most appropriate way.
\\[0.5cm]
Most people are met with uncertainties on a daily basis. We are fully aware that decisions have to be made even though information is not exact or complete. Often this results in decisions based on guesses and assumptions \cite{UncertainInformation}. When dealing with high-risk decisions it is necessary to be aware of the uncertainties so that it can be accounted for when taking the actual decision. 
There are different types of uncertainties and it can be characterized as both subjective and objective (see figure~\ref{fig:typesOfUncertainty}). 
\begin{figure}[h!]
\centering
\includegraphics[width=0.7\linewidth,natwidth=898,natheight=587]{billeder/TypesOfUncertainInformation.png}
\caption{Types of uncertainty from \cite{UncertainInformation}}
\label{fig:typesOfUncertainty}
\end{figure}  
The objective uncertainties is the acquisition and processing of the data, and the output. Is the data acquired from a trusted source, is it comprehensive enough and does processing impact the data precision. We are combining and comparing information from different sources during the data processing which can bring uncertainties such as conflicting data or just bad estimates. This information needs to be brought to the users attention so they can act on it. The subjective uncertainties arise in the mind of the user when he is to make a decision. People perceive, interpret and process information differently depending on their background. It is not enough to simply present the information. It is essential to visualize and present the information specifically to the users of the system. Decisions also depend on the specific task, the context, needed accuracy, time constraint, level of risk and experience. 
\\[0.5cm]
Five strategies for user decision making in an uncertain environment is presented in \cite{UncertainInformation}. 
\begin{itemize}
\item Reduction: collecting further information to reduce uncertainty.
\item Assumption-based reasoning: Fill in gaps by relying on experience and imagination, or make sense of factual information.
\item Weighting pros and cons of alternatives
\item Forestalling: Prepare what to do in case of a potential negative result
\item Suppressing uncertainty by simply ignoring it. Last resort.
\end{itemize}  
This emphasizes the need for making the users aware of potential uncertainties in the information e.g. situation awareness. What strategy to use relies greatly on how much uncertainty. The uncertainties should always be made aware to the users so that it is not necessary for them to check the validity themselves. It is furthermore important that the user understands the basic logic of the algorithm so that the answer is not just "black magic". They need to trust it.
\\[0.5cm]
The way of presenting information greatly impacts the decision making process. People can process more data and do it more quickly when it is presented graphically rather than in text \cite{UncertainInformation}. This does not imply that the information can be overloaded with graphical information; it is still important to only show what is most critical for the user to make the best decision. An example of a graphical representation that is perceived quickly is \textit{blurred or degraded graphical images} where the images in a natural way visualizes the amount of uncertainty by blurring it accordingly (see figure ~\ref{fig:blurryIcons}). 
\begin{figure}[h!]
\centering
\includegraphics[width=0.7\linewidth,natwidth=898,natheight=587]{billeder/blurryIcons.png}
\caption{Blurry icons from \cite{UncertainInformation}}
\label{fig:blurryIcons}
\end{figure}  
\\[0.5cm]
Based on the discussions in \cite{UncertainInformation} they present a number of high level guidelines for presenting uncertainty. To summarise the most prominent:
\begin{itemize}
\item Always define and present uncertainty and accuracy of data.
\item Present uncertainty at all time and not only when the decision has to be made.
\item Identify the user so that the system can be aligned with their way of reasoning and thinking. This include basic understanding of the underlying algorithm.
\item Provide conflict in data if it is present.
\item Graphical distortion can be improve performance when showing uncertainties.
\item Graphical presentation is not as precise a numerical but they can support each other in achieving precision and performance.
\item Users must be able to restructure the information environment themselves so that it fits their need and way of thinking.
\end{itemize}
The guidelines can be used when developing a prototype for a Decision Support System (DSS) interface.
\subsection{Summary}
This dissertation think of the feasibility of Artificial Neural Networks as dependent on both obtaining accurate results but also in its potential for practical use. Decision Support Systems (DSS) help users in complex domains that fits well with the electricity market. They mention the formulation, solution and analysis stages that incorporates intelligent systems that do big calculations and present them to the user/agent in a ready to use fashion. According to this we are conducting the first two steps of the implementation of the DSS. The third step deals with how to show the calculations to a user or agent through a Human Computer Interface. In the domain of the electricity market this would include showing uncertain information to the user to always make room for qualified guesses if the underlying data is not trustworthy. It is presented in \cite{UncertainInformation} that ignoring uncertain information in complex markets can result in creating uncertainties and thereby taking bad decisions. \cite{UncertainInformation} furthermore presents the types of uncertainty and how we are dealing with the objective ones in this thesis (see Figure~\ref{fig:typesOfUncertainty}. What is very important for us (as well as others relying on our prediction) is the reliability of the inputs in terms of data source but also how the different data is aggregated, presented and analysed.

%%%%%%%%%%%%%%%%%%%%%%%%%%%%%%%%%%%%%%%%%%%%%%%%%%%%%%%%%%%%%%%%%%%%%%%

\chapter{Dataset Analysis}
\label{ch:theANNs}
The purpose of this chapter will be to analyse what influences the price and wind production. The influences are based on concepts and knowledge from Section \ref{sec:predictionSection} and need further validation by analysing them in regard to the Danish electricity market. Furthermore the objective uncertainties presented in Section \ref{sec:uncertainInformation} will be elaborated on when analysing what influences the electricity price/wind power. The analysis will contribute to locating the potential input parameters to be used in the Artificial Neural Network.
\section{Data Collection}
\label{sec:dataCollection}
Historical weather data has been collected for various Danish weather stations from NOAA\footnote{\url{http://www.ncdc.noaa.gov/}}. The data is in a fixed length format and contains all the necessary weather data for prediction. Format can be seen in BILAG 11111. 
The historical hourly price and wind production data has been downloaded from nordpoolspot\footnote{\url{www.nordpoolspot.com}} in excel format BILAG2.
The data will be aggregated into one file for each of the predictions containing the needed input and output parameters.

For the sake of simplicity we will be focusing on West Denmark and Funen which is also known as DK1 in the electricity market [find ref]. The collected data will be validated through plot diagrams that show and establish the relationships between weather conditions and the power generation/price of DK1.

The weather data is collected from all available weather stations in DK1 from NOAA~\ref{fig:stations4average} - some stations have been omitted due to missing data. The collected data will be averaged and used as basis for creating training sets to be used as input parameters for the Artificial Neural Networks. It is necessary to average the weather data to get a representative picture for all regions included in DK1 because only one price and production exist for all of DK1.

\begin{figure}[H]
\centering
\includegraphics[width=0.85\linewidth,natwidth=898,natheight=587]{billeder/stations4average.png}
\caption{Selected weather stations in DK1}
\label{fig:stations4average}
\end{figure}


We will remove data that is obscure and is related to conditions we cannot predict. (See trimming~\ref{sec:Trimming})

we need to discuss the fact that all production data stems from nordpoolspot. The impact is obscure values a times of very low consumption. The production would not make sense in decision making if not predicted based on market conditions. The consumption will decide production together with wind speed.
\newpage
\section{Wind Power Analysis}
\label{sec:windPowerAnalysis}
The purpose is to predict the total hourly wind production available on the energy market from all of western Denmark (DK1). It is important to note that it is the wind production available on the market and not the entire production of Western Denmark. This can also be seen by the correlation between demand and wind production in Table~\ref{table:pearsonCoeficientWindProduction} which indicates that the production is moved into the market when the demand for electricity is high. Because the wind production follows demand and thereby the market there is a potential in considering  economical and statistical factors of the current market situation when predicting the wind power. It is our intention to get a close enough estimate so that it can be used as an indicator for the hourly amount of wind production in the market for the next 24 hours.
Most others have been forecasting the wind production for specific wind farms where exact weather conditions and wind mill throughput of the site is known beforehand. The assumption is that we loose accuracy without the actual throughputs of the windmills because it is directly proportional to the meteorological factors as presented in~\ref{sec:windmillPlacement}. 

The Pearson Correlation Coefficient\footnote{\url{http://en.wikipedia.org/wiki/Pearson_product-moment_correlation_coefficient}} has been used to establish the linear dependency between meteorological factors, demand and wind production. Table ~\ref{table:pearsonCoeficientWindProduction} shows this correlation coefficients. The influential factors will be elaborated in this section. Demand and wind speed has the immediate best correlation to wind power. The analysis will look further into all of the parameters.

\begin{table}[H]
\centering  % used for centering table
\begin{tabular}{|c|c|} % centered columns (3 columns)
\hline
Input factor & Pearson Correlation to Wind Power \\ 
%heading
\hline                  % inserts single horizontal line
Demand & 0.61 \\ \hline % inserting body of the table
Wind Speed & 0.94 \\ \hline
Temperature & -0.09 \\ \hline
Wind Direction & 0.21 \\
\hline %inserts single line
\end{tabular}
\caption{Table showing Pearson correlation coefficient between various factors and the wind production.} % title of Table
\label{table:pearsonCoeficientWindProduction} % is used to refer this table in the text
\end{table}

\subsection{Meteorological factors}
It is not surprising that weather conditions directly impact the wind power generation. The typical input parameters for wind power prediction are wind speed, air density, temperature and pressure \cite{WindPowerGenerationUsingANN} with the most influential factor being wind speed because it is directly converted to power in the wind turbine. The following will describe the parameters relationship to wind production and how it is used in the modelled ANN.

\subsubsection{Wind Speed}
\label{sec:windPowerWindSpeed}
Wind speed is directly proportional to wind production as described in Section~\ref{sec:windmillPlacement}. To illustrate this relationship 300 hours that are considered representative is showed in Figure~\ref{fig:windVsProd}.  The graph shows the expected relationship in the way wind power follows the trend of the wind speed (this must be differentiated from the predictions graphs so have extra attention on the two axes with different values here). Finally a plot diagram all of 2012 with the correlation coefficient is shown in Figure~\ref{fig:windSpeedWindProductionPlot} which makes no room for doubting the influence of wind speed on wind power. A final remark is how wind speed covers a wide range of wind power productions over the year --- as an example consider wind speed 15 and how it covers wind power in the interval of approximately 700 to 1900. The ANN generalization will move the function towards the majority of the interval but when facing the less represented values they will be hard to predict. It can become an issue that must be looked into during experiments.  

\begin{figure}[H]
\centering
\includegraphics[width=0.99\linewidth]{billeder/windSpeedWindProductionPlot.png}
\caption{Wind speed and wind production in diagram}
\label{fig:windSpeedWindProductionPlot}
\end{figure}

\begin{figure}[H]
\centering
\includegraphics[width=0.95\linewidth]{billeder/WindSpeedVsProduction.png}
\caption{Wind speed vs. wind production from 2012}
\label{fig:windVsProd}
\end{figure}

\subsubsection{Air Density}
\label{sec:airDensity}
It is described in Section~\ref{sec:windmillPlacement} that wind energy is proportional to air density where a higher density means more power for a specific wind speed. This is expressed in the formula $Power from Wind=\frac{1}{(2)}*\rho*A*V^3$ where $\rho$ = air density, A = area of wind turbine and V = wind speed. The wind speed is obviously more influential and we must find out the influence of air density on all different wind speeds in order to identify its significance, e.g. consider the hypothetical examples of wind power calculations below:
\\[0.5cm]
\noindent Example 1:

\begin{center}
$wind power = \frac{1}{(2)}*0.9*1*10^3 = 450$
\end{center}

\noindent where where V = 10, A = 1 and $\rho$ =  0,9.
\\[0.5cm]
\noindent Example 2: 

\begin{center}
$wind power = \frac{1}{(2)}*0.1*1*22^3 = 532$
\end{center}

\noindent where V = 22, A = 1 and $\rho$ =  0,1.
\\[0.5cm]
\noindent It illustrates why air density cannot be compared directly to wind power. Even though air density is much higher in the first example it still has a lower wind power than the second example due to the wind speed.

Air density depends directly on temperature and pressure which can be described by $Air Density=\frac{P*M}{(R*T)}$ where R is a gas constant, M is the mass of dry air, P is pressure and T is temperature. The monthly pressure in Denmark has low variation compared to the temperature as shown in Figure~\ref{fig:pressureTemperatureVariance}. For this reason, the temperature will have the most influence on the air density in Denmark and it might even be enough to leave out pressure and only include temperature. The formula express that when temperature decreases the air density will increase, e.g. the wind power production for a specific wind speed will be higher in times of low temperature. The air density has been calculated for every hour in the training set and the correlation has been established for each wind speed in Table~\ref{table:pearsonCoeficientAirDensity}. The wind power production increases with the air density for each wind speed as described by the formula but how much varies from wind to wind speed --- wind speed 24 and the various air densities have no influence on wind power but in average the influence of air density is seen in a correlation of 0,28.

\begin{figure}[H]
\centering
\includegraphics[width=0.95\linewidth]{billeder/pressureTemperatureVariance.png}
\caption{Temperature and Pressure variance for 2012}
\label{fig:pressureTemperatureVariance}
\end{figure}

\footnotesize
\begin{table}[H]
\centering  % used for centering table
\begin{tabular}{|c|c|} % centered columns (3 columns)
\hline
Wind Speed (mph) & Co-relation between air density and wind power \\ % inserts table 
%heading
\hline                  % inserts single horizontal line
2 & 0,29\\ \hline
3 & 0,38 \\ \hline
4 & 0,28 \\ \hline
5 & 0,18 \\ \hline
6 & 0,27 \\ \hline
7 & 0,26 \\ \hline
8 & 0,29 \\ \hline
9 & 0,31 \\ \hline
10 & 0,28  \\ \hline
11 & 0,18 \\ \hline
12 & 0,19 \\ \hline
13 & 0,19 \\ \hline
14 & 0,19 \\ \hline
15 & 0,12 \\ \hline
16 & 0,10 \\ \hline
17 & 0,22 \\ \hline
18 & 0,11 \\ \hline
19 & 0,28 \\ \hline
20 & 0,11 \\ \hline
21 & 0,18 \\ \hline
22 & 0,18 \\ \hline
23 & 0,26 \\ \hline
24 & 0,01 \\ \hline
25 & 0,12 \\ \hline
26 & 0,20 \\ \hline
27 & 0,72 \\ \hline
28 & 0,87 \\ \hline
29 & 0,61 \\ \hline
30 & 0,76 \\ \hline  
Average: & 0,28 \\ % [1ex] adds vertical space      
\hline %inserts single line
\end{tabular}
\caption{Table showing Pearson correlation coefficient between the various wind speeds in the dataset and the air density.} % title of Table
\label{table:pearsonCoeficientAirDensity} % is used to refer this table in the text
\end{table}
\normalsize

\subsubsection{Wind Direction}
Figure~\ref{fig:windDirVsProd} shows that the wind direction in general has a slight impact on the wind production (even though it is hard to see from the plot) which can also be seen in the correlation constant 0,21 from Table~\ref{table:pearsonCoeficientWindDirection}. The wind direction is expected to have a small influence based on Section~\ref{sec:windmillPlacement} where it is described how change in wind direction affect placement of windmills. The wind speed could be more powerful when coming from a specific direction which differs according to the physical location. It needs to be validated in experiments if the wind direction makes sense for western Denmark.
 
\begin{table}[H]
\centering  % used for centering table
\begin{tabular}{|c|c|} % centered columns (3 columns)
\hline
Input factor & Pearson Correlation Coefficient \\ % inserts table 
%heading
\hline                  % inserts single horizontal line
Wind Production & 0.21 \\ \hline % inserting body of the table
Wind Speed & 0.20 \\ \hline % [1ex] adds vertical space
\hline %inserts single line
\end{tabular}
\caption{Table showing Pearson correlation coefficient between various factors and the wind direction.} % title of Table
\label{table:pearsonCoeficientWindDirection} % is used to refer this table in the text
\end{table}

\begin{figure}[H]
\centering
\includegraphics[width=0.95\linewidth]{billeder/productionVsWindDirection.png}
\caption{Wind Direction vs. wind production in 2012}
\label{fig:windDirVsProd}
\end{figure}

\subsection{Demand}
\label{sec:demandWindProduction}
According to the correlation constant of 0.61 from Table~\ref{table:pearsonCoeficientWindProduction} it is expected that the available wind power on the electricity market will be low at times with low demand, e.g if no energy is needed the wind power can't be sold.  The relationship is shown in Figure~\ref{fig:demandVsWindProduction} where wind power illustrates its relationship with demand in the plot diagram. It tells us that the wind production is connected to the market and cannot be deduced only from meteorological factors.

\begin{figure}[H]
\centering
\includegraphics[width=0.95\linewidth]{billeder/consumptionVsWindProduction.png}
\caption{Demand vs. Wind Production in 2012}
\label{fig:demandVsWindProduction}
\end{figure}

\noindent Demand would be a predicted value in our ANN if used in a real setting. In our experiments we use actual values from nord pool spot to simulate the demand input --- it is out of scope for this thesis to predict it. A final remark is the potential of substituting demand with temperature because it is highly influenced by Heating Degree and Cooling Degree Days as presented in Section~\ref{sec:ElectricityDemand}. The relationship between temperature and demand is expressed in Figure~\ref{fig:consump_temp_green} and we will investigate if this substitution is possible. This must be mentioned in relation to the absence of predicted demand because using the actual value makes the substitution less likely --- in a real life setting the demand would be predicted based on temperature (and other meteorological factors) which could potentially be directly substituted in the network instead of the prediction. What speaks for temperature as a substitution is that it in one parameter can reflect both demand and air density and thereby make the network more simple. We will compare the two approaches in experiments to come. 

\begin{figure}[H]
\centering
\includegraphics[width=0.95\linewidth]{billeder/energy_price_plots/consump_temp.png}
\caption{Demand and temperature plot.}
\label{fig:consump_temp_green}
\end{figure}

\subsubsection{Time of day}
\label{sec:greenTOD}
The dataset used for prediction consists of hourly observations for all calendar days of the year. It is worth identifying if there is a pattern within these days. The time of day an its associated average wind power is illustrated in the pillar diagram of Figure~\ref{fig:hourly_wind_production}. During 2012 the hours between 8-20 have a higher wind power production in general. It indicates what have been stated before with the correlation to demand, namely the amount of wind power is adjusted to the actual market demand. Time of day will be included in the experiments to verify the connection that has been established here. 

\begin{figure}[H]
\centering
\includegraphics[width=0.8\linewidth]{billeder/hourly_wind_production.png}
\caption{Time of day vs. Wind Production in 2012}
\label{fig:hourly_wind_production}
\end{figure}

\subsection{Seasonality}
\label{sec:windProdSeasonality}
The wind speed follows seasonal changes. It is highly probable that the wind power on the market will follow this change due to its obvious relation to wind speed. Figure~\ref{fig:windProductionMonths} visualizes this relationship for the different months of 2012. The wind power is lower from April to August compared to all other months of the year. A clear distinction between the actual seasons can be seen in the pillar diagram of Figure~\ref{fig:windProductionSeasons} where the former is confirmed with summer and spring being lowest in wind power. Seasonality must be considered as input and tested in experiments.

\begin{figure}[H]
\centering
\includegraphics[width=0.85\linewidth]{billeder/Seasons/windProductionMonths.png}
\caption{Wind production in relation to months in 2012}
\label{fig:windProductionMonths}
\end{figure}

\begin{figure}[H]
\centering
\includegraphics[width=0.85\linewidth]{billeder/Seasons/windProdctionSeasons.png}
\caption{Wind production in relation to the four seasons in 2012}
\label{fig:windProductionSeasons}
\end{figure}

\subsection{Wind Production Development}
\label{sec:windProductionDev}
The above analysis shows the wind production development to be much dependent on wind speed and air density together with the general trend on the market in relation to demand. 
\\[0.5cm]
Wind power is highly volatile. This statement is clear when observing the development curve for the first 400 of a year in Figure~\ref{fig:windHourDevelopment400Hours} --- these hours are much representative for the overall development of wind power. 

\begin{figure}[H]
\centering
\includegraphics[width=0.99\linewidth]{billeder/productionTendency400Hours.png}
\caption{Wind production development for 400 hours in 2011}
\label{fig:windHourDevelopment400Hours}
\end{figure}

\noindent We hope to predict 24-hours ahead and in order to do so we must be able to identify the trends of the dataset, e.g. are we moving up or down based analysing the previous hours. The issue is discussed in detail in Section~\ref{sec:usingStatisticalInput} but in short more characteristics about trends in the dataset will allow us to approach our target more accurately. The ANN function can run into problem when trying to generalize on meteorological factors alone due to the many wind power productions associated with the same wind speed. Hopefully consumption and air density can help wind speed within this interval but attempting to add even more information of wind power is worth a try. To exemplify consider a situation where similar inputs (in this case temperature, demand and wind speed) will result in many different wind power productions during the year. The similar days are defined by having similar wind speeds, temperatures and demands during the day --- the margins for what defines a similar days can be seen in Table~\ref{table:similarHoursLimitsWindProd} just below.

\begin{table}[H]
\centering  % used for centering table
\begin{tabular}{|c|c|c|c|} % centered columns (3 columns)
\hline
 & \#1 Windspeed & \#2 Temperature (Celsius) & \#3 Demand \\ \hline % inserts table 
High margin: & 16.5 & 3.3 & 2359.5  \\ \hline
Low margin: & 13.5 & 2.7 & 1930.5 \\ \hline % [1ex] adds vertical space
\hline %inserts single line
\end{tabular}
\caption{This is the high and low margins used for the the similar input to output distribution.} % title of Table
\label{table:similarHoursLimitsWindProd} % is used to refer this table in the text
\end{table}

\noindent The similar days result in an amount of different wind power productions which also emphasizes its volatility. The different productions can be seen in Figure~\ref{fig:inputParameterDistribution} and range from 848-1398. The generalization will move towards 848-1048 within this interval but if we are facing a similar day during prediction but on a rising trend with a value of 1198 coming just before then we would still guess between 848-1048 instead of the higher value that is much more likely. The point of the simple example is to illustrate the importance of considering the the immediate previous hours because of their significant for the movements to come. Identifying these previous movements will give us the possibility of a more accurate prediction. The simplest way to include information about the past is to include the wind power from past hours and let the network itself calculate the impact of former wind power productions --- Figure~\ref{fig:windProductionVsLastWindProduction} shows the relationship between the current production and the last known production with a Pearson constant being 0,99   

\begin{figure}[H]
\centering
\includegraphics[width=0.85\linewidth]{billeder/Equal_wind.png}
\caption{Same input parameter to wind production output distribution}
\label{fig:inputParameterDistribution}
\end{figure}

\begin{figure}[H]
\centering
\includegraphics[width=0.95\linewidth]{billeder/windProductionVsLastWindProduction.png}
\caption{Wind power vs. last known wind power}
\label{fig:windProductionVsLastWindProduction}
\end{figure}

\noindent The relationship between the individual calculated approaches and wind power will not be analysed further here due to the many adjustable settings in the calculations (different hours and smoothing factor as described in Section~\ref{sec:usingStatisticalInput}) but the analysis show the potential for capturing trendlike behaviour throughout the dataset which will be tested in the experiments.

\subsubsection{Trimming} 
Trimming is for removing irregularities that we are not able to predict. Intuitively this makes no sense for wind power since it is constrained by natural forces and the windmills ability to produce power --- at some point nothing more can be produced. What could give rise to irregularities is the wind power and its connection to the electricity market where a sudden rise in demand could result in the need for pushing a huge amount of power into the market. Figure~\ref{fig:windProductionTrimming} attempts to illustrate the different hours that would be cut off when using percentile trim. The most noticeable thing is that a larger trim will cut the tops of the graphs which could potentially result in the creation of irregularities. 

\begin{figure}[H]
\centering
\includegraphics[width=0.99\linewidth]{billeder/windProductionTrimming.jpg}
\caption{Wind power production trimming}
\label{fig:windProductionTrimming}
\end{figure}

The graph containing the trims for the entire year makes it hard to see what is actually cut-off but is still gives the sense of potential values being removed from the dataset. The graph in Figure~\ref{fig:pointingOutPlaceWhereTrim} highlights the ours between 6200-7200 to give a better sense of what is removed. It is observable that there are not sudden drops but all wind power productions have intermediary steps --- we want to be able to predict all of these values and since they are not irregular it makes no sense to remove them. The only irregularity we can identify happens in around 1100 in Figure~\ref{fig:windProductionTrimming} where the wind power suddenly drops to zero, e.g. we will require the values to be above zero. It will be discussed later if it in the hands of a trader could make sense to trim data based on expertise, e.g. if the trader has knowledge about trend movements and experience tells that within the next 24 hours only certain values are plausible to occur then trimming could be a possibility. It will of course only make sense if trimming away values do not create more noise and achieves more accuracy but this will be tested in experiments.

\begin{figure}[H]
\centering
\includegraphics[width=0.99\linewidth]{billeder/pointingOutPlaceWhereTrim.png}
\caption{Wind power production vs. last known wind power production}
\label{fig:pointingOutPlaceWhereTrim}
\end{figure}

\subsection{Conclusion}
The above analysis has clarified to what extend the different parameters affect the wind power. Not surprisingly wind speed has the greatest influence on wind power, but also demand has shown a good correlation with the production which suggest a close link to the electricity market. The wind power vary from hour to hour (highest between 8-20) which makes the time of day a significant factor. The wind power is decreased significantly during winter so the season will also need to be included in test. The analysis of the yearly wind power development showed the need for including past wind power production to better identify the movements to come.
The air density is according to its formula proportional to wind power and it showed a correlation of 0,28 in average which makes it applicable for testing. The current wind direction and wind power is related with a correlation of 0,21 but further validation needs to be carried out in experiments.
\newpage
\section{Electricity Price Analysis}
\label{sec:ElectricityPriceAnalysis}
Energy price prediction is a cumbersome task to handle because of the highly volatile nature of such predictions \cite{pjmForecast, yamin2004adaptive} and the plethora of factors that influence the energy price\cite{singhal2011electricity}. In this section we will show the factors that influence the price by analyzing relevant data that will later become input in our ANN. Not too surprisingly; the most important factor in any market is \fnurl{demand}{http://en.wikipedia.org/wiki/Supply_and_demand} and this greatly influences the price. Time of day, day of the week, wind speed and temperature (that are the quantifiable factors) also play a big role. Sociocultural influences affects the price as well but are hard to measure \cite{singhal2011electricity}. This section will show the connection between electricity price and the influential factors.

\noindent The price throughout this thesis is represented in DKK the danish currency.

\subsection{Price}\label{sec:Price}
We present some of the influences on the price in this section and argue why these parameters have an impact on the price. We also account for the time perspective and show the non-linear nature of energy prices. To accompany the graphs we also calculated the Pearson's correlations (Section ~\ref{sec:Pearsons}) that can be seen in Table~\ref{table:pearsonsPriceVariables}.

Pearson's correlations:
\begin{table}[H]
\centering  % used for centering table
\begin{tabular}{|c|c|c|} % centered columns
 \hline
 Parameter 1 & Parameter 2 & Pearsons \\ [0.5ex] % inserts table 
%heading
\hline                  % inserts single horizontal line
Price & Demand & 0.31 \\ \hline
Price & Wind speed & -0.28  \\ \hline
Price & Temperature & -0.18 \\ \hline
Price & Last known price & 0.89 \\ \hline
Demand & Wind speed & 0.57 \\ \hline
Demand & Temperature & -0.59 \\ \hline
\end{tabular}
\caption{Pearson's correlations for price prediction variables} % title of Table
\label{table:pearsonsPriceVariables} % is used to refer this table in the text
\end{table}

The last known price has a great correlation to the price we are going to predict that can be seen in Figure~\ref{fig:price_price} and by the Pearsons correlation of 0.89. Even though the price is volatile and have spikes it follows a pattern; that clearly shows us that the historical price development is very important when predicting the next prices.

\begin{figure}[H]
\centering
\includegraphics[width=0.99\textwidth]{billeder/priceVsLastKnownPrice.png}
\caption{Price and last known price}
\label{fig:price_price}
\end{figure}

The correlation between the last known price and the price to be predicted shows how important the movement of the curve is when predicting the price can also be seen in Figure \ref{fig:priceGraphFirst400Hours}. Therefore further analysis about curve movement might help the ANN to better predictions if we can foresee the most likely way for the graph to move e.g. if the price is close to the max price it might soon start to drop again 

\begin{figure}[H]
\centering
\includegraphics[width=0.99\textwidth]{billeder/priceGraph400.png}
\caption{The price movement in the first 400 hours of 2012.}
\label{fig:priceGraphFirst400Hours}
\end{figure}

%------------------------------------------------------------------------------------------------------------------------------
\subsubsection{Weather conditional influences}
\label{sec:priceWeatherInfluence}
The electricity price is affected by different weather conditions. We present temperature and wind speed and show how the correlation between them and the price is.

\begin{figure}[H]
\centering
\includegraphics[width=0.99\textwidth]{billeder/energy_price_plots/price_temp.png}
\caption{Price and temperature plot.}
\label{fig:price_temp}
\end{figure}

In Figure~\ref{fig:price_temp} and Table~\ref{table:pearsonsPriceVariables} we see that there is nearly no correlation (coefficient is -0,18) between temperature and the price. This indicates that we probably cannot use the temperature as a direct influence on the price. The temperature variable says something about heating degree days and cooling degree days mentioned in \cite{19}. These days indicate when the need for electrical heating or cooling is present thus influencing the electricity price. This influence is discussed later in this analysis.

\begin{figure}[H]
\centering
\includegraphics[width=0.99\textwidth]{billeder/energy_price_plots/price_wind.png}
\caption{Price and wind speed plot.}
\label{fig:price_wind}
\end{figure}

In Figure ~\ref{fig:price_wind} we see how the wind impacts the price. If the wind speed increases the energy price decreases. This agrees with paper \cite{dayAheadImpactOfWindPowerForecasts} where they show that the wind influences the wind power production. The share of green energy produced from wind mills are approximately 25\% as off 2008\cite{windPowerDanishLiberalized} so we expect to see that the wind will have an impact on the price. When the production is high and the demand is moderate --- the price will drop because of overproduction and because we cannot store the energy for later use.

%------------------------------------------------------------------------------------------------------------------------------
\subsection{Seasonality}\label{sec:seasonality}
The price is influenced greatly by the seasonality. Seasonality covers the time of year but also the time-of-day and the day-of-the-week in this section. Seasonality plays a significant role in price prediction because the demand is affected by the behaviour of people.

\begin{figure}[H]
\centering
\includegraphics[width=0.99\textwidth ]{billeder/energy_price_plots/Average_price_over_weekdays.png}
\caption{Daily price dispersion}
\label{fig:price_over_weekdays}
\end{figure}

Figure ~\ref{fig:price_over_weekdays} shows us how the trend of the price varies over the different days in a week. The most noticeable here is how the price is decreasing in the weekend (Saturday and Sunday) and are somewhat steady for the weekdays. We are looking at two options for our neural network. The graph shows us a difference between highest (Wednesday) and lowest (Sunday) of 80 which is pretty much when the price fluctuates between 632 and 61 (with 1\% top and bottom trim see section ~\ref{sec:Trimming}).

\begin{figure}[H]
\centering
\includegraphics[width=0.99\textwidth ]{billeder/energy_price_plots/price_per_hour.png}
\caption{Hourly price dispersion}
\label{fig:price_per_hour}
\end{figure}

In Figure~\ref{fig:price_per_hour} we see the average price per hour over a whole day. We see a trend where the price is highest from 08 to 10 and again from 17 to 19. This is because the electricity follows the same pattern as people e.g. 17 to 19 is when most people prepares dinners and thus the need for electricity price rises. The price fluctuates between 335 and 190 that gives us a difference of 145. This is a pretty significant difference between highest and lowest and will help us point the prediction in the right direction.

\begin{figure}[H]
\centering
\includegraphics[width=0.99\textwidth ]{billeder/energy_price_plots/seasons.png}
\caption{Seasonal price dispersion}
\label{fig:seasons}
\end{figure}

The sesonality is also reflected in what time of year it is. In the winter time the electrical heating and need for electric light plays a significant role on how much electricity is consumed and thus the price goes up \cite{crowley2005weather}. This is shown in figure ~\ref{fig:seasons} where it clearly shows the average price is higher in winter and fall than in summer and spring.

\begin{figure}[H]
\centering
\includegraphics[width=0.99\textwidth ]{billeder/energy_price_plots/averageMonthlyPrice.png}
\caption{Monthly price dispersion}
\label{fig:monthlyAveragePrice}
\end{figure}

To get a more fine grained representation of the seasonal influences on the price; we have created a monthly distribution of the price. The Figure~\ref{fig:monthlyAveragePrice} shows that the highest average price is 360 (February) and lowest 190 (July). This is a significant price jump and shows the same trend as for the seasonal Figure~\ref{fig:seasons} but here we see the effects of the holidays in July; where the price is significantly lower than the rest of the year.

%------------------------------------------------------------------------------------------------------------------------------
\subsection{Volatility and High-Frequency}
\label{sec:volatility}
The electricity prices are very volatile and changes on an hourly basis. The electricity price is one of the most volatile entities in the commodity markets \cite{pjmForecast} if not the most volatile as they state in \cite{yamin2004adaptive}. The spikes are often reflected by the same input variables resulting in completely different results. The price has a very high frequency of spike prices. A spike price is a price that elevates very quickly and drops in a matter of a few hours. These conditions can be tricky to handle and we will now address some of the problems in handling volatility and spike prices.

In \cite{singhal2011electricity} they describe the factors that influence the price and causes them to be highly volatile:
\begin{itemize}
	\item Volatility in fuel price
 	\item Load uncertainty
 	\item Fluctuations in hydroelectricity production
 	\item Generation uncertainty (outages)
 	\item Transmission congestion
 	\item Behavior of market participant (based on anticipated price)
 	\item Market manipulation (market power, counterparty risk)
\end{itemize}
All of these factors influence the price and makes it highly volatile. These factors are very hard to account for and are not included in our analysis but has to be kept in mind when trying to predict the prices. 

Some of these factors will automatically be accounted for when we do the data analysis. They describe in \cite{yamin2004adaptive} that the behavior of the market participants (or the bidding strategies as they call it) are already a part of the training dataset thus already accounted for when we do the analysis of the data.

Spikes are hard to predict and often impossible because of the limited occurrence of these (under 2\% in our dataset) and because they are so much higher than the rest of the max prices. Spikes can add an error factor to the dataset thus making it harder to predict the rest of the datasets values because the ANN will make adjustments for these high values. Trimming can be used to reduce the number of irregular spikes in the datasets and have been used in \cite{singhal2011electricity} where they remove the electricity prices above \$70 and also in \cite{yamin2004adaptive} where they trim the dataset if the price is above \$50. The trimming of course only makes sense to use if it actually improves the predictions. It must be included in experiments to see if it has any improvement.

\begin{figure}[H]
\centering
\includegraphics[width=0.99\textwidth ]{billeder/energy_price_plots/same_hour_distribution.png}
\caption{Price ditribution over equal hours}
\label{fig:same_hour_distribution}
\end{figure}

In earlier figures we have shown how the inputs influence the electricity price and given some reasons to why this influences the price. In figure~\ref{fig:same_hour_distribution} we have chosen similar hours across all days of a year and we show how different the price can be for similar inputs. The similar days have been chosen from the 'core' inputs (wind speed, temperature and demand) and the margins seen in table~\ref{table:similarHoursLimits} reflects the upper and lower bound of the inputs. The margins are calculated from a randomly chosen day and the interval is 10\% up and down. Even though the price is volatile we can see a trend in the graph that shows us that the price most of the time will be about 265.

\begin{table}[H]
\centering  % used for centering table
\begin{tabular}{|c|c|c|c|} % centered columns (3 columns)
	\hline
 & Windspeed & Temperature & Demand \\ [0.5ex] % inserts table 
%heading
\hline                  % inserts single horizontal line
High margin: & 12.1 & 6.6 & 2510  \\ \hline
Low margin: & 9.9 & 5.4 & 2053 \\  \hline
\end{tabular}
\caption{This is the high and low margins for our similar hours comparison.} % title of Table
\label{table:similarHoursLimits} % is used to refer this table in the text
\end{table}

\begin{figure}[H]
\centering
\includegraphics[width=\textwidth ]{billeder/energy_price_plots/plotGraph.jpg}
\caption{A plot graph of the price development over the first 1500 hours in 2012}
\label{fig:plotGraph}
\end{figure}

The volatility of the price is clearly shown in Figure~\ref{fig:same_hour_distribution} where the price on similar hours fluctuates from 215 to 345. To address the volatile nature of the price we need to look at the trends of the known historical price. If we take a look at the price plot in Figure~\ref{fig:plotGraph} it clearly shows how volatile the price actually is. Another thing we can derive from the graph is how the price moves. It does not just jump from top to bottom of the interval but it takes (in most cases) intermediary steps to get there. This tendency can be used alongside the last known price to give the neural network a direction to follow and an approximate price as a point of origin. For more details on this see section~\ref{sec:usingStatisticalInput}.

In the same manner we can address the high-frequency spike prices that occurs in the time series. We have to identify a tendency in the historical data and use this to make the predictions follow the steep spikes in the dataset.

%------------------------------------------------------------------------------------------------------------------------------
\subsection{Demand}
Demand is directly connected to the energy prices (which comes as no suprise) since every market is driven by demand \cite{singhal2011electricity}. Since demand needs to be predicted we chose to include from \fnurl{Nord Pool Spot}{http://www.nordpoolspot.com/} since it is out of scope for this thesis to predict it. We can though attempt to compute the demand in our Artificial Neural Network by using the influential factors of demand as inputs, i.e. substitute the demand input with the factors that plays a role in predicting the demand. The last two methods requires us to have an idea of what the input parameters for the prediction of demand is. As mentioned in related work we have seen a function that calculates the demand based on CDD(Cooling degree days), HDD(Heating degree days), ELD(Humidity), V$_w$ (Wind speed), M$_s$ (Sunshine), M$_r$ (Rainfall) \cite{19}.

\begin{figure}[H]
\centering
\includegraphics[width=0.99\textwidth ]{billeder/energy_price_plots/price_consump_graph.png}
\caption{Demand and price graph.}
\label{fig:consump_price_graph}
\end{figure}

The price and consumption if mapped out in a graph shows that they follow the same trends see Figure ~\ref{fig:consump_price_graph}. The demand and price does not follow the exact same pattern that also states that there are more factors that influence the price.

\begin{figure}[H]
\centering
\includegraphics[width=0.99\textwidth ]{billeder/energy_price_plots/consump_price.png}
\caption{Demand and price plot.}
\label{fig:consump_price}
\end{figure}

In Figure ~\ref{fig:consump_price} we see the connection between energy price and the demand. The model shows us that if the demand rises the electricity price rises as well. This is a common tendency in a market where there is not endless supply.

\begin{figure}[H]
\centering
\includegraphics[width=0.99\textwidth ]{billeder/energy_price_plots/consump_temp.png}
\caption{Demand and temperature plot.}
\label{fig:consump_temp}
\end{figure}

In Figure ~\ref{fig:consump_temp} we see the connection between demand and temperature. The temperature itself has an influence when it is very cold or very warm. This connection is expressed as CDD(Cooling degree days) and HDD(Heating degree days). In Figure ~\ref{fig:consump_temp} we see that if temperature decreases the demand will go up. This is because the people of Denmark use a lot of energy on heating their homes and lighting them (HDD) in the winter time. In \cite{19} they also describe CDD which indicates the need for cooling. In Denmark it is limited how high the temperature goes and how often we actually need to cool our homes see table~\ref{table:CDD_HDD}. The table shows that we only have two days that are characterized as cooling degree days. On the other hand we have a lot of heating degree days where we need heating. As they are the majority of the days we will use temperature as a measure for this. CDD and HDD are normally calculated as an index for how much of a need there is for cooling or heating respectively. Here we just present what days are categorized as either and not the index.

\begin{table}[H]
\centering
\begin{tabular}{|c|c|c|} 
	\hline
HDDs seen & CDDs seen & Others \\ [0.5ex]
\hline
315 & 2 & 48 \\  \hline
\end{tabular}
\caption{Number of HDD, CDD and other days seen throughout 2012.} % title of Table
\label{table:CDD_HDD} % is used to refer this table in the text
\end{table}

\begin{figure}[H]
\centering
\includegraphics[width=0.99\textwidth ]{billeder/energy_price_plots/consump_wind.png}
\caption{Demand and wind speed plot.}
\label{fig:consump_wind}
\end{figure}

Wind speed plays a role in predicting electrical demand. The wind affects the electrical heating needed in homes across the country. The wind cools down the houses and thus need for heating arises \cite{19}. As seen in Figure ~\ref{fig:consump_wind} and in Table ~\ref{table:pearsonsPriceVariables} there is a pretty good correlation between wind speed and demand. The graph clearly shows us that; when the wind speed increases (from 15 and up) the overall demand increases aswell. This indicates that high wind speeds will have the greatest influence on the demand.


%------------------------------------------------------------------------------------------------------------------------------
\subsection{Conclusion}
The above analysis of the data inputs have shown what parameters are more likely to influence the price than others. The last-known price and the price to predict showed a 0.89 pearsons correlation and a graph that showed a close to linear correlation between the two. The weather conditional factors did not show the most convincing correlations with the price. The wind speed showed the best correlation and are the most likely to show and impact. The seasonality of the data was pictured in graphs and the connection between time of day, day of week, month of year and season of year was pretty obvious. We therefore expect them to have a great impact on the price. The volatility and high frequency section showed us that there might be a need for trimming and a need for graph analysis to make the predictions more accurate. The demand section showed how demand and price followed the same pattern and that they are connected. The Pearsons correlation between the two are not the best but we still expect it to be very important for the accuracy of the predictions.
\section{Discussion}
\todo{discuss the important factors that came out of this ---> what should be taken into consideration in the tests.}

Also talk about differences and similarities in the two data sets. Same data. Seasonality and hours impact both. ect.

%%%%%%%%%%%%%%%%%%%%%%%%%%%%%%%

\chapter{Forecasting Model}
\label{ch:forecastingModel}
The wind power production and electricity price time-series have similarities in relation to volatility, seasonality and are both related to the electricity market seen in the co-relation to consumption. We will use the same network structure and learning algorithm in both cases. This Chapter will describe the used Artificial Neural Network, the learning algorithm and the different strategies that will be experimented with in our model.
\section{Overview}
The wind power production and electricity price time-series have similarities in relation to volatility, seasonality and are both related to the electricity market seen in the co-relation to consumption. Therefore the same techniques have been attempted on both.

\subsection{LITTERATURE?}
USE MAIN CONCEPTS TO DISCUSS RESULTS!
\todo{related to neural networks and what it is also used for and why this is a good approach here}
\todo{relate it to machine learning and why this fits our problem}
\todo{what makes this ai or machine learning}
This section will describe our considerations when building and modelling our network. The forecasting models will be implemented using an Artificial Neural Network with back propagation where it takes as input a time series and in return outputs the predicted value. The specific input parameters will be described in the sections of the two specific prediction networks - price and wind production. 
The simplest way to model the network is to simply take all input parameters from the time-series without altering or analysing them - this is a naive approach. In the following subsections we will describe different strategies to data set manipulation from the most naive approach to more refined approaches where normalizing and analysis of values is considered.



  EASY TO DO NAIVE

we add a module that will 



do some figures of the network design/model. Also talk about the number of hidden layers must be experimented with to find the best result. - underfit and overfit as well.
TALK ABOUT MULTIPLE-STEP-AHEAD FORECASTING --> see karlbranting
\subsection{Data manipulation}
In the following subsections we will describe different strategies for data set manipulation from the most naive approach to more refined approaches such as trimming and analysis. These approaches are applied to the data before sending the input to the network. These strategies are used in experiments with the actual networks to locate the best possible strategy.

\subsubsection{Normalization}
For artificial neural networks to do the best work; the data should be normalized to either bipolar data(-1 to 1) or binary data(0 to 1). This ensures the best performance by the activation functions since the sigmoid activation function has the steepest gradient (see figure ~\ref{fig:Sigmoid}) between -1 and 1 thus giving the finest granulated outputs. The same applies for the hyperbolic tangent(tanh) (see figure ~\ref{fig:Tanh}) activation function. The difference between the two are the output it generates from the same input. The hyperbolic tangent generates output that ranges from -1 to 1 and it has a steeper gradient than the sigmoid function thus giving tanh a bit more granulated outputs than the sigmoid function.
\begin{figure}[H]
\centering
\includegraphics[width=0.8\linewidth,natwidth=898,natheight=587]{billeder/activationFunctions/sigmoid.png}
\caption{Sigmoid}
\label{fig:Sigmoid}
\end{figure}

\begin{figure}[H]
\centering
\includegraphics[width=0.8\linewidth,natwidth=898,natheight=587]{billeder/activationFunctions/tanh.png}
\caption{Hyperbolic tangent(tanh)}
\label{fig:Tanh}
\end{figure}

The normalization of the data is done using the following functions:
\begin{table}[H]
\centering  % used for centering table
\renewcommand{\arraystretch}{2}
\begin{tabular}{c c} % centered columns
 \#Normalization & \#Function \\ [0.5ex] % inserts table 
%heading
\hline                  % inserts single horizontal line
Zero-to-one & $ X_{norm} = \frac{X_i - X_{min}}{X_{max} - X_{min}}$ \\
Minus-one-to-one & $ X_{norm} = \frac{X_i - (\frac{X_{max} - X_{min}}{2})}{\frac{X_{max} - X_{min}}{2}}$ \\
[1ex]
\hline %inserts single line
\end{tabular}
\caption{Normalization functions} % title of Table
\label{table:naiveTrainingApproach} % is used to refer this table in the text
\end{table}
Where $X_i$ is the data entry, $X_min$ is the lowest value in the data set and $X_max$ is the highest value in the dataset.

\subsubsection{The naive approach}
First of all we tried out the most simple approach available, the naive approach, since this has almost no overhead and takes no preprocessing of the data. The method is simple; you take one neuron per input and one neuron as output. You add all of the data in your training set without doing any preprocessing of data. This kind of training set is good enough for simple problems e.g. the XOR problem or likewise. When we talk about more complex real-world problems like forecasting the energy prices this method does not perform to well and gives less than satisfactory results. Another factor is that a huge part of getting a neural network to perform well is the manipulation of the dataset to get rid of outliers and some of the noise in the dataset. Nevertheless it still gives an indication whether you have some kind of coherence between your input data and the output data even though the peromance is less than satisfactory. \todo{Mere generelt}

\subsubsection{Trimming}
As mentioned before we need to get rid of some of the outliers and some of the noise in the data set to make it easier for the network to approximate a function based on the input data. There are different approaches to trimming but the two we use are standard trimming and percentile trimming of the dataset. Standard trimming is a simple way of getting rid of the worst outliers. The way to do it is; take a low and a high number in your data set and remove everything below and above these limits. Of course this can be very arbitrary but with a simple plot diagram you will be able to see where the limits should be set. Percentile trimming is a statistic approach where you cut of x\% from the top and the bottom of your dataset. You make a percentwise distribution over your dataset with the lowest values in the beginning and the highest values in the end. Then you can take the 5th percentile which represents the lowest 5\% of your dataset and the remove these values. The same is done for the top 5\% and then you have removed the most extreme outliers and made your dataset more robust in statistical analysis.
\todo{LAV GRAF DER VISER TRIMMING}
\todo{trimming og matrix skal skrives i lidt mindre hverdagssprog og saa er det fint}

\subsubsection{Matrix}
When analysing the data it is possible to find connection between different rows of data eg. what time of the day or what day of the week it is. These kind of data can be describe using a single node where you normalize the data between -1 and 1 but this will only give this neuron one weight to change for ALL the hours in the day. One approach to making these inputs more important than others are splitting them up into a simple matrix. This means that we have a matrix with one row and 24 inputs(one for each hour); we set a 1 in the input representing the time of day and set the rest to 0. This way the neural network will have a specific weight to apply according to which hour of the day it is. This of course adds alot of overhead in terms of how many input nodes you need to have and how many nodes you need in the hidden layers. This will increase the processing time of the neural network iterations so you have to do some cost/benefit analysis on it. \todo{Matrix er ogsaa ok, selvom der skal kontekstualiseres en anelse.}

\subsubsection{Historical Data}
It is the intention to achieve day-ahead forecasting by predicting the next 24 hours based on the past --- this can be defined as multiple step-ahead forecasting \todo{FIND REF}. It is necessary to investigate how to include input information about the past hours from step to step in order to give an idea of what will come next. This becomes clear when considering price and wind production development curves where the immediate development of the price or production is significant factors for the movements to come. When looking at the wind production in Figure~\ref{fig:windHourDevelopment400Hours} it is obvious that the production has a tendency and based on this immediate history moves either up or down. It does not suddenly drop from 0 to 1500 without having any intermediary steps. It is necessary to identify these tendencies in every hour in order to predict it. One approach is to use statistical methods to capture the past as used in economics \todo{FIND REF}. Another approach could be to always include prices or wind productions at different times from the last 24 hours. What is evident is the need for analysing the price and wind production curves both to capture the tendencies but at the same time locating irregularities --- prices or productions that are so obscure that we won't be able to predict them.

\begin{figure}[H]
\centering
\includegraphics[width=0.99\linewidth,natwidth=898,natheight=587]{billeder/productionTendency400Hours.png}
\caption{Wind production development for 400 hours in 2011}
\label{fig:windHourDevelopment400Hours}
\end{figure}

\todo{price irregular price curves}

Price and wind production curves must 
It is the expectation that the further we get from the start time the less accurate a prediction will be because it uses data from its immediate past to do the prediction. 

\todo{Mention the calculation of the outer percentiles with a neural network for itself}

\section{Statistical Input}
\todo{Talk about statistical input features like historical volatility (EWMA), skewness and basic calculation of line slope. Skewness is a more sophisticated way of calculating if the distribution is leaning to one sine of the mean - the simple line slope calculation is meant to calculate if we are on the way up or down}

TEGNING MULTIPE STEP

TALK ABOUT MULTIPLE-STEP-AHEAD FORECASTING --> see karlbranting
\section{Strategies for Prediction}
\label{sec:stratsForPrediction}
\todo{present some of the strategies that we have towards prediction - maybe a summary of the above things?}
\subsection{Standard Inputs}

\subsection{Using Statistical Inputs}
\label{sec:usingStatisticalInput}
As described in the Section~\ref{sec:annSection} and in Figure~\ref{fig:overfitting} the Neural Network strives for a generalized function without overfitting so that it also applies for data outside of the trained set. In order to achieve such a function it is necessary to include enough input parameters to get a close enough fit. Every input parameter should narrow down the number of possible output values or else it makes no sense to include them. What can become problematic is when the input parameters simply result in too many output values, e.g. if similar wind speeds, air densities and temperatures would correspond to wind productions between 800-1300 \todo{concrete example from data set}. The generalization would move towards the majority of the wind productions in the interval but because the purpose of prediction is to come as close as possible to the ideal value this is not enough when the interval is too big. One way to solve this is to include a "snapshot" of the current situation and add it as input to the network so that it takes into account market trends at that time. The purpose is to add additional characteristics of the time-series that are currently used for training. As discussed in Historical Data~\ref{sec:historicalData} the market has certain trends and if the prediction knows the production or price from one hour ago and the current market trend in general, it will possibly have a better chance of guessing within the given interval. If putting this into context of the wind production interval 800-1300; the last hour wind production being 1100 and a rising current trend should with high probability not guess around 800 even if the majority of the wind productions is placed here. The concept is illustrated in Figure~\ref{fig:WP}\todo{make drawing}. Including the current trend as input can help the ANN to better approach the target to predict because it  would calculate how current trend in general influences the output over the entire dataset. The trend input parameter would be a probability for the curve to either go up or down at every step. This can potentially cause problematic situations when facing steep rising or falling slopes, provided that such exist, because the ANN would not predict high or low enough since the trend says nothing about the slope but only if the trend is rising or falling. A possible solution could be to include inputs telling something about the current slope. One such input could be a simple curve analysis where the slope of the last prices or productions are taken into consideration as well. The slope says something about how much the curve in general is rising whereas the statistics will reflect the general trend, are we moving up or down, over time. If the majority of the slopes in the dataset is falling the network would predict according to that. If we are facing a steep slope we won't be able to predict high enough since the slope is out of the ordinary. Events could be related to socio-cultural events that are beyond the scope of this thesis like the financial crisis or the breakdown of power plants.

Other problems arise together with the inclusion of tendency --- when have we moved enough in one direction? We need to rely on the other input parameters to pull us either up or down so that we can identify the next trend. The dataset used for training must be big enough to reflect most trends or else problems can occur when meeting new trends in the unseen data for prediction, e.g. if most trends in general are rising in the training set and then trying to predict 24 hours that are not would result in a disproportion between the two sets. The possibility of over-shooting the first targets would be high since it has never seen such trends before.    It will need to be tested thoroughly during our experiments. 

One thing to keep in mind here is that the trend and slope are only a minority of the input parameters. Without these the Artificial Neural Network would have made one generalization and the assumptions is that it will help the function to approach the target better when a lot of output possibilities exist. The result will be a new function where the immediate past is considered at every hour and the possibility of a improved generalization.

\begin{figure}[H]
\centering
\includegraphics[width=0.99\linewidth,natwidth=898,natheight=587]{billeder/WP_000057.jpg}
\caption{A) Shows the generalized function. B) The approaching of output values}
\label{fig:WP}
\end{figure}

We will work with several statistical methods as input and with combinations of them all. 

\subsection{Statistical Input}
\subsubsection{Curve Analysis}
\label{sec:curveAnalysis}
\subsection{Scatter}
\subsubsection{EWMA - Historical Volatility}
\label{sec:ewmaVolatility}
\todo{Talk about statistical input features like historical volatility (EWMA), skewness and basic calculation of line slope. Skewness is a more sophisticated way of calculating if the distribution is leaning to one sine of the mean - the simple line slope calculation is meant to calculate if we are on the way up or down}

EWMA should be used for time series that do not have a clear trend direction\cite[Chapter~7.3.2]{econometrics} which is exactly what we have. \todo{see page 588 in econometrics book}. EWMA is a latent trend model where a latent variable is included to describe a discrete choice model. The latent variable is called the smoothing factor and if this factor is close to one then the last trend has a higher weight than the recent observation and when it is close to zero the new has higher priority and thereby letting the EWMA follow trends more rapidly - in our case the observation would be either production or price and the smoothing factor is calculated based on the last calculated trend. Choice of smoothing factor must be tested in experiments but since there is high volatility in both price and wind power production a lower factor could be expected. Every hour must calculate the historical volatility based on a defined number of previous hours. The exact number will also be found in experiments.

\subsubsection{Skewness}
\label{sec:skewness}

\subsection{Similar Days Approach}
\label{sec:sdmApproach}
\cite{pjmForecast} and described in related work.

\subsection{Re-calculating output}
The re-calculation concept works by letting other neural networks re-calculate the prediction from the original network. The purpose is to identify places where the generalization function has problems and then divide it into new neural networks that only have the purpose of focusing on these problematic situations. For simplicity we are going to apply this to the small numbers and big numbers of the dataset. We will define what is small and big by taking the 2 and 98 percentile. Whenever the "original" network predicts something in or close to the big or small interval then the responsibility of that prediction is forwarded to one of the two. It works like a second opinion but here the network has been specialized to only focus on a specific part of the dataset instead of everything\todo{give example of corrections}. It creates a normalization function on a smaller dataset and the idea is to leave out a lot of unnecessary information. It does not need to account for anything other than its interval in the generalization function so it should be able to predict its target group better. We let the original network decide if it actually thinks we are in the low numbers --- if we are then we can possibly make that prediction even more accurate. These thoughts could basically we applied on all other parts of possible wind production values. 

Problem can occur if the intervals are too small which will cause the dataset to be too small and hard for the ANN to generalize upon.



TEGNING MULTIPE STEP

TALK ABOUT MULTIPLE-STEP-AHEAD FORECASTING --> see karlbranting

\section{Conclusion}
When modeling the network it is necessary to experiment with different types of inputs, hidden layers and neurons to find the best combinations. Furthermore, it is important to investigate the number of training epochs in order to find the best fit so that the networks training will not be under-fitted or over-fitted. We also have to experiment with the best data manipulation to find the best representation for the input parameters. When doing multiple step-ahead forecasting the immediate trend must be transferred as input to the next step, e.g. by including statistical economical features.

 

%%%%%%%%%%%%%%%%%%%%%%%%%%%%%%%

\chapter{Experimental Results}
\label{ch:experimentalResults}
This section describes the experimental results of the prediction simulations for both wind power and electricity prices. These experiments are done to verify or reject the assumptions from the dataset analysis. Before the actual experiments we will introduce the procedure for testing.
\section{Test Procedure}
\label{sec:testProcedure}
The following test procedure describes the test process and environment for the experiments of both wind power and electricity prices. 
\\[0.5cm]
The construction of the Feedforward Neural Network topology is done by incremental pruning as described in Section~\ref{sec:pruning}. The tool helps to decide the structure of the hidden layers and neurons in the network based on the pre-defined number of inputs and the output. The returned configuration is considered to be the optimal structuring. Every time the input combination change pruning will be applied to discover the best possible configuration. All predictions will be based on a simulation over an entire year where 8600 hours is predicted 24 hours at a time. It is important to emphasize that the experiments have been conducted on year to simulate a real-life use --- this is important because we want the results to realistically reflect practical use. In the following experiments we consider those numbers of hours to be satisfying. For testing purposes and simplicity the top-3 of an experiment will be used as basis for the next. 

A more accurate prediction will always be preferred but trade-offs will be discussed and analysed. The starting number of epochs and size of the dataset has been selected through simple trial and error but will be tested more thoroughly as testing proceeds to validate it. It is not possible or realistic to show all prediction graphs in full extend and therefore all experiments will point out only parts of the prediction graphs to highlight explanatory situations or problems relevant to the discussion or analysis at hand. Additional 1000 hours of the highlighted predictions will be shown in the appendix where a link to a digitalized version of all hours can be found. 

We are using two datasets when predicting with the Artificial Neural Network. The training set is what the network is trained with. The testing set is unseen data that is NOT contained in the training set that the network must be able to predict. The need to generalize beyond the training set is described in Section~\ref{sec:machineLearning} about Machine Learning and and makes sense in a prediction context --- the necessity for doing "out-of-sample"-forecasting is also stated in Section~\ref{sec:svmPrediction} about Support Vector Machines. The data collection from Section~\ref{sec:dataCollection} describes how our testing set does not contain weather forecasts but actual values from 2012 but also that 24 hours weather forecasts showed an accuracy of 97\% in 2012. It must of course be taken into consideration when discussing our results.
\\[0.5cm]
The procedure for every experiment:
\begin{itemize}
\item Describe the purpose.
\item Describe the expected outcome (Hypothesis).
\item Identify variables to be used.
\item For all predictions in the experiment do:
\begin{itemize}
	\item Prune the network based on input parameters.
	\item Simulate the prediction of 8600 hours 24 hours at a time.
	\item Show results.
	\item Analyse results.
	\item Point out indicative parts of the prediction graphs.
\end{itemize}
\item Conclusion
\end{itemize}
\noindent In short, the experiments will be conducted in five steps for both wind power and electricity price; 1) Selection of Input Parameter; 2) Data Manipulation; 3) Calculated Inputs; 4) Black Box Optimization; 5) Step-Ahead Forecasting.
\newpage
\section{Statistical Evaluation Methods}
\label{sec:statisticalEvaluation}
STATISTIKKER
	Data Cleaning:
		Pearsons
		Trimming
		Percentile trimming

	Error evaluation:
		Mean Average Error
		Mean Squared Error
		Root 

\newpage
\section{Wind Power Experiments}
\label{sec:windProductionExperiments}
\subsection{Experimental Results}
This subsection describes the experiments that uncovers the best combination of hidden layers, neurons and epochs as well as the different strategies. Based on the above analysis of wind production influences the network will be tested with the following input parameters; 1) Wind speed; 2) Air Density; 3) Time of day 4) Temperature and 5) Consumption;. Furthermore, experiments are needed to investigate the statistical inputs as presented in section~\ref{sec:usingStatisticalInput}.

\todo{DRAWING OF NETWORK}

I FOUND RECURRENT LINK IN ARTICLE REF ID

Black art --- experimenting with hidden layers, momentum and learning rate. 

\todo{remember to compare temperature inclusion in relation to consumption}

\subsubsection{Experiment Series One - Selection of input parameters}
The first experiment series is an attempt to find the best input parameters of the network. The input parameters to experiment with is wind speed, air density, time of day, consumption and temperature. The most significant tests will be to determine if meteorological factors can stand as a substitute for consumption and if wind direction improves the prediction.


\begin{table}[H]
\centering  % used for centering table
\resizebox{\textwidth}{!}{
\begin{tabular}{c c c c c c c c c c} % centered columns (3 columns)
 & Wind & Air &  &  & Wind & Time & Last & \\ 
Test & Speed & Dens & Consump & Temp & Direction & of Day & Production & MAE & Rank \\
[0.5ex] % inserts table 
%heading
\hline                  % inserts single horizontal line
1. & x &  &  &  &  &  &  & 153.6 & 0 \\ %
2. & x & & x & & & & & 139.8 & 0 \\ 
3. & x & & x & & & x & & 138.7 & 0 \\
4. & x & & x & & & x & x & 139.0 & 0 \\ 
5. & x & & x & & x & x & & 147.7 & 0 \\
7. & x & & x & x &  &  & x & x 141.8 & 0 \\
8. & x & x & & & & & & 140.4 & 0 \\
9. & x & x & & & & x & & 139.2 & 0 \\
10. & x & x & x & & & x & & 138.3 & 0 \\
11. & x & x & x & & & x & x & 138.3 & 0 \\ [1ex] % [1ex] adds vertical space
\hline %inserts single line
\end{tabular}}
\caption{Table showing the difference between using consumption and meteorological factors} % title of Table
\label{table:consumptionInclusionTable} % is used to refer this table in the text
\end{table}

Since the co-relation between wind production and wind speed is 0.94 it has been tested on its own. The result clearly shows this relationship by only missing the ideal wind production with 143.9 in average in test number 1. Wind speed has been tested in combination with different input parameters. The addition of air density stands out with a MAE of 116.3 which can somehow be expected since it is directly proportional to the wind speed as described in~\ref{sec:airDensity}. 

The table clearly shows that wind direction does not have a correlation to the other parameters and therefore must be omitted in tests to come.
The importance of the wind production development is described in~\ref{sec:windProductionDev} and to account for this the network has been supplied with the wind production from one hour ago - more sophisticated attempts with statistics will come in other experiments.
It is obvious that using the consumption is much more accurate than substituting it with just meteorological factors. The wind direction does not give better prediction so this is omitted from the experiments to come.

\todo{The hours can be reflected as described in matrix section. Now each of the values will reflect one hour instead of one value reflecting all of the hours.}

The following experiments will be based on the input parameters:
\begin{itemize}
\item Wind Speed;
\item Air Density;
\item Wind production from last hour;
\item Consumption;
\end{itemize}


\subsubsection{Experiment Two - Data Manipulation}
The first experiment will show the difference in performance due to the manipulation of the time series data set. The three approaches are naive, trimming and matrix. These experiments do not include any statistical features. The point of this experiment is to find the best approach to simple data manipulation as presented in~\ref{sec:DataManipulation}.

\todo{also different size of dataset}

\begin{table}[H]
\centering  % used for centering table
\begin{tabular}{c c c c} % centered columns (3 columns)
ANN Type & MAE & MPE & Rank \\ [0.5ex] % inserts table 
%heading
\hline                  % inserts single horizontal line
ANN Naive & 0 & 0 & 0 \\ % inserting body of the table
ANN Trimming & 0 & 0 & 0 \\
ANN Matrix  & 0 & 0 & 0\\
ANN Matrix and Trimming  & 0 & 0 & 0 \\ 
ANN Statistical features  & 0 & 0 & 0\\ [1ex] % [1ex] adds vertical space
\hline %inserts single line
\end{tabular}
\caption{Table showing the performance of different data manipulation approaches when all run with 1500 epochs.} % title of Table
\label{table:dataManipulationApproaches} % is used to refer this table in the text
\end{table}

\todo{better with or without trimming?}

The purpose is to locate the 

\todo{do 1-12-24 step ahead to show the "carry-with"-error} 

Taking the most naive approach to prediction by simply including the important normalized input parameters and trying to predict the prices for an entire year. 

\subsubsection{Experiment Three - Size of data set}

\subsubsection{Experiment Four - Prediction Strategies}
Statistics


\subsubsection{Experiment Five - Performance Optimization}
The impact of different Layers/Neurons/Epochs combinations is huge and the results is shown here.

\begin{table}[H]
\centering  % used for centering table
\begin{tabular}{c c c c c} % centered columns (3 columns)
ANN Type & Epochs & MAE & MPE & Rank \\ [0.5ex] % inserts table 
%heading
\hline                  % inserts single horizontal line
ANN & 500 & 0 & 0 & 0 \\ % inserting body of the table
ANN & 1000 & 0 & 0 & 0 \\
ANN & 2000 & 0 & 0 & 0 \\
ANN & 2500 & 0 & 0 & 0 \\ [1ex] % [1ex] adds vertical space
\hline %inserts single line
\end{tabular}
\caption{Table showing the performance of different data manipulation approaches.} % title of Table
\label{table:performanceOpti} % is used to refer this table in the text
\end{table} 

\subsection{testlol}

\begin{center}
\begin{longtable}{|c|c|c|c|c|c|c|c|c|c|}
\caption{Input parameters test}\\
\label{table:windProdInputParams}
\hline
\textbf{Wind Speed} & \textbf{Air Density} & \textbf{Consumption} & \textbf{Temperature} & \textbf{Wind Direction} & \textbf{Time of Day} & \textbf{Last Production}& \textbf{Time Matrix}& \textbf{MAE} & \textbf{Rank} \\
\hline
\endfirsthead
\multicolumn{10}{c}%
{\tablename\ \thetable\ -- \textit{Continued from previous page}} \\
\hline
\textbf{Wind Speed} & \textbf{Air Density} & \textbf{Consumption} & \textbf{Temperature} & \textbf{Wind Direction} & \textbf{Time of Day} & \textbf{Last Production}& \textbf{Time Matrix}& \textbf{MAE} & \textbf{Rank} \\
\hline
\endhead
\hline \multicolumn{10}{r}{\textit{Continued on next page}} \\
\endfoot
\hline
\endlastfoot

%\arrayrulecolor{light-gray}
 x &  x &  &  &  x &  &  x &  x & 119.7 & \#1 \\ \hline
 x &  x &  &  &  &  &  x &  & 119.8 & \#2 \\ \hline
 x &  x &  x &  x &  &  &  &  x & 119.97 & \#3 \\ \hline
 x &  &  x &  x &  &  &  &  x & 120.39 & \#4 \\ \hline
 x &  x &  &  &  x &  x &  x &  & 120.71 & \#5 \\ \hline
 x &  &  x &  &  &  &  &  x & 121.5 & \#6 \\ \hline
 x &  &  x &  &  &  &  x &  & 121.72 & \#7 \\ \hline
 x &  &  &  &  &  x &  x &  & 121.75 & \#8 \\ \hline
 x &  &  &  &  &  &  x &  x & 122.24 & \#9 \\ \hline
 x &  &  x &  x &  &  &  x &  x & 122.28 & \#10 \\ \hline
 x &  &  x &  x &  x &  &  x &  x & 123.29 & \#11 \\ \hline
 x &  &  x &  &  x &  &  &  x & 123.52 & \#12 \\ \hline
 x &  x &  x &  &  &  &  &  x & 123.91 & \#13 \\ \hline
 x &  &  &  &  x &  x &  x &  & 124.53 & \#14 \\ \hline
 x &  x &  &  &  &  x &  x &  & 124.77 & \#15 \\ \hline
 x &  x &  &  &  &  &  &  x & 124.9 & \#16 \\ \hline
 x &  &  &  &  x &  &  x &  & 125.16 & \#17 \\ \hline
 x &  &  &  x &  &  &  &  x & 125.59 & \#18 \\ \hline
 x &  x &  &  &  x &  &  &  x & 125.84 & \#19 \\ \hline
 x &  x &  x &  &  x &  &  &  x & 125.91 & \#20 \\ \hline
 x &  x &  x &  &  x &  &  x &  & 125.99 & \#21 \\ \hline
 x &  x &  &  x &  &  &  x &  & 126.69 & \#22 \\ \hline
 x &  x &  &  x &  &  &  &  x & 126.95 & \#23 \\ \hline
 x &  x &  x &  &  x &  x &  x &  & 127.28 & \#24 \\ \hline
 x &  x &  x &  x &  x &  &  x &  x & 127.75 & \#25 \\ \hline
 x &  &  &  &  &  &  &  x & 127.94 & \#26 \\ \hline
 x &  x &  x &  &  x &  &  x &  x & 128.31 & \#27 \\ \hline
 x &  &  &  &  x &  &  x &  x & 128.49 & \#28 \\ \hline
 x &  &  &  &  x &  &  &  x & 128.85 & \#29 \\ \hline
 x &  x &  &  &  x &  &  x &  x & 128.86 & \#30 \\ \hline
 x &  x &  x &  &  x &  &  x &  & 129.01 & \#31 \\ \hline
 x &  x &  &  x &  &  &  x &  x & 129.28 & \#32 \\ \hline
 x &  &  x &  &  x &  &  x &  & 129.32 & \#33 \\ \hline
 x &  x &  x &  x &  x &  x &  x &  & 129.72 & \#34 \\ \hline
 x &  x &  &  x &  &  x &  x &  & 130.27 & \#35 \\ \hline
 x &  &  &  x &  &  &  x &  x & 131.17 & \#36 \\ \hline
 x &  x &  &  x &  x &  &  x &  & 131.21 & \#37 \\ \hline
 x &  x &  x &  x &  &  &  x &  & 131.61 & \#38 \\ \hline
 x &  x &  &  x &  &  &  x &  x & 132.7 & \#39 \\ \hline
 x &  x &  x &  &  &  &  x &  x & 132.85 & \#40 \\ \hline
 x &  x &  &  &  x &  x &  x &  & 132.97 & \#41 \\ \hline
 x &  x &  &  &  &  &  x &  x & 133.11 & \#42 \\ \hline
 x &  &  x &  x &  x &  x &  x &  & 133.79 & \#43 \\ \hline
 x &  x &  x &  &  &  x &  x &  & 134.07 & \#44 \\ \hline
 x &  x &  &  x &  &  x &  x &  & 134.3 & \#45 \\ \hline
 x &  x &  x &  &  &  &  x &  & 135.03 & \#46 \\ \hline
 x &  x &  x &  x &  x &  &  x &  x & 135.51 & \#47 \\ \hline
 x &  x &  x &  &  &  &  &  & 135.78 & \#48 \\ \hline
 x &  x &  &  &  x &  x &  &  & 136.45 & \#49 \\ \hline
 x &  &  x &  &  x &  x &  x &  & 136.57 & \#50 \\ \hline
 x &  x &  x &  &  x &  x &  &  & 136.82 & \#51 \\ \hline
 x &  x &  x &  &  &  x &  &  & 136.98 & \#52 \\ \hline
 x &  x &  &  x &  x &  x &  x &  & 136.99 & \#53 \\ \hline
 x &  &  x &  &  x &  &  x &  x & 137.08 & \#54 \\ \hline
 x &  &  x &  &  &  x &  x &  & 137.33 & \#55 \\ \hline
 x &  &  &  x &  &  x &  &  & 137.52 & \#56 \\ \hline
 x &  x &  x &  x &  x &  &  x &  & 137.71 & \#57 \\ \hline
 x &  &  x &  x &  x &  &  x &  x & 137.76 & \#58 \\ \hline
 x &  x &  &  x &  &  x &  &  & 137.96 & \#59 \\ \hline
 x &  &  x &  &  &  &  x &  x & 138.42 & \#60 \\ \hline
 x &  x &  &  &  x &  &  &  & 138.43 & \#61 \\ \hline
 x &  &  x &  &  x &  x &  &  & 138.52 & \#62 \\ \hline
 x &  x &  x &  x &  &  &  &  & 138.59 & \#63 \\ \hline
 x &  x &  x &  x &  &  x &  &  & 138.88 & \#64 \\ \hline
 x &  x &  &  &  &  &  &  & 139.12 & \#65 \\ \hline
 x &  &  x &  &  x &  &  &  & 139.3 & \#66 \\ \hline
 x &  &  &  x &  &  &  &  & 139.33 & \#67 \\ \hline
 x &  &  &  x &  &  &  x &  & 139.41 & \#68 \\ \hline
 x &  &  x &  x &  x &  x &  x &  & 139.48 & \#69 \\ \hline
 x &  &  x &  x &  &  &  &  & 139.59 & \#70 \\ \hline
 x &  x &  &  &  x &  &  x &  & 139.73 & \#71 \\ \hline
 x &  &  &  &  x &  x &  x &  & 139.99 & \#72 \\ \hline
 x &  x &  &  x &  &  &  &  & 140.08 & \#73 \\ \hline
 x &  &  &  x &  &  x &  x &  & 140.13 & \#74 \\ \hline
 x &  &  &  &  &  x &  &  & 140.24 & \#75 \\ \hline
 x &  x &  x &  &  x &  &  &  & 140.92 & \#76 \\ \hline
 x &  x &  &  &  &  x &  &  & 141.19 & \#77 \\ \hline
 x &  x &  x &  x &  x &  &  x &  & 141.41 & \#78 \\ \hline
 x &  &  x &  &  &  &  &  & 141.77 & \#79 \\ \hline
 x &  &  &  &  x &  x &  &  & 141.79 & \#80 \\ \hline
 x &  &  &  x &  x &  &  x &  & 141.85 & \#81 \\ \hline
 x &  &  &  &  &  &  &  & 141.87 & \#82 \\ \hline
 x &  &  x &  &  &  x &  &  & 141.92 & \#83 \\ \hline
 x &  &  x &  x &  &  x &  &  & 142.4 & \#84 \\ \hline
 x &  &  x &  x &  &  x &  x &  & 142.76 & \#85 \\ \hline
 x &  &  &  x &  &  &  x &  & 143.35 & \#86 \\ \hline
 x &  &  &  &  x &  &  &  & 143.58 & \#87 \\ \hline
 x &  &  &  &  x &  &  x &  & 143.76 & \#88 \\ \hline
 x &  &  &  &  x &  &  x &  x & 143.96 & \#89 \\ \hline
 x &  x &  x &  x &  &  x &  x &  & 144.3 & \#90 \\ \hline
 x &  x &  x &  x &  &  &  x &  x & 144.9 & \#91 \\ \hline
 x &  &  x &  x &  x &  &  x &  & 145.07 & \#92 \\ \hline
 x &  x &  &  x &  &  &  x &  & 145.37 & \#93 \\ \hline
 x &  x &  x &  x &  x &  &  x &  x & 145.9 & \#94 \\ \hline
 x &  x &  x &  &  x &  &  x &  x & 146.31 & \#95 \\ \hline
 x &  &  &  x &  &  &  x &  x & 146.65 & \#96 \\ \hline
 x &  &  x &  x &  x &  x &  x &  & 146.78 & \#97 \\ \hline
 x &  x &  &  &  x &  &  x &  & 147.92 & \#98 \\ \hline
 x &  &  x &  &  x &  x &  x &  & 148.84 & \#99 \\ \hline
 x &  x &  x &  &  x &  x &  x &  & 151.94 & \#100 \\ \hline
 x &  &  x &  x &  x &  &  x &  & 152.48 & \#101 \\ \hline
 x &  &  x &  &  x &  &  x &  x & 153.36 & \#102 \\ \hline
 x &  x &  x &  x &  x &  x &  x &  & 155.57 & \#103 \\ \hline
 x &  &  x &  x &  x &  &  x &  x & 155.83 & \#104 \\ \hline
 x &  &  &  x &  x &  x &  x &  & 159.09 & \#105 \\ \hline
 x &  &  &  x &  &  x &  x &  & 159.3 & \#106 \\ \hline
 x &  x &  &  x &  x &  &  x &  x & 159.47 & \#107 \\ \hline
 x &  &  &  &  &  &  x &  & 159.57 & \#108 \\ \hline
 x &  x &  x &  x &  x &  &  x &  & 160.91 & \#109 \\ \hline
 x &  &  x &  &  x &  &  x &  & 161.28 & \#110 \\ \hline
 x &  &  &  x &  x &  &  x &  x & 167.37 & \#111 \\ \hline
 x &  &  x &  x &  &  &  x &  & 173.48 & \#112 \\ \hline
 x &  x &  x &  x &  x &  x &  x &  & 191.6 & \#113 \\ \hline
 x &  &  x &  x &  x &  &  x &  & 202.93 & \#114 \\ \hline
\end{longtable}
\end{center}


\newpage
\section{Electricity Price Experiments}
\label{sec:priceExperiments}
\subsection{Experimental results}
This section contains experimental results to find the best possible price predicting solution. The results are divided into 5 experiments. The first experiment finds the best combination of input parameters for the network. These input parameters count meteorological, social and seasonal factors we identified in section~\ref{sec:Price}. The next experiment identifies the need for trimming and reasons why it is needed in this particular dataset. The third experiment tests different statistical strategies to incorporate historical prices as a part of the dataset. This includes historical prices, curve behavior analysis, skewness analysis and historical EWMA(Exponentially-Weighted Moving Average). The fourth experiment takes care of the Artificial Neural Network parameters. This includes black-box optimization like pruning of the network and optimization of epochs.

\subsubsection{Experiment 1: Inputs}
In this section we experimented with the basic input parameters that we identified in section~\ref{sec:Price}. We did a cross comparison of all the inputs and possible combinations (For those that made sense). The Price and the Demand is such a basic measure when you define a price in any markets that they are included in every prediction. We did not make a cross comparison of the Month of Year and the Seasons of Year since they say the same but with different granularity. We did a cross comparison both with and without matrix inputs and with a mix of non-matrix and matrix inputs. This is done to determine whether an input makes better sense on matrix form or just as a simple normalized value. 

Experiment 1 is based on a dataset consisting of the last 3 months averaging to about 2189 hours. We use 200 epochs for each training iteration.

This includes last hours Price (P), The Demand (D), Wind Speed(WS), Temperature(T), The Hourly Time of Day (ToD), The Day of The Week(DoW), The Month of The Year(MoY), The Season of The Year(SoY). The (M) is for Matrix input and states whether the input in the seasonal rows(ToD, WoD, MoY, SoY) are on matrix form or not.

As we saw in the wind production experiments(WPE) section~\ref{sec:windPowerAnalysis} there was a problem regarding the way we conducted tests for the seasonality; specifically for the MoY and the SoY. We only use the last 3 months to train the network and that sort of eliminates the obvious purpose of the MoY and SoY. As we saw in the results for the first experiment in WPE if we include the same month as we are in from last year and reintroduce the purpose of the SoY and MoY it gave us a worse result than leaving it out. To make sure the same thing applies on the price prediction experiments we conducted to runs of all the combinations of inputs. First we ran it with the last 3 months including the same month that we are in from last year and after that we ran the experiment with only the last 3 months.

Table~\ref{table:Top20Prices} shows us the top 20 MAE from the experiment with a training set containing the last 3 months(3Month). We have compared this to the experiment with a training set containing the last 3 months and the same month from last year(4Month). As we see in the table there isn't the biggest deviation between the two datasets. This also shows in \todo{Ref Appendix for both} that the distribution of the MAE over the two datasets are quite similar. From this we can conclude that the effect of using last years month in our dataset does not make a significant difference and can be left out. This raises the question if the MoY and the SoY can be left out as well, since the obvious use for it is eliminated by never having a full month or full season in the training set - that is equal to the one we predict.

First we conducted an experiment containing a training set with a full year (to test the effect of seasonality on a full year) we included the rest of the parameters equal to rank \#1 in table~\ref{table:Top20Prices} and shifted the seasonality. The results can be seen in table~\ref{table:1YearTrain} where we can see the full effect of seasonality over a year. The table clearly shows that the MoY combination is the best of the three. If we compare the results to the results from table~\ref{table:Top20Prices} it shows us that with monthly seasonality the neural network will be able to do the same predictions as the predictions that only have a training set of 3 months. With the possibility of over training in too large datasets(see citation~\cite{1}) and that the neural network takes longer time training on a larger dataset. We can conclude that the smaller dataset of 3 months are better to use than the full year (We will elaborate on this in experiment X)\todo{Saet rigtigt experiment}. Further more we have to test if seasonality being an input parameter makes sense when our dataset is only 3 months big. 

\begin{table}[H]
\centering  % used for centering table
\resizebox{\textwidth}{!}{
\begin{tabular}{c c c c c c c c c c c c} % centered columns (7 columns)
P & D & WS & T & ToD & DoW & MoY & SoY & 3Month & 4Month & Rank\\ [0.5ex] % inserts table 
%heading
\hline                  % inserts single horizontal line
 \x    & \x    & \x    & \x    & \x\m  & \x\m  &       & \x\m  & 57.12 & 61.70 & \#1 \\
 \x    & \x    & \x    & \x    & \x\m  & \x    &       & \x\m  & 58.09 & 68.23 & \#2 \\
 \x    & \x    & \x    & \x    & \x\m  &       & \x\m  &       & 58.79 & 63.80 & \#3 \\
 \x    & \x    & \x    &       & \x\m  & \x\m  & \x\m  &       & 60.14 & 67.82 & \#4 \\
 \x    & \x    & \x    & \x    & \x\m  & \x    &       &       & 62.19 & 74.89 & \#5 \\
 \x    & \x    & \x    & \x    & \x\m  &       &       & \x\m  & 62.26 & 61.86 & \#6 \\
 \x    & \x    & \x    & \x    & \x\m  & \x    & \x\m  &       & 62.84 & 62.62 & \#7 \\
 \x    & \x    & \x    & \x    & \x    & \x    &       & \x\m  & 63.94 & 72.37 & \#8 \\
 \x    & \x    & \x    & \x    & \x    & \x\m  & \x\m  &       & 64.19 & 84.12 & \#9 \\
 \x    & \x    & \x    &       & \x\m  & \x    & \x\m  &       & 64.72 & 72.29 & \#10 \\

 \x    & \x    & \x    &       & \x    & \x    & \x\m  &       & 65.07 & 79.71 & \#11 \\
 \x    & \x    & \x    & \x    & \x\m  & \x\m  &       &       & 65.95 & 58.19 & \#12 \\
 \x    & \x    & \x    & \x    & \x\m  & \x\m  & \x\m  &       & 66.55 & 67.31 & \#13 \\
 \x    & \x    & \x    &       & \x    &       &       & \x\m  & 67.21 & 74.67 & \#14 \\
 \x    & \x    & \x    & \x    & \x    & \x    &       &       & 67.88 & 77.29 & \#15 \\
 \x    & \x    & \x    &       & \x    & \x    &       & \x    & 68.21 & 72.45 & \#16 \\
 \x    & \x    & \x    &       & \x\m  & \x\m  &       &       & 68.34 & 72.31 & \#17 \\
 \x    & \x    & \x    &       & \x\m  &       &       & \x\m  & 68.35 & 76.25 & \#18 \\
 \x    & \x    & \x    & \x    & \x    & \x    & \x    &       & 68.43 & 75.10 & \#19 \\
 \x    & \x    & \x    & \x    & \x    &       &       & \x\m  & 68.45 & 75.97 & \#20 \\ 
 \hline %inserts single line
\end{tabular}
}
\caption{The top 20 results on training set 3 last months} % title of Table
\label{table:Top20Prices} % is used to refer this table in the text
\end{table}

\begin{table}[H]
\centering  % used for centering table
\begin{tabular}{c c c c c c c c c c c} % centered columns (7 columns)
P & D & WS & T & ToD & DoW & MoY & SoY & MAE & Rank\\ [0.5ex] % inserts table 
\hline
\x    & \x    & \x    & \x    & \x\m  & \x\m  & \x\m  &       & 63.74 & \#1 \\
\x    & \x    & \x    & \x    & \x\m  & \x\m  &       &       & 90.79 & \#2 \\
\x    & \x    & \x    & \x    & \x\m  & \x\m  &       & \x\m  & 94.75 & \#3 \\
\hline
\end{tabular}
\caption{The top 20 results on training set 3 last months} % title of Table
\label{table:1YearTrain} % is used to refer this table in the text
\end{table}

\begin{table}[H]
\centering  % used for centering table
\begin{tabular}{c c c c c c c c c c c} % centered columns (7 columns)
P & D & WS & T & ToD & DoW & MoY & SoY & MAE & Rank\\ [0.5ex] % inserts table 
\hline
\x    & \x    & \x    & \x    & \x\m  & \x\m  & \x\m  &       & 63.74 & \#1 \\
\x    & \x    & \x    & \x    & \x\m  & \x\m  &       &       & 90.79 & \#2 \\
\x    & \x    & \x    & \x    & \x\m  & \x\m  &       & \x\m  & 94.75 & \#3 \\
\hline
\end{tabular}
\caption{The top 20 results on training set 3 last months} % title of Table
\label{table:1YearTrain} % is used to refer this table in the text
\end{table}

If we take a look at the top 20 best input combinations shown in table~\ref{table:Top20Prices} we see some clear tendencies. If we start from the beginning the price and demand are static as mentioned earlier since they are fundamental market forces and thus not a changing factor in this analysis. The next input parameter is the Wind Speed. We see that every input combination in the top 20 includes the wind production and it is therefore a must for the prediction of the energy prices. Also we saw in section~\ref{sec:windPowerAnalysis}(table~\ref{table:pearsonCoeficientWindProduction}) that the Wind Speed heavily influences the green energy production and thus influencing the energy prices. \todo{Lav en sammenligning af wind speed og wind production for at udelukke den ene}

The temperature is a less obvious candidate for the prediction of price since the Pearson's correlation between the two only are 0.17. Nevertheless it is showing up in 8/10 top combinations in table~\ref{table:Top20Prices} and this might be because of the correlation between temperature/demand which is -0.59. The temperature is scattered all over the 144 combinations and thus it is hard to say anything about this input with confidence. \todo{Lav en analyse med og uden temperatur paa den bedste}.

The Hourly Time of Day (ToD) is included in every single combination in the top 20. This clearly shows that this input parameter is important for the prediction of the price. This is kind of obvious since what we are predicting is the hourly price. In section~\ref{sec:seasonality}(figure~\ref{fig:price_per_hour}) we saw that the price varied from 190 to 335 which strengthens the importance of the relationship between time of day and the price. Also the top 7 all have the ToD on matrix form which indicates that this is the best way of representing the ToD.

Next we have the Day of the Week (DoW) parameter. This parameter are present in 75\% of the 20 best results (8/10 and in 15/20 best combinations). We have to believe that it plays a significant role in the prediction of price. If we look at the analysis of the average price over weekdays in section~\ref{sec:seasonality}(figure~\ref{fig:price_over_weekdays}) we see that there is a significant difference in price on the different days especially the weekdays compared to the weekend. This parameter is mixed between matrix input and standard input. This might be due to the fact that the biggest difference between days are weekend and weekday thus minimizing the effect of a matrix representation. \todo{Maaske lav et forsøg med weekday/weekend matrix.}

The last two parameters - Month of Year(MoY) and Season of Year (SoY) - are codependent and will be covered together. As mentioned before they cover the same information and we therefore only need one of them at any time \todo{Lav forsoeg der viser at det ikke giver mening at have baade MoY og SoY paa samme tid.}. The values are present in 9/10 of the best combinations in ~\ref{table:Top20Prices}. This is an indicator that the seasonality in the form of MoY and SoY plays a role in predicting the electricity price. Also we saw in the analysis in section~\ref{sec:seasonality}(figure~\ref{fig:monthlyAveragePrice} and ~\ref{fig:seasons}) that the price changes with seasonality and that it especially was more expensive in the winther than the rest of the year.


\begin{table}[H]
\centering  % used for centering table
\resizebox{\textwidth}{!}{
\begin{tabular}{c c c c c c c c c c c} % centered columns (7 columns)
P & D & WS & T & ToD & DoW & MoY & SoY & MAE & Rank\\ [0.5ex] % inserts table 
%heading
\hline                  % inserts single horizontal line
 \x    & \x    & \x    &       &       & \x\m  & \x\m  &       & 105.70 & \#134 \\
 \x    & \x    &       &       &       & \x\m  &       & \x\m  & 107.85 & \#135 \\
 \x    & \x    &       & \x    & \x    &       & \x    &       & 108.37 & \#136 \\
 \x    & \x    &       & \x    &       &       & \x    &       & 111.14 & \#137 \\
 \x    & \x    &       & \x    &       &       & \x\m  &       & 111.65 & \#138 \\
 \x    & \x    &       & \x    &       & \x\m  & \x\m  &       & 113.56 & \#139 \\
 \x    & \x    &       & \x    &       &       &       & \x    & 115.29 & \#140 \\
 \x    & \x    &       &       &       &       & \x\m  &       & 115.81 & \#141 \\
 \x    & \x    &       & \x    &       & \x\m  &       & \x\m  & 115.83 & \#142 \\
 \x    & \x    &       & \x    &       & \x    & \x    &       & 117.17 & \#143 \\
 \x    & \x    &       &       &       & \x\m  & \x\m  &       & 117.34 & \#144 \\
\end{tabular}
}
\caption{The bottom 10 input combinations for price prediction} % title of Table
\label{table:Bottom10Prices} % is used to refer this table in the text
\end{table}

If we take a look at the bottom 10 input combinations in terms of MAE we see some tendencies as well. We see that Wind Speed only appears in one of the combinations and so does the Time of Day(And not in the same combination). This further strengthens that these two variables are very important for the price prediction.

%\begin{table}[H]
%\centering  % used for centering table
%\resizebox{\textwidth}{!}{
%	\begin{tabular}{c c c c c c c c c c c} % centered columns (7 columns)
%	P & D & WS & T & ToD & WoD & MoY & SoY & MAE & Rank\\ [0.5ex] % inserts table 
%	\hline                  % inserts single horizontal line
%	x & x & x & x & x    & x(M) & x(M) &      & 61,95 & \#1 \\ %newPredictions/TEN__MIXEDPrice_Consump_windSpeed_temperatureRow_timeOfDay_weekdaysMATRIX_monthOfYearMATRIX
%	x & x & x & x & x(M) & x    &      & x(M) & 62,76 & \#2 \\ %newPredictions/TEN__MIXEDPrice_Consump_windSpeed_temperatureRow_timeOfDayMATRIX_weekdays",%
%	x & x & x & x & x(M) & x(M) &      & x(M) & 62,87 & \#3 \\ %newPredictions/TEN__MIXEDPrice_Consump_windSpeed_temperatureRow_timeOfDayMATRIX_weekdays_monthOfYearMATRIX",
%	x & x & x &   & x(M) & x    & x(M) &      & 62,99 & \#4 \\ %newPredictions/TEN__MIXEDPrice_Consump_windSpeed_timeOfDayMATRIX_weekdays_monthOfYearMATRIX",
%	x & x & x & x & x(M) & x    & x(M) &      & 64,24 & \#5 \\ %newPredictions/TEN__MATRIX_Price_Consump_windSpeed_temperatureRow_timeOfDay_weekdays_seasonOfYear",
%	x & x & x & x & x(M) & x    &      &      & 65,18 & \#6 \\ %newPredictions/TEN__MIXEDPrice_Consump_windSpeed_temperatureRow_timeOfDayMATRIX_weekdays_seasonOfYearMATRIX",
%	x & x & x & x & x(M) &      & x(M) &      & 65,53 & \#7 \\ %newPredictions/TEN__MIXEDPrice_Consump_windSpeed_temperatureRow_timeOfDayMATRIX_monthOfYearMATRIX",
%	x & x & x & x & x    & x    &      & x(M) & 65,80 & \#8 \\ %newPredictions/TEN__MIXEDPrice_Consump_windSpeed_temperatureRow_timeOfDayMATRIX_seasonOfYearMATRIX",
%	x & x & x & x & x(M) &      &      & x(M) & 67,21 & \#9 \\ %newPredictions/TEN__MIXEDPrice_Consump_windSpeed_temperatureRow_timeOfDay_weekdays_seasonOfYearMATRIX",
%	x & x & x &   & x(M) & x(M) & x(M) &      & 70,25 & \#10 \\ %newPredictions/TEN__MATRIX_Price_Consump_windSpeed_timeOfDay_weekdays_monthOfYear"
%	\hline %inserts single line
%	\end{tabular}
%}
%\caption{Average MAE of ten runs per entry} % title of Table
%\label{table:Top10Average} % is used to refer this table in the text
%\end{table}

\subsubsection{Experiment 2: Trimming}
\begin{figure}[H]
\centering
\includegraphics[width=0.85\linewidth,natwidth=898,natheight=587]{billeder/PriceExperimentalAnalysis/NoTrimming.jpg}
\caption{The \#1 forecast with no trimming of the dataset}
\label{fig:NoTrim}
\end{figure}

\begin{figure}[H]
\centering
\includegraphics[width=0.85\linewidth,natwidth=898,natheight=587]{billeder/PriceExperimentalAnalysis/1PTrim.jpg}
\caption{The \#1 forecast with 1\% trimming in both ends of the dataset}
\label{fig:1PTrim}
\end{figure}

\begin{table}[H]
\centering  % used for centering table
\resizebox{0.6\textwidth}{!}{
	\begin{tabular}{c c c c c c} % centered columns (7 columns)
	1PTrim & 2PTrim & 3PTrim & 4PTrim & 5PTrim & Number\\ [0.5ex] % inserts table 
	\hline                  % inserts single horizontal line
	47,21 & 42,90 & 44,48 & 42,46 & 41,79 & \#1 \\
	46,15 & 43,67 & 43,07 & 39,16 & 40,28 & \#2 \\
	47,14 & 45,10 & 43,38 & 40,50 & 39,41 & \#3 \\
	46,70 & 43,96 & 43,21 & 40,03 & 40,29 & \#4 \\
	45,96 & 43,25 & 45,51 & 40,74 & 40,42 & \#5 \\
	47,27 & 45,96 & 44,98 & 41,39 & 39,98 & \#6 \\
	45,93 & 44,66 & 43,39 & 41,02 & 40,40 & \#7 \\
	46,64 & 42,69 & 44,48 & 41,69 & 40,61 & \#8 \\
	45,98 & 44,51 & 43,71 & 40,74 & 41,08 & \#9 \\
	45,60 & 46,07 & 45,81 & 42,52 & 41,32 & \#10 \\
	\hline %inserts single line
	\end{tabular}
}
\caption{Trims} % title of Table
\label{table:Top10Trimming} % is used to refer this table in the text
\end{table}

363 entries gets removed per percent.

\subsubsection{Experiment 3: Statistical strategies}

\subsubsection{Experiment 4: Black box optimization}


\newpage

%%%%%%%%%%%%%%%%%%%%%%%%%%%%%%%%%%%%%%%%%%%%%%%%%%%%%%%%%%%%%%%%%%%%%

\chapter{Experimental Result Discussion}
\label{ch:experimentalResultDiscussions}
This section discusses the results ..........

\subsection{The importance of the underlying data}

\subsubsection{Input parameters}

\subsubsection{Data Manipulation - k}
De andre tekster undlader i stor omfang at snakke om input i forhold til konkret data, trimming og manipulation af det konkrete data. Sammenstil det med vores eget. Reflekter og konkluder.

== the best possible prediction. Hvad er med til at give det ebdsre result.

\subsubsection{Black Box optimization}



\subsection{Trust - b}
We have obtained the best possible prediction through data manipulation and black box optimization. The network outputs a prediction which in itself could be used as decision support by the traders. The problematic here is for the users to know when it can be trusted and why to even trust it. People are not in the habit of blindly trusting everything that is placed in front of them and especially when it comes to money.


Communicate what we are building the prediction on. Show how we have performed etc.

\todo{show examples with in low and high numbers}

Or maybe rather say how much can this actually be trusted.

 
the best possible prediction, how do you get people to trust it. "Black"-magic. Hvad sker der.
How do we communicate it.

\subsubsection{The underlying data}
Artificial Neural Networks are known to be a black box\todo{ref}. Knowledge about the underlying data is only known by the ones who built the system. First and foremost the system must communicate the input parameters and where it comes from.

\subsubsection{Uncertain information}
f.x. the difference in accuracy when in the lowest and highest numbers.

\subsubsection{The output}


\subsection{WHAT SHOULD BE DONE}
\begin{itemize}
\item Improve understanding of the underlying data and the output;
\item Transparency is the goal;
\item Let the user take decisions based on the uncertain information;
\item Visualize errors;
\end{itemize}

\subsubsection{Transparency}



%%%%%%%%%%%%%%%%%%%%%%%%%%%%%%%%%%%%%%%%%%%%%%%%%%%%%%%%%%%%%%%%%%%%%%%


\chapter{Conclusion}
\label{ch:conclusion}
The ambition of this dissertation has been to examine the feasibility of a Back Propagation Artificial Neural Network as a technology for prediction of the electricity price and wind power production in the Danish electricity market. The feasibility is derived by the dataset analysis, experimental results and the potential for practical use. Feedforward Neural Network with Resilient Backpropagation has been used as prediction model. We divide the hypotheses into three categories and describe how they have been fulfilled one by one.

\begin{itemize}
\item Dataset analysis --- The purpose of the dataset analysis was to identify influential factors to be tested in the experiments. The dataset analysis have been done by identifying the characteristics of the electricity price and wind power based on related work within the field. The correlation between the individual properties for price and wind power have been established through a comprehensive analysis of the collected data. These data consists of hourly meteorological factors, demand and electricity prices as well as wind power. The analysis was conducted by establishing correlations statistically and systematically investigating the characteristics in relation to the electricity price and wind power in the Danish electricity market. The electricity price showed to be influenced by last known price, demand, time of day, seasonality and wind speed. Wind power was as expected highly influenced by wind speed and last known production but also by demand, air density, wind direction, time of day and seasonality. A volatile nature was established for both but price more significantly than wind power. Furthermore the need for transparency in input parameters and their representation was identified to be important for understanding the context. The dataset analysis approach clarified the influential factors sufficiently and the results helped shape the hypotheses of the experiments. 
\item Experimental results --- The experimental results have the purpose of either verifying or rejecting the established hypotheses to obtain the best Artificial Neural Network models for both wind power and electricity prediction. The experiments were divided into sub-experiments with the purpose of handling different aspects of the analysis such as best input and data representation. Further fine-tuning was included in sub-experiments such as black box optimization and the impact of step-ahead forecasting. The experiments was conducted to uncover as many scenarios as possible but also designed to improve from experiment to experiment. The best input combination for wind power was wind speed, temperature, last known wind power and time of day where temperature was found to substitute consumption. All inputs were normalized and used directly as input, but time of day was divided into a matrix representation. The black box optimization showed a training set size of three months and 300 epochs of training with an accuracy in MAE of 121,02 MWh out of an interval of 0-2800 MWh. The best inputs for electricity price was last known price, demand, temperature, wind speed, time of day, day of week and season of year. The influence of wind speed on the electricity price shows the connection to wind power in the Danish market. All timely and seasonal inputs were represented as matrices. The dataset was trimmed with 1\% in top and bottom to remove irregularities. Best number of epochs was 500 on a six months training set with a MAE of 45,11 DKK out of an interval of 61-632 DKK. The many experiments have made it possible to cover all input combinations and thereby establishing the correct combinations and their representations. The experiments was conducted on unseen data and contributed to the transparency of and trust in the system by simulating day-ahead predictions over an entire year.
\item Practical use --- The analysis and experiments have been used to relate the results to the potential of practical use. The experimental results indicate practical use by simulating day-ahead prediction for a whole year and still maintaining a steady result. The need for transparency and trust is of utmost importance when taking a decision in the electricity market due to the high-risk nature of the financial markets and therefore it is closely related to the feasibility. The price or wind power cannot stand alone without being accompanied by elaborating information of what and how it was obtained. The problem that we are facing while modelling can be transferred to the subject of decision making in terms of trusting the dataset and its representation by communicating the choices behind. It comes down to making the black box more transparent.
\end{itemize}

\noindent Our results show that the day-ahead predictions in both cases are capable of approaching its target over an entire year. Both wind power and electricity price follows the correct direction of the curve in 70\% of the predictions. The achieved Mean Average Error for wind power and electricity prices are 121,02 MWh and 45,11 DKK, respectively. The findings throughout the thesis show the feasibility of Artificial Neural Network as a technology for prediction as well as for decision support.  

%%%%%%%%%%%%%%%%%%%%%%%%%%%%%%%%%%%%%%%%%%%%%%%%%%%%%%%%%%%%%%%%%%%%%%%%

\addcontentsline{toc}{chapter}{Bibliography}
\bibliographystyle{plain} 
\bibliography{refs}

%%%%%%%%%%%%%%%%%%%%%%%%%%%%%%%%%%%%%%%%%

\chapter{Appendix}
\label{ch:appendix}
\section{Abbreviation List}
\label{sec:abbreviationList}
This is a list of the abbreviations used in this thesis.

\begin{itemize}
	\item \% CD - How many percentage of the predictions in the correct direction  
	\item \% Rank - Percentage rank from the best prediction
	\item AD - Air Density
	\item AI - Artificial Intelligence
	\item ARIMA - Autoregressive Integrated Moving Average
	\item RPROP - Resilient Backpropagation
	\item BPNN - Backpropagation Neural Network
	\item CDD - Cooling Degree Days
	\item D - Demand
	\item DSS - Decision Support Systems
	\item H1 - Neurons in hidden layer 1
	\item H2 - Neurons in hidden layer 2
	\item HDD - Heating Degree Days
	\item L-P - Last known production
	\item m - Matrix
	\item MAE - Mean Average Error
	\item MAPE - Mean Absolute Percentage Error
	\item Mo - Month
	\item MPE - Mean Percentage Error
	\item PCC - Pearson Correlation Coefficient
	\item RMSE - Root Mean Square Error
	\item SVM - Support Vector Machine
	\item S - Skewness
	\item SC - Simple slope calculation
	\item Scatter/Sc - Historical productions as input
	\item T - Temperature
	\item ToD - Time of Day
	\item V - Historical volatility
	\item WD - Wind direction
	\item WS - Wind Speed
\end{itemize}
\newpage
\section{Weather Data}
\subsection{Stations}
\label{sec:weatherStations}
The station list was received together when the received data from The National Oceanic and Atmospheric Administration's (NOAA) National Climatic Data Center (NCDC)\footnote{\url{http://www.ncdc.noaa.gov/}}.
\newline 

\noindent   USAF-WBAN-ID STATION NAME                   COUNTRY                                            STATE 			      LATITUDE LONGITUDE ELEVATION
------------ ------------------------------ -------------------------------------------------- ------------------------------ -------- --------- ---------
060190 99999 SILSTRUP                       DENMARK                                                                             +55.933  +008.633   +0041.0\newline
060310 99999 TYLSTRUP                       DENMARK                                                                            +57.183  +009.950   +0009.0\newline
060340 99999 SINDAL                         DENMARK                                                                            +57.500  +010.217   +0028.0\newline
060410 99999 SKAGEN                         DENMARK                                                                            +57.733  +010.633   +0005.0\newline
060490 99999 HALD V                         DENMARK                                                                            +56.567  +010.100   +0089.0\newline
060520 99999 THYBOROEN                      DENMARK                                                                            +56.700  +008.217   +0004.0\newline
060580 99999 HVIDE SANDE                    DENMARK                                                                            +56.000  +008.150   +0005.0\newline
060600 99999 KARUP                          DENMARK                                                                            +56.300  +009.117   +0053.0\newline
060690 99999 FOULUM                         DENMARK                                                                            +56.500  +009.567   +0058.0\newline
060700 99999 AARHUS LUFTHAVN                DENMARK                                                                            +56.317  +010.633   +0023.0\newline
060740 99999 AARHUS SYD                     DENMARK                                                                            +56.083  +010.133   +0055.0\newline
060790 99999 ANHOLT                         DENMARK                                                                            +56.717  +011.517   +0004.0\newline
060800 99999 ESBJERG                        DENMARK                                                                            +55.533  +008.567   +0029.0\newline
060810 99999 BLAAVANDSHUK                   DENMARK                                                                            +55.550  +008.083   +0018.0\newline
060960 99999 ROEMOE/JUVRE                   DENMARK                                                                            +55.183  +008.567   +0009.0\newline
061040 99999 BILLUND                        DENMARK                                                                            +55.733  +009.167   +0080.0\newline
061080 99999 KOLDINGEGNENS LUFTHA           DENMARK                                                                            +55.433  +009.333   +0045.0\newline
061100 99999 SKRYDSTRUP                     DENMARK                                                                            +55.233  +009.267   +0047.0\newline
061180 99999 SOENDERBORG                    DENMARK                                                                            +54.967  +009.783   +0018.0\newline
061190 99999 KEGNAES                        DENMARK                                                                            +54.850  +009.983   +0017.0\newline
061200 99999 ODENSE/BELDRINGE               DENMARK                                                                            +55.483  +010.333   +0021.0\newline
061230 99999 TOROE                          DENMARK                                                                            +55.250  +009.883   +0003.0\newline
060300 99999 FLYVESTATION AALBOR            DENMARK                                                                            +57.100  +009.850   +0013.0\newline
\subsection{Format} 
\label{sec:weatherDataFormat}
The format description is taken directly from The National Oceanic and Atmospheric Administration's (NOAA) National Climatic Data Center (NCDC)\footnote{\url{http://www.ncdc.noaa.gov/}} when receiving the data. The format is in fixed length and the highlighted lines are used in this thesis. \newline 

\noindent SURFACE HOURLY ABBREVIATED FORMAT \newline

\noindent ONE HEADER RECORD FOLLOWED BY DATA RECORDS: \newline

\noindent COLUMN  DATA DESCRIPTION \newline

\noindent 01-06   USAF = AIR FORCE CATALOG STATION NUMBER \newline 
08-12   WBAN = NCDC WBAN NUMBER \newline
14-25   YR--MODAHRMN = YEAR-MONTH-DAY-HOUR-MINUTE IN GREENWICH MEAN TIME (GMT) \newline
\textbf{27-29   DIR = WIND DIRECTION IN COMPASS DEGREES, 990 = VARIABLE, REPORTED AS
        '***' WHEN AIR IS CALM (SPD WILL THEN BE 000)} \newline
\textbf{31-37   SPD \& GUS = WIND SPEED \& GUST IN MILES PER HOUR} \newline   
39-41   CLG = CLOUD CEILING--LOWEST OPAQUE LAYER
        WITH 5/8 OR GREATER COVERAGE, IN HUNDREDS OF FEET,
        722 = UNLIMITED \newline
43-45   SKC = SKY COVER -- CLR-CLEAR, SCT-SCATTERED-1/8 TO 4/8,
        BKN-BROKEN-5/8 TO 7/8, OVC-OVERCAST,
        OBS-OBSCURED, POB-PARTIAL OBSCURATION \newline  
47-47   L = LOW CLOUD TYPE, SEE BELOW\newline
49-49   M = MIDDLE CLOUD TYPE, SEE BELOW\newline
51-51   H = HIGH CLOUD TYPE, SEE BELOW  \newline
53-56   VSB = VISIBILITY IN STATUTE MILES TO NEAREST TENTH
        NOTE: FOR SOME STATIONS, VISIBILITY IS REPORTED ONLY UP TO A
        MAXIMUM OF 7 OR 10 MILES IN METAR OBSERVATIONS, BUT TO HIGHER
        VALUES IN SYNOPTIC OBSERVATIONS, WHICH CAUSES THE VALUES TO 
        FLUCTUATE FROM ONE DATA RECORD TO THE NEXT.  ALSO, VALUES
        ORIGINALLY REPORTED AS '10' MAY APPEAR AS '10.1' DUE TO DATA
        BEING ARCHIVED IN METRIC UNITS AND CONVERTED BACK TO ENGLISH.\newline
58-68   MW MW MW MW = MANUALLY OBSERVED PRESENT WEATHER--LISTED BELOW IN PRESENT WEATHER TABLE\newline
70-80   AW AW AW AW = AUTO-OBSERVED PRESENT WEATHER--LISTED BELOW IN PRESENT WEATHER TABLE\newline
82-82   W = PAST WEATHER INDICATOR, SEE BELOW\newline
\textbf{84-92   TEMP \& DEWP = TEMPERATURE \& DEW POINT IN FAHRENHEIT} \newline
\textbf{94-99   SLP = SEA LEVEL PRESSURE IN MILLIBARS TO NEAREST TENTH} \newline
101-105   ALT = ALTIMETER SETTING IN INCHES TO NEAREST HUNDREDTH \newline
\textbf{107-112   STP = STATION PRESSURE IN MILLIBARS TO NEAREST TENTH}\newline
114-116  MAX = MAXIMUM TEMPERATURE IN FAHRENHEIT (TIME PERIOD VARIES)\newline
118-120 MIN = MINIMUM TEMPERATURE IN FAHRENHEIT (TIME PERIOD VARIES)\newline
122-126 PCP01 = 1-HOUR LIQUID PRECIP REPORT IN INCHES AND HUNDREDTHS --
        THAT IS, THE PRECIP FOR THE PRECEDING 1 HOUR PERIOD\newline
128-132 PCP06 = 6-HOUR LIQUID PRECIP REPORT IN INCHES AND HUNDREDTHS --
        THAT IS, THE PRECIP FOR THE PRECEDING 6 HOUR PERIOD\newline
134-138 PCP24 = 24-HOUR LIQUID PRECIP REPORT IN INCHES AND HUNDREDTHS
        THAT IS, THE PRECIP FOR THE PRECEDING 24 HOUR PERIOD\newline
140-144 PCPXX = LIQUID PRECIP REPORT IN INCHES AND HUNDREDTHS, FOR
        A PERIOD OTHER THAN 1, 6, OR 24 HOURS (USUALLY FOR 12 HOUR PERIOD
        FOR STATIONS OUTSIDE THE U.S., AND FOR 3 HOUR PERIOD FOR THE U.S.)
        T = TRACE FOR ANY PRECIP FIELD\newline
146-147 SD = SNOW DEPTH IN INCHES \newline 
\section{Price experimental results}
\subsection{Price}
\label{sec:priceResultAppendix}
\subsection{Experiment One}
The figure is a 1000 hour graph is shown in \ref{fig:fullPageExperiment1} for the best prediction from Experiment One. The full graph can be seen in the digitalized version. In the directory "PriceGraphExperiment1".

\begin{sidewaysfigure}[h]
\centering
\includegraphics[width=\linewidth]{billeder/PriceGraphs/Experiment1.png}
\caption{First 1000 hours of the best prediction in experiment 1}
\label{fig:fullPageExperiment1}
\end{sidewaysfigure}

This includes last hours Price (P), The Demand (D), Wind Speed(WS), Temperature(T), The Hourly Time of Day (ToD), The Day of The Week(DoW), The Month of The Year(MoY), The Season of The Year(SoY),

\footnotesize
\begin{longtable}{|c|c|c|c|c|c|c|c|c|c|c|c|c|c|}
\caption{Input parameters test}\\
\hline
\textbf{\#} & \textbf{P} & \textbf{C} & \textbf{WS} & \textbf{T} & \textbf{ToD} & \textbf{DoW} & \textbf{MoY}& \textbf{SoY} & \textbf{Trend} & \textbf{H1} & \textbf{H2} & \textbf{MAE} & \textbf{\% Deviation} \\
\hline
\endfirsthead
\multicolumn{10}{c}%
{\tablename\ \thetable\ -- \textit{Continued from previous page}} \\
\hline
\textbf{\#} & \textbf{P} & \textbf{C} & \textbf{WS} & \textbf{T} & \textbf{ToD} & \textbf{DoW} & \textbf{MoY}& \textbf{SoY} & \textbf{Trend} & \textbf{H1} & \textbf{H2} & \textbf{MAE} & \textbf{\% Deviation} \\
\hline
\endhead
\hline \multicolumn{10}{r}{\textit{Continued on next page}} \\
\endfoot
\hline
\endlastfoot
1   &  \x    & \x    & \x    & \x    & \x\m  & \x\m  &       & \x\m  & 66,8\% &  7  & 8  & 57,12 & - \\ \hline
2   &  \x    & \x    & \x    & \x    & \x\m  & \x    &       & \x\m  & 67,9\% &  5  & 9  & 58,09 & 1,7\% \\ \hline
3   &  \x    & \x    & \x    & \x    & \x\m  &       & \x\m  &       & 66,5\% &  6  & 11 & 58,79 & 2,92\% \\ \hline
4   &  \x    & \x    & \x    &       & \x\m  & \x\m  & \x\m  &       & 66,5\% &  9  & 3  & 60,14 & 5,29\% \\ \hline
5   &  \x    & \x    & \x    & \x    & \x\m  & \x    &       &       & 66,9\% &  16 & 8  & 62,19 & 8,88\% \\ \hline
6   &  \x    & \x    & \x    & \x    & \x\m  &       &       & \x\m  & 67,2\% &  6  & 5  & 62,26 & 9,0\% \\ \hline
7   &  \x    & \x    & \x    & \x    & \x\m  & \x    & \x\m  &       & 67,0\% &  5  & 8  & 62,84 & 10,01\% \\ \hline
8   &  \x    & \x    & \x    & \x    & \x    & \x    &       & \x\m  & 62,9\% &  5  & 0  & 63,94 & 11,94\% \\ \hline
9   &  \x    & \x    & \x    & \x    & \x    & \x\m  & \x\m  &       & 62,2\% &  5  & 4  & 64,19 & 12,38\% \\ \hline
10  &  \x    & \x    & \x    &       & \x\m  & \x    & \x\m  &       & 67,4\% &  6  & 8  & 64,72 & 13,31\% \\ \hline
11  &  \x    & \x    & \x    &       & \x    & \x    & \x\m  &       & 63,3\% &  5  & 5  & 65,07 & 13,92\% \\ \hline
12  &  \x    & \x    & \x    & \x    & \x\m  & \x\m  &       &       & 67,0\% &  6  & 6  & 65,95 & 15,46\% \\ \hline
13  &  \x    & \x    & \x    & \x    & \x\m  & \x\m  & \x\m  &       & 66,8\% &  7  & 12 & 66,55 & 16,51\% \\ \hline
14  &  \x    & \x    & \x    &       & \x    &       &       & \x\m  & 62,4\% &  7  & 0  & 67,21 & 17,66\% \\ \hline
15  &  \x    & \x    & \x    & \x    & \x    & \x    &       &       & 62,6\% &  5  & 0  & 67,88 & 18,84\% \\ \hline
16  &  \x    & \x    & \x    &       & \x    & \x    &       & \x    & 62,0\% &  6  & 0  & 68,21 & 19,42\% \\ \hline
17  &  \x    & \x    & \x    &       & \x\m  & \x\m  &       &       & 67,2\% &  8  & 6  & 68,34 & 19,64\% \\ \hline
18  &  \x    & \x    & \x    &       & \x\m  &       &       & \x\m  & 68,2\% &  7  & 4  & 68,35 & 19,66\% \\ \hline
19  &  \x    & \x    & \x    & \x    & \x    & \x    & \x    &       & 62,2\% &  6  & 0  & 68,43 & 19,8\% \\ \hline
20  &  \x    & \x    & \x    & \x    & \x    &       &       & \x\m  & 62,4\% &  5  & 0  & 68,45 & 19,84\% \\ \hline
21  &  \x    & \x    & \x    &       & \x    &       &       &       & 63,0\% &  5  & 0  & 69,09 & 20,96\% \\ \hline
22  &  \x    & \x    & \x    & \x    &       & \x    &       &       & 62,0\% &  5  & 0  & 69,14 & 21,04\% \\ \hline
23  &  \x    & \x    & \x    &       & \x    &       & \x    &       & 62,2\% &  5  & 0  & 69,34 & 21,39\% \\ \hline
24  &  \x    & \x    & \x    & \x    & \x\m  & \x    & \x    &       & 66,7\% &  7  & 8  & 69,37 & 21,45\% \\ \hline
25  &  \x    & \x    & \x    & \x    & \x\m  &       &       & \x\m  & 67,5\% &  9  & 6  & 69,46 & 21,6\% \\ \hline
26  &  \x    & \x    & \x    & \x    & \x    & \x\m  & \x    &       & 61,5\% &  6  & 0  & 69,47 & 21,62\% \\ \hline
27  &  \x    & \x    & \x    & \x    &       & \x\m  &       &       & 60,7\% &  8  & 0  & 69,50 & 21,67\% \\ \hline
28  &  \x    & \x    & \x    & \x    & \x    &       &       &       & 62,5\% &  11 & 0  & 69,62 & 21,88\% \\ \hline
29  &  \x    & \x    & \x    & \x    &       & \x    & \x    &       & 61,4\% &  5  & 0  & 69,70 & 22,02\% \\ \hline
30  &  \x    & \x    & \x    & \x    & \x\m  &       &       & \x    & 66,6\% &  6  & 0  & 69,71 & 22,04\% \\ \hline
31  &  \x    & \x    & \x    &       & \x    &       &       & \x    & 62,1\% &  6  & 0  & 69,75 & 22,11\% \\ \hline
32  &  \x    & \x    & \x    & \x    & \x    &       & \x    &       & 62,7\% &  7  & 0  & 69,82 & 22,23\% \\ \hline
33  &  \x    & \x    & \x    &       &       &       &       &       & 58,9\% &  6  & 0  & 69,90 & 22,37\% \\ \hline
34  &  \x    & \x    & \x    & \x    &       & \x\m  & \x    &       & 61,3\% &  5  & 0  & 70,29 & 23,06\% \\ \hline
35  &  \x    & \x    & \x    &       & \x    & \x    & \x    &       & 63,0\% &  5  & 0  & 70,58 & 23,56\% \\ \hline
36  &  \x    & \x    & \x    & \x    & \x    & \x    &       & \x    & 61,8\% &  6  & 0  & 70,75 & 23,86\% \\ \hline
37  &  \x    & \x    & \x    &       & \x    & \x    &       & \x\m  & 62,4\% &  5  & 0  & 70,96 & 24,23\% \\ \hline
38  &  \x    & \x    & \x    & \x    &       &       &       &       & 61,3\% &  5  & 0  & 71,31 & 24,84\% \\ \hline
39  &  \x    & \x    &       & \x    & \x\m  & \x\m  &       &       & 67,5\% &  9  & 19 & 71,39 & 24,98\% \\ \hline
40  &  \x    & \x    & \x    &       & \x    & \x    &       &       & 62,2\% &  5  & 0  & 71,49 & 25,16\% \\ \hline
41  &  \x    & \x    & \x    & \x    &       &       &       &       & 60,6\% &  9  & 0  & 71,58 & 25,32\% \\ \hline
42  &  \x    & \x    & \x    & \x    &       & \x    &       & \x\m  & 61,5\% &  6  & 0  & 71,74 & 25,6\% \\ \hline
43  &  \x    & \x    & \x    &       & \x\m  &       & \x\m  &       & 66,7\% &  6  & 11 & 71,81 & 25,72\% \\ \hline
44  &  \x    & \x    & \x    &       & \x\m  & \x\m  &       & \x\m  & 66,8\% &  7  & 5  & 71,89 & 25,86\% \\ \hline
45  &  \x    & \x    & \x    & \x    & \x\m  & \x    &       & \x    & 66,4\% &  8  & 7  & 71,92 & 25,91\% \\ \hline
46  &  \x    & \x    & \x    &       & \x\m  &       &       & \x\m  & 67,2\% &  7  & 10 & 72,70 & 27,28\% \\ \hline
47  &  \x    & \x    & \x    &       & \x\m  & \x    &       & \x    & 66,4\% &  5  & 9  & 72,74 & 27,35\% \\ \hline
48  &  \x    & \x    & \x    & \x    & \x\m  &       &       &       & 66,7\% &  7  & 0  & 73,70 & 29,03\% \\ \hline
49  &  \x    & \x    & \x    & \x    &       & \x    & \x\m  &       & 62,1\% &  6  & 0  & 73,80 & 29,2\% \\ \hline
50  &  \x    & \x    & \x    & \x    &       & \x\m  & \x\m  &       & 61,2\% &  8  & 0  & 74,19 & 29,88\% \\ \hline
51  &  \x    & \x    & \x    & \x    & \x    &       & \x\m  &       & 62,9\% &  10 & 0  & 74,29 & 30,06\% \\ \hline
52  &  \x    & \x    & \x    &       & \x\m  & \x    &       &       & 67,5\% &  8  & 0  & 74,56 & 30,53\% \\ \hline
53  &  \x    & \x    & \x    &       & \x    & \x\m  & \x\m  &       & 62,5\% &  5  & 7  & 74,89 & 31,11\% \\ \hline
54  &  \x    & \x    & \x    &       & \x\m  & \x    &       & \x\m  & 68,0\% &  5  & 3  & 75,25 & 31,74\% \\ \hline
55  &  \x    & \x    & \x    & \x    &       & \x\m  &       & \x\m  & 60,7\% &  7  & 0  & 75,31 & 31,85\% \\ \hline
56  &  \x    & \x    & \x    & \x    & \x    & \x    & \x\m  &       & 62,1\% &  7  & 0  & 75,40 & 32,0\% \\ \hline
57  &  \x    & \x    & \x    & \x    & \x    & \x\m  &       &       & 62,5\% &  5  & 0  & 75,45 & 32,09\% \\ \hline
58  &  \x    & \x    & \x    &       &       &       &       & \x\m  & 59,3\% &  12 & 0  & 75,52 & 32,21\% \\ \hline
59  &  \x    & \x    & \x    &       &       & \x    &       &       & 59,2\% &  5  & 0  & 75,63 & 32,41\% \\ \hline
60  &  \x    & \x    & \x    &       & \x\m  &       &       &       & 67,4\% &  5  & 11 & 75,77 & 32,65\% \\ \hline
61  &  \x    & \x    & \x    & \x    &       & \x\m  &       & \x\m  & 61,6\% &  8  & 0  & 76,12 & 33,26\% \\ \hline
62  &  \x    & \x    & \x    & \x    &       &       & \x\m  &       & 62,5\% &  8  & 0  & 76,18 & 33,37\% \\ \hline
63  &  \x    & \x    & \x    & \x    & \x    & \x\m  &       & \x\m  & 62,2\% &  6  & 0  & 76,19 & 33,39\% \\ \hline
64  &  \x    & \x    & \x    &       &       &       &       &       & 59,7\% &  10 & 0  & 76,32 & 33,61\% \\ \hline
65  &  \x    & \x    & \x    &       & \x    &       & \x\m  &       & 63,0\% &  8  & 0  & 76,39 & 33,74\% \\ \hline
66  &  \x    & \x    & \x    & \x    & \x\m  &       & \x\m  &       & 66,3\% &  8  & 7  & 77,70 & 36,03\% \\ \hline
67  &  \x    & \x    & \x    &       & \x\m  & \x    & \x    &       & 66,9\% &  5  & 0  & 77,72 & 36,06\% \\ \hline
68  &  \x    & \x    & \x    & \x    & \x\m  &       & \x    &       & 67,1\% &  5  & 0  & 78,05 & 36,64\% \\ \hline
69  &  \x    & \x    & \x    & \x    &       & \x\m  &       & \x    & 61,1\% &  9  & 0  & 78,49 & 37,41\% \\ \hline
70  &  \x    & \x    &       & \x    & \x\m  & \x\m  &       & \x\m  & 66,9\% &  12 & 10 & 78,57 & 37,55\% \\ \hline
71  &  \x    & \x    & \x    & \x    & \x    &       &       & \x    & 61,8\% &  7  & 0  & 78,69 & 37,76\% \\ \hline
72  &  \x    & \x    & \x    &       & \x    & \x\m  &       & \x\m  & 61,8\% &  6  & 0  & 78,74 & 37,85\% \\ \hline
73  &  \x    & \x    & \x    &       & \x    & \x\m  &       &       & 62,2\% &  7  & 0  & 78,88 & 38,1\% \\ \hline
74  &  \x    & \x    &       & \x    & \x    & \x    &       &       & 58,2\% &  5  & 0  & 79,01 & 38,32\% \\ \hline
75  &  \x    & \x    &       &       & \x\m  & \x\m  & \x\m  &       & 65,8\% &  16 & 6  & 79,71 & 39,55\% \\ \hline
76  &  \x    & \x    & \x    & \x    &       &       & \x    &       & 62,1\% &  6  & 0  & 79,89 & 39,86\% \\ \hline
77  &  \x    & \x    & \x    &       &       & \x\m  &       & \x\m  & 59,3\% &  8  & 0  & 80,20 & 40,41\% \\ \hline
78  &  \x    & \x    & \x    & \x    & \x    & \x\m  &       & \x    & 62,7\% &  5  & 7  & 81,11 & 42,0\% \\ \hline
79  &  \x    & \x    &       &       &       &       &       &       & 53,1\% &  13 & 0  & 81,28 & 42,3\% \\ \hline
80  &  \x    & \x    & \x    & \x    &       & \x    &       & \x    & 61,1\% &  6  & 0  & 81,34 & 42,4\% \\ \hline
81  &  \x    & \x    &       &       & \x\m  & \x\m  &       & \x\m  & 66,4\% &  15 & 9  & 81,58 & 42,82\% \\ \hline
82  &  \x    & \x    & \x    & \x    &       &       &       & \x\m  & 61,2\% &  9  & 0  & 82,03 & 43,61\% \\ \hline
83  &  \x    & \x    & \x    &       & \x    & \x\m  & \x    &       & 62,4\% &  5  & 0  & 82,04 & 43,63\% \\ \hline
84  &  \x    & \x    &       & \x    & \x    &       &       &       & 59,0\% &  11 & 0  & 82,29 & 44,07\% \\ \hline
85  &  \x    & \x    & \x    &       &       & \x\m  &       &       & 58,7\% &  8  & 0  & 82,32 & 44,12\% \\ \hline
86  &  \x    & \x    &       & \x    &       &       &       &       & 53,9\% &  8  & 0  & 82,72 & 44,82\% \\ \hline
87  &  \x    & \x    &       &       & \x    &       &       &       & 59,6\% &  8  & 0  & 83,35 & 45,92\% \\ \hline
88  &  \x    & \x    & \x    &       & \x\m  &       & \x    &       & 66,6\% &  7  & 0  & 83,82 & 46,74\% \\ \hline
89  &  \x    & \x    & \x    &       & \x    & \x\m  &       & \x    & 62,0\% &  6  & 0  & 84,08 & 47,2\% \\ \hline
90  &  \x    & \x    &       & \x    & \x\m  & \x\m  & \x\m  &       & 65,4\% &  9  & 12 & 84,50 & 47,93\% \\ \hline
91  &  \x    & \x    &       &       & \x\m  & \x\m  &       &       & 67,0\% &  11 & 9  & 85,15 & 49,07\% \\ \hline
92  &  \x    & \x    &       & \x    & \x\m  &       &       & \x\m  & 66,5\% &  11 & 12 & 85,51 & 49,7\% \\ \hline
93  &  \x    & \x    & \x    &       &       &       & \x    &       & 60,7\% &  9  & 0  & 85,55 & 49,77\% \\ \hline
94  &  \x    & \x    & \x    &       &       & \x    &       & \x    & 59,4\% &  6  & 0  & 85,57 & 49,81\% \\ \hline
95  &  \x    & \x    & \x    &       &       &       &       & \x    & 59,7\% &  6  & 0  & 85,66 & 49,96\% \\ \hline
96  &  \x    & \x    & \x    &       & \x\m  &       & \x\m  &       & 66,1\% &  10 & 6  & 85,77 & 50,16\% \\ \hline
97  &  \x    & \x    & \x    &       &       & \x    & \x    &       & 59,9\% &  5  & 0  & 85,99 & 50,54\% \\ \hline
98  &  \x    & \x    &       &       & \x    & \x    &       &       & 58,0\% &  5  & 0  & 86,00 & 50,56\% \\ \hline
99  &  \x    & \x    & \x    &       & \x\m  &       &       & \x    & 66,4\% &  7  & 0  & 86,13 & 50,79\% \\ \hline
100 &  \x    & \x    &       & \x    &       & \x    &       &       & 53,5\% &  5  & 0  & 86,33 & 51,14\% \\ \hline
101 &  \x    & \x    &       &       & \x\m  &       &       & \x\m  & 66,9\% &  15 & 4  & 87,00 & 52,31\% \\ \hline
102 &  \x    & \x    & \x    &       &       & \x    &       & \x\m  & 59,0\% &  7  & 0  & 87,63 & 53,41\% \\ \hline
103 &  \x    & \x    & \x    & \x    &       & \x\m  & \x\m  &       & 61,6\% &  7  & 0  & 87,87 & 53,83\% \\ \hline
104 &  \x    & \x    &       &       &       & \x    &       &       & 51,7\% &  5  & 3  & 87,89 & 53,87\% \\ \hline
105 &  \x    & \x    &       & \x    &       &       &       &       & 53,6\% &  6  & 0  & 87,94 & 53,96\% \\ \hline
106 &  \x    & \x    &       &       &       &       &       &       & 52,6\% &  12 & 0  & 89,03 & 55,86\% \\ \hline
107 &  \x    & \x    & \x    &       &       & \x    & \x\m  &       & 60,2\% &  6  & 0  & 89,14 & 56,06\% \\ \hline
108 &  \x    & \x    & \x    & \x    &       &       &       & \x    & 60,9\% &  7  & 0  & 89,55 & 56,78\% \\ \hline
109 &  \x    & \x    &       &       & \x    & \x    & \x    &       & 57,1\% &  6  & 0  & 90,73 & 58,84\% \\ \hline
110 &  \x    & \x    &       & \x    & \x\m  &       &       &       & 65,7\% &  11 & 0  & 91,06 & 59,42\% \\ \hline
111 &  \x    & \x    &       &       & \x\m  &       &       &       & 67,4\% &  11 & 0  & 91,23 & 59,72\% \\ \hline
112 &  \x    & \x    &       &       &       &       &       & \x\m  & 52,6\% &  13 & 0  & 91,67 & 60,49\% \\ \hline
113 &  \x    & \x    & \x    &       &       & \x\m  &       & \x\m  & 59,2\% &  7  & 0  & 91,91 & 60,91\% \\ \hline
114 &  \x    & \x    & \x    &       &       & \x\m  & \x\m  &       & 59,6\% &  6  & 0  & 92,65 & 62,2\% \\ \hline
115 &  \x    & \x    &       & \x    & \x    & \x    &       & \x    & 56,5\% &  5  & 0  & 92,95 & 62,73\% \\ \hline
116 &  \x    & \x    & \x    &       &       &       & \x\m  &       & 60,2\% &  9  & 0  & 93,63 & 63,92\% \\ \hline
117 &  \x    & \x    & \x    &       &       & \x\m  & \x    &       & 59,3\% &  9  & 0  & 94,58 & 65,58\% \\ \hline
118 &  \x    & \x    &       &       &       &       &       & \x    & 52,3\% &  6  & 0  & 96,70 & 69,29\% \\ \hline
119 &  \x    & \x    &       &       &       & \x\m  &       &       & 50,0\% &  7  & 0  & 96,78 & 69,43\% \\ \hline
120 &  \x    & \x    &       &       & \x    &       &       & \x    & 57,2\% &  6  & 0  & 97,09 & 69,98\% \\ \hline
121 &  \x    & \x    &       & \x    & \x    & \x    & \x    &       & 56,7\% &  5  & 0  & 98,17 & 71,87\% \\ \hline
122 &  \x    & \x    &       &       & \x    & \x    &       & \x    & 57,1\% &  6  & 0  & 99,10 & 73,49\% \\ \hline
123 &  \x    & \x    &       &       &       &       & \x    &       & 52,6\% &  8  & 0  & 99,38 & 73,98\% \\ \hline
124 &  \x    & \x    &       &       & \x    &       & \x    &       & 56,8\% &  6  & 0  & 99,78 & 74,68\% \\ \hline
125 &  \x    & \x    & \x    &       &       & \x\m  &       & \x    & 59,0\% &  5  & 0  & 100,63 & 76,17\% \\ \hline
126 &  \x    & \x    &       & \x    &       &       &       & \x\m  & 53,1\% &  8  & 0  & 102,76 & 79,9\% \\ \hline
127 &  \x    & \x    &       & \x    & \x\m  &       & \x\m  &       & 64,9\% &  13 & 16 & 103,01 & 80,34\% \\ \hline
128 &  \x    & \x    &       &       &       & \x    & \x    &       & 51,2\% &  6  & 0  & 103,56 & 81,3\% \\ \hline
129 &  \x    & \x    &       &       &       & \x    &       & \x    & 52,3\% &  5  & 0  & 103,71 & 81,57\% \\ \hline
130 &  \x    & \x    &       & \x    &       & \x    &       & \x    & 52,2\% &  5  & 0  & 104,19 & 82,41\% \\ \hline
131 &  \x    & \x    &       & \x    &       & \x\m  &       &       & 53,0\% &  9  & 0  & 104,33 & 82,65\% \\ \hline
132 &  \x    & \x    &       &       & \x\m  &       & \x\m  &       & 66,0\% &  12 & 14 & 104,93 & 83,7\% \\ \hline
133 &  \x    & \x    &       & \x    & \x    &       &       & \x    & 56,3\% &  5  & 0  & 105,04 & 83,89\% \\ \hline
134 &  \x    & \x    & \x    &       &       & \x\m  & \x\m  &       & 59,8\% &  5  & 0  & 105,70 & 85,05\% \\ \hline
135 &  \x    & \x    &       &       &       & \x\m  &       & \x\m  & 51,4\% &  7  & 0  & 107,85 & 88,81\% \\ \hline
136 &  \x    & \x    &       & \x    & \x    &       & \x    &       & 56,5\% &  5  & 0  & 108,37 & 89,72\% \\ \hline
137 &  \x    & \x    &       & \x    &       &       & \x    &       & 54,4\% &  7  & 0  & 111,14 & 94,57\% \\ \hline
138 &  \x    & \x    &       & \x    &       &       & \x\m  &       & 53,5\% &  8  & 0  & 111,65 & 95,47\% \\ \hline
139 &  \x    & \x    &       & \x    &       & \x\m  & \x\m  &       & 52,9\% &  7  & 0  & 113,56 & 98,81\% \\ \hline
140 &  \x    & \x    &       & \x    &       &       &       & \x    & 52,7\% &  8  & 0  & 115,29 & 101,84\% \\ \hline
141 &  \x    & \x    &       &       &       &       & \x\m  &       & 53,0\% &  11 & 0  & 115,81 & 102,75\% \\ \hline
142 &  \x    & \x    &       & \x    &       & \x\m  &       & \x\m  & 53,0\% &  5  & 0  & 115,83 & 102,78\% \\ \hline
143 &  \x    & \x    &       & \x    &       & \x    & \x    &       & 53,2\% &  6  & 0  & 117,17 & 105,13\% \\ \hline
144 &  \x    & \x    &       &       &       & \x\m  & \x\m  &       & 52,0\% &  7  & 0  & 117,34 & 105,43\% \\ \hline
\end{longtable}
\normalsize

\footnotesize
\begin{longtable}{|c|c|c|c|c|c|c|c|c|c|c|c|c|c|}
\caption{Input parameters test}\\
\hline
\textbf{\#} & \textbf{P} & \textbf{C} & \textbf{WS} & \textbf{T} & \textbf{ToD} & \textbf{DoW} & \textbf{MoY}& \textbf{SoY} & \textbf{Trend} & \textbf{H1} & \textbf{H2} & \textbf{MAE} & \textbf{\% Deviation} \\
\hline
\endfirsthead
\multicolumn{10}{c}%
{\tablename\ \thetable\ -- \textit{Continued from previous page}} \\
\hline
\textbf{\#} & \textbf{P} & \textbf{C} & \textbf{WS} & \textbf{T} & \textbf{ToD} & \textbf{DoW} & \textbf{MoY}& \textbf{SoY} & \textbf{Trend} & \textbf{H1} & \textbf{H2} & \textbf{MAE} & \textbf{\% Deviation} \\
\hline
\endhead
\hline \multicolumn{10}{r}{\textit{Continued on next page}} \\
\endfoot
\hline
\endlastfoot
1  &  \x    & \x    & \x    & \x    & \x\m  & \x\m  &       &       & 66,9\% &  6  & 5  & 58,19 & - \\ \hline
2  &  \x    & \x    & \x    & \x    & \x\m  & \x\m  &       & \x\m  & 66,9\% &  6  & 0  & 61,70 & 6,03\% \\ \hline
3  &  \x    & \x    & \x    & \x    & \x\m  &       &       & \x\m  & 67,7\% &  9  & 0  & 61,86 & 6,31\% \\ \hline
4  &  \x    & \x    & \x    & \x    & \x\m  & \x    & \x\m  &       & 65,8\% &  5  & 0  & 62,62 & 7,61\% \\ \hline
5  &  \x    & \x    & \x    & \x    & \x    &       &       &       & 60,2\% &  5  & 0  & 62,98 & 8,23\% \\ \hline
6  &  \x    & \x    & \x    & \x    & \x\m  &       & \x\m  &       & 66,6\% &  10 & 0  & 63,80 & 9,64\% \\ \hline
7  &  \x    & \x    & \x    &       & \x\m  &       &       & \x\m  & 66,6\% &  6  & 0  & 64,90 & 11,53\% \\ \hline
8  &  \x    & \x    & \x    &       & \x\m  & \x    &       & \x\m  & 66,4\% &  6  & 0  & 65,32 & 12,25\% \\ \hline
9  &  \x    & \x    & \x    & \x    & \x\m  & \x    & \x    &       & 65,9\% &  5  & 0  & 66,42 & 14,14\% \\ \hline
10 &  \x    & \x    & \x    &       & \x    &       &       &       & 61,0\% &  5  & 2  & 66,48 & 14,25\% \\ \hline
11 &  \x    & \x    & \x    &       & \x\m  & \x\m  &       & \x\m  & 66,7\% &  5  & 2  & 67,06 & 15,24\% \\ \hline
12 &  \x    & \x    & \x    & \x    & \x\m  & \x\m  & \x\m  &       & 65,5\% &  7  & 11 & 67,31 & 15,67\% \\ \hline
13 &  \x    & \x    & \x    &       & \x\m  & \x    &       &       & 67,7\% &  6  & 0  & 67,60 & 16,17\% \\ \hline
14 &  \x    & \x    & \x    & \x    & \x\m  &       & \x\m  &       & 65,2\% &  5  & 17 & 67,76 & 16,45\% \\ \hline
15 &  \x    & \x    & \x    &       & \x\m  & \x\m  & \x\m  &       & 65,6\% &  10 & 4  & 67,82 & 16,55\% \\ \hline
16 &  \x    & \x    & \x    &       & \x\m  & \x    & \x    &       & 67,1\% &  5  & 0  & 68,13 & 17,08\% \\ \hline
17 &  \x    & \x    & \x    & \x    & \x\m  & \x    &       & \x\m  & 66,4\% &  5  & 8  & 68,23 & 17,25\% \\ \hline
18 &  \x    & \x    & \x    & \x    & \x\m  &       &       & \x    & 66,4\% &  7  & 6  & 68,41 & 17,56\% \\ \hline
19 &  \x    & \x    & \x    &       & \x    & \x    & \x    &       & 59,7\% &  5  & 0  & 68,43 & 17,6\% \\ \hline
20 &  \x    & \x    &       & \x    & \x\m  & \x\m  &       & \x\m  & 66,2\% &  6  & 0  & 68,63 & 17,94\% \\ \hline
21 &  \x    & \x    & \x    & \x    & \x\m  &       &       & \x\m  & 66,0\% &  5  & 0  & 68,83 & 18,28\% \\ \hline
22 &  \x    & \x    & \x    &       & \x\m  &       & \x    &       & 66,6\% &  6  & 14 & 69,19 & 18,9\% \\ \hline
23 &  \x    & \x    & \x    & \x    & \x\m  & \x    &       & \x    & 67,0\% &  6  & 23 & 69,31 & 19,11\% \\ \hline
24 &  \x    & \x    & \x    &       & \x\m  &       & \x\m  &       & 66,5\% &  6  & 6  & 69,89 & 20,11\% \\ \hline
25 &  \x    & \x    & \x    & \x    & \x    & \x    &       & \x    & 59,0\% &  5  & 0  & 70,28 & 20,78\% \\ \hline
26 &  \x    & \x    & \x    &       & \x\m  &       & \x\m  &       & 66,1\% &  8  & 0  & 70,53 & 21,21\% \\ \hline
27 &  \x    & \x    & \x    & \x    & \x    & \x\m  &       & \x    & 59,5\% &  5  & 0  & 70,60 & 21,33\% \\ \hline
28 &  \x    & \x    & \x    &       & \x    & \x    &       &       & 59,1\% &  5  & 0  & 70,98 & 21,98\% \\ \hline
29 &  \x    & \x    & \x    &       & \x\m  & \x    &       & \x    & 66,0\% &  5  & 0  & 71,30 & 22,53\% \\ \hline
30 &  \x    & \x    & \x    &       & \x    & \x\m  &       &       & 60,2\% &  5  & 3  & 71,44 & 22,77\% \\ \hline
31 &  \x    & \x    & \x    &       &       &       &       &       & 56,6\% &  6  & 0  & 71,96 & 23,66\% \\ \hline
32 &  \x    & \x    & \x    & \x    & \x    &       &       & \x    & 59,6\% &  5  & 0  & 72,02 & 23,77\% \\ \hline
33 &  \x    & \x    & \x    & \x    & \x\m  &       & \x    &       & 66,1\% &  5  & 0  & 72,24 & 24,15\% \\ \hline
34 &  \x    & \x    & \x    &       & \x\m  & \x    & \x\m  &       & 66,3\% &  8  & 0  & 72,29 & 24,23\% \\ \hline
35 &  \x    & \x    & \x    &       & \x\m  & \x\m  &       &       & 66,9\% &  5  & 0  & 72,31 & 24,27\% \\ \hline
36 &  \x    & \x    & \x    & \x    & \x    & \x    &       & \x\m  & 58,8\% &  5  & 0  & 72,37 & 24,37\% \\ \hline
37 &  \x    & \x    & \x    &       & \x    & \x    &       & \x    & 58,7\% &  5  & 0  & 72,45 & 24,51\% \\ \hline
38 &  \x    & \x    & \x    & \x    & \x    & \x\m  &       & \x\m  & 58,9\% &  7  & 0  & 72,91 & 25,3\% \\ \hline
39 &  \x    & \x    &       &       & \x\m  & \x\m  & \x\m  &       & 66,3\% &  8  & 0  & 73,10 & 25,62\% \\ \hline
40 &  \x    & \x    & \x    & \x    & \x\m  &       &       &       & 66,6\% &  8  & 0  & 73,35 & 26,05\% \\ \hline
41 &  \x    & \x    & \x    &       & \x    &       &       & \x    & 58,7\% &  7  & 0  & 73,52 & 26,34\% \\ \hline
42 &  \x    & \x    & \x    &       & \x    & \x\m  &       & \x\m  & 58,9\% &  7  & 0  & 73,57 & 26,43\% \\ \hline
43 &  \x    & \x    & \x    &       &       & \x    &       & \x\m  & 55,4\% &  8  & 0  & 74,26 & 27,62\% \\ \hline
44 &  \x    & \x    &       &       & \x\m  & \x\m  &       &       & 67,1\% &  10 & 0  & 74,48 & 27,99\% \\ \hline
45 &  \x    & \x    & \x    &       & \x    &       &       & \x\m  & 59,9\% &  8  & 0  & 74,67 & 28,32\% \\ \hline
46 &  \x    & \x    & \x    & \x    & \x\m  & \x    &       &       & 67,0\% &  5  & 0  & 74,89 & 28,7\% \\ \hline
47 &  \x    & \x    & \x    &       & \x    &       & \x    &       & 58,6\% &  5  & 0  & 74,96 & 28,82\% \\ \hline
48 &  \x    & \x    & \x    & \x    & \x    & \x    & \x    &       & 59,4\% &  5  & 0  & 75,10 & 29,06\% \\ \hline
49 &  \x    & \x    &       & \x    & \x    &       &       & \x    & 56,4\% &  5  & 0  & 75,31 & 29,42\% \\ \hline
50 &  \x    & \x    & \x    & \x    & \x    & \x    & \x\m  &       & 59,7\% &  5  & 0  & 75,70 & 30,09\% \\ \hline
51 &  \x    & \x    & \x    & \x    & \x    &       &       & \x\m  & 59,3\% &  5  & 4  & 75,97 & 30,56\% \\ \hline
52 &  \x    & \x    &       & \x    & \x\m  &       & \x\m  &       & 65,9\% &  6  & 0  & 76,08 & 30,74\% \\ \hline
53 &  \x    & \x    &       & \x    & \x\m  & \x\m  & \x\m  &       & 65,6\% &  7  & 0  & 76,22 & 30,98\% \\ \hline
54 &  \x    & \x    & \x    &       & \x    & \x\m  & \x    &       & 58,9\% &  7  & 4  & 76,24 & 31,02\% \\ \hline
55 &  \x    & \x    & \x    &       & \x\m  &       &       & \x\m  & 66,2\% &  8  & 0  & 76,25 & 31,04\% \\ \hline
56 &  \x    & \x    &       &       & \x\m  &       &       & \x\m  & 67,2\% &  6  & 9  & 76,30 & 31,12\% \\ \hline
57 &  \x    & \x    &       &       & \x\m  & \x\m  &       & \x\m  & 66,9\% &  8  & 0  & 76,70 & 31,81\% \\ \hline
58 &  \x    & \x    & \x    & \x    &       &       &       &       & 58,8\% &  8  & 0  & 76,74 & 31,88\% \\ \hline
59 &  \x    & \x    &       &       & \x\m  &       &       &       & 67,5\% &  5  & 23 & 76,79 & 31,96\% \\ \hline
60 &  \x    & \x    &       & \x    & \x\m  & \x\m  &       &       & 65,8\% &  5  & 0  & 76,85 & 32,07\% \\ \hline
61 &  \x    & \x    &       & \x    & \x\m  &       &       & \x\m  & 67,2\% &  5  & 0  & 77,05 & 32,41\% \\ \hline
62 &  \x    & \x    & \x    & \x    &       & \x    & \x    &       & 57,5\% &  5  & 0  & 77,20 & 32,67\% \\ \hline
63 &  \x    & \x    & \x    & \x    & \x    & \x    &       &       & 59,3\% &  5  & 0  & 77,29 & 32,82\% \\ \hline
64 &  \x    & \x    & \x    &       &       & \x\m  & \x    &       & 57,1\% &  5  & 0  & 77,32 & 32,88\% \\ \hline
65 &  \x    & \x    & \x    &       & \x    & \x    &       & \x\m  & 59,3\% &  5  & 0  & 77,46 & 33,12\% \\ \hline
66 &  \x    & \x    & \x    & \x    &       &       &       &       & 57,3\% &  8  & 0  & 77,66 & 33,46\% \\ \hline
67 &  \x    & \x    &       &       & \x    &       &       &       & 59,7\% &  9  & 7  & 77,69 & 33,51\% \\ \hline
68 &  \x    & \x    & \x    &       & \x    & \x\m  & \x\m  &       & 60,0\% &  5  & 0  & 77,73 & 33,58\% \\ \hline
69 &  \x    & \x    &       & \x    & \x\m  &       &       &       & 67,1\% &  7  & 0  & 77,79 & 33,68\% \\ \hline
70 &  \x    & \x    & \x    & \x    & \x    & \x\m  & \x    &       & 59,0\% &  5  & 0  & 77,85 & 33,79\% \\ \hline
71 &  \x    & \x    & \x    & \x    &       & \x    &       &       & 56,8\% &  6  & 0  & 78,04 & 34,11\% \\ \hline
72 &  \x    & \x    &       &       & \x\m  &       & \x\m  &       & 65,9\% &  10 & 0  & 78,13 & 34,27\% \\ \hline
73 &  \x    & \x    & \x    &       & \x    & \x\m  &       & \x    & 59,3\% &  5  & 0  & 78,18 & 34,35\% \\ \hline
74 &  \x    & \x    & \x    & \x    &       & \x\m  & \x\m  &       & 58,8\% &  5  & 0  & 78,33 & 34,61\% \\ \hline
75 &  \x    & \x    & \x    & \x    &       &       &       & \x\m  & 57,0\% &  7  & 0  & 79,14 & 36,0\% \\ \hline
76 &  \x    & \x    & \x    &       &       &       &       &       & 55,6\% &  6  & 0  & 79,29 & 36,26\% \\ \hline
77 &  \x    & \x    & \x    &       &       & \x\m  &       & \x\m  & 55,8\% &  6  & 3  & 79,48 & 36,59\% \\ \hline
78 &  \x    & \x    & \x    & \x    &       & \x\m  &       & \x\m  & 57,5\% &  5  & 9  & 79,55 & 36,71\% \\ \hline
79 &  \x    & \x    & \x    &       &       &       &       & \x\m  & 56,6\% &  8  & 0  & 79,64 & 36,86\% \\ \hline
80 &  \x    & \x    &       &       & \x    & \x    &       &       & 58,2\% &  5  & 0  & 79,66 & 36,9\% \\ \hline
81 &  \x    & \x    & \x    &       & \x    & \x    & \x\m  &       & 59,2\% &  6  & 0  & 79,71 & 36,98\% \\ \hline
82 &  \x    & \x    & \x    & \x    &       &       & \x    &       & 57,4\% &  6  & 0  & 80,26 & 37,93\% \\ \hline
83 &  \x    & \x    & \x    &       &       &       & \x    &       & 56,7\% &  5  & 1  & 80,36 & 38,1\% \\ \hline
84 &  \x    & \x    & \x    &       & \x\m  &       &       & \x    & 66,4\% &  5  & 1  & 80,56 & 38,44\% \\ \hline
85 &  \x    & \x    & \x    & \x    &       &       &       & \x    & 58,1\% &  8  & 0  & 81,11 & 39,39\% \\ \hline
86 &  \x    & \x    & \x    &       &       & \x    & \x    &       & 56,0\% &  7  & 0  & 81,45 & 39,97\% \\ \hline
87 &  \x    & \x    & \x    &       &       & \x\m  &       & \x\m  & 55,4\% &  6  & 0  & 81,97 & 40,87\% \\ \hline
88 &  \x    & \x    & \x    & \x    &       & \x\m  &       & \x\m  & 57,1\% &  9  & 0  & 81,98 & 40,88\% \\ \hline
89 &  \x    & \x    & \x    &       & \x\m  &       &       &       & 67,2\% &  9  & 0  & 82,05 & 41,0\% \\ \hline
90 &  \x    & \x    &       &       &       &       &       &       & 51,5\% &  6  & 0  & 82,21 & 41,28\% \\ \hline
91 &  \x    & \x    & \x    & \x    &       & \x    &       & \x    & 57,5\% &  5  & 0  & 82,66 & 42,05\% \\ \hline
92 &  \x    & \x    &       & \x    & \x    & \x    &       &       & 56,9\% &  8  & 0  & 82,70 & 42,12\% \\ \hline
93 &  \x    & \x    & \x    & \x    &       & \x\m  & \x\m  &       & 57,9\% &  7  & 0  & 82,85 & 42,38\% \\ \hline
94 &  \x    & \x    &       &       & \x    & \x    &       & \x    & 55,7\% &  5  & 0  & 82,88 & 42,43\% \\ \hline
95 &  \x    & \x    &       &       & \x    &       &       & \x    & 57,7\% &  5  & 0  & 83,00 & 42,64\% \\ \hline
96 &  \x    & \x    & \x    &       &       & \x    & \x\m  &       & 57,6\% &  6  & 0  & 83,59 & 43,65\% \\ \hline
97 &  \x    & \x    & \x    & \x    & \x    &       & \x\m  &       & 59,1\% &  5  & 0  & 83,76 & 43,94\% \\ \hline
98 &  \x    & \x    &       &       & \x    & \x    & \x    &       & 57,4\% &  10 & 1  & 83,97 & 44,3\% \\ \hline
99 &  \x    & \x    &       & \x    &       & \x    &       &       & 52,4\% &  11 & 0  & 83,98 & 44,32\% \\ \hline
100 &  \x    & \x    & \x    &       &       &       & \x\m  &       & 56,9\% &  10 & 0  & 83,99 & 44,34\% \\ \hline
101 &  \x    & \x    & \x    & \x    & \x    & \x\m  & \x\m  &       & 59,2\% &  8  & 0  & 84,12 & 44,56\% \\ \hline
102 &  \x    & \x    & \x    &       &       & \x    &       & \x    & 55,1\% &  9  & 0  & 84,22 & 44,73\% \\ \hline
103 &  \x    & \x    & \x    & \x    &       & \x    &       & \x\m  & 56,2\% &  8  & 0  & 84,23 & 44,75\% \\ \hline
104 &  \x    & \x    & \x    &       &       & \x    &       &       & 54,2\% &  7  & 0  & 84,31 & 44,89\% \\ \hline
105 &  \x    & \x    & \x    & \x    &       & \x\m  & \x    &       & 57,1\% &  10 & 0  & 85,00 & 46,07\% \\ \hline
106 &  \x    & \x    &       & \x    & \x    &       &       &       & 58,2\% &  12 & 0  & 85,11 & 46,26\% \\ \hline
107 &  \x    & \x    & \x    & \x    &       & \x\m  &       &       & 57,1\% &  8  & 0  & 85,13 & 46,3\% \\ \hline
108 &  \x    & \x    & \x    & \x    &       & \x    & \x\m  &       & 58,5\% &  7  & 0  & 85,22 & 46,45\% \\ \hline
109 &  \x    & \x    &       & \x    & \x    &       & \x    &       & 57,6\% &  5  & 0  & 85,30 & 46,59\% \\ \hline
110 &  \x    & \x    & \x    &       &       & \x\m  & \x\m  &       & 56,8\% &  6  & 9  & 85,35 & 46,67\% \\ \hline
111 &  \x    & \x    & \x    &       &       & \x\m  &       & \x    & 55,9\% &  7  & 0  & 85,99 & 47,77\% \\ \hline
112 &  \x    & \x    &       &       &       & \x\m  &       & \x\m  & 50,2\% &  7  & 0  & 86,42 & 48,51\% \\ \hline
113 &  \x    & \x    & \x    &       &       &       &       & \x    & 56,0\% &  7  & 0  & 86,64 & 48,89\% \\ \hline
114 &  \x    & \x    & \x    & \x    &       &       & \x\m  &       & 58,5\% &  7  & 0  & 87,26 & 49,96\% \\ \hline
115 &  \x    & \x    &       &       &       &       & \x    &       & 51,8\% &  7  & 0  & 87,88 & 51,02\% \\ \hline
116 &  \x    & \x    &       & \x    & \x    & \x    & \x    &       & 56,9\% &  6  & 0  & 88,25 & 51,66\% \\ \hline
117 &  \x    & \x    & \x    &       &       &       &       &       & 55,5\% &  5  & 0  & 89,02 & 52,98\% \\ \hline
118 &  \x    & \x    &       &       &       & \x    & \x    &       & 52,0\% &  6  & 0  & 89,42 & 53,67\% \\ \hline
119 &  \x    & \x    & \x    & \x    &       & \x\m  &       & \x    & 56,5\% &  7  & 0  & 89,94 & 54,56\% \\ \hline
120 &  \x    & \x    &       &       & \x    &       & \x    &       & 55,8\% &  6  & 14 & 90,09 & 54,82\% \\ \hline
121 &  \x    & \x    & \x    &       &       & \x\m  & \x\m  &       & 56,7\% &  5  & 0  & 90,12 & 54,87\% \\ \hline
122 &  \x    & \x    &       & \x    &       &       &       &       & 53,3\% &  8  & 9  & 90,95 & 56,3\% \\ \hline
123 &  \x    & \x    &       &       &       &       &       & \x\m  & 51,8\% &  9  & 0  & 91,05 & 56,47\% \\ \hline
124 &  \x    & \x    &       & \x    &       & \x\m  &       &       & 52,0\% &  6  & 9  & 91,30 & 56,9\% \\ \hline
125 &  \x    & \x    &       & \x    & \x    & \x    &       & \x    & 56,1\% &  6  & 0  & 91,32 & 56,93\% \\ \hline
126 &  \x    & \x    &       & \x    &       &       &       & \x    & 52,0\% &  7  & 0  & 91,69 & 57,57\% \\ \hline
127 &  \x    & \x    & \x    &       &       & \x\m  &       &       & 53,9\% &  8  & 3  & 92,73 & 59,36\% \\ \hline
128 &  \x    & \x    &       & \x    &       &       & \x    &       & 52,4\% &  7  & 4  & 92,82 & 59,51\% \\ \hline
129 &  \x    & \x    &       & \x    &       & \x    & \x    &       & 51,7\% &  6  & 5  & 93,37 & 60,46\% \\ \hline
130 &  \x    & \x    &       & \x    &       &       & \x\m  &       & 53,3\% &  6  & 0  & 93,56 & 60,78\% \\ \hline
131 &  \x    & \x    &       & \x    &       & \x    &       & \x    & 52,2\% &  5  & 0  & 93,59 & 60,84\% \\ \hline
132 &  \x    & \x    &       & \x    &       &       &       &       & 52,9\% &  9  & 0  & 93,69 & 61,01\% \\ \hline
133 &  \x    & \x    &       &       &       &       &       & \x    & 51,6\% &  9  & 0  & 94,34 & 62,12\% \\ \hline
134 &  \x    & \x    &       &       &       &       &       &       & 51,2\% &  15 & 0  & 94,99 & 63,24\% \\ \hline
135 &  \x    & \x    &       & \x    &       &       &       & \x\m  & 53,1\% &  10 & 0  & 96,45 & 65,75\% \\ \hline
136 &  \x    & \x    &       &       &       & \x\m  &       &       & 50,8\% &  9  & 0  & 96,79 & 66,33\% \\ \hline
137 &  \x    & \x    &       &       &       &       & \x\m  &       & 52,1\% &  11 & 0  & 96,94 & 66,59\% \\ \hline
138 &  \x    & \x    &       &       &       & \x    &       &       & 50,0\% &  5  & 0  & 97,94 & 68,31\% \\ \hline
139 &  \x    & \x    &       & \x    &       & \x\m  &       & \x\m  & 51,0\% &  5  & 10 & 100,13 & 72,07\% \\ \hline
140 &  \x    & \x    &       & \x    &       & \x\m  & \x\m  &       & 52,5\% &  5  & 0  & 100,61 & 72,9\% \\ \hline
141 &  \x    & \x    &       &       &       & \x\m  & \x\m  &       & 51,0\% &  9  & 0  & 102,47 & 76,1\% \\ \hline
142 &  \x    & \x    &       &       &       & \x    &       & \x    & 50,9\% &  5  & 0  & 104,56 & 79,69\% \\ \hline
\end{longtable}
\normalsize

\footnotesize
\begin{longtable}{|c|c|c|c|}
\caption{Own set vs Unseen set, Including MAPE}\\
\hline
\textbf{\#} & \textbf{MAE(Training dataset)} & \textbf{MAE(Unseen dataset)} & \textbf{MAPE(Unseen dataset)} \\
\hline
\endfirsthead
\multicolumn{4}{c}%
{\tablename\ \thetable\ -- \textit{Continued from previous page}} \\
\hline
\textbf{\#} & \textbf{MAE(Training dataset)} & \textbf{MAE(Unseen dataset)} & \textbf{MAPE(Unseen dataset)} \\
\hline
\endhead
\hline \multicolumn{4}{r}{\textit{Continued on next page}} \\
\endfoot
\hline
\endlastfoot
	1  & 20,9 & 57,12 & 21,72\% \\ \hline
	2  & 29,61 & 58,09 & 22,09\% \\ \hline
	3  & 25,49 & 58,79 & 22,35\% \\ \hline
	4  & 18,94 & 60,14 & 22,87\% \\ \hline
	5  & 31,1 & 62,19 & 23,65\% \\ \hline
	6  & 22,97 & 62,26 & 23,67\% \\ \hline
	7  & 20,64 & 62,84 & 23,89\% \\ \hline
	8  & 23,87 & 63,94 & 24,31\% \\ \hline
	9  & 17,76 & 64,19 & 24,41\% \\ \hline
	10 & 19,79 & 64,72 & 24,61\% \\ \hline
	11 & 20,85 & 65,07 & 24,74\% \\ \hline
	12 & 25,71 & 65,95 & 25,08\% \\ \hline
	13 & 30,02 & 66,55 & 25,3\% \\ \hline
	14 & 31,32 & 67,21 & 25,56\% \\ \hline
	15 & 31,8 & 67,88 & 25,81\% \\ \hline
	16 & 23,11 & 68,21 & 25,94\% \\ \hline
	17 & 30,39 & 68,34 & 25,98\% \\ \hline
	18 & 40,54 & 68,35 & 25,99\% \\ \hline
	19 & 29,4 & 68,43 & 26,02\% \\ \hline
	20 & 34,38 & 68,45 & 26,03\% \\ \hline
	21 & 43,0 & 69,09 & 26,27\% \\ \hline
	22 & 32,98 & 69,14 & 26,29\% \\ \hline
	23 & 34,93 & 69,34 & 26,36\% \\ \hline
	24 & 79,23 & 69,37 & 26,38\% \\ \hline
	25 & 23,4 & 69,46 & 26,41\% \\ \hline
	26 & 36,0 & 69,47 & 26,41\% \\ \hline
	27 & 40,8 & 69,50 & 26,42\% \\ \hline
	28 & 33,68 & 69,62 & 26,47\% \\ \hline
	29 & 22,59 & 69,70 & 26,5\% \\ \hline
	30 & 22,06 & 69,71 & 26,51\% \\ \hline
	31 & 34,26 & 69,75 & 26,52\% \\ \hline
	32 & 26,2 & 69,82 & 26,55\% \\ \hline
	33 & 45,9 & 69,90 & 26,58\% \\ \hline
	34 & 31,96 & 70,29 & 26,73\% \\ \hline
	35 & 27,24 & 70,58 & 26,84\% \\ \hline
	36 & 22,86 & 70,75 & 26,9\% \\ \hline
	37 & 27,66 & 70,96 & 26,98\% \\ \hline
	38 & 32,6 & 71,31 & 27,11\% \\ \hline
	39 & 43,63 & 71,39 & 27,15\% \\ \hline
	40 & 35,22 & 71,49 & 27,18\% \\ \hline
	41 & 40,97 & 71,58 & 27,22\% \\ \hline
	42 & 26,29 & 71,74 & 27,28\% \\ \hline
	43 & 22,29 & 71,81 & 27,31\% \\ \hline
	44 & 27,08 & 71,89 & 27,33\% \\ \hline
	45 & 27,16 & 71,92 & 27,35\% \\ \hline
	46 & 26,23 & 72,70 & 27,64\% \\ \hline
	47 & 21,52 & 72,74 & 27,66\% \\ \hline
	48 & 28,94 & 73,70 & 28,02\% \\ \hline
	49 & 30,41 & 73,80 & 28,06\% \\ \hline
	50 & 31,81 & 74,19 & 28,21\% \\ \hline
	51 & 22,83 & 74,29 & 28,25\% \\ \hline
	52 & 30,33 & 74,56 & 28,35\% \\ \hline
	53 & 18,98 & 74,89 & 28,48\% \\ \hline
	54 & 47,92 & 75,25 & 28,61\% \\ \hline
	55 & 20,64 & 75,31 & 28,63\% \\ \hline
	56 & 22,16 & 75,40 & 28,67\% \\ \hline
	57 & 33,23 & 75,45 & 28,69\% \\ \hline
	58 & 33,07 & 75,52 & 28,71\% \\ \hline
	59 & 55,33 & 75,63 & 28,76\% \\ \hline
	60 & 29,14 & 75,77 & 28,81\% \\ \hline
	61 & 21,59 & 76,12 & 28,94\% \\ \hline
	62 & 23,15 & 76,18 & 28,97\% \\ \hline
	63 & 23,25 & 76,19 & 28,97\% \\ \hline
	64 & 54,54 & 76,32 & 29,02\% \\ \hline
	65 & 21,57 & 76,39 & 29,05\% \\ \hline
	66 & 21,36 & 77,70 & 29,54\% \\ \hline
	67 & 29,17 & 77,72 & 29,55\% \\ \hline
	68 & 23,47 & 78,05 & 29,68\% \\ \hline
	69 & 25,74 & 78,49 & 29,85\% \\ \hline
	70 & 25,37 & 78,57 & 29,87\% \\ \hline
	71 & 28,61 & 78,69 & 29,92\% \\ \hline
	72 & 23,02 & 78,74 & 29,94\% \\ \hline
	73 & 48,13 & 78,88 & 29,99\% \\ \hline
	74 & 43,03 & 79,01 & 30,04\% \\ \hline
	75 & 38,71 & 79,71 & 30,31\% \\ \hline
	76 & 26,58 & 79,89 & 30,38\% \\ \hline
	77 & 25,86 & 80,20 & 30,5\% \\ \hline
	78 & 19,39 & 81,11 & 30,84\% \\ \hline
	79 & 40,53 & 81,28 & 30,91\% \\ \hline
	80 & 22,61 & 81,34 & 30,93\% \\ \hline
	81 & 27,88 & 81,58 & 31,02\% \\ \hline
	82 & 28,72 & 82,03 & 31,19\% \\ \hline
	83 & 28,02 & 82,04 & 31,19\% \\ \hline
	84 & 35,39 & 82,29 & 31,29\% \\ \hline
	85 & 34,14 & 82,32 & 31,3\% \\ \hline
	86 & 42,64 & 82,72 & 31,45\% \\ \hline
	87 & 70,52 & 83,35 & 31,69\% \\ \hline
	88 & 32,93 & 83,82 & 31,87\% \\ \hline
	89 & 29,08 & 84,08 & 31,97\% \\ \hline
	90 & 42,86 & 84,50 & 32,13\% \\ \hline
	91 & 37,84 & 85,15 & 32,38\% \\ \hline
	92 & 27,01 & 85,51 & 32,51\% \\ \hline
	93 & 29,8 & 85,55 & 32,53\% \\ \hline
	94 & 34,82 & 85,57 & 32,54\% \\ \hline
	95 & 33,33 & 85,66 & 32,57\% \\ \hline
	96 & 38,43 & 85,77 & 32,61\% \\ \hline
	97 & 33,74 & 85,99 & 32,7\% \\ \hline
	98 & 28,01 & 86,00 & 32,7\% \\ \hline
	99 & 27,74 & 86,13 & 32,75\% \\ \hline
	100 & 72,29 & 86,33 & 32,83\% \\ \hline
	101 & 42,09 & 87,00 & 33,08\% \\ \hline
	102 & 41,73 & 87,63 & 33,32\% \\ \hline
	103 & 20,61 & 87,87 & 33,41\% \\ \hline
	104 & 89,21 & 87,89 & 33,42\% \\ \hline
	105 & 46,33 & 87,94 & 33,44\% \\ \hline
	106 & 46,68 & 89,03 & 33,85\% \\ \hline
	107 & 21,15 & 89,14 & 33,89\% \\ \hline
	108 & 26,33 & 89,55 & 34,05\% \\ \hline
	109 & 50,28 & 90,73 & 34,5\% \\ \hline
	110 & 29,26 & 91,06 & 34,62\% \\ \hline
	111 & 50,6 & 91,23 & 34,69\% \\ \hline
	112 & 35,34 & 91,67 & 34,85\% \\ \hline
	113 & 26,64 & 91,91 & 34,95\% \\ \hline
	114 & 20,56 & 92,65 & 35,23\% \\ \hline
	115 & 30,36 & 92,95 & 35,34\% \\ \hline
	116 & 24,38 & 93,63 & 35,6\% \\ \hline
	117 & 25,54 & 94,58 & 35,96\% \\ \hline
	118 & 55,71 & 96,70 & 36,77\% \\ \hline
	119 & 50,05 & 96,78 & 36,8\% \\ \hline
	120 & 31,17 & 97,09 & 36,92\% \\ \hline
	121 & 145,57 & 98,17 & 37,33\% \\ \hline
	122 & 31,64 & 99,10 & 37,68\% \\ \hline
	123 & 76,82 & 99,38 & 37,79\% \\ \hline
	124 & 42,84 & 99,78 & 37,94\% \\ \hline
	125 & 28,13 & 100,63 & 38,26\% \\ \hline
	126 & 36,09 & 102,76 & 39,07\% \\ \hline
	127 & 27,48 & 103,01 & 39,17\% \\ \hline
	128 & 103,76 & 103,56 & 39,38\% \\ \hline
	129 & 43,66 & 103,71 & 39,44\% \\ \hline
	130 & 35,31 & 104,19 & 39,62\% \\ \hline
	131 & 44,97 & 104,33 & 39,67\% \\ \hline
	132 & 32,04 & 104,93 & 39,9\% \\ \hline
	133 & 41,23 & 105,04 & 39,94\% \\ \hline
	134 & 22,81 & 105,70 & 40,19\% \\ \hline
	135 & 42,85 & 107,85 & 41,01\% \\ \hline
	136 & 73,87 & 108,37 & 41,21\% \\ \hline
	137 & 65,42 & 111,14 & 42,26\% \\ \hline
	138 & 43,75 & 111,65 & 42,45\% \\ \hline
	139 & 37,56 & 113,56 & 43,18\% \\ \hline
	140 & 44,47 & 115,29 & 43,84\% \\ \hline
	141 & 38,98 & 115,81 & 44,03\% \\ \hline
	142 & 49,19 & 115,83 & 44,04\% \\ \hline
	143 & 45,62 & 117,17 & 44,55\% \\ \hline
	144 & 39,28 & 117,34 & 44,62\% \\ \hline
	\end{longtable}
\normalsize

\subsection{Experiment Two}
The figure is a 1000 hour graph is shown in \ref{fig:fullPageExperiment2} for the best prediction from Experiment Two. The full graph can be seen in the digitalized version. In the directory "PriceGraphExperiment2".

\begin{sidewaysfigure}[h]
\centering
\includegraphics[width=\linewidth]{billeder/PriceGraphs/Experiment2.png}
\caption{First 1000 hours of the best prediction in experiment two}
\label{fig:fullPageExperiment2}
\end{sidewaysfigure}

\subsection{Experiment Three}
The figure is a 1000 hour graph is shown in \ref{fig:fullPageExperiment3} for the best prediction from Experiment Three. The full graph can be seen in the digitalized version. In the directory "PriceGraphExperiment3".

\begin{sidewaysfigure}[h]
\centering
\includegraphics[width=\linewidth]{billeder/PriceGraphs/Experiment3.png}
\caption{First 1000 hours of the best prediction in experiment 3}
\label{fig:fullPageExperiment3}
\end{sidewaysfigure}

\footnotesize
\begin{longtable}{|c|c|c|c|c|c|c|}
\caption{Own set vs Unseen set, Including MAPE}\\
\hline
\textbf{\#} & \textbf{Stack size} & \textbf{Smoothing factor} & \textbf{\% CD} & \textbf{MAPE} & \textbf{MAE} & \textbf{\% ranking} \\
\hline
\endfirsthead
\multicolumn{7}{c}%
{\tablename\ \thetable\ -- \textit{Continued from previous page}} \\
\hline
\textbf{\#} & \textbf{Stack size} & \textbf{Smoothing factor} & \textbf{\% CD} & \textbf{MAPE} & \textbf{MAE} & \textbf{\% ranking} \\
\hline
\endhead
\hline \multicolumn{7}{r}{\textit{Continued on next page}} \\
\endfoot
\hline
\endlastfoot
1  & 16 & 0,9 &  67,1\% & 17,56\% & 45,96 & - \\ \hline
2  & 12 & 0,1 &  66,7\% & 17,63\% & 46,13 & 0,38\% \\ \hline
3  & 20 & 0,9 &  66,9\% & 17,76\% & 46,47 & 1,11\% \\ \hline
4  & 8 & 0,9 &  67,1\% & 17,79\% & 46,55 & 1,28\% \\ \hline
5  & 20 & 0,5 &  66,7\% & 17,79\% & 46,56 & 1,31\% \\ \hline
6  & 24 & 0,4 &  67,3\% & 17,80\% & 46,58 & 1,36\% \\ \hline
7  & 20 & 0,7 &  66,4\% & 17,80\% & 46,59 & 1,37\% \\ \hline
8  & 12 & 0,9 &  67,9\% & 17,81\% & 46,61 & 1,41\% \\ \hline
9  & 24 & 0,8 &  67,8\% & 17,88\% & 46,78 & 1,79\% \\ \hline
10 & 24 & 0,5 &  67,1\% & 17,91\% & 46,88 & 2,0\% \\ \hline
11 & 20 & 0,4 &  66,2\% & 17,97\% & 47,03 & 2,33\% \\ \hline
12 & 24 & 0,2 &  67,5\% & 17,97\% & 47,04 & 2,34\% \\ \hline
13 & 12 & 0,3 &  67,6\% & 17,99\% & 47,07 & 2,43\% \\ \hline
14 & 16 & 0,8 &  67,8\% & 18,02\% & 47,17 & 2,63\% \\ \hline
15 & 6 & 0,9 &  67,5\% & 18,10\% & 47,36 & 3,04\% \\ \hline
16 & 16 & 0,6 &  67,3\% & 18,20\% & 47,64 & 3,66\% \\ \hline
17 & 6 & 0,7 &  67,8\% & 18,22\% & 47,69 & 3,77\% \\ \hline
18 & 20 & 0,1 &  67,2\% & 18,25\% & 47,76 & 3,92\% \\ \hline
19 & 24 & 0,3 &  68,0\% & 18,29\% & 47,86 & 4,13\% \\ \hline
20 & 4 & 0,9 &  67,7\% & 18,29\% & 47,86 & 4,14\% \\ \hline
21 & 24 & 0,6 &  67,5\% & 18,29\% & 47,87 & 4,15\% \\ \hline
22 & 4 & 0,5 &  68,9\% & 18,30\% & 47,88 & 4,19\% \\ \hline
23 & 8 & 0,2 &  67,3\% & 18,32\% & 47,94 & 4,3\% \\ \hline
24 & 20 & 0,6 &  67,4\% & 18,32\% & 47,96 & 4,35\% \\ \hline
25 & 6 & 0,2 &  66,9\% & 18,35\% & 48,03 & 4,5\% \\ \hline
26 & 8 & 0,7 &  66,4\% & 18,37\% & 48,08 & 4,61\% \\ \hline
27 & 20 & 0,8 &  66,9\% & 18,39\% & 48,14 & 4,74\% \\ \hline
28 & 20 & 0,2 &  66,6\% & 18,40\% & 48,15 & 4,78\% \\ \hline
29 & 12 & 0,5 &  67,9\% & 18,43\% & 48,22 & 4,92\% \\ \hline
30 & 16 & 0,3 &  66,7\% & 18,43\% & 48,23 & 4,94\% \\ \hline
31 & 16 & 0,7 &  67,1\% & 18,45\% & 48,28 & 5,05\% \\ \hline
32 & 12 & 0,8 &  66,7\% & 18,46\% & 48,32 & 5,13\% \\ \hline
33 & 16 & 0,5 &  66,7\% & 18,50\% & 48,40 & 5,32\% \\ \hline
34 & 24 & 0,7 &  67,0\% & 18,52\% & 48,47 & 5,46\% \\ \hline
35 & 8 & 0,3 &  67,2\% & 18,58\% & 48,63 & 5,81\% \\ \hline
36 & 20 & 0,3 &  67,1\% & 18,58\% & 48,63 & 5,82\% \\ \hline
37 & 8 & 0,6 &  66,4\% & 18,61\% & 48,70 & 5,97\% \\ \hline
38 & 16 & 0,4 &  67,7\% & 18,61\% & 48,71 & 5,98\% \\ \hline
39 & 8 & 0,4 &  67,7\% & 18,63\% & 48,76 & 6,1\% \\ \hline
40 & 6 & 0,1 &  66,3\% & 18,69\% & 48,92 & 6,44\% \\ \hline
41 & 8 & 0,1 &  67,5\% & 18,70\% & 48,94 & 6,48\% \\ \hline
42 & 16 & 0,2 &  68,2\% & 18,75\% & 49,07 & 6,77\% \\ \hline
43 & 16 & 0,1 &  68,0\% & 18,78\% & 49,14 & 6,92\% \\ \hline
44 & 8 & 0,8 &  67,4\% & 18,86\% & 49,35 & 7,37\% \\ \hline
45 & 12 & 0,7 &  67,2\% & 18,89\% & 49,43 & 7,55\% \\ \hline
46 & 6 & 0,5 &  66,8\% & 18,89\% & 49,44 & 7,57\% \\ \hline
47 & 12 & 0,6 &  67,1\% & 18,93\% & 49,54 & 7,79\% \\ \hline
48 & 6 & 0,4 &  67,0\% & 18,96\% & 49,63 & 7,99\% \\ \hline
49 & 24 & 0,1 &  66,9\% & 18,98\% & 49,68 & 8,1\% \\ \hline
50 & 4 & 0,2 &  68,5\% & 19,01\% & 49,74 & 8,22\% \\ \hline
51 & 4 & 0,3 &  67,8\% & 19,05\% & 49,85 & 8,47\% \\ \hline
52 & 6 & 0,6 &  66,4\% & 19,06\% & 49,89 & 8,55\% \\ \hline
53 & 4 & 0,7 &  67,6\% & 19,06\% & 49,89 & 8,55\% \\ \hline
54 & 24 & 0,9 &  66,8\% & 19,07\% & 49,90 & 8,57\% \\ \hline
55 & 4 & 0,8 &  67,5\% & 19,14\% & 50,08 & 8,97\% \\ \hline
56 & 12 & 0,4 &  67,2\% & 19,17\% & 50,18 & 9,18\% \\ \hline
57 & 6 & 0,8 &  67,2\% & 19,22\% & 50,29 & 9,42\% \\ \hline
58 & 8 & 0,5 &  67,5\% & 19,28\% & 50,44 & 9,76\% \\ \hline
59 & 4 & 0,1 &  67,5\% & 19,31\% & 50,52 & 9,94\% \\ \hline
60 & 12 & 0,2 &  67,2\% & 19,37\% & 50,70 & 10,33\% \\ \hline
61 & 4 & 0,4 &  67,6\% & 19,38\% & 50,71 & 10,34\% \\ \hline
62 & 4 & 0,6 &  67,1\% & 19,43\% & 50,84 & 10,62\% \\ \hline
63 & 6 & 0,3 &  67,3\% & 19,63\% & 51,38 & 11,8\% \\ \hline
	\end{longtable}
\normalsize


\subsection{Experiment Four}
The figure is a 1000 hour graph is shown in \ref{fig:fullPageExperiment4} for the best prediction from Experiment Four. The full graph can be seen in the digitalized version. In the directory "PriceGraphExperiment4".

\begin{sidewaysfigure}[h]
\centering
\includegraphics[width=\linewidth]{billeder/PriceGraphs/Experiment4.png}
\caption{First 1000 hours of the best prediction in experiment four}
\label{fig:fullPageExperiment4}
\end{sidewaysfigure}


\subsection{Experiment Five}
The figure is a 1000 hour graph is shown in \ref{fig:fullPageExperiment5} for the best prediction from Experiment Five. The full graph can be seen in the digitalized version. In the directory "PriceGraphExperiment5".

\begin{sidewaysfigure}[h]
\centering
\includegraphics[width=\linewidth]{billeder/PriceGraphs/Experiment5.png}
\caption{First 1000 hours of the best prediction in experiment five}
\label{fig:fullPageExperiment5}
\end{sidewaysfigure} 
\section{Wind power experimental results}
\label{sec:windResultsAppendix}
\subsection{Experiment One: Best Input Combination}

\subsubsection{Best Input Combination Results}
\label{sec:simpleInputTest}
The simple input test based on 3 month historical data and 200 epochs.

WS = Wind Speed
AD = Air Density
C = Consumption
T = Temperature
WD = Wind Direction
L-P = Last Known Production
D = Date
ToD = Time of Day
M = Time of Day as matrix
CD = Correct Direction

\footnotesize
\begin{center}
\begin{longtable}{|c|c|c|c|c|c|c|c|c|c|c|c|c|}
\caption{Wind Production Input Parameter Test}\\
\hline
\textbf{WS} & \textbf{AD} & \textbf{C} & \textbf{T} & \textbf{WD} & \textbf{L-P} & \textbf{D}& \textbf{ToD} & \textbf{MAE} & \textbf{\% from \#1} & \textbf{H1} & \textbf{H2} & \textbf{CD} \\
\hline
\endfirsthead
\multicolumn{13}{c}%
{\tablename\ \thetable\ -- \textit{Continued from previous page}} \\
\hline
\textbf{WS} & \textbf{AD} & \textbf{C} & \textbf{T} & \textbf{WD} & \textbf{L-P} & \textbf{D}& \textbf{ToD} & \textbf{MAE} & \textbf{\% from \#1} & \textbf{H1} & \textbf{H2} & \textbf{CD}  \\
\hline
\endhead
\hline \multicolumn{13}{r}{\textit{Continued on next page}} \\
\endfoot
\hline
\endlastfoot
\arrayrulecolor{light-gray}
 \x &  &  &  \x &  &  \x &  &  \x & 125,75 & 0,0\% & 19 & 13 & 73\% \\ \hline
 \x &  \x &  &  &  \x &  \x &  &  \x & 131,59 & 4,64\% & 20 & 17 & 73\% \\ \hline
 \x &  \x &  &  &  &  \x &  &  \x & 131,89 & 4,88\% & 16 & 20 & 73\% \\ \hline
 \x &  \x &  \x &  \x &  \x &  \x &  &  \x & 133,18 & 5,91\% & 13 & 20 & 72\% \\ \hline
 \x &  \x &  \x &  \x &  \x &  \x &  &  \x & 133,77 & 6,38\% & 16 & 17 & 73\% \\ \hline
 \x &  \x &  \x &  &  &  \x &  &  \x & 134,14 & 6,67\% & 12 & 17 & 73\% \\ \hline
 \x &  \x &  \x &  &  \x &  \x &  &  \x & 135,4 & 7,67\% & 6 & 13 & 72\% \\ \hline
 \x &  \x &  \x &  &  &  \x &  &  \x & 136,33 & 8,41\% & 21 & 10 & 73\% \\ \hline
 \x &  \x &  &  &  &  \x &  \x &  \x & 136,65 & 8,67\% & 5 & 24 & 73\% \\ \hline
 \x &  &  &  &  &  \x &  &  \x & 137,33 & 9,21\% & 9 & 15 & 74\% \\ \hline
 \x &  &  &  \x &  \x &  \x &  &  \x & 137,8 & 9,58\% & 17 & 16 & 73\% \\ \hline
 \x &  \x &  &  &  \x &  \x &  &  \x & 138,16 & 9,87\% & 15 & 21 & 72\% \\ \hline
 \x &  &  \x &  &  &  &  &  \x & 138,33 & 10,0\% & 23 & 13 & 64\% \\ \hline
 \x &  \x &  &  \x &  &  \x &  &  \x & 138,94 & 10,49\% & 10 & 25 & 72\% \\ \hline
 \x &  \x &  &  &  &  &  &  \x & 139,25 & 10,74\% & 23 & 12 & 54\% \\ \hline
 \x &  \x &  &  &  &  \x &  &  \x & 139,35 & 10,82\% & 16 & 14 & 73\% \\ \hline
 \x &  \x &  \x &  &  &  &  &  \x & 139,67 & 11,07\% & 17 & 20 & 64\% \\ \hline
 \x &  &  \x &  &  \x &  \x &  &  \x & 139,78 & 11,16\% & 21 & 9 & 71\% \\ \hline
 \x &  \x &  \x &  \x &  &  &  &  \x & 139,99 & 11,32\% & 11 & 19 & 64\% \\ \hline
 \x &  &  &  &  &  &  &  \x & 140,14 & 11,44\% & 16 & 9 & 41\% \\ \hline
 \x &  &  &  &  \x &  &  &  \x & 140,44 & 11,68\% & 17 & 20 & 61\% \\ \hline
 \x &  &  &  \x &  &  \x &  &  \x & 140,55 & 11,77\% & 13 & 13 & 73\% \\ \hline
 \x &  &  \x &  &  &  \x &  &  \x & 140,82 & 11,98\% & 18 & 16 & 73\% \\ \hline
 \x &  \x &  \x &  &  &  &  &  \x & 140,85 & 12,01\% & 16 & 18 & 64\% \\ \hline
 \x &  \x &  \x &  \x &  &  \x &  &  \x & 140,91 & 12,06\% & 18 & 13 & 73\% \\ \hline
 \x &  \x &  &  &  &  &  &  \x & 141,09 & 12,2\% & 23 & 9 & 53\% \\ \hline
 \x &  &  \x &  &  &  &  &  \x & 141,28 & 12,35\% & 19 & 15 & 64\% \\ \hline
 \x &  &  \x &  &  \x &  &  &  \x & 141,38 & 12,43\% & 16 & 22 & 65\% \\ \hline
 \x &  &  &  &  \x &  &  &  \x & 141,5 & 12,52\% & 18 & 1 & 60\% \\ \hline
 \x &  &  &  &  \x &  \x &  &  \x & 142,22 & 13,1\% & 21 & 9 & 72\% \\ \hline
 \x &  \x &  \x &  \x &  &  \x &  \x &  \x & 142,35 & 13,2\% & 5 & 20 & 72\% \\ \hline
 \x &  \x &  &  \x &  \x &  &  &  \x & 142,44 & 13,27\% & 2 & 23 & 61\% \\ \hline
 \x &  \x &  \x &  &  &  \x &  \x &  \x & 142,57 & 13,38\% & 15 & 14 & 73\% \\ \hline
 \x &  \x &  &  \x &  &  \x &  &  \x & 142,64 & 13,43\% & 17 & 20 & 72\% \\ \hline
 \x &  &  \x &  &  &  \x &  &  \x & 142,67 & 13,46\% & 18 & 13 & 72\% \\ \hline
 \x &  \x &  \x &  &  \x &  &  &  \x & 142,75 & 13,52\% & 18 & 18 & 63\% \\ \hline
 \x &  &  \x &  \x &  \x &  \x &  &  \x & 142,97 & 13,69\% & 25 & 12 & 71\% \\ \hline
 \x &  \x &  &  &  \x &  &  &  \x & 143,07 & 13,77\% & 2 & 24 & 61\% \\ \hline
 \x &  \x &  \x &  &  &  &  \x &  \x & 143,23 & 13,9\% & 6 & 25 & 64\% \\ \hline
 \x &  &  \x &  \x &  &  \x &  &  \x & 143,24 & 13,91\% & 20 & 9 & 72\% \\ \hline
 \x &  \x &  &  \x &  &  &  &  \x & 143,53 & 14,14\% & 21 & 10 & 54\% \\ \hline
 \x &  \x &  &  &  \x &  &  &  \x & 143,98 & 14,5\% & 17 & 14 & 61\% \\ \hline
 \x &  \x &  \x &  &  \x &  &  &  \x & 144,06 & 14,56\% & 12 & 21 & 63\% \\ \hline
 \x &  \x &  \x &  &  &  \x &  \x &  \x & 144,24 & 14,7\% & 25 & 10 & 71\% \\ \hline
 \x &  &  &  \x &  \x &  \x &  &  \x & 144,95 & 15,27\% & 12 & 20 & 72\% \\ \hline
 \x &  \x &  \x &  &  \x &  \x &  &  \x & 144,96 & 15,28\% & 9 & 18 & 71\% \\ \hline
 \x &  &  &  &  &  \x &  &  \x & 145,11 & 15,4\% & 18 & 13 & 72\% \\ \hline
 \x &  \x &  &  \x &  \x &  \x &  &  \x & 145,26 & 15,51\% & 16 & 19 & 72\% \\ \hline
 \x &  \x &  \x &  \x &  &  &  &  \x & 145,49 & 15,7\% & 2 & 23 & 65\% \\ \hline
 \x &  &  \x &  &  \x &  \x &  \x &  \x & 145,92 & 16,04\% & 9 & 22 & 73\% \\ \hline
 \x &  &  \x &  &  \x &  &  \x &  \x & 145,98 & 16,09\% & 11 & 18 & 64\% \\ \hline
 \x &  &  \x &  &  &  &  \x &  \x & 146,01999999999998 & 16,12\% & 16 & 16 & 64\% \\ \hline
 \x &  \x &  \x &  \x &  &  &  \x &  \x & 146,32 & 16,36\% & 16 & 9 & 64\% \\ \hline
 \x &  &  \x &  \x &  &  &  &  \x & 146,48 & 16,49\% & 20 & 17 & 64\% \\ \hline
 \x &  &  \x &  &  \x &  &  &  \x & 146,82 & 16,76\% & 15 & 19 & 64\% \\ \hline
 \x &  \x &  &  &  &  &  \x &  \x & 147,08 & 16,96\% & 1 & 25 & 54\% \\ \hline
 \x &  &  &  \x &  &  &  &  \x & 147,41 & 17,22\% & 21 & 1 & 51\% \\ \hline
 \x &  \x &  &  \x &  \x &  \x &  \x &  \x & 147,99 & 17,69\% & 15 & 19 & 72\% \\ \hline
 \x &  \x &  \x &  \x &  &  \x &  \x &  \x & 148,27 & 17,91\% & 20 & 10 & 72\% \\ \hline
 \x &  &  \x &  \x &  \x &  \x &  &  \x & 148,61 & 18,18\% & 24 & 10 & 71\% \\ \hline
 \x &  &  &  \x &  \x &  &  &  \x & 148,67 & 18,23\% & 13 & 22 & 60\% \\ \hline
 \x &  &  &  \x &  &  &  &  \x & 149,27 & 18,7\% & 14 & 19 & 51\% \\ \hline
 \x &  &  &  &  &  &  &  \x & 149,72 & 19,06\% & 9 & 25 & 41\% \\ \hline
 \x &  &  \x &  &  &  &  \x &  \x & 149,8 & 19,13\% & 22 & 14 & 64\% \\ \hline
 \x &  \x &  \x &  \x &  &  \x &  &  \x & 149,85 & 19,17\% & 25 & 10 & 71\% \\ \hline
 \x &  &  &  &  &  \x &  \x &  \x & 150,3 & 19,52\% & 13 & 20 & 72\% \\ \hline
 \x &  \x &  \x &  \x &  &  &  \x &  \x & 150,45 & 19,64\% & 22 & 16 & 64\% \\ \hline
 \x &  &  \x &  \x &  \x &  &  &  \x & 150,63 & 19,79\% & 14 & 24 & 63\% \\ \hline
 \x &  &  \x &  \x &  &  \x &  &  \x & 150,96 & 20,05\% & 14 & 13 & 71\% \\ \hline
 \x &  &  \x &  &  \x &  \x &  &  \x & 151,54 & 20,51\% & 14 & 18 & 71\% \\ \hline
 \x &  &  &  &  \x &  \x &  &  \x & 151,55 & 20,52\% & 21 & 9 & 72\% \\ \hline
 \x &  &  \x &  &  &  \x &  \x &  \x & 151,92 & 20,81\% & 16 & 12 & 72\% \\ \hline
 \x &  \x &  &  &  \x &  \x &  \x &  \x & 152,61 & 21,36\% & 2 & 18 & 71\% \\ \hline
 \x &  \x &  \x &  &  \x &  &  \x &  \x & 152,89 & 21,58\% & 22 & 13 & 63\% \\ \hline
 \x &  \x &  &  &  &  &  \x &  \x & 153,3 & 21,91\% & 13 & 13 & 54\% \\ \hline
 \x &  \x &  &  &  \x &  &  \x &  \x & 153,31 & 21,92\% & 7 & 19 & 61\% \\ \hline
 \x &  &  &  &  &  &  \x &  \x & 153,92 & 22,4\% & 22 & 9 & 41\% \\ \hline
 \x &  &  \x &  \x &  &  &  &  \x & 154,02 & 22,48\% & 13 & 19 & 63\% \\ \hline
 \x &  \x &  \x &  \x &  \x &  &  &  \x & 154,25 & 22,66\% & 9 & 24 & 63\% \\ \hline
 \x &  \x &  \x &  &  \x &  &  \x &  \x & 154,51 & 22,87\% & 1 & 11 & 64\% \\ \hline
 \x &  &  &  &  \x &  &  \x &  \x & 154,58 & 22,93\% & 7 & 22 & 59\% \\ \hline
 \x &  \x &  &  &  &  \x &  \x &  \x & 154,85 & 23,14\% & 5 & 12 & 72\% \\ \hline
 \x &  \x &  &  \x &  &  &  \x &  \x & 154,97 & 23,24\% & 15 & 15 & 54\% \\ \hline
 \x &  \x &  &  \x &  \x &  &  \x &  \x & 155,07 & 23,32\% & 13 & 15 & 61\% \\ \hline
 \x &  \x &  &  \x &  &  &  \x &  \x & 155,55 & 23,7\% & 1 & 25 & 54\% \\ \hline
 \x &  &  &  &  \x &  &  \x &  \x & 156,13 & 24,16\% & 22 & 19 & 60\% \\ \hline
 \x &  \x &  \x &  \x &  \x &  &  &  \x & 156,26 & 24,26\% & 15 & 21 & 62\% \\ \hline
 \x &  &  \x &  \x &  &  &  \x &  \x & 156,52 & 24,47\% & 21 & 10 & 64\% \\ \hline
 \x &  &  \x &  \x &  \x &  &  \x &  \x & 156,72 & 24,63\% & 9 & 23 & 63\% \\ \hline
 \x &  \x &  &  \x &  \x &  &  &  \x & 156,94 & 24,8\% & 20 & 13 & 60\% \\ \hline
 \x &  &  \x &  &  &  \x &  \x &  \x & 157,19 & 25,0\% & 7 & 22 & 72\% \\ \hline
 \x &  &  &  &  &  &  \x &  \x & 157,36 & 25,14\% & 20 & 11 & 40\% \\ \hline
 \x &  &  \x &  \x &  \x &  &  &  \x & 157,44 & 25,2\% & 15 & 16 & 62\% \\ \hline
 \x &  \x &  &  &  \x &  &  \x &  \x & 157,56 & 25,3\% & 9 & 18 & 60\% \\ \hline
 \x &  \x &  \x &  &  &  &  \x &  \x & 157,71 & 25,42\% & 24 & 9 & 61\% \\ \hline
 \x &  &  &  \x &  \x &  &  \x &  \x & 158,06 & 25,69\% & 23 & 14 & 60\% \\ \hline
 \x &  \x &  &  \x &  &  &  &  \x & 158,3 & 25,88\% & 19 & 14 & 50\% \\ \hline
 \x &  &  &  \x &  &  &  \x &  \x & 158,4 & 25,96\% & 15 & 19 & 51\% \\ \hline
 \x &  &  &  &  &  \x &  \x &  \x & 158,51 & 26,05\% & 13 & 16 & 71\% \\ \hline
 \x &  \x &  \x &  \x &  \x &  &  \x &  \x & 159,52 & 26,85\% & 15 & 14 & 62\% \\ \hline
 \x &  &  \x &  \x &  &  &  \x &  \x & 159,55 & 26,88\% & 19 & 19 & 64\% \\ \hline
 \x &  &  \x &  \x &  \x &  &  \x &  \x & 160,37 & 27,53\% & 13 & 21 & 62\% \\ \hline
 \x &  \x &  &  \x &  \x &  \x &  &  \x & 160,53 & 27,66\% & 16 & 20 & 71\% \\ \hline
 \x &  &  &  &  \x &  \x &  \x &  \x & 160,56 & 27,68\% & 5 & 19 & 71\% \\ \hline
 \x &  \x &  \x &  \x &  \x &  \x &  \x &  \x & 160,62 & 27,73\% & 18 & 14 & 71\% \\ \hline
 \x &  &  \x &  \x &  \x &  \x &  \x &  \x & 161,1 & 28,11\% & 18 & 17 & 70\% \\ \hline
 \x &  &  &  \x &  \x &  &  &  \x & 161,21 & 28,2\% & 21 & 15 & 60\% \\ \hline
 \x &  &  \x &  &  \x &  \x &  \x &  \x & 161,23 & 28,21\% & 20 & 14 & 70\% \\ \hline
 \x &  &  \x &  &  \x &  &  \x &  \x & 161,32 & 28,29\% & 21 & 14 & 63\% \\ \hline
 \x &  &  &  \x &  \x &  &  \x &  \x & 165,26 & 31,42\% & 24 & 11 & 59\% \\ \hline
 \x &  &  &  \x &  \x &  \x &  \x &  \x & 166,87 & 32,7\% & 1 & 16 & 71\% \\ \hline
 \x &  &  &  \x &  &  \x &  \x &  \x & 167,74 & 33,39\% & 9 & 22 & 70\% \\ \hline
 \x &  &  &  \x &  &  &  \x &  \x & 167,83 & 33,46\% & 17 & 17 & 49\% \\ \hline
 \x &  \x &  &  \x &  \x &  \x &  \x &  \x & 168,73 & 34,18\% & 13 & 21 & 69\% \\ \hline
 \x &  \x &  &  \x &  \x &  &  \x &  \x & 168,81 & 34,24\% & 17 & 15 & 60\% \\ \hline
 \x &  \x &  \x &  &  \x &  \x &  \x &  \x & 169,41 & 34,72\% & 3 & 21 & 71\% \\ \hline
 \x &  \x &  \x &  \x &  \x &  &  \x &  \x & 169,83 & 35,05\% & 12 & 21 & 62\% \\ \hline
 \x &  &  &  \x &  \x &  \x &  \x &  \x & 169,92 & 35,13\% & 21 & 13 & 70\% \\ \hline
 \x &  \x &  &  \x &  &  \x &  \x &  \x & 170,93 & 35,93\% & 9 & 18 & 71\% \\ \hline
 \x &  &  \x &  \x &  \x &  \x &  \x &  \x & 171,61 & 36,47\% & 9 & 13 & 71\% \\ \hline
 \x &  &  \x &  \x &  &  \x &  \x &  \x & 172,72 & 37,35\% & 14 & 15 & 71\% \\ \hline
 \x &  &  &  &  \x &  \x &  \x &  \x & 173,12 & 37,67\% & 18 & 9 & 71\% \\ \hline
 \x &  &  \x &  \x &  &  \x &  \x &  \x & 173,65 & 38,09\% & 13 & 17 & 69\% \\ \hline
 \x &  \x &  \x &  \x &  \x &  \x &  \x &  \x & 174,26 & 38,58\% & 15 & 14 & 68\% \\ \hline
 \x &  \x &  &  \x &  &  \x &  \x &  \x & 174,55 & 38,81\% & 3 & 17 & 71\% \\ \hline
 \x &  &  &  \x &  &  \x &  \x &  \x & 174,85 & 39,05\% & 1 & 17 & 70\% \\ \hline
 \x &  \x &  \x &  &  \x &  \x &  \x &  \x & 180,89 & 43,85\% & 20 & 11 & 68\% \\ \hline
 \x &  \x &  &  &  \x &  \x &  \x &  \x & 199,22 & 58,43\% & 18 & 19 & 70\% \\ \hline
\end{longtable}
\label{table:windProdInputParams}
\end{center}
\normalsize

\subsubsection{Best Input Combination Prediction Graphs}
\label{sec:bestCombiPredictionsGraphs}
The best prediction graphs for the input combination experiment.

\begin{sidewaysfigure}[h!]
\centering
\includegraphics[width=0.99\linewidth]{billeder/bestInputCombi0-2500.png}
\caption{Wind Power prediction for 0-2500 hours in 2012 with the best combination}
\label{fig:bestInputCombi0-2500}
\end{sidewaysfigure} 

\begin{sidewaysfigure}[h!]
\centering
\includegraphics[width=0.99\linewidth]{billeder/bestInputCombi2500-5000.png}
\caption{Wind Power prediction for 2500-5000 hours in 2012 with the best combination}
\label{fig:bestInputCombi2500-5000}
\end{sidewaysfigure} 

\subsection{Experiment Two:}

\subsubsection{Simple input test with seasonality}
\label{sec:simpleInputTestSeason}
The simple input test based on 3 months before the prediction day and the same month in the previous year. All with 200 epochs.

WS = Wind Speed
AD = Air Density
C = Consumption
T = Temperature
WD = Wind Direction
L-P = Last Known Production
D = Date
ToD = Time of Day
M = Time of Day as matrix
CD = Correct direction in \%

\footnotesize
\begin{center}
\begin{longtable}{|c|c|c|c|c|c|c|c|c|c|c|c|c|}
\caption{Wind Production Input Parameter Test}\\
\hline
\textbf{WS} & \textbf{AD} & \textbf{C} & \textbf{T} & \textbf{WD} & \textbf{L-P} & \textbf{D}& \textbf{ToD} & \textbf{MAE} & \textbf{\% from \#1} & \textbf{H1} & \textbf{H2} & \textbf{H2} \\
\hline
\endfirsthead
\multicolumn{13}{c}%
{\tablename\ \thetable\ -- \textit{Continued from previous page}} \\
\hline
\textbf{WS} & \textbf{AD} & \textbf{C} & \textbf{T} & \textbf{WD} & \textbf{L-P} & \textbf{D}& \textbf{ToD} & \textbf{MAE} & \textbf{\% from \#1} & \textbf{H1} & \textbf{H2} & \textbf{CD} \\
\hline
\endhead
\hline \multicolumn{13}{r}{\textit{Continued on next page}} \\
\endfoot
\hline
\endlastfoot
\arrayrulecolor{light-gray}
 \x &  \x &  \x &  &  \x &  \x &  &  \x & 142,88 & 0,0\% & 7 & 17 & 73\% \\ \hline
 \x &  &  &  \x &  \x &  \x &  &  \x & 142,89 & 0,01\% & 14 & 11 & 73\% \\ \hline
 \x &  \x &  &  &  \x &  \x &  &  \x & 143,37 & 0,34\% & 3 & 21 & 73\% \\ \hline
 \x &  \x &  \x &  \x &  \x &  \x &  &  \x & 143,97 & 0,76\% & 4 & 25 & 72\% \\ \hline
 \x &  &  &  &  &  \x &  &  \x & 143,98 & 0,77\% & 2 & 21 & 72\% \\ \hline
 \x &  \x &  \x &  \x &  &  \x &  \x &  \x & 144,11 & 0,86\% & 1 & 17 & 73\% \\ \hline
 \x &  \x &  &  &  &  \x &  &  \x & 144,12 & 0,87\% & 7 & 12 & 73\% \\ \hline
 \x &  &  &  &  &  &  &  \x & 144,28 & 0,98\% & 12 & 17 & 41\% \\ \hline
 \x &  &  \x &  &  \x &  \x &  &  \x & 144,42 & 1,08\% & 11 & 10 & 72\% \\ \hline
 \x &  \x &  &  \x &  \x &  \x &  &  \x & 144,48 & 1,12\% & 1 & 17 & 73\% \\ \hline
 \x &  &  \x &  \x &  &  \x &  &  \x & 144,7 & 1,27\% & 21 & 6 & 72\% \\ \hline
 \x &  \x &  \x &  \x &  &  \x &  &  \x & 144,79 & 1,34\% & 18 & 13 & 72\% \\ \hline
 \x &  &  &  &  &  \x &  &  \x & 144,93 & 1,43\% & 18 & 1 & 73\% \\ \hline
 \x &  &  \x &  \x &  \x &  \x &  \x &  \x & 144,98 & 1,47\% & 14 & 9 & 73\% \\ \hline
 \x &  &  \x &  \x &  &  \x &  \x &  \x & 145,02 & 1,5\% & 18 & 11 & 73\% \\ \hline
 \x &  \x &  \x &  \x &  &  \x &  \x &  \x & 145,19 & 1,62\% & 2 & 15 & 73\% \\ \hline
 \x &  &  &  \x &  &  \x &  \x &  \x & 145,29 & 1,69\% & 13 & 6 & 73\% \\ \hline
 \x &  \x &  &  &  &  &  &  \x & 145,51 & 1,84\% & 1 & 22 & 54\% \\ \hline
 \x &  &  \x &  &  \x &  \x &  &  \x & 145,76 & 2,02\% & 13 & 17 & 73\% \\ \hline
 \x &  \x &  \x &  &  \x &  &  &  \x & 145,79 & 2,04\% & 18 & 9 & 62\% \\ \hline
 \x &  \x &  \x &  &  \x &  \x &  \x &  \x & 145,98 & 2,17\% & 1 & 9 & 73\% \\ \hline
 \x &  &  \x &  &  &  &  &  \x & 145,99 & 2,18\% & 16 & 13 & 64\% \\ \hline
 \x &  &  &  &  \x &  &  &  \x & 146,24 & 2,35\% & 24 & 9 & 61\% \\ \hline
 \x &  \x &  \x &  &  \x &  \x &  &  \x & 146,49 & 2,53\% & 2 & 13 & 72\% \\ \hline
 \x &  &  \x &  &  &  \x &  &  \x & 146,53 & 2,55\% & 1 & 18 & 73\% \\ \hline
 \x &  \x &  \x &  &  &  &  &  \x & 146,57 & 2,58\% & 24 & 11 & 64\% \\ \hline
 \x &  \x &  \x &  &  \x &  \x &  \x &  \x & 146,61 & 2,61\% & 5 & 15 & 72\% \\ \hline
 \x &  \x &  &  \x &  \x &  \x &  \x &  \x & 146,64 & 2,63\% & 1 & 13 & 73\% \\ \hline
 \x &  \x &  \x &  \x &  \x &  \x &  \x &  \x & 146,78 & 2,73\% & 2 & 16 & 73\% \\ \hline
 \x &  \x &  \x &  \x &  \x &  \x &  \x &  \x & 146,89 & 2,81\% & 4 & 16 & 72\% \\ \hline
 \x &  \x &  &  &  &  &  &  \x & 146,96 & 2,86\% & 25 & 5 & 54\% \\ \hline
 \x &  \x &  \x &  &  &  \x &  &  \x & 147,18 & 3,01\% & 14 & 15 & 72\% \\ \hline
 \x &  &  \x &  \x &  &  \x &  &  \x & 147,3 & 3,09\% & 17 & 7 & 72\% \\ \hline
 \x &  &  &  &  &  \x &  \x &  \x & 147,64 & 3,33\% & 11 & 11 & 73\% \\ \hline
 \x &  &  \x &  &  &  &  &  \x & 147,7 & 3,37\% & 11 & 22 & 63\% \\ \hline
 \x &  \x &  \x &  &  &  &  &  \x & 147,86 & 3,49\% & 10 & 25 & 63\% \\ \hline
 \x &  &  &  &  \x &  \x &  \x &  \x & 147,95 & 3,55\% & 4 & 17 & 73\% \\ \hline
 \x &  &  \x &  &  &  \x &  &  \x & 147,98 & 3,57\% & 14 & 17 & 72\% \\ \hline
 \x &  &  &  &  \x &  \x &  \x &  \x & 148,23 & 3,74\% & 1 & 23 & 73\% \\ \hline
 \x &  \x &  &  \x &  &  \x &  \x &  \x & 148,28 & 3,78\% & 5 & 13 & 73\% \\ \hline
 \x &  &  &  \x &  \x &  \x &  \x &  \x & 148,4 & 3,86\% & 13 & 13 & 73\% \\ \hline
 \x &  \x &  &  \x &  &  \x &  &  \x & 148,57 & 3,98\% & 21 & 14 & 72\% \\ \hline
 \x &  \x &  \x &  \x &  &  &  &  \x & 148,66 & 4,05\% & 14 & 20 & 63\% \\ \hline
 \x &  \x &  \x &  \x &  &  &  &  \x & 149,0 & 4,28\% & 7 & 25 & 63\% \\ \hline
 \x &  \x &  &  &  \x &  &  &  \x & 149,04 & 4,31\% & 16 & 12 & 61\% \\ \hline
 \x &  &  &  \x &  &  \x &  &  \x & 149,08 & 4,34\% & 19 & 15 & 72\% \\ \hline
 \x &  &  \x &  &  &  \x &  \x &  \x & 149,13 & 4,37\% & 3 & 16 & 73\% \\ \hline
 \x &  \x &  \x &  &  &  \x &  \x &  \x & 149,28 & 4,48\% & 2 & 16 & 73\% \\ \hline
 \x &  \x &  &  \x &  \x &  \x &  &  \x & 149,31 & 4,5\% & 1 & 17 & 72\% \\ \hline
 \x &  &  \x &  &  \x &  &  &  \x & 149,4 & 4,56\% & 14 & 18 & 63\% \\ \hline
 \x &  \x &  \x &  &  &  \x &  &  \x & 149,95 & 4,95\% & 1 & 16 & 73\% \\ \hline
 \x &  &  \x &  &  \x &  \x &  \x &  \x & 150,28 & 5,18\% & 1 & 22 & 72\% \\ \hline
 \x &  \x &  \x &  \x &  &  \x &  &  \x & 150,33 & 5,21\% & 9 & 19 & 72\% \\ \hline
 \x &  \x &  &  &  &  \x &  \x &  \x & 150,36 & 5,24\% & 4 & 15 & 73\% \\ \hline
 \x &  \x &  &  &  &  \x &  &  \x & 150,98 & 5,67\% & 21 & 1 & 72\% \\ \hline
 \x &  &  &  &  \x &  &  &  \x & 151,09 & 5,75\% & 11 & 21 & 61\% \\ \hline
 \x &  &  \x &  \x &  &  &  &  \x & 151,13 & 5,77\% & 20 & 10 & 63\% \\ \hline
 \x &  &  &  \x &  &  &  &  \x & 151,3 & 5,89\% & 20 & 15 & 50\% \\ \hline
 \x &  \x &  &  \x &  &  \x &  &  \x & 151,45 & 6,0\% & 7 & 20 & 72\% \\ \hline
 \x &  \x &  \x &  \x &  \x &  \x &  &  \x & 151,46 & 6,01\% & 10 & 24 & 72\% \\ \hline
 \x &  \x &  &  &  \x &  \x &  \x &  \x & 151,48 & 6,02\% & 2 & 16 & 73\% \\ \hline
 \x &  \x &  &  \x &  \x &  \x &  \x &  \x & 151,67 & 6,15\% & 4 & 17 & 72\% \\ \hline
 \x &  &  \x &  \x &  &  &  &  \x & 151,7 & 6,17\% & 17 & 13 & 63\% \\ \hline
 \x &  &  \x &  \x &  \x &  \x &  \x &  \x & 151,76 & 6,22\% & 2 & 16 & 72\% \\ \hline
 \x &  &  \x &  \x &  \x &  \x &  &  \x & 151,91 & 6,32\% & 1 & 9 & 71\% \\ \hline
 \x &  &  \x &  \x &  &  \x &  \x &  \x & 152,06 & 6,42\% & 7 & 22 & 72\% \\ \hline
 \x &  \x &  &  &  \x &  &  &  \x & 152,25 & 6,56\% & 19 & 12 & 61\% \\ \hline
 \x &  &  &  &  \x &  \x &  &  \x & 152,28 & 6,58\% & 19 & 14 & 72\% \\ \hline
 \x &  \x &  &  \x &  &  &  &  \x & 152,35 & 6,63\% & 23 & 12 & 54\% \\ \hline
 \x &  \x &  &  &  &  \x &  \x &  \x & 152,36 & 6,63\% & 10 & 22 & 73\% \\ \hline
 \x &  \x &  \x &  \x &  \x &  &  \x &  \x & 152,39 & 6,66\% & 18 & 9 & 63\% \\ \hline
 \x &  &  &  \x &  \x &  \x &  &  \x & 152,43 & 6,68\% & 6 & 23 & 72\% \\ \hline
 \x &  \x &  &  &  \x &  \x &  &  \x & 153,17 & 7,2\% & 2 & 10 & 72\% \\ \hline
 \x &  \x &  \x &  &  &  &  \x &  \x & 153,2 & 7,22\% & 7 & 13 & 63\% \\ \hline
 \x &  &  \x &  \x &  \x &  &  \x &  \x & 153,3 & 7,29\% & 13 & 20 & 63\% \\ \hline
 \x &  &  \x &  \x &  &  &  \x &  \x & 153,52 & 7,45\% & 16 & 15 & 63\% \\ \hline
 \x &  &  &  \x &  &  &  &  \x & 153,6 & 7,5\% & 22 & 14 & 51\% \\ \hline
 \x &  \x &  \x &  &  &  \x &  \x &  \x & 153,82 & 7,66\% & 8 & 14 & 73\% \\ \hline
 \x &  &  \x &  \x &  \x &  &  &  \x & 153,84 & 7,67\% & 9 & 20 & 62\% \\ \hline
 \x &  &  &  &  \x &  \x &  &  \x & 153,95 & 7,75\% & 15 & 12 & 73\% \\ \hline
 \x &  &  \x &  &  \x &  &  &  \x & 154,16 & 7,89\% & 14 & 13 & 61\% \\ \hline
 \x &  \x &  &  \x &  &  &  &  \x & 154,18 & 7,91\% & 12 & 24 & 54\% \\ \hline
 \x &  \x &  \x &  \x &  &  &  \x &  \x & 154,84 & 8,37\% & 13 & 18 & 63\% \\ \hline
 \x &  &  &  \x &  \x &  \x &  \x &  \x & 154,94 & 8,44\% & 1 & 18 & 72\% \\ \hline
 \x &  \x &  \x &  \x &  &  &  \x &  \x & 154,95 & 8,45\% & 2 & 21 & 62\% \\ \hline
 \x &  \x &  &  \x &  &  &  \x &  \x & 155,0 & 8,48\% & 18 & 12 & 54\% \\ \hline
 \x &  &  &  &  &  &  &  \x & 155,07 & 8,53\% & 7 & 21 & 41\% \\ \hline
 \x &  \x &  \x &  &  \x &  &  \x &  \x & 155,46 & 8,8\% & 16 & 15 & 63\% \\ \hline
 \x &  &  \x &  \x &  \x &  &  \x &  \x & 155,7 & 8,97\% & 16 & 18 & 62\% \\ \hline
 \x &  \x &  \x &  \x &  \x &  &  &  \x & 155,79 & 9,04\% & 19 & 11 & 62\% \\ \hline
 \x &  \x &  \x &  &  &  &  \x &  \x & 155,82 & 9,06\% & 7 & 22 & 62\% \\ \hline
 \x &  &  \x &  \x &  &  &  \x &  \x & 155,89 & 9,11\% & 13 & 22 & 63\% \\ \hline
 \x &  \x &  &  &  &  &  \x &  \x & 156,04 & 9,21\% & 22 & 8 & 54\% \\ \hline
 \x &  &  &  \x &  &  \x &  &  \x & 156,14 & 9,28\% & 1 & 21 & 72\% \\ \hline
 \x &  \x &  &  &  \x &  &  \x &  \x & 156,34 & 9,42\% & 14 & 16 & 61\% \\ \hline
 \x &  &  &  \x &  \x &  &  \x &  \x & 156,59 & 9,6\% & 12 & 22 & 61\% \\ \hline
 \x &  &  &  &  &  \x &  \x &  \x & 156,6 & 9,6\% & 22 & 1 & 72\% \\ \hline
 \x &  &  \x &  &  \x &  \x &  \x &  \x & 156,84 & 9,77\% & 1 & 10 & 71\% \\ \hline
 \x &  &  \x &  &  &  &  \x &  \x & 156,89 & 9,81\% & 14 & 19 & 62\% \\ \hline
 \x &  \x &  &  \x &  \x &  &  \x &  \x & 157,48 & 10,22\% & 12 & 17 & 61\% \\ \hline
 \x &  &  &  \x &  &  &  \x &  \x & 157,61 & 10,31\% & 15 & 14 & 50\% \\ \hline
 \x &  &  &  \x &  \x &  &  &  \x & 158,37 & 10,84\% & 25 & 9 & 60\% \\ \hline
 \x &  \x &  &  &  \x &  &  \x &  \x & 158,46 & 10,9\% & 5 & 23 & 61\% \\ \hline
 \x &  \x &  &  \x &  \x &  &  &  \x & 158,64 & 11,03\% & 22 & 9 & 61\% \\ \hline
 \x &  \x &  &  &  &  &  \x &  \x & 158,74 & 11,1\% & 24 & 1 & 54\% \\ \hline
 \x &  \x &  &  &  \x &  \x &  \x &  \x & 159,01 & 11,29\% & 2 & 23 & 72\% \\ \hline
 \x &  &  \x &  &  \x &  &  \x &  \x & 159,25 & 11,46\% & 12 & 23 & 62\% \\ \hline
 \x &  \x &  \x &  \x &  \x &  &  \x &  \x & 159,32 & 11,51\% & 14 & 24 & 62\% \\ \hline
 \x &  \x &  \x &  &  \x &  &  \x &  \x & 159,44 & 11,59\% & 14 & 12 & 63\% \\ \hline
 \x &  &  \x &  &  &  \x &  \x &  \x & 159,6 & 11,7\% & 2 & 13 & 71\% \\ \hline
 \x &  \x &  \x &  &  \x &  &  &  \x & 159,74 & 11,8\% & 12 & 14 & 61\% \\ \hline
 \x &  &  &  \x &  &  &  \x &  \x & 159,94 & 11,94\% & 11 & 15 & 51\% \\ \hline
 \x &  \x &  &  \x &  &  \x &  \x &  \x & 160,11 & 12,06\% & 10 & 14 & 71\% \\ \hline
 \x &  \x &  \x &  \x &  \x &  &  &  \x & 160,55 & 12,37\% & 15 & 19 & 62\% \\ \hline
 \x &  &  &  &  \x &  &  \x &  \x & 160,79 & 12,53\% & 17 & 13 & 61\% \\ \hline
 \x &  &  &  &  &  &  \x &  \x & 160,98 & 12,67\% & 10 & 21 & 41\% \\ \hline
 \x &  \x &  &  \x &  \x &  &  \x &  \x & 161,12 & 12,77\% & 13 & 13 & 61\% \\ \hline
 \x &  &  &  &  &  &  \x &  \x & 161,18 & 12,81\% & 18 & 12 & 41\% \\ \hline
 \x &  &  \x &  &  &  &  \x &  \x & 161,25 & 12,86\% & 3 & 25 & 62\% \\ \hline
 \x &  &  \x &  \x &  \x &  &  &  \x & 161,5 & 13,03\% & 18 & 17 & 62\% \\ \hline
 \x &  &  &  &  \x &  &  \x &  \x & 161,59 & 13,09\% & 16 & 20 & 60\% \\ \hline
 \x &  \x &  &  \x &  &  &  \x &  \x & 162,1 & 13,45\% & 6 & 22 & 54\% \\ \hline
 \x &  &  \x &  &  \x &  &  \x &  \x & 162,54 & 13,76\% & 12 & 13 & 62\% \\ \hline
 \x &  \x &  &  \x &  \x &  &  &  \x & 162,55 & 13,77\% & 19 & 15 & 60\% \\ \hline
 \x &  &  &  \x &  &  \x &  \x &  \x & 162,57 & 13,78\% & 4 & 23 & 71\% \\ \hline
 \x &  &  &  \x &  \x &  &  &  \x & 162,87 & 13,99\% & 15 & 19 & 60\% \\ \hline
 \x &  &  &  \x &  \x &  &  \x &  \x & 163,34 & 14,32\% & 15 & 19 & 61\% \\ \hline
 \x &  &  \x &  \x &  \x &  \x &  &  \x & 165,37 & 15,74\% & 13 & 18 & 71\% \\ \hline
\end{longtable}
\label{table:windProdInputParamsSeasonal}
\end{center}
\normalsize

\subsubsection{Historical volatility results}
\label{sec:historicalVolatiltiyResultsAppendix}
Historical volatility results with all combinations of previous hours between 4-24 and different smoothing factors.

\footnotesize
\begin{center}
\begin{longtable}{|c|c|c|c|}
\caption{Wind Production Input Parameter Test}\\
\hline
\textbf{Previous Hours} & \textbf{Smoothing factor} & \textbf{MAE} & \textbf{\% Deviation} \\
\hline
\endfirsthead
\multicolumn{4}{c}%
{\tablename\ \thetable\ -- \textit{Continued from previous page}} \\
\hline
\textbf{Previous Hours} & \textbf{Smoothing factor} & \textbf{MAE} & \textbf{\% Deviation} \\
\hline
\endhead
\hline \multicolumn{4}{r}{\textit{Continued on next page}} \\
\endfoot
\hline
\endlastfoot
\arrayrulecolor{light-gray}
6 & 0,70 & 121,81 & 0,0\% \\ \hline
20 & 0,80 & 122,9 & 0,89\% \\ \hline
24 & 0,60 & 123,13 & 1,08\% \\ \hline
24 & 0,80 & 124,02 & 1,81\% \\ \hline
16 & 0,20 & 125,65 & 3,15\% \\ \hline
12 & 0,30 & 127,0 & 4,26\% \\ \hline
24 & 0,70 & 127,06 & 4,31\% \\ \hline
16 & 0, 60 & 127,38 & 4,57\% \\ \hline
24 & 0,40 & 127,51 & 4,68\% \\ \hline
16 & 0,90 & 127,77 & 4,89\% \\ \hline
16 & 0,30 & 127,86 & 4,97\% \\ \hline
4 & 0,60 & 128,49 & 5,48\% \\ \hline
4 & 0,10 & 128,51 & 5,5\% \\ \hline
20 & 0,50 & 129,36 & 6,2\% \\ \hline
8 & 0,10 & 130,06 & 6,77\% \\ \hline
24 & 0,20 & 131,05 & 7,59\% \\ \hline
12 & 0,90 & 131,87 & 8,26\% \\ \hline
20 & 0,10 & 132,63 & 8,88\% \\ \hline
20 & 0,30 & 133,32 & 9,45\% \\ \hline
6 & 0,20 & 133,67 & 9,74\% \\ \hline
4 & 0,30 & 133,84 & 9,88\% \\ \hline
12 & 0,60 & 134,02 & 10,02\% \\ \hline
4 & 0,20 & 134,18 & 10,16\% \\ \hline
16 & 0,40 & 134,28 & 10,24\% \\ \hline
4 & 0,40 & 134,71 & 10,59\% \\ \hline
12 & 0,40 & 135,05 & 10,87\% \\ \hline
16 & 0,80 & 135,14 & 10,94\% \\ \hline
20 & 0,20 & 135,3 & 11,07\% \\ \hline
12 & 0,50 & 135,42 & 11,17\% \\ \hline
24 & 0,90 & 135,71 & 11,41\% \\ \hline
16 & 0,50 & 136,44 & 12,01\% \\ \hline
4 & 0,50 & 136,58 & 12,13\% \\ \hline
6 & 0,30 & 136,6 & 12,14\% \\ \hline
20 & 0,70 & 136,63 & 12,17\% \\ \hline
20 & 0,60 & 136,86 & 12,36\% \\ \hline
12 & 0,20 & 136,98 & 12,45\% \\ \hline
6 & 0,10 & 137,16 & 12,6\% \\ \hline
24 & 0,10 & 137,34 & 12,75\% \\ \hline
6 & 0,50 & 137,45 & 12,84\% \\ \hline
24 & 0,50 & 137,91 & 13,22\% \\ \hline
20 & 0,90 & 137,92 & 13,23\% \\ \hline
8 & 0,20 & 138,26 & 13,5\% \\ \hline
8 & 0,40 & 138,69 & 13,86\% \\ \hline
6 & 0,60 & 138,87 & 14,01\% \\ \hline
8 & 0,50 & 138,93 & 14,05\% \\ \hline
8 & 0,60 & 139,2 & 14,28\% \\ \hline
8 & 0,90 & 139,84 & 14,8\% \\ \hline
8 & 0,70 & 140,01 & 14,94\% \\ \hline
24 & 0,30 & 140,45 & 15,3\% \\ \hline
6 & 0,70 & 141,3 & 16,0\% \\ \hline
8 & 0,30 & 141,6 & 16,25\% \\ \hline
4 & 0,70 & 142,19 & 16,73\% \\ \hline
12 & 0,80 & 142,2 & 16,74\% \\ \hline
6 & 0,40 & 142,98 & 17,38\% \\ \hline
6 & 0,80 & 143,21 & 17,57\% \\ \hline
20 & 0,40 & 143,59 & 17,88\% \\ \hline
8 & 0,80 & 143,62 & 17,9\% \\ \hline
12 & 0,10 & 144,37 & 18,52\% \\ \hline
16 & 0,70 & 144,43 & 18,57\% \\ \hline
16 & 0,10 & 146,77 & 20,49\% \\ \hline
4 & 0,80 & 146,9 & 20,6\% \\ \hline
6 & 0,90 & 147,48 & 21,07\% \\ \hline
4 & 0,90 & 151,15 & 24,09\% \\ \hline
\end{longtable}
\label{table:historicalVolatilityAppendixTable}
\end{center}
\normalsize 	


\end{document}

